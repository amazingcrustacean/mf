% Authors:  Nils Carqueville, Daniel Murfet
 
\documentclass{compositio}
\usepackage{stmaryrd}
\usepackage{amsmath, amscd, amssymb, mathrsfs, accents, amsfonts}
\usepackage{url}
\usepackage[all]{xy}
\usepackage{longtable}
\usepackage{dsfont}
\usepackage{tikz}
\def\nicenocolourscheme{\shadedraw[top color=gray!2, bottom color=gray!25, draw=gray!50!black, dashed]}
\definecolor{Myblue}{rgb}{0,0,0.6}
\usepackage[a4paper,colorlinks,citecolor=Myblue,linkcolor=Myblue,urlcolor=Myblue,pdfpagemode=None]{hyperref}

\SelectTips{cm}{}

\newtheorem{theorem}{Theorem}[section]
\newtheorem{proposition}[theorem]{Proposition}
\newtheorem{lemma}[theorem]{Lemma}
\newtheorem{corollary}[theorem]{Corollary}
\newtheorem*{theoremn}{Theorem}

\theoremstyle{definition}
\newtheorem{definition}[theorem]{Definition}
\newtheorem{example}[theorem]{Example}
\newtheorem{remark}[theorem]{Remark}
\newtheorem{s}[theorem]{}
\newtheorem*{setup}{Setup}
\newtheorem*{propositionn}{Proposition}

\numberwithin{equation}{section}

% Operators
\def\eval{\operatorname{ev}}
\def\coev{\operatorname{coev}}
\def\res{\operatorname{Res}}
\def\sg{\operatorname{sg}}
\def\Inj{\operatorname{Inj}}
\def\inc{\operatorname{inc}}
\def\Proj{\operatorname{Proj}}
\def\Coker{\operatorname{Coker}}
\def\Ker{\operatorname{Ker}}
\def\Im{\operatorname{Im}}
\def\free{\operatorname{free}}
\def\can{\operatorname{can}}
\def\ac{\operatorname{ac}}
\def\HH{\operatorname{HH}}
\def\K{\mathbf{K}}
\def\D{\mathbf{D}}
\def\N{\mathbf{N}}
\def\sing{\operatorname{Sg}}
\def\Hom{\operatorname{Hom}}
\def\uHom{\underline{\Hom}}
\def\modd{\operatorname{mod}}
\def\Modd{\operatorname{Mod}}
\def\Grmodd{\operatorname{GrMod}}
\def\CM{\operatorname{CM}}
\def\Ker{\operatorname{Ker}}
\def\Spec{\operatorname{Spec}}
\def\straightK{\operatorname{K}}
\def\straightC{\operatorname{C}}
\def\holim{\operatorname{hocolim}}
\DeclareMathOperator{\Ext}{Ext}
\DeclareMathOperator{\coh}{coh}
\DeclareMathOperator{\serre}{S}
\DeclareMathOperator{\Flat}{Flat}
\DeclareMathOperator{\qc}{qc}
\DeclareMathOperator{\Perf}{Perf}
\DeclareMathOperator{\Map}{Map}
\DeclareMathOperator{\Qco}{Qco}
\DeclareMathOperator{\Tr}{Tr}
\DeclareMathOperator{\End}{End}
\DeclareMathOperator{\rank}{rank}
\DeclareMathOperator{\tot}{Tot}
\DeclareMathOperator{\skos}{K}
\DeclareMathOperator{\hht}{ht}
\DeclareMathOperator{\depth}{depth}
\DeclareMathOperator{\STr}{STr}
\DeclareMathOperator{\tr}{tr}
\DeclareMathOperator{\ch}{ch}
\DeclareMathOperator{\str}{str}
\DeclareMathOperator{\hmf}{hmf}
\DeclareMathOperator{\HMF}{HMF}
\DeclareMathOperator{\HF}{HF}
\DeclareMathOperator{\pr}{pr}
\DeclareMathOperator{\At}{At}
\DeclareMathOperator{\mff}{mf}
\DeclareMathOperator{\MF}{MF}
\DeclareMathOperator{\Sh}{Sh}

\begin{document}

% Commands
\def\Res{\res\!}
\newcommand{\cat}[1]{\mathcal{#1}}
\newcommand{\lto}{\longrightarrow}
\newcommand{\xlto}[1]{\stackrel{#1}\lto}
\newcommand{\mf}[1]{\mathfrak{#1}}
\newcommand{\md}[1]{\mathscr{#1}}
\newcommand{\intvar}{\bs{x}_{\textup{int}}}
\newcommand{\extvar}{\bs{x}_{\textup{ext}}}
\newcommand{\qderu}[2]{\mathbf{D}^{#1}(#2)}
\newcommand{\ud}{\mathrm{d}}
\def\l{\,|\,}
\def\cf{\boldsymbol{cf}}
\def\bx{\boldsymbol{x}}
\def\by{\boldsymbol{y}}
\def\ba{\boldsymbol{a}}
\def\bb{\boldsymbol{b}}
\def\totimes{\otimes}
\def\di{Q}
\newcommand{\cotimes}[1]{\,\widehat{\otimes}_{#1}\,}
\def\QQ{\mathds{Q}}
\def\krc{C}
\def\diffm{d}
\def\diffh{d_{\chi}}
\def\redh{\overline{H}}
\def\ZZ{\mathds{Z}}
\def\bs{\boldsymbol}
\def\Ztwo{\mathds{Z}_2}
\def\mdual{^{\vee}}
\def\KR{\operatorname{KR}}
\def\I{\!\operatorname{i}\!}
\def\E{\operatorname{e}\!}
\def\sln{\mathfrak{sl}(N)}
\def\nN{\mathds{N}}
\def\nZ{\mathds{Z}}
\def\nQ{\mathds{Q}}
\def\nR{\mathds{R}}
\def\nC{\mathds{C}}
\def\Bar{\mathds{B}}
\def\cBar{\widehat{\mathds{B}}}
\def\Re{R^{\operatorname{e}}}
\def\lra{\longrightarrow}
\def\lmt{\longmapsto}
\def\LG{\mathcal{LG}_k}
\def\dual{\dagger}
\def\dlangle{\big\langle\!\big\langle}
\def\drangle{\big\rangle\!\big\rangle}
\def\bigdlangle{\Big\langle\!\Big\langle}
\def\bigdrangle{\Big\rangle\!\Big\rangle}

\newcommand{\be}{\begin{equation}}
\newcommand{\ee}{\end{equation}}
\def\Xcirc{%
\begin{tikzpicture}[inner sep=0mm]
\node (X) at (0,0) {$X$};
\node (0) at (0,0) [circle,inner sep=0.99pt, thin,draw=black,fill= white] {};
\end{tikzpicture}%
}
\def\Xbul{%
\begin{tikzpicture}[inner sep=0mm]
\node (X) at (0,0) {$X$};
\node (0) at (0,0) [circle,inner sep=0.99pt, thin,draw=black,fill= black] {};
\end{tikzpicture}%
}

\allowdisplaybreaks

\usetikzlibrary{arrows,calc,decorations.pathreplacing,decorations.markings,shapes.geometric,shadows}
\tikzset{
    string/.style={draw=#1, postaction={decorate}, decoration={markings,mark=at position .51 with {\arrow[draw=#1]{>}}}},
    costring/.style={draw=#1, postaction={decorate}, decoration={markings,mark=at position .51 with {\arrow[draw=#1]{<}}}},
    ostring/.style={draw=#1, postaction={decorate}, decoration={markings,mark=at position .47 with {\arrow[draw=#1]{>}}}},
    ustring/.style={draw=#1, postaction={decorate}, decoration={markings,mark=at position .56 with {\arrow[draw=#1]{>}}}},
    oostring/.style={draw=#1, postaction={decorate}, decoration={markings,mark=at position .43 with {\arrow[draw=#1]{>}}}},
    uustring/.style={draw=#1, postaction={decorate}, decoration={markings,mark=at position .59 with {\arrow[draw=#1]{>}}}},
    directed/.style={string=blue!50!black}, 
    odirected/.style={ostring=blue!50!black}, 
    udirected/.style={ustring=blue!50!black}, 
    oodirected/.style={oostring=blue!50!black}, 
    uudirected/.style={uustring=blue!50!black},     
    redirected/.style={costring= blue!50!black},
}

\usetikzlibrary{fadings,decorations.pathreplacing}

\newcommand\pgfmathsinandcos[3]{%
  \pgfmathsetmacro#1{sin(#3)}%
  \pgfmathsetmacro#2{cos(#3)}%
}
\newcommand\LongitudePlane[3][current plane]{%
  \pgfmathsinandcos\sinEl\cosEl{#2} % elevation
  \pgfmathsinandcos\sint\cost{#3} % azimuth
  \tikzset{#1/.estyle={cm={\cost,\sint*\sinEl,0,\cosEl,(0,0)}}}
}
\newcommand\LatitudePlane[3][current plane]{%
  \pgfmathsinandcos\sinEl\cosEl{#2} % elevation
  \pgfmathsinandcos\sint\cost{#3} % latitude
  \pgfmathsetmacro\yshift{\cosEl*\sint}
  \tikzset{#1/.estyle={cm={\cost,0,0,\cost*\sinEl,(0,\yshift)}}} %
}
\newcommand\DrawLongitudeCircle[2][1]{
  \LongitudePlane{\angEl}{#2}
  \tikzset{current plane/.prefix style={scale=#1}}
  \pgfmathsetmacro\angVis{atan(sin(#2)*cos(\angEl)/sin(\angEl))} %
  \draw[redirected,current plane,color=blue!50!black, very thick] (\angVis:1) arc (\angVis:\angVis+180:1);
  \draw[current plane,dotted,color=blue!50!gray, very thick] (\angVis-180:1) arc (\angVis-180:\angVis:1);
}
\newcommand\DrawLatitudeCircle[2][1]{
  \LatitudePlane{\angEl}{#2}
  \tikzset{current plane/.prefix style={scale=#1}}
  \pgfmathsetmacro\sinVis{sin(#2)/cos(#2)*sin(\angEl)/cos(\angEl)}
  \pgfmathsetmacro\angVis{asin(min(1,max(\sinVis,-1)))}
  \draw[directed,current plane, color=blue!50!black] (\angVis:1) arc (\angVis:-\angVis-180:1);
  \draw[current plane,dashed, color=blue!50!gray] (180-\angVis:1) arc (180-\angVis:\angVis:1);
}
\newcommand\DrawLatitudeCircleU[2][1]{
  \LatitudePlane{\angEl}{#2}
  \tikzset{current plane/.prefix style={scale=#1}}
  \pgfmathsetmacro\sinVis{sin(#2)/cos(#2)*sin(\angEl)/cos(\angEl)}
  \pgfmathsetmacro\angVis{asin(min(1,max(\sinVis,-1)))}
  \draw[redirected,current plane, color=blue!50!black] (\angVis:1) arc (\angVis:-\angVis-180:1);
  \draw[current plane,dashed, color=blue!50!gray] (180-\angVis:1) arc (180-\angVis:\angVis:1);
}





\title{Rise of the Planet of the Coevaluations}
\author{Nils Carqueville}
\email{nils.carqueville@physik.uni-muenchen.de}
\address{Arnold Sommerfeld Center for Theoretical Physics, LMU M\"unchen \& Excellence Cluster Universe}

\author{Daniel Murfet}
\email{daniel.murfet@math.ucla.edu}
\address{Department of Mathematics, UCLA}

\classification{TODO}

\begin{abstract}
They take over. 
\end{abstract}

\maketitle


\section{Background}\label{sec:Background}

\subsection{Bicategory of Landau-Ginzburg models}\label{subsec:bicatLG}

Let us fix the ring $R=k[x_1,\ldots,x_n]$, where~$k$ is a commutative noetherian $\nQ$-algebra; relevant examples are $k=\nC$ and $k=\nC[t_1,\ldots,t_d]$. Typically we will simply write $k[x]$ for $k[x_1,\ldots,x_n]$. We call an element $W\in R$ a \textsl{potential} if $R/(\partial_{x_i}W)$ is finitely generated projective over~$k$ and the $\partial_{x_i}W$ form a regular sequence. 

A \textsl{linear factorisation} of a potential $W\in R$ is a $\nZ_2$-graded $R$-module $X=X^0\oplus X^1$ together with an odd $R$-linear endomorphism~$d_X$ such that $d_X^2=W\cdot 1_X$. If~$X$ is a free $R$-module then the pair $(X,d_X)$ is called a \textsl{matrix factorisation}, and we often refer to it simply by~$X$ without explicitly mentioning the \textsl{differential} $d_X$. 

A \textsl{morphism} of linear factorisations $(X,d_X)$ and $(Y,d_Y)$ is an even $R$-linear map $\varphi: X \longrightarrow Y$ such that $d_Y \varphi = \varphi d_X$. Two morphisms $\varphi, \psi: X\lra Y$ are \textsl{homotopic} if there exists an odd $R$-linear map $\lambda:X\lra Y$ such that $d_Y\lambda + \lambda d_X = \psi-\varphi$. Equality up to homotopy is an equivalence relation. 

The \textsl{(homotopy) category of linear factorisations} $\HF(R,W)$ is the category of linear factorisations of $W\in R$ modulo homotopy relations. We denote by $\HMF(R,W)$ its full subcategory of matrix factorisations, and we write $\hmf(R,W)$ for the full subcategory of matrix factorisations whose objects are isomorphic to matrix factorisations with finitely-generated underlying $R$-modules. These three categories have standard triangulated structures whose shift functor we denote as~$[1]$. 

Given a potential $W\in R=k[x]$ there is always the \textsl{unit matrix factorisation} $\Delta_W \in \hmf(\Re, \widetilde W)$ where $\Re = R\otimes_k R$ and $\widetilde W = W\otimes 1 - 1\otimes W$. Introducing~$n$ formal symbols $\theta_i$ as a convenient notational device, we have 
$$
\Delta_W = \bigwedge \Big( \bigoplus_{i=1}^n \Re \theta_i \Big)
$$
as an $\Re$-module whose $\nZ_2$-grading is given by $\theta$-degree modulo~2. We write $\pi: \Delta_W \lra \Re$ for the projection to the $\theta$-degree~$0$ component and $\mu:\Re \lra R$ for multiplication. Typically we will omit the wedge product and write  e.\,g.~$\theta_i\wedge \theta_j$ simply as $\theta_i \theta_j$. To describe the differential $d_{\Delta_W}$ we further need the variable-changing map
$$
{}^{t_i}(-): k[x,y] \lra k[x,y] \, , \qquad p \lmt p\big|_{x_i\lmt y_i}
$$
in terms of which we can define difference quotient operators
$$
\partial_{[i]}: k[x,y] \lra k[x,y] \, , \qquad p \lmt \frac{{}^{t_1\ldots t_{i-1}}p - {}^{t_1\ldots t_i}p}{x_i-y_i} \, . 
$$
In terms of these the differential on $\Delta_W$ is given by
\be\label{DeltaW}
d_{\Delta_W} = \delta_+ + \delta_- \, , \qquad \delta_+ = \sum_{i=1}^n \partial_{[i]}W\cdot \theta_i\, , \qquad \delta_- =  \sum_{i=1}^n (x_i-y_i) \cdot \theta_i^*  \,. 
\ee

TODO: Put lemma on Leibniz rule for $\partial_{[i]}$ here or later? 

We call $\Delta_W$ the unit matrix factorisation as it is the unit with respect to the tensor product of matrix factorisations. For our purposes it will be sufficient to consider matrix factorisations $X\in \HMF(R_1\otimes_k R_2, W_2-W_1)$ and $Y\in \HMF(R_2\otimes_k R_3, W_3-W_2)$ (where of course $R_i = k$ is a possibility for any~$i$). Then the \textsl{tensor product matrix factorisation} $Y\otimes X \in \HMF(R_1\otimes_k R_3, W_3-W_1)$ is the module 
$$
Y\otimes X = \Big( (Y^0\otimes_{R_2} X^0) \oplus (Y^1\otimes_{R_2} X^1) \Big) \oplus \Big( (Y^0\otimes_{R_2} X^1) \oplus (Y^1\otimes_{R_2} X^0) \Big) 
$$
together with the differential
$$
d_{Y\otimes X} = d_Y \otimes 1 + 1 \otimes d_X
$$
where the second term comes with the usual Koszul signs when applied to elements. We  observe that e.\,g.~by the argument of~\cite[Section~12]{dm1102.2957} the tensor product does not lead out of the categories $\hmf(R_i \otimes_k R_j, W_j - W_i)$. Also note that there are obvious natural isomorphisms $\alpha_{X,Y,Z}: (X\otimes Y)\otimes Z \lra X\otimes (Y \otimes Z)$. 

Now we can be more specific about the tensor action of the unit matrix factorisation. For $X\in \hmf(R_1 \otimes_k R_2,W_2-W_1)$ there are natural maps
\be\label{lambdarho}
\lambda_X = \mu\pi \otimes 1_X: \Delta_{W_2} \otimes X \lra X \, , \qquad \rho_X = 1_X \otimes \mu\pi : X \otimes \Delta_{W_1} \lra X
\ee
which are isomorphisms in $\hmf(R_1 \otimes_k R_2,W_2-W_1)$. Later in Section~\ref{TODO} we will give a description of their explicit homotopy inverses. 

\begin{definition}
The \textsl{bicategory of Landau-Ginzburg models} $\LG$ consists of the following data: 
\begin{itemize}
\item Objects are pairs $(R,W)$ with $W\in R=k[x]$ a potential. 
\item 1- and 2-morphisms are the objects and morphisms of the categories $\hmf(R_1 \otimes_k R_2,W_2-W_1)$, respectively. 
\item The unit 1-morphisms are $\Delta_W \in \hmf(\Re,\widetilde W)$. 
\item The composition functor is the tensor product
$$
\otimes : \hmf(R_2 \otimes_k R_3,W_3-W_2) \times \hmf(R_2 \otimes_k R_3,W_3-W_2) \lra \hmf(R_1 \otimes_k R_3,W_3-W_1) \, .
$$
\item There are natural 2-isomorphisms $\alpha, \lambda, \rho$ as above. 
\end{itemize}�
\end{definition}

\begin{proposition}[\cite{McNameethesis, Calinetal, cr0909.4381}] 
$\LG$ really is a bicategory, i.\,e.~$\alpha, \lambda, \rho$ are natural isomorphisms up to homotopy, and they satisfy the coherence axioms for bicategories. 
\end{proposition}

One of the main results of the present paper is that $\LG$ is endowed with an additional duality structure. Let us recall the relevant notion. 

\begin{definition}\label{def:bicatduals}
We say a bicategory~$\mathcal B$ \textsl{has duals} (or equivalently, that \textsl{every} 1-morphism is \textsl{dualisable}) if for each 1-morphism $X\in \mathcal B(A,B)$ there exists a 1-morphism $X^\dual \in \mathcal B(B,A)$ together with 2-morphisms 
\be\label{evcoev}
\eval_X : X^\dual \otimes X \lra 1_A \, , \qquad \coev_X : 1_B \lra X \otimes X^\dual
\ee
satisfying 
\begin{align}
\rho_X \circ (1_X \otimes \eval_X) \circ \alpha_{X,X^\dual,X} \circ (\coev_X \otimes 1_X) \lambda_X^{-1} & = 1_X \, , \label{uglyZorro1}\\
\lambda_{X^\dual} \circ (\eval_X \otimes 1_{X^\dual}) \circ \alpha^{-1}_{X^\dual,X,X^\dual} \circ (1_{X^\dual} \otimes \coev_X) \circ \rho_{X^\dual}^{-1} & = 1_{X^\dual} \, . \label{uglyZorro2}
\end{align}
\end{definition}

It is convenient (and also suggested by the motivation discussed in the Introduction) to denote identities in bicategories with duals in string diagram notation, see e.\,g.~\cite{TODO} for more details. In this language the evaluation and coevaluation maps~\eqref{evcoev} are written as
$$
\eval_{X} = 
\begin{tikzpicture}[very thick,scale=1.0,color=blue!50!black, baseline=.6cm]
\draw[line width=0pt] 
(2.5,1.6) node[line width=0pt] (I) {{\small$1_A$}}
(3,0) node[line width=0pt] (D) {{\small $X\vphantom{X^\dual}$}}
(2,0) node[line width=0pt] (s) {\small{$X^\dual$}}; 
\draw[directed] (D) .. controls +(0,1) and +(0,1) .. (s);
\draw[dashed] (2.5,0.81) -- (I);
\end{tikzpicture}
\equiv
\begin{tikzpicture}[very thick,scale=1.0,color=blue!50!black, baseline=.6cm]
\draw[line width=0pt] 
(3,0) node[line width=0pt] (D) {{\small$X\vphantom{X^\dual}$}}
(2,0) node[line width=0pt] (s) {{\small$X^\dual$}}; 
\draw[directed] (D) .. controls +(0,1) and +(0,1) .. (s);
\end{tikzpicture} , 
\qquad
\coev_{X} = 
\begin{tikzpicture}[very thick,scale=1.0,color=blue!50!black, baseline=-.6cm,rotate=180]
\draw[line width=0pt] 
(2.5,1.6) node[line width=0pt] (I) {{\small$1_B$}}
(3,0) node[line width=0pt] (D) {{\small$X\vphantom{X^\dual}$}}
(2,0) node[line width=0pt] (s) {{\small$X^\dual$}}; 
\draw[redirected] (D) .. controls +(0,1) and +(0,1) .. (s);
\draw[dashed] (2.5,0.81) -- (I);
\end{tikzpicture}
\equiv
\begin{tikzpicture}[very thick,scale=1.0,color=blue!50!black, baseline=-.6cm,rotate=180]
\draw[line width=0pt] 
(3,0) node[line width=0pt] (D) {{\small$X\vphantom{A^\dual}$}}
(2,0) node[line width=0pt] (s) {{\small$X^\dual$}}; 
\draw[redirected] (D) .. controls +(0,1) and +(0,1) .. (s);
\end{tikzpicture} \, . 
$$
Note that such diagrams are always to be read from bottom to top. Then the defining relations~\eqref{uglyZorro1} and~\eqref{uglyZorro2} translate into the \textsl{Zorro moves}
\be\label{Zorros}
\begin{tikzpicture}[very thick,scale=1.0,color=blue!50!black, baseline=0cm]
\draw[line width=0] 
(-1,1.25) node[line width=0pt] (A) {{\small $X$}}
(1,-1.25) node[line width=0pt] (A2) {{\small $X$}}; 
\draw[directed] (0,0) .. controls +(0,-1) and +(0,-1) .. (-1,0);
\draw[directed] (1,0) .. controls +(0,1) and +(0,1) .. (0,0);
\draw (-1,0) -- (A); 
\draw (1,0) -- (A2); 
\end{tikzpicture}
=
\begin{tikzpicture}[very thick,scale=1.0,color=blue!50!black, baseline=0cm]
\draw[line width=0] 
(0,1.25) node[line width=0pt] (A) {{\small $X$}}
(0,-1.25) node[line width=0pt] (A2) {{\small $X$}}; 
\draw (A2) -- (A); 
\end{tikzpicture}
\, , \qquad
\begin{tikzpicture}[very thick,scale=1.0,color=blue!50!black, baseline=0cm]
\draw[line width=0] 
(1,1.25) node[line width=0pt] (A) {{\small $X^\dual$}}
(-1,-1.25) node[line width=0pt] (A2) {{\small $X^\dual$}}; 
\draw[directed] (0,0) .. controls +(0,1) and +(0,1) .. (-1,0);
\draw[directed] (1,0) .. controls +(0,-1) and +(0,-1) .. (0,0);
\draw (-1,0) -- (A2); 
\draw (1,0) -- (A); 
\end{tikzpicture}
=
\begin{tikzpicture}[very thick,scale=1.0,color=blue!50!black, baseline=0cm]
\draw[line width=0] 
(0,1.25) node[line width=0pt] (A) {{\small $X^\dual$}}
(0,-1.25) node[line width=0pt] (A2) {{\small $X^\dual$}}; 
\draw (A2) -- (A); 
\end{tikzpicture} \, .
\ee

The dual~$X^\dual$ of~$X$ as described in Definition~\ref{def:bicatduals} is more precisely called the~\textsl{right dual} to~$X$, though we will mostly continue to simply speak of duals without a further attribute. However, let us mention that~$X^\dual$ is called the \textsl{left dual} to~$X$ if there are maps 
$$
\widetilde\eval_{X} = 
\begin{tikzpicture}[very thick,scale=1.0,color=blue!50!black, baseline=.6cm]
\draw[line width=0pt] 
(2.5,1.6) node[line width=0pt] (I) {{\small$1_B$}}
(3,0) node[line width=0pt] (D) {{\small $X^\dual\vphantom{X^\dual}$}}
(2,0) node[line width=0pt] (s) {\small{$X\vphantom{X^\dual}$}}; 
\draw[redirected] (D) .. controls +(0,1) and +(0,1) .. (s);
\draw[dashed] (2.5,0.81) -- (I);
\end{tikzpicture}
\equiv
\begin{tikzpicture}[very thick,scale=1.0,color=blue!50!black, baseline=.6cm]
\draw[line width=0pt] 
(3,0) node[line width=0pt] (D) {{\small$X^\dual$}}
(2,0) node[line width=0pt] (s) {{\small$X\vphantom{X^\dual}$}}; 
\draw[redirected] (D) .. controls +(0,1) and +(0,1) .. (s);
\end{tikzpicture} , 
\qquad
\widetilde\coev_{X} = 
\begin{tikzpicture}[very thick,scale=1.0,color=blue!50!black, baseline=-.6cm,rotate=180]
\draw[line width=0pt] 
(2.5,1.6) node[line width=0pt] (I) {{\small$1_A$}}
(3,0) node[line width=0pt] (D) {{\small$X^\dual$}}
(2,0) node[line width=0pt] (s) {{\small$X\vphantom{X^\dual}$}}; 
\draw[directed] (D) .. controls +(0,1) and +(0,1) .. (s);
\draw[dashed] (2.5,0.81) -- (I);
\end{tikzpicture}
\equiv
\begin{tikzpicture}[very thick,scale=1.0,color=blue!50!black, baseline=-.6cm,rotate=180]
\draw[line width=0pt] 
(3,0) node[line width=0pt] (D) {{\small$X^\dual$}}
(2,0) node[line width=0pt] (s) {{\small$X\vphantom{X^\dual}$}}; 
\draw[directed] (D) .. controls +(0,1) and +(0,1) .. (s);
\end{tikzpicture}
$$
satisfying the associated Zorro moves
\be\label{otherZorros}
\begin{tikzpicture}[very thick,scale=1.0,color=blue!50!black, baseline=0cm]
\draw[line width=0] 
(1,1.25) node[line width=0pt] (A) {{\small $X$}}
(-1,-1.25) node[line width=0pt] (A2) {{\small $X$}}; 
\draw[redirected] (0,0) .. controls +(0,1) and +(0,1) .. (-1,0);
\draw[redirected] (1,0) .. controls +(0,-1) and +(0,-1) .. (0,0);
\draw (-1,0) -- (A2); 
\draw (1,0) -- (A); 
\end{tikzpicture}
=
\begin{tikzpicture}[very thick,scale=1.0,color=blue!50!black, baseline=0cm]
\draw[line width=0] 
(0,1.25) node[line width=0pt] (A) {{\small $X$}}
(0,-1.25) node[line width=0pt] (A2) {{\small $X$}}; 
\draw (A2) -- (A); 
\end{tikzpicture}
\, , \qquad
\begin{tikzpicture}[very thick,scale=1.0,color=blue!50!black, baseline=0cm]
\draw[line width=0] 
(-1,1.25) node[line width=0pt] (A) {{\small $X^\dual$}}
(1,-1.25) node[line width=0pt] (A2) {{\small $X^\dual$}}; 
\draw[redirected] (0,0) .. controls +(0,-1) and +(0,-1) .. (-1,0);
\draw[redirected] (1,0) .. controls +(0,1) and +(0,1) .. (0,0);
\draw (-1,0) -- (A); 
\draw (1,0) -- (A2); 
\end{tikzpicture}
=
\begin{tikzpicture}[very thick,scale=1.0,color=blue!50!black, baseline=0cm]
\draw[line width=0] 
(0,1.25) node[line width=0pt] (A) {{\small $X^\dual$}}
(0,-1.25) node[line width=0pt] (A2) {{\small $X^\dual$}}; 
\draw (A2) -- (A); 
\end{tikzpicture} \, .
\ee

Checking that the Zorro moves hold in $\LG$ with the evaluation and coevaluation maps~\eqref{TODO} and duals $X^\dual = X^\vee[n]$ (all to be discussed in detail in Section~\ref{TODO}) is no easy task. Instead of tackling it directly we will in an intermediate step first prove the Zorro moves using a different model for the unit matrix factorisation, namely the completed bar complex of the ring~$R$. As preparation we shall next discuss some background on non-commutative forms. 

\subsection{Bar complex}\label{subsec:Bar}

Let us for the moment be slightly more general and consider an arbitrary unital associative $k$-algebra~$A$. In our applications to matrix factorisations we will set $A=R=k[x]$. 

\textsl{Non-commutative $n$-forms over~$A$} are elements in 
$$
\Omega^n A = A\otimes \bar A^{\otimes n} 
$$
where $\bar A = A/k$ and in this section by ``$\otimes$'' we mean ``$\otimes_k$''. We denote the projection of $a_0\otimes a_1 \otimes \ldots \otimes a_n \in A^{\otimes n}$ to $\Omega^n A$ as $(a_0,a_1,\ldots,a_n)$. The direct sum 
$$
\Omega A = \bigoplus_{n\geq 0} \Omega^n A
$$
is a differential graded algebra $(\Omega A, d, \cdot)$ with multiplication given by
$$
(a_0,\ldots,a_m) \cdot (a_{m+1},\ldots, a_{m+n}) = \sum_{i=0}^m (-1)^{m-i}(a_0,\ldots,a_{i-1},a_i a_{i+1},a_{i+2},\ldots, a_{m+n}) 
$$
and differential
$$
d: (a_0,\ldots,a_n) \lmt (1,a_0,\ldots,a_n)
$$
where $a_i\in A$. We will write $(a_0,a_1,\ldots,a_n)$ also as $a_0da_1\ldots da_n$. 

More generally one can consider relative non-commutative forms: for a subalgebra $B\subset A$ they are elements in $\Omega_B A = \bigoplus_{n\geq 0} A\otimes_B (A/B)^{\otimes_B n}$, which has a differential graded structure analogous to $\Omega A = \Omega_k A$. We refer to the book~\cite{Loday} for further details. 

A central role is played by the \textsl{(normalised) bar complex} 
$$
\Bar = \bigoplus_{n\geq 0} \Bar_n \, , \qquad \Bar_n = \Omega^n A \otimes A \, .
$$
It is an $(A^{\text{op}} \otimes A)$-module via $(a\otimes a').(a_0da_1\ldots da_n\otimes a_{n+1}) = aa_0da_1\ldots da_n\otimes a_{n+1}a'$. Together with the differential $d\otimes 1_A$ (which by standard abuse of notation we usually simply write~$d$) and the product induced from $\Omega A$ and~$A$, the bar complex~$\Bar$ is a differential graded algebra $(\Bar,d,\cdot)$ as well. 

There is a second differential graded structure on~$\Bar$ if the algebra~$A$ is commutative. To describe it let us first recall that (still for arbitrary~$A$) the bar complex is the standard resolution 
$$
\xymatrix{%
\cdots \ar[r]^-{b'} & A \otimes \bar A^{\otimes 2} \otimes A \ar[r]^-{b'} & A \otimes \bar A \otimes A \ar[r]^-{b'} & A\otimes A \ar[r]^-{b'} & A \ar[r] & 0
}%
$$
of~$A$, where the degree-lowering differential~$b'$ is the $A$-bilinear map
$$
b': (a_0,\ldots,a_n)\otimes a_{n+1} \lmt \sum_{i=0}^{n-1} (-1)^i (a_0,\ldots,a_i a_{i+1},\ldots, a_n) \otimes a_{n+1} + (-1)^n (a_0,\ldots,a_{n-1}) \otimes a_n a_{n+1} \, . 
$$
%Equivalently, in differential form notation~$b'$ acts as
%$$
%b': da_0\ldots da_n\otimes a_{n+1} \lmt (-1)^{n-1} a_0da_1\ldots da_{n-1} \cdot a_n \otimes a_{n+1} + (-1)^n a_0 da_1\ldots da_{n-1} \otimes a_n a_{n+1} \, . 
%$$
From this it is straightforward to check that we have the identity
\be\label{b'd+db'}
b'd+db'=1_{\Bar} \, .
\ee

From now on we assume that~$A$ is commutative. Recall that $(m,n)$-shuffles are permutations in
$$
\operatorname{Sh}(m,n) = \big\{ \sigma\in S_{m+n} \,|\, \sigma(1)<\sigma(2)<\ldots<\sigma(m), \, \sigma(m+1)<\sigma(m+2)<\ldots<\sigma(m+n) \big\} \, . 
$$
We use them to define the $A$-bilinear \textsl{shuffle product}~$\times$ on~$\Bar$ as
\begin{align*}
& (a_0da_1\ldots da_m \otimes a_{m+1}) \times (b_0db_1\ldots db_n \otimes b_{n+1}) \\
& \qquad 
= \sum_{\sigma_{\operatorname{Sh}(m,n)}} (-1)^{|\sigma|} a_0 b_0 \, \sigma_\bullet (da_1\ldots da_m db_1 \ldots db_n) \otimes a_{m+1} b_{n+1}
\end{align*}
where $\sigma_\bullet(da_1\ldots da_j) = da_{\sigma(1)}\ldots da_{\sigma(j)}$. One finds that $(\Bar,b',\times)$ is a graded-commutative differential graded algebra. Note that for $\omega\in A\otimes A=\Bar_0$ it follows immediately that $\omega\times(-) = \omega\cdot(-)$. 

We now return to the $k$-algebra $A=R=k[x_1,\ldots,x_n]$. Earlier we set $\widetilde W = W\otimes 1 - 1\otimes W \in \Re$ for a potential $W\in R$, in terms of which we now define the bar complex endomorphism
$$
d_{\Bar} = b' + d\widetilde W \times (-) \, . 
$$
\begin{lemma}
$(\Bar,d_{\Bar})$ is a linear factorisation of $\widetilde W\in \Re$. 
\end{lemma}

\begin{proof}
The bar complex $\Bar=\Bar^0 \oplus \Bar^1$ is $\nZ_2$-graded with $\Bar^i = \bigoplus_{n\in 2\nN+i}\Bar_n$. Since~$b'$ and~$\times$ are both $R$-bilinear, $d_{\Bar}$ is indeed $\Re$-linear. Furthermore, we have $b'^2=0$ and $d\widetilde W \times d\widetilde W = (dW\otimes 1) \times (dW\otimes 1) = dWdW\otimes 1 - dWdW \otimes 1 = 0$, so that for $\omega\in\Bar$ we find 
\begin{align*}
d_{\Bar}^2 (\omega) & = b'(d\widetilde W \times \omega) + d\widetilde W \times b'(\omega) \\
& = b'(d\widetilde W) \times \omega - d\widetilde W \times b'(\omega) + d\widetilde W \times b'(\omega) \\
& = \widetilde W \times \omega \\
& = \widetilde W \cdot \omega
\end{align*}
where in the second last step we used~\eqref{b'd+db'} together with $b'(\widetilde W)=0$. 
\end{proof}

If we use~$\pi$ also to denote the projection $\Bar�\lra \Bar_0 = \Re$, then $\mu\pi \otimes 1_X : \Bar \otimes X \lra X$ and $1_X \otimes \mu\pi: X \otimes \Bar \lra X$ give left and right actions of~$\Bar$ as in~\eqref{lambdarho}. These maps have homotopy inverses too and we will construct them in Section~\ref{TODO}. For the moment we take the fact that~$\Bar$ is another model for the unit action on matrix factorisations as motivation to discuss its relation to the Koszul matrix factorisation~$\Delta_W$. Before we do this on the level of linear factorisations we shall consider the special case $W=0$. We write $\Delta = \Delta_0 = \bigwedge( \bigoplus_{i=1}^n \Re \theta_i)$ and observe that now $(\Delta_W, d_{\Delta_W})$ reduces to the ordinary Koszul complex $(\Delta, \delta_-)$, see~\eqref{DeltaW}

There are two $\Re$-linear maps between~$\Bar$ and~$\Delta$ which will be important to us: 
\begin{align*}
\Phi & : \Delta \lra \Bar \, , \qquad \theta_{i_1}\ldots \theta_{i_p} \lmt \sum_{\sigma\in S_p} (-1)^{|\sigma|} dx_{i_{\sigma(1)}} \ldots dx_{i_{\sigma(p)}} \otimes 1 \, , \\
\Psi & : \Bar \lra \Delta \, , \qquad df_1\ldots df_p \otimes 1 \lmt \sum_{1\leq i_1<\ldots<i_p\leq n} \Big( \prod_{k=1}^p \partial_{[i_k]} f_k \Big) \, \theta_{i_1} \ldots \theta_{i_p} \, .
\end{align*}
These maps were studied in~\cite{sw0911.0917}, we only rephrase the presentation of~$\Psi$ in terms of the difference quotient operators~$\partial_{[i]}$ suitable for our setting. One easily verifies that $\Psi\Phi = 1_\Delta$. 

\begin{lemma}\label{PhiPsiDG}
Both~$\Phi$ and~$\Psi$ are maps of differential graded algebras between $(\Delta, \delta_-, \wedge)$ and $(\Bar, b', \times)$. 
\end{lemma}

\begin{proof}
We refer to~\cite{sw0911.0917} for the case of~$\Phi$; since our expression for~$\Psi$ is not manifestly the same as in loc.~cit.~we spell out the proof. Let us first show that~$\Psi$ is compatible with the differentials. On the one hand we compute $(\delta_- \Psi) (df_1\ldots df_p \otimes 1)$ to be 
\begin{align}
& \delta_ -\Big( \sum_{i_1<\ldots <i_p} (\partial_{[i_1]} f_1) \ldots (\partial_{[i_p]} f_p) \, \theta_{i_1} \ldots \theta_{i_p} \Big) \nonumber \\
& =  \sum_{j=1}^n \sum_{i_1<\ldots <i_p} (\partial_{[i_1]} f_1) \ldots (\partial_{[i_p]} f_p) \cdot (x_j - y_j) \sum_{k=1}^p (-1)^{k+1} \delta_{j i_k} \theta_{i_1} \ldots \widehat{\theta_{i_k}} \ldots \theta_{i_p}  \nonumber \\
& = \sum_{k=1}^p (-1)^{k+1}\sum_{i_1<\ldots <i_p} (\partial_{[i_1]} f_1) \ldots ({}^{t_1\ldots t_{i_{k-1}}} f_k - {}^{t_1\ldots t_{i_{k}}} f_k) \ldots (\partial_{[i_p]} f_p) \, \theta_{i_1} \ldots \widehat{\theta_{i_k}} \ldots \theta_{i_p}  \nonumber \\
& = \sum_{2\leq t_2<\ldots< i_p} (f_1 - {}^{t_1\ldots t_{i_{2}-1}} f_1) (\partial_{[i_2]} f_2) \ldots (\partial_{[i_p]} f_p) \, \theta_{i_2} \ldots \theta_{i_p} \nonumber \\
& \qquad + \sum_{k=2}^{p-1} (-1)^{k+1}\sum_{i_1,\ldots,i_p} (\partial_{[i_1]} f_1) \ldots ({}^{t_1\ldots t_{i_{k-1}}} f_k - {}^{t_1\ldots t_{i_{k+1}-1}} f_k) \ldots (\partial_{[i_p]} f_p) \, \theta_{i_1} \ldots \widehat{\theta_{i_k}} \ldots \theta_{i_p}  \nonumber \\
& \qquad + (-1)^{p+1} \sum_{i_1<\ldots< i_{p-1}\leq n-1} (\partial_{[i_1]} f_1) \ldots (\partial_{[i_{p-1}]} f_{p-1}) ({}^{t_1\ldots t_{i_{p-1}}} f_p - {}^{t_1\ldots t_n} f_p)  \, \theta_{i_1} \ldots \theta_{i_{p-1}} \nonumber \\
& = \sum_{2\leq t_2<\ldots <i_p} f_1 (\partial_{[i_2]} f_2) \ldots (\partial_{[i_p]} f_p) \, \theta_{i_2} \ldots \theta_{i_p} \nonumber \\
& \qquad + (-1)^{p} \sum_{i_1<\ldots< i_{p-1}\leq n-1} (\partial_{[i_1]} f_1) \ldots (\partial_{[i_{p-1}]} f_{p-1}) \, {}^{t_1\ldots t_n} f_p  \, \theta_{i_1} \ldots \theta_{i_{p-1}}  \label{delPsi} 
\end{align}
while on the other hand $(\Psi b') (df_1\ldots df_p \otimes 1)$ equals
\begin{align*}
& \Psi \Big( f_1 df_2 \ldots df_p \otimes 1 + \sum_{k=1}^{p-1} (-1)^k df_1 \ldots d(f_k f_{k+1}) \ldots df_p \otimes 1 + (-1)^p df_1 \ldots df_{p-1} \otimes f_p \Big) \\
& = \sum_{i_1<\ldots< i_{p-1}} f_1 (\partial_{[i_1]} f_2) \ldots (\partial_{[i_{p-1}]} f_p) \, \theta_{i_1} \ldots \theta_{i_{p-1}} \\
& \qquad + \sum_{k=1}^{p-1} (-1)^k \sum_{i_1<\ldots< i_{p-1}} (\partial_{[i_1]} f_1) \ldots (\partial_{[i_k]} (f_k f_{k+1}) )\ldots (\partial_{[i_{p-1}]} f_p) \, \theta_{i_1} \ldots \theta_{i_{p-1}} \\
& \qquad + (-1)^p \sum_{i_1<\ldots< i_{p-1}} (\partial_{[i_1]} f_1) \ldots (\partial_{[i_{p-1}]} f_{p-1}) \, {}^{t_1\ldots t_n} f_p \, \theta_{i_1} \ldots \theta_{i_{p-1}} \\
& = \sum_{2\leq i_1<\ldots< i_{p-1}} f_1 (\partial_{[i_1]} f_2) \ldots (\partial_{[i_{p-1}]} f_p) \, \theta_{i_1} \ldots \theta_{i_{p-1}} \\ 
& \qquad + \sum_{2\leq i_2 \ldots< i_{p-1}} f_1 (\partial_{[1]} f_2) (\partial_{[i_2]} f_3) \ldots (\partial_{[i_{p-1}]} f_p) \, \theta_{i_1} \ldots \theta_{i_{p-1}} \\ 
& \qquad + \sum_{k=1}^{p-1} (-1)^k \sum_{i_1<\ldots< i_{p-1}} (\partial_{[i_1]} f_1) \ldots \big\{ ({}^{t_1\ldots t_{i_k -1}}f_k)(\partial_{[i_k]} f_{k+1}) \\
& \qquad\quad + (\partial_{[i_k]} f_k) ({}^{t_1\ldots t_{i_k}} f_{k+1}) \big\} \ldots (\partial_{[i_{p-1}]} f_p) \, \theta_{i_1} \ldots \theta_{i_{p-1}} \\
& \qquad + (-1)^p \sum_{i_1<\ldots< i_{p-1}\leq n-1} (\partial_{[i_1]} f_1) \ldots (\partial_{[i_{p-1}]} f_{p-1}) \, {}^{t_1\ldots t_n} f_p \, \theta_{i_1} \ldots \theta_{i_{p-1}} \\
& \qquad\quad + (-1)^p \sum_{i_1<\ldots< i_{p-2}\leq n-1} (\partial_{[i_1]} f_1) \ldots (\partial_{[i_{p-1}]} f_{p-1}) \, {}^{t_1\ldots t_n} f_p \, \theta_{i_1} \ldots \theta_{i_{p-1}} \\
& = \sum_{2\leq i_1<\ldots< i_{p-1}} f_1 (\partial_{[i_1]} f_2) \ldots (\partial_{[i_{p-1}]} f_p) \, \theta_{i_1} \ldots \theta_{i_{p-1}} \\ 
& \qquad + (-1)^p \sum_{i_1<\ldots< i_{p-1}\leq n-1} (\partial_{[i_1]} f_1) \ldots (\partial_{[i_{p-1}]} f_{p-1}) \, {}^{t_1\ldots t_n} f_p \, \theta_{i_1} \ldots \theta_{i_{p-1}} 
\end{align*}
which agrees with~\eqref{delPsi}. 

To establish compatibility with the products we compute 
\begin{align*}
& \Psi( df_1\ldots df_p \otimes 1) \wedge \Psi( df_{p+1}\ldots df_{p+q} \otimes 1) \\ 
=\, & \Big\{ \sum_{i_1<\ldots< i_p} \Big( \prod_{k=1}^p \partial_{[i_k]} f_k \Big) \theta_{i_1} \ldots \theta_{i_p}\Big\} 
\wedge 
\Big\{ \sum_{i_{p+1}<\ldots< i_{p+q}} \Big( \prod_{k=p+1}^{p+q} \partial_{[i_k]} f_k \Big) \theta_{i_{p+1}} \ldots \theta_{i_{p+q}} \Big\} \\
= \, & \sum_{i_1<\ldots< i_p} \sum_{i_{p+1}<\ldots< i_{p+q}} \Big( \prod_{k=1}^{p+q} \partial_{[i_k]} f_k \Big) \theta_{i_{1}} \ldots \theta_{i_{p+q}} \\
= \, & \sum_{i_1<\ldots< i_{p+q}} \sum_{\sigma\in\operatorname{Sh}(p,q)} (-1)^{|\sigma|} \Big( \prod_{k=1}^{p+q} \partial_{[i_{\sigma(k)}]} f_k \Big) \theta_{i_{1}} \ldots \theta_{i_{p+q}} \\
= \, & \Psi \big((df_1\ldots df_p \otimes 1) \times( df_{p+1}\ldots df_{p+q} \otimes 1) \big)
\end{align*}
where in the third step the anti-commutativity of the~$\theta_i$ allowed us to sum over the longer sequences $i_1<\ldots< i_{p+q}$ by introducing an additional sum over shuffles. 
\end{proof}

Now we come back to consider any potential $W\in R$. The map~$\Psi$ continues to be a good map on the level of linear factorisations: 
\begin{lemma}\label{PsiHF}
$\Psi: (\Bar, d_\Bar) \lra (\Delta_W, d_{\Delta_W})$ is a morphism in $\HF(\Re,\widetilde W)$. 
\end{lemma}

\begin{proof}
We need to show $d_{\Delta_W} \Psi = \Psi d_\Bar$. But since $d_{\Delta_W} = \delta_+ + \delta_-$ and $d_\Bar = b' + d\widetilde W \times (-)$ by Lemma~\eqref{PhiPsiDG} what remains to be checked is $\delta_+ \Psi = \Psi (d\widetilde W\times (-))$. This can be done: 
$$
\delta_+ \Psi = \Big( \sum_{i=1}^n \partial_{[i]} W \cdot \theta_i^* \Big) \wedge \Psi(-) = \Psi(dW\otimes 1) \wedge \Psi(-) = \Psi (d\widetilde W\times (-)) \, . 
$$
\end{proof}


TODO: Say something about $\Phi$ here too


\begin{remark}
Instead of the bar complex $\Bar = \bigoplus_{n\geq 0} \Bar_n$ one can also consider its completed version
$$
\cBar = \prod_{n\geq 0} \Bar_n \, .
$$
It has differential graded structures $(\cBar,d,\cdot)$ and $(\cBar,b',\times)$ analogous to~$\Bar$, and the maps~$\Phi$ and~$\Psi$ lift to maps to and from the completed bar complex~$\cBar$ with the same properties as in Lemmas~\ref{PhiPsiDG} and~\ref{PsiHF}. 
\end{remark}









\begin{itemize}
\item It's silly to have both $y$ and $x$ in the same domain, this is some artifact of bad decisions in the notes.
\item Dropped subscripts from $\rho, \lambda$ in the diagram.
\end{itemize}

\newpage

\section{Zorro moves}\label{sec:Zorro}

In this section we will show that the bicategory $\LG$ of Landau-Ginzburg models has duals for any noetherian $\mathbb{Q}$-algebra $k$. Let us fix two arbitrary potentials $W\in k[x] = k[x_1,\ldots,x_n]$ and $V\in k[z] = k[z_1,\ldots,z_m]$. Then we want to prove that for any matrix factorisation $X\in \hmf(k[z,x], V-W)$ and its dual $X^\dual = X^\vee[n]\in \hmf(k[z,x], W-V)$ the Zorro moves~\eqref{Zorros} and~\eqref{otherZorros} are satisfied. Let $\{ e_i \}_{i \in I}$ denote a homogeneous $k[z,x]$-basis of $X$ and $e_i^*$ the dual basis of $X^\vee$.

Let us consider the first identity of~\eqref{otherZorros} in more detail: 
\be\label{Zorro1detail}
\begin{tikzpicture}[very thick,scale=1.0,color=blue!50!black, baseline=0cm]

\fill (0.2,1.6) circle (0pt) node {{\small $\Delta$}};
\fill (-0.2,-1.6) circle (0pt) node {{\small $\Delta$}};

\fill (1,1.8) circle (2.5pt) node[right] {{\small $\lambda$}};
\fill (-1,-1.8) circle (2.5pt) node[left] {{\small $\rho^{-1}$}};

\fill (-1.25,-2.25) circle (0pt) node {{\footnotesize $z$}};
\fill (-0.75,-2.25) circle (0pt) node {{\footnotesize $x'$}};

\fill (-1.5,0) circle (0pt) node {{\footnotesize $z\vphantom{z'}$}};
\fill (-0.5,0) circle (0pt) node {{\footnotesize $x$}};
\fill (0.5,0) circle (0pt) node {{\footnotesize $z'$}};
\fill (1.5,0) circle (0pt) node {{\footnotesize $x'\vphantom{z'}$}};

\fill (1.25,2.25) circle (0pt) node {{\footnotesize $x'$}};
\fill (0.75,2.25) circle (0pt) node {{\footnotesize $z\vphantom{y}$}};

\draw[dashed] (-0.5,0.75) .. controls +(0,0.75) and +(-0.25,-0.75) .. (1,1.8);
\draw[dashed] (0.5,-0.75) .. controls +(0,-0.75) and +(0.25,0.75) .. (-1,-1.8);

\draw[line width=0] 
(1,2.7) node[line width=0pt] (A) {{\small $X$}}
(-1,-2.7) node[line width=0pt] (A2) {{\small $X$}}; 
\draw[redirected] (0,0) .. controls +(0,1) and +(0,1) .. (-1,0);
\draw[redirected] (1,0) .. controls +(0,-1) and +(0,-1) .. (0,0);
\draw (-1,0) -- (A2); 
\draw (1,0) -- (A); 
\end{tikzpicture}
=
\begin{tikzpicture}[very thick,scale=1.0,color=blue!50!black, baseline=0cm]
\draw[line width=0] 
(0,2.7) node[line width=0pt] (A) {{\small $X$}}
(0,-2.7) node[line width=0pt] (A2) {{\small $X$}}; 
\draw (A2) -- (A); 
\end{tikzpicture}
\ee
Here we have indicated the variable names in the various domains (\textbf{fix} we are still very messy with the variables). We call the left hand side of (\ref{Zorro1detail}) the \emph{Zorro map} and denote it $\mathcal Z$. It is the composite
\be\label{eq:zorro1a1}
\xymatrix@C+2pc{
X \ar[r]^-{\rho^{-1}} & X \otimes \Delta \ar[r]^-{1 \otimes \coev} & X \otimes X^{\dagger} \otimes X \ar[r]^-{\eval \otimes 1} & \Delta \otimes X \ar[r]^-{\lambda} & X\,,
}
\ee
which we prove is homotopic to the identity on $X$. Our strategy is to use the nondegenerate pairing on $\hmf(k[z,x], V-W)$. If $k$ is a field this is the Kapustin-Li pairing of \cite{??,??} but in general we use the contents of Section \ref{??} (\textbf{more}). The upshot is that there is a homotopy equivalence of $\mathbb{Z}_2$-graded complexes over $k$
\be
\Hom_{k[z,x]}(X,X) \lto \Hom_k( \Hom_{k[z,x]}(X,X), k )[m+n]\,, \qquad \varphi \mapsto \langle \varphi, - \rangle_{\textup{KL}}\,.
\ee
To prove that $\mathcal{Z} \simeq 1_X$ it is therefore enough to prove that the functionals $\langle \mathcal{Z}, - \rangle_{\textup{KL}}$ and $\langle 1_X, - \rangle_{\textup{KL}}$ on $\Hom_{k[z,x]}(X,X)$ are \emph{homotopic}. When $k$ is a field this equivalent to showing that the two functionals take the same value on any closed endomorphism of $X$, and in general it amounts to showing that $\langle \mathcal{Z}, \varphi \rangle_{\textup{KL}}$ and $\langle 1_X, \varphi \rangle_{\textup{KL}}$ differ, as functionals of $\varphi$, by a functional of $D(\varphi)$. In light of the formula for $\langle -, - \rangle_{\textup{KL}}$ it is natural that our first step should therefore be to understand the map $\str( \mathcal{Z} \circ - )$.

Let us introduce the notation
$$
\langle\!\langle-\rangle\!\rangle = (-1)^{n+1\choose 2} \Res_{k[x,y,z]/k[y,z]} \left[ \frac{ \varepsilon\Psi(-) \underline{\operatorname{d}\!x}}{\partial_{x_1}W \ldots \partial_{x_n} W} \right]: \Bar \lto k[z,y]\,.
$$
where $\varepsilon\Psi: \Bar \lto k[z,y][n]$ is the chain map of Section \ref{??} and $\underline{\operatorname{d}\!x}=\operatorname{d}\!x_1\ldots \operatorname{d}\!x_n$. Let $\At$ denote the Atiyah class of $\End(X)$, which is an operator on $\End(X) \otimes_{k[x]} \Bar$. Let $\lambda_i$ be a null-homotopy on $X$ for the action of $\partial_{x_i} W$, for example $\lambda_i=\partial_{x_i}d_X(z,x)$, and set $\underline\lambda=\lambda_1\ldots \lambda_n$. Our first order of business is to establish the following expression for the Zorro map.

\begin{lemma}\label{lemma:Zorrointermediate}
\be\label{Zorrointermediate}
\mathcal Z(-) = \sum_j (-1)^{|e_j| + n} \big\langle\!\big\langle \str \big( \underline\lambda \circ \At^n (-\circ e_j^*) \big) \big\rangle\!\big\rangle \cdot e_j
\ee
\end{lemma}

%Given the explicit formula for the pairing in (\ref{??}) it is certainly enough to prove that the closed $k$-linear map
%\[
%\str(\mathcal{Z} \circ -): \Hom_{k[z,x]}(X,X) \lto k[z,x]/(\partial_{x_i} W, \partial_{z_i} V)
%\]
%is homotopic to $\str(-)$. That is, we prove that modulo the ideal generated by the partial derivatives $\partial_{y_i} W, \partial_{z_i} V$, the difference between $\str(\mathcal{Z} \circ -)$ and $\str(-)$ factors via a degree one $k$-linear operator on $\Hom_{k[z,x]}(X,X)$. The first step is to find an explicit formula for $\mathcal{Z}$.

%We will denote equality modulo the derivatives $f_i(y):= \partial_{y_i}W$ and $g_i(z):= \partial_{z_i}V$ by ``$\equiv$'',

 Before proceeding with the proof, let us recall the explicit formulas for the maps involved
\begin{align*}
\rho^{-1}: X \lto X \otimes \Delta, \qquad \rho^{-1}(a) &= \sum_{i \ge 0}(-1)^i \Psi\! \At^i_X(a)\,,\\
\coev: \Delta \lto X^{\dagger} \otimes X, \qquad \coev(\gamma) &= \varepsilon\Big( \gamma \wedge \sum_{i \ge 0}(-1)^i (\Psi\!\At^i_{X^{\dagger}} \otimes 1)(\iota_X) \Big)\,,\\
&= \sum_{i,j} (-1)^{i+|e_j|} \varepsilon \Big( \gamma \wedge \Psi\!\At^i_{X^{\dagger}}( e_j^* ) \otimes e_j \Big)\\
\eval: X \otimes X^{\dagger} \lto \Delta, \qquad \eval(\alpha) &= (-1)^{\binom{n+1}{2}}\Res_{k[x,y,z]/k[y,z]}\! \left[ \frac{\str ( \alpha|_{z=z'} \underline{\lambda}) \underline{\operatorname{d}\! x}}{\partial_{x_1}W \ldots \partial_{x_n} W} \right] + \mathcal{O}(\theta)\,.
\end{align*}
Note that tensoring with the diagonal defines a map $(-)|_{z=z'}: X \otimes X^{\dagger} \lto \Hom_{k[x,z]}(X,X)$ (\textbf{more}) and $\mathcal{O}(\theta)$ denotes higher order terms which do not survive $\lambda$ and therefore may be ignored for present purposes. Also define $\nabla$ and Atiyah classes (\textbf{todo}).

Let us consider the image of a basis element $e_q$ under the first two maps of (\ref{eq:zorro1a1})
\begin{align}
(1 \otimes \coev) \circ \rho^{-1}( e_q ) &= \sum_{i,k,j} (-1)^{i+k} \varepsilon\Big( \Psi\! \At^i_X( e_q ) \wedge \Psi \!\At^k_{X^{\dagger}}( e_j^* ) \otimes e_j \Big) \nonumber\\
&= \sum_{j} \sum_{i+k = n} (-1)^n \varepsilon\Psi\Big( \At^i_X( e_q ) \times \At^k_{X^{\dagger}}( e_j^* ) \otimes e_j \Big) \nonumber\\
&= \sum_{j} (-1)^n \varepsilon\Psi\Big( \At^n_{X \otimes X^{\dagger}}( e_q \otimes e_j^* ) \otimes e_j \Big)\,. \label{firstzorromaps}
\end{align} 
In the last step we use the following lemma about associative Atiyah classes.

\begin{lemma} $\At \times \At = \At$.
\end{lemma}

The proof is given at the end of this section. 

\begin{proof}[Proof of Lemma \ref{lemma:Zorrointermediate}]
Applying $\eval$ in the first two components of $X \otimes X^{\dagger} \otimes X$ to the expression in (\ref{firstzorromaps}) has the effect of identifying $z$ with $z'$. This changes the associative Atiyah class of $X \otimes X^{\dagger}$ to the associative Atiyah class of $\End(X)$ and so
\begin{align}
\mathcal{Z}(e_q) &= \lambda \circ (\eval \otimes 1) \circ (1 \otimes \coev) \circ \rho^{-1}( e_q )\\
&= \sum_j (-1)^{|e_j|+n+\binom{n+1}{2}}\Res_{k[x]/k}\! \left[ \frac{ \varepsilon\Psi \str\Big( \At^n( e_q \circ e_j^* ) \underline{\lambda} \Big) \underline{\operatorname{d}\! x} }{\partial_{x_1}W \ldots \partial_{x_n} W} \right] \cdot e_j
\end{align}
Since the Zorro map~$\mathcal Z$ is $k[z,x]$-linear it is fixed by its action on basis elements~$e_q$, which proves that $\mathcal{Z}$ is given by (\ref{Zorrointermediate}).
\end{proof}

Then we can compute for $\varphi \in \End(X)$ homogeneous
\begin{align}
\str(\mathcal Z\varphi) & = \sum_i (-1)^{|e_i|} e^*_i \big( \mathcal Z(\varphi(e_i)) \big) \nonumber \\
& = \sum_{i,j} (-1)^{|e_i| + |e_j| + n} e^*_i \Big( \big\langle\!\big\langle \str( \underline\lambda \At^n (\varphi(e_i)\otimes e^*_j \otimes e_j)) \big\rangle\!\big\rangle \Big) \nonumber \\
& = \sum_{i,j} (-1)^{|e_i| + |e_j| + n + n|e_j|} e^*_i \Big( \big\langle\!\big\langle \str( \underline\lambda \At^n (\varphi(e_i)\otimes e^*_j )) \big\rangle\!\big\rangle e_j \Big) \nonumber \\
& = \big\langle\!\big\langle \str \big( \underline\lambda \At^n \big(\sum_i \varphi(e_i)\otimes e^*_j \big)\big) \big\rangle\!\big\rangle \nonumber \\
& = \big\langle\!\big\langle \str \big( \underline\lambda \At^n (\varphi)\big) \big\rangle\!\big\rangle \label{strlambdaAtn} \, .
\end{align}
The next step of the proof that $\mathcal{Z} \simeq 1_X$ is to show that $\str(\mathcal{Z} \circ -)$ is homotopic to $\str(-)$. In outline, here is the argument: the Atiyah class is ``anti-self-adjoint'' with respect to $\langle\!\langle \str( - ) \rangle\!\rangle$, so
\begin{equation}\label{eq:trueproofeq}
\str(\mathcal{Z} \circ -) = \big\langle\!\big\langle \str \big( \underline\lambda \circ \At^n (-)\big) \big\rangle\!\big\rangle \simeq (-1)^n \big\langle\!\big\langle \str \big( \At^n( \underline\lambda ) \circ -\big) \big\rangle\!\big\rangle\,.
\end{equation}
We will see that $\At^n( \underline\lambda )$ is a transition determinant in the sense of the calculus of residues, so that the right hand side can be identified with $\str(-)$. Conceptually this parallels the proof of nondegeneracy of the Kapustin-Li formula given in Section \ref{??}. However we proceed a little differently here, since we are working with noncommutative differential forms and this introduces various subtleties.

The form of self-adjointness of the Atiyah class that we will use is contained in the next result. The $p = 0$ case is the statement that
\[
\big\langle\!\big\langle \str( \lambda_1 \cdots \lambda_n \circ \At(\varphi) ) \big\rangle\!\big\rangle = \sum_{i=1}^n (-1)^{i-1} \Big\langle\!\Big\langle \nabla \left\{ f_i \str( \lambda_1 \cdots \widehat{\lambda_i} \cdots \lambda_n \circ \varphi ) \right\} \Big\rangle\!\Big\rangle\,,
\]
where as usual $\widehat{\lambda_i}$ means that $\lambda_i$ is omitted. Applying the lemma inductively will interpolate between the left and right hand sides of (\ref{eq:trueproofeq}). For the general statement we will need some notation: given a sequence $\bs{i} = (i_1,\ldots,i_p)$ in $[1,n]$ (not necessarily in ascending order) we define
\[
\ell(\bs{i}) = p, \qquad w(\bs{i}) = i_1 + \cdots + i_p, \qquad |\bs{i}| = \sum_{1 \le a < b \le n} \delta_{i_a > i_b}\, \qquad \gamma(\bs{i}) = w(\bs{i}) + |\bs{i}| + \binom{p+1}{2}\,.
\]
We write $\underline{\lambda}_{i_1,\ldots,i_p}$ for the product $\underline{\lambda} = \lambda_1 \cdots \lambda_n$ with the $\lambda_{i_1},\ldots,\lambda_{i_p}$ omitted. We deliberately confuse an endomorphism of $X$ with the operator on $\End(X) \otimes_{k[x]} \Bar$ which acts by post-composition with the endomorphism. Let $D$ denote the differential on $\End(X)$.

\begin{lemma}\label{lemma:zorromain} For $\varphi \in \End(X) \otimes_{k[x]} \Bar$ and $0 \le p < n$ we have
\begin{align*}
\sum_{\ell(\bs{i}) = p} (-1)^{\gamma(\bs{i}) + p} & \Big\langle\!\Big\langle \nabla f_{i_1} \nabla f_{i_2} \cdots \nabla f_{i_p} \str\big( \underline{\lambda}_{i_1,\ldots,i_p} \circ \At(\varphi) \big) \Big\rangle\!\Big\rangle\\
&= \sum_{\ell(\bs{i}) = p+1} (-1)^{\gamma(\bs{i}) + p+1} \Big\langle\!\Big\langle \nabla f_{i_1} \nabla f_{i_2} \cdots \nabla f_{i_{p+1}} \str\big( \underline{\lambda}_{i_1,\ldots,i_p,i_{p+1}} \circ \varphi \big) \Big\rangle\!\Big\rangle + h_p( D( \varphi ) )
\end{align*}
where the sums are over all sequences of length $p, p+1$ respectively and
\[
h_p( \psi ) = \sum_{\ell(\bs{i}) = p} (-1)^{\gamma(\bs{i}) +p} \Big\langle\!\Big\langle \nabla f_{i_1} \cdots \nabla f_{i_p} \str\big( [ \underline{\lambda}_{i_1,\ldots,i_p}, \nabla](\psi) \big) \Big\rangle\!\Big\rangle\,.
\]
\end{lemma}
\begin{proof}
Fix a sequence $\bs{i}$ of length $p$. The super Jacobi identity for operators on $\End(X) \otimes \Bar$ gives
\be\label{superJacobi}
\big[ \underline{\lambda}_{i_1,\ldots,i_p}, \At \big] = (-1)^{n-p} \big[ D, \big[ \underline{\lambda}_{i_1,\ldots,i_p}, \nabla \big] \big] - \big[ \nabla, \big[ D, \underline{\lambda}_{i_1,\ldots,i_p} \big] \big]\,.
\ee
It is easy to see that $\str \circ \nabla = \nabla \circ \str$ and hence $\str \circ \At = 0$ so if we apply the previous identity to $\varphi$ and then apply $\dlangle \eta \str( - ) \drangle$, with $\eta = \nabla \circ f_{i_1} \circ \cdots \circ \nabla \circ f_{i_p}$, we have
\begin{align}
\dlangle \eta \str\big( \underline{\lambda}_{i_1,\ldots,i_p} \circ \At(\varphi) \big) \drangle &= (-1)^{n-p} \dlangle \eta \str\big( \big[ D, \big[ \underline{\lambda}_{i_1,\ldots,i_p}, \nabla \big] \big] \varphi \big) \drangle \nonumber\\
&\qquad - \dlangle \eta \str\big( \big[ \nabla, \big[ D, \underline{\lambda}_{i_1,\ldots,i_p} \big] \big] \varphi \big) \drangle \nonumber\\
&= \dlangle \eta \str\big( \big[ \underline{\lambda}_{i_1,\ldots,i_p}, \nabla \big] ( D\varphi ) \big) \drangle \label{eq:hutch1}\\
&\qquad- \dlangle \eta \nabla \str\big( \big[ D, \underline{\lambda}_{i_1,\ldots,i_p} \big] \varphi \big) \drangle \label{eq:hutch2}\\
&\qquad+ (-1)^{n-p+1} \bigdlangle \eta \str\big( \big[ D, \underline{\lambda}_{i_1,\ldots,i_p} \big]( \nabla \varphi ) \big) \drangle \label{eq:hutch3}
\end{align}
For compactness let us write $f_i = \partial_{x_i} W$. Since $[d_X, \lambda_i] = f_i$ we have
\[
\big[ D, \underline{\lambda}_{i_1,\ldots,i_p} \big] = \sum_{a \notin \{ i_1, \ldots, i_p \} } (-1)^{a-1 + \#\{ b \l i_b < a \}} f_{a} \cdot \underline{\lambda}_{i_1,\ldots,i_p,a}
\]
where the sign counts the number of $\lambda$'s to the left of $\lambda_a$ in $\underline{\lambda}_{i_1,\ldots,i_p}$. Subsituting, the last summand (\ref{eq:hutch3}) in the above is, up to a sign
\[
\sum_{a \notin \{ i_1,\ldots, i_p \}} (-1)^{a-1 + \#\{ b \l i_b < a \}} \bigdlangle \nabla f_{i_1} \cdots \nabla f_{i_p} f_a \str\big( \big[ D, \underline{\lambda}_{i_1,\ldots,i_p,a} \big]( \nabla \varphi ) \big) \bigdrangle
\]
If we sum over all sequences $\bs{i}$ with the signs given in the statement of the lemma it is straightforward to check that the above term vanishes, because $f_{i_p} f_a$ and $f_a f_{i_p}$ appear in the sum with opposite signs. This completes the proof, since the sum over $\bs{i}$ of (\ref{eq:hutch1}) and (\ref{eq:hutch2}) gives us the right hand side of the equation in the statement of the lemma.
\end{proof}

For the proof of the main theorem we will also need the following.

\begin{lemma}
Let $\omega\in\Omega(k[x_1,\ldots,x_n])$ be and $n$-form. Then $\Psi(\omega f)=\Psi(\omega) f(y)$ for any $f\in k[x]$. 
\end{lemma}

\begin{proof}
We may assume that $\omega$ is of the form $da_1\ldots da_n$ for some $a_i\in k[x]$. Then
\begin{align*}
\Psi(\omega f) & = \Psi \Big( \sum_{i=1}^n (-1)^{n-i} da_1\ldots da_{i-1} d(a_i a_{i+1}) da_{i+2}\ldots da_n df + (-1)^n a_1 da_2\ldots da_n df \Big) \\
& = \sum_{i=1}^n (-1)^{n-i} \partial_{[1]} a_1\ldots \partial_{[i-1]} a_{i-1} \partial_{[i]} (a_i a_{i+1}) \partial_{[i+1]} a_{i+2} \ldots \partial_{[n-1]} a_n \partial_{[n]} f \\
& \qquad + (-1)^n a_1 \partial_{[1]} a_2 \ldots \partial_{[n-1]} a_n \partial_{[n]} f \\
& = \sum_{i=1}^n (-1)^{n-i} \partial_{[1]} a_1\ldots \partial_{[i-1]} a_{i-1} \Big(\partial_{[i]} a_i {}	^{t_1\ldots t_i}a_{i+1} + {}^{t_1\ldots t_{i-1}} a_i \partial_{[i]} a_{i+1} \Big) \\
& \qquad \cdot  \partial_{[i+1]} a_{i+2} \ldots \partial_{[n-1]} a_n \partial_{[n]} f + (-1)^n a_1 \partial_{[1]} a_2 \ldots \partial_{[n-1]} a_n \partial_{[n]} f \\
& = \partial_{[1]} a_1 \ldots \partial_{[n]} a_n f(y) \\
& = \Psi(\omega) f(y) \, ,
\end{align*}
where we used Lemma~\ref{TODO} in the third step. 
\end{proof}

\begin{theorem}\label{theorem:mainzorro} The Zorro map $\mathcal{Z}$ is homotopic to $1_X$.
\end{theorem}
\begin{proof}
Applying Lemma \ref{lemma:zorromain} repeatedly and using (\ref{strlambdaAtn} we find that for $\varphi \in \End(X)$
\begin{align*}
\str( \mathcal{Z} \varphi ) &= \big\langle\!\big\langle \str \big( \underline\lambda \At^n (\varphi)\big) \big\rangle\!\big\rangle\\
&= (-1)^n \sum_{\sigma \in S_n} (-1)^{|\sigma|} \big\langle\!\big\langle \nabla f_{\sigma(1)} \cdots \nabla f_{\sigma(n)} \str(\varphi) \big\rangle\!\big\rangle + \sum_{p=0}^{n-1} h_i( D \At^{n-p-1}(\varphi) )\,.
\end{align*}
%where 
%$$
%\delta = 
%\det 
%\begin{pmatrix}
%df_1 & df_2 & \cdots & df_n \\
%\vdots & \vdots & & \vdots \\
%df_1 & df_2 & \cdots & df_n 
%\end{pmatrix} .
%$$
%This concludes the proof that the Zorro map~\eqref{Zorro1superdetail} is homotopic to the identity. 
But by a small argument involving transition determinants this shows that
\[
\str( \mathcal{Z} \varphi ) = \str( \varphi )|_{x \mapsto y} + \sum_{p=0}^{n-1} h_i( D \At^{n-p-1}(\varphi) ).
\]
This shows that $\str( \mathcal{Z} \circ - )$ is homotopic to $\str( - )$ but this is not quite what we need. After all
\begin{equation}\label{eq:bracketkl}
\langle -, - \rangle_{\textup{KL}} = \frac{1}{n!} \sum_{\sigma \in S_n} \textup{sgn}(\sigma) \Res_{k[y,z]/k} \left[ \frac{ \str( - \circ - \circ \underline{\lambda} \underline{\mu}) \underline{\operatorname{d}\!y} \underline{\operatorname{d}\!z}}{\partial_{x_1}W \ldots \partial_{x_n} W \partial_{z_1} V \ldots \partial_{z_n} V} \right]
\end{equation}
But given $\varphi' \in \End(X)$ if we set $\varphi = \varphi' \circ \underline{\lambda} \underline{\mu}$ and substitute in (\ref{??}) we find that the two functionals $\langle \mathcal{Z}, - \rangle_{\textup{KL}}$ and $\langle 1_X, - \rangle_{\textup{KL}}$ differ by a term which is an expression in residues of
\begin{equation}\label{eq:mainzorro1}
h_p( D \At^{n-p-1}(\varphi) ) = h_p( D \At^{n-p-1}(\varphi' \underline{\lambda} \underline{\mu}) )
\end{equation}
with a sum over all $p$. So to prove that these two functionals are homotopic, it is enough to prove that the residue of (\ref{eq:mainzorro1}) is a function of $D(\varphi')$. But $\At$ is closed, so
\begin{align}
h_p( D \At^{n-p-1}(\varphi' \underline{\lambda} \underline{\mu}) ) &= h_p \At^{n-p-1}D(\varphi' \underline{\lambda} \underline{\mu}) ) \nonumber\\
&= h_p \At^{n-p-1}\big( D(\varphi') \underline{\lambda} \underline{\mu} )\\
&\qquad + h_p \At^{n-p-1}\big( \varphi' D(\underline{\lambda}) \underline{\mu} ) \label{eq:mainzorro4}\\
&\qquad + h_p \At^{n-p-1}\big( \varphi' \underline{\lambda} D(\underline{\mu}) ) \label{eq:mainzorro5}
\end{align}
Now $D(\underline{\mu})$ is a linear combination of terms divisible by the $\partial_{z_j} V$, and since $h_p$ and $\At$ are $k[z]$-linear these coefficients pass through to annihilate with the denominator in the residue in  (\ref{eq:bracketkl}). 

Similarly $\varphi' D(\underline{\lambda}) \underline{\mu}$ is a sum of terms of the form $f_j \alpha$ for various $j$. But the Atiyah class and the commutator $[\underline{\lambda}_{i_1,\ldots,i_p}, \nabla]$ are right linear (explain right action on $\Bar$) so
\begin{align*}
h_p \At^{n-p-1}\big( \alpha f_j ) &= h_p\left( \At^{n-p-1}(\alpha) \cdot f_j \right)\\
&= \sum_{\ell(\bs{i}) = p} (-1)^{\gamma(\bs{i}) +p} \Big\langle\!\Big\langle \nabla \left\{ f_{i_1} \cdots \nabla f_{i_p} \str\big( [ \underline{\lambda}_{i_1,\ldots,i_p}, \nabla]\At^{n-p-1}(\alpha) \big) \cdot f_j \right\} \Big\rangle\!\Big\rangle\,.
\end{align*}
But $\nabla( f_{i_1} \beta f_j ) = f_{i_1} \nabla( \beta f_j ) + df_{i_1} \beta f_j$. The first summand vanishes in the residue for obvious reasons, and the second vanishes since by Lemma \ref{??}
\[
\dlangle - \cdot f_j \drangle = \dlangle - \drangle \cdot f_j(y)
\]
which vanishes in the residue.

The upshot is that (\ref{eq:mainzorro4}) and (\ref{eq:mainzorro5}) do not contribute under the residue, so that
\[
\langle \mathcal{Z}, - \rangle_{\textup{KL}} - \langle 1_X, - \rangle_{\textup{KL}} = 
\frac{1}{n!} \sum_{\sigma \in S_n}\sum_{p=0}^{n-1} \textup{sgn}(\sigma) \Res_{k[y,z]/k} \left[ \frac{ h_p \At^{n-p-1}( D(-) \underline{\lambda} \underline{\mu} ) \underline{\operatorname{d}\!y} \underline{\operatorname{d}\!z}}{\partial_{x_1}W \ldots \partial_{x_n} W \partial_{z_1} V \ldots \partial_{z_n} V} \right]\,.
\]
This proves that $\langle \mathcal{Z}, - \rangle_{\textup{KL}}$ and $\langle 1_X, - \rangle_{\textup{KL}}$ are homotopic functionals, and so by the nondegeneracy of Theorem \ref{??} we conclude that $\mathcal{Z}$ and $1_X$ are homotopic.
\end{proof}

The three other Zorro moves are proven analogously. What remains to be done is to show that the same is true if we replace the completed bar complex~$\cBar$ by the unit object~$\Delta_W$ of the monoidal category $\HMF(k[x,y], \widetilde W)$.

\newpage

, as well as the fact that we use the completed bar complex~$\cBar$ as a model for the unit endomorphism of the object~$W$. Writing out the coevaluation~\eqref{TODO} the left-hand side of~\eqref{Zorro1detail} becomes 
\be\label{Zorro1superdetail}
\mathcal Z := 
\begin{tikzpicture}[very thick,scale=0.8,color=blue!50!black, baseline=1.7cm,line/.style={&gt;=latex}]

\draw (-1.3,0) -- (-1.3,4.8); 
\draw (-4.9,-3) -- (-4.9,4.8); 
\draw[redirected] (-1.3,4.8) .. controls +(0,1.75) and +(0,1.75) .. (-4.9,4.8);

\draw (0,0) -- (0,8.5); 

\draw[dashed] (1.3,0.4) -- (1.3,4.5);

\draw[dashed] (-2.6,0.4) -- (-2.6,1.1);
\draw[dashed] (-2.6,1.1) .. controls +(1,0.7) and +(-1,-0.7) .. (2.6,1.9);
\draw[dashed] (2.6,1.9) -- (2.6,3);

\draw[line width=0.75pt] 
(0,4.5) node[fill=white,draw,text width=5cm,align=center]{{\footnotesize$1_{X^\dual} \otimes 1_X \otimes (\varepsilon\circ\Psi)\vphantom{1_{\cBar} \otimes \sum_{b\geq 0} (-1)^b (\At_{X^\dual\otimes X})^b(\iota)}$}} 
(0,3) node[fill=white,draw,text width=5cm,align=center]{{\footnotesize$1_{X^\dual} \otimes 1_X \otimes \times\vphantom{1_{\cBar} \otimes \sum_{b\geq 0} (-1)^b (\At_{X^\dual\otimes X})^b(\iota)}$}}
(0,1.5) node[draw,text width=5cm,align=center]{{\footnotesize$\vphantom{1_{\cBar} \otimes \sum_{b\geq 0} (-1)^b \At(d_{X^\dual\otimes X})^b(\iota)}$}}
(0,0) node[fill=white,draw,text width=5cm,align=center]{{\footnotesize$1_{\cBar} \otimes \sum_{b\geq 0} (-1)^b (\At_{X^\dual\otimes X})^b(\iota)$}};

\fill (3.2,4.5) circle (0pt) node[right] {{\small $\gamma_4$}};
\fill (3.2,3) circle (0pt) node[right] {{\small $\gamma_3$}};
\fill (3.2,1.5) circle (0pt) node[right] {{\small $\gamma_2$}};
\fill (3.2,0) circle (0pt) node[right] {{\small $\gamma_1$}};

\fill (0,7.8) circle (2.5pt) node[right] {{\small $\lambda_X$}};
\fill (-4.9,-2.2) circle (2.5pt) node[left] {{\small $\rho_X^{-1}$}};

\draw[dashed] (-4.9,-2.2) .. controls +(0.25,1) and +(0,-1) .. (-2.6,-0.4);
\draw[dashed] (-3.1,6) .. controls +(0,1) and +(-0.25,-1) .. (0,7.8);

\end{tikzpicture}
\ee
where $\iota = \sum_j (-1)^{|e_j|} e_j^* \otimes e_j$, with $\{e_j\}$ a $k[z,x]$-basis of~$X$, corresponds to $1_E\in\End(X)$ in~$\gamma_1$, and the twist map~$\gamma_2$ produces Koszul signs coming from commuting non-commutative forms from the left to the very right. 

To prove that~\eqref{Zorro1superdetail} is indeed homotopic to the identity we first concentrate on the map 
$$
\Gamma := \gamma_3 \circ \gamma_2 \circ \gamma_1 \circ \rho_X^{-1}: X \longrightarrow X \otimes X^\dual \otimes X \otimes \cBar \, . 
$$
If we understand the bar complex as $\Omega_{k[x]}k[z,x]$ then as explained in Section~\ref{TODO} only $d_{X^\dual}$ contributes to the Atiyah class in~$\gamma_1$. As a result~$\Gamma$ is basically the shuffle product of two telescopic series of the Atiyah classes for~$X$ and $X^\dual$. This can be expressed in terms of the Atiyah class of the tensor product $X^\dual \otimes X$: 

\begin{lemma}
$\Gamma(e_q) = \sum_{n\geq 0} \sum_j (-1)^{|e_j| + n} (\At_{X\otimes X^\dual})^n(e_q \otimes e_j^* \otimes e_j)$. 
\end{lemma}

\begin{proof}
We compute $\Gamma(e_q)$, using $\At_X(e_q) = (-1)^{|e_q|+1} e_k \otimes d(d_X)_{kq}$ and paying attention to Koszul signs: 
\begin{align}
e_q & \stackrel{\rho_X^{-1}}{\longmapsto} \sum_{a\geq 0} (-1)^a \At_X^a(e_q) \nonumber \\
& \qquad\; = \sum_{a\geq 0} (-1)^{a + (|e_q|+1)+\ldots+(|e_q|+a)} e_{k_a} \otimes d(d_X)_{k_a k_{a-1}} \ldots d(d_X)_{k_1 q} \nonumber \\
& \stackrel{\gamma_1}{\longmapsto} \Big( \sum_{a\geq 0} (-1)^{a + a|e_q| + {a+1\choose 2}} e_{k_a} \otimes d(d_X)_{k_a k_{a-1}} \ldots d(d_X)_{k_1 q} \Big) \nonumber \\
& \qquad\; \otimes \Big( \sum_{b\geq 0} \sum_j (-1)^{|e_j|+b} (-1)^{(|e_j|+1)+\ldots+(|e_j|+b) + b|e_j|} e^*_{l_b} \otimes e_j \otimes d(d_{X^\dual})_{l_b l_{b-1}} \ldots d(d_{X^\dual})_{l_1 j} \Big) \nonumber \\
& \stackrel{\gamma_3 \circ \gamma_2}{\longmapsto} \sum_{n\geq 0} \sum_j (-1)^{|e_j|+n} \sum_{a=0}^n(-1)^{a|e_q| + {a+1\choose 2} + {n-a+1\choose 2}} \sum_{\sigma \in \Sh(n-a,a)} (-1)^{|\sigma|} e_{k_a} \otimes e^*_{l_{n-a}} \otimes e_j \nonumber \\
& \qquad\;   \otimes \sigma_\bullet \left( d(d_{X^\dual})_{l_{n-a} l_{n-a-1}} \ldots d(d_{X^\dual})_{l_1 j} d(d_X)_{k_a k_{a-1}} \ldots d(d_X)_{k_1 q} \right) \nonumber \\
& \qquad\; =  \sum_{n\geq 0} \sum_j (-1)^{|e_j|+n} \sum_{a=0}^n (-1)^{a|e_q| + {a+1\choose 2} + {n-a+1\choose 2} + a(n-a)} \sum_{\sigma \in \Sh(a,n-a)} (-1)^{|\sigma|}  \label{31} \\
& \qquad\qquad \cdot e_{k_a} \otimes e^*_{l_{n-a}} \otimes e_j \otimes \sigma_\bullet \left( d(d_X)_{k_a k_{a-1}} \ldots d(d_X)_{k_1 q} d(d_{X^\dual})_{l_{n-a} l_{n-a-1}} \ldots d(d_{X^\dual})_{l_1 j} \right) \nonumber \\
& \qquad\; =  \sum_{n\geq 0} \sum_j (-1)^{|e_j|+n} ( \At_X + \At_{X^\dual} )^n (e_q \otimes e^*_j \otimes e_j) \label{32} \\
& \qquad\; =  \sum_{n\geq 0} \sum_j (-1)^{|e_j|+n} (\At_{X\otimes X^\dual})^n (e_q \otimes e^*_j \otimes e_j) \, . \nonumber
\end{align}
To understand the penultimate step we note that the sign $(-1)^{a|e_q| + {a+1\choose 2} + {n-a+1\choose 2} + a(n-a)}$ with $\sigma=\operatorname{id}$ in~\eqref{31} is precisely that of the contribution to~\eqref{32} where $\At_{X^\dual}$ first acts $n-a$ times on $e_q\otimes e^*_j\otimes e_j$, followed by $\At_X^a$. The sign $(-1)^{|\sigma|}$ appears in~\eqref{32} if some $\At_X$ acts before some of the $\At_{X^\dual}$. 
\end{proof}







[TODO: better write this bit only once the section on $\cBar$ as a MF is in place] 

\begin{theorem}
TODO: $\LG$ has duals\ldots
\end{theorem}


\section{Defect action on bulk fields}\label{sec:defectaction}

In any bicategory with duals there are natural maps between the endomorphism spaces of unit 1-morphisms. Roughly, these maps are constructed by capturing a 2-morphism of a unit 1-morphism inside a loop labelled by an arbitrary 1-morphism (and its dual). Below we present the details for the case of the bicategory $\LG$. We will also give the interpretation in terms of defect actions on bulk fields in Landau-Ginzburg models. 

Let $X\in \hmf(k[z,x], V-W)$ as before. In this section when we write $\End$ we mean the spaces of 2-endomorphisms in $\LG$. We define maps 
$$
\mathcal D_l(X): \End(\Delta_V) \longrightarrow \End(\Delta_W) \, , \qquad
\mathcal D_r(X): \End(\Delta_W) \longrightarrow \End(\Delta_V)
$$
in terms of the morphisms encoding the monoidal and duality structures as follows. For $\phi\in \End(\Delta_V)$ and $\psi\in \End(\Delta_W)$ we set
\begin{align*}
\mathcal D_l(X)(\phi) & = \eval_X \circ (1_{X^\dual}\otimes (\lambda_X \circ (\phi\otimes 1_X)\circ \lambda_X^{-1})) \circ \widetilde\coev_X \, , \\ 
\mathcal D_r(X)(\phi) & = \widetilde\eval_X \circ (1_{X}\otimes (\lambda_{X^\dual} \circ (\phi\otimes 1_{X^\dual})\circ \lambda_{X^\dual}^{-1})) \circ \coev_X \, .
\end{align*}
Diagrammatically these definitions read 
\be\label{defectaction}
\mathcal D_l(X)(\phi) = 
\begin{tikzpicture}[very thick,scale=0.8,color=blue!50!black, baseline]

\fill (1.5,1.3) circle (2.5pt) node[right] {{\small $\lambda_X$}};
\fill (1.5,-1.3) circle (2.5pt) node[right] {{\small $\lambda^{-1}_X$}};
\fill (0,0) circle (2.5pt) node[left] {{\small $\phi$}};

\fill (0.6,1) circle (0pt) node {{\small $\Delta_V$}};
\fill (0.6,-0.95) circle (0pt) node {{\small $\Delta_V$}};

\draw[directed] (1.5,1.3) .. controls +(0,1.5) and +(0,1.5) .. (-1.5,1.3);
\draw[directed] (-1.5,-1.3) .. controls +(0,-1.5) and +(0,-1.5) .. (1.5,-1.3);
\draw (1.5,-1.3) -- (1.5,1.3)
node[midway,left] {{{\footnotesize$z'\vphantom{y}$}}}
node[midway,right] {{{\footnotesize$y\vphantom{yz'}$}}};
\draw (-1.5,-1.3) -- (-1.5,1.3)
node[midway,left] {{{\footnotesize$x\vphantom{yz'}$}}}
node[midway,right] {{{\footnotesize$z\vphantom{yz'}$}}};
\draw[dashed] (0,0) .. controls +(0,1) and +(-0.5,-1) .. (1.5,1.3);
\draw[dashed] (0,0) .. controls +(0,-1) and +(-0.5,1) .. (1.5,-1.3);
\draw[dashed] (0,-2.5) -- (0,-3.5)
node[near end,right] {{{\small$\Delta_W$}}};
\draw[dashed] (0,2.47) -- (0,3.5)
node[near end,right] {{{\small$\Delta_W$}}};
\end{tikzpicture}
\, , \qquad 
\mathcal D_r(X)(\psi) = 
\begin{tikzpicture}[very thick,scale=0.8,color=blue!50!black, baseline]

\fill (1.5,1.3) circle (2.5pt) node[right] {{\small $\lambda_{X^\dual}$}};
\fill (1.5,-1.3) circle (2.5pt) node[right] {{\small $\lambda^{-1}_{X^\dual}$}};
\fill (0,0) circle (2.5pt) node[left] {{\small $\psi$}};

\fill (0.6,1) circle (0pt) node {{\small $\Delta_W$}};
\fill (0.6,-0.95) circle (0pt) node {{\small $\Delta_W$}};

\draw[redirected] (1.5,1.3) .. controls +(0,1.5) and +(0,1.5) .. (-1.5,1.3);
\draw[redirected] (-1.5,-1.3) .. controls +(0,-1.5) and +(0,-1.5) .. (1.5,-1.3);
\draw (1.5,-1.3) -- (1.5,1.3)
node[midway,left] {{{\footnotesize$y\vphantom{yz'}$}}}
node[midway,right] {{{\footnotesize$z'\vphantom{yz'}$}}};
\draw (-1.5,-1.3) -- (-1.5,1.3)
node[midway,left] {{{\footnotesize$z\vphantom{yz'}$}}}
node[midway,right] {{{\footnotesize$x\vphantom{yz'}$}}};
\draw[dashed] (0,0) .. controls +(0,1) and +(-0.5,-1) .. (1.5,1.3);
\draw[dashed] (0,0) .. controls +(0,-1) and +(-0.5,1) .. (1.5,-1.3);
\draw[dashed] (0,-2.5) -- (0,-3.5)
node[near end,right] {{{\small$\Delta_V$}}};
\draw[dashed] (0,2.47) -- (0,3.5)
node[near end,right] {{{\small$\Delta_V$}}};
\end{tikzpicture}
\ee
where again we indicated our choice of variable names in the four domains. 

\begin{remark}
$\End(\Delta_W) = k[x]/(\partial_{x_i}W)$ is the Hochschild cohomology of $\hmf(k[x], W)$~\cite{d0904.4713}. This space also precisely describes bulk fields of Landau-Ginzburg models with potential~$W$, it is a commutative Frobenius algebra whose non-degenerate pairing
\be\label{bulktopmet}
\langle \phi, \psi \rangle_W = (-1)^n \Res_{k[x]/k} \left[ \frac{\phi \psi \, \underline{\operatorname{d}\!x}}{\partial_{x_1}W\ldots\partial_{x_n} W}\right]
\ee
describes two-point correlators. Furthermore, matrix factorisations of $V-W$ describe defect conditions between different Landau-Ginzburg models. Hence the maps~\eqref{defectaction} have the natural interpretations in terms of defect operators on bulk fields: for example, a bulk field~$\phi$ in the theory with potential~$V$ is mapped to the bulk field $\mathcal D_l(X)(\phi)$ in the theory with potential~$W$ by wrapping around its insertion on the worldsheet a defect line labelled by~$X$, and then collapsing this loop onto the insertion point. This limiting process is non-singular as the bicategory $\LG$ describes the purely topological sector of Landau-Ginzburg models. 
\end{remark}

Using the ``folding trick'' (which relates defects to boundary conditions in a product theory) one can argue for explicit expressions for $\mathcal D_l(X)$ and $\mathcal D_r(X)$. This was done in~\cite{cr1006.5609} for the case $V=W$. Here we use the duality structure to directly prove it for the general case: 

\begin{proposition}
For any $X\in \hmf(k[z_1,\ldots,z_m,x_1,\ldots,x_n], V-W)$, $\phi\in \End(\Delta_V)$ and $\psi\in \End(\Delta_W)$ we have
\begin{align*}
\mathcal D_l(X)(\phi) & = (-1)^{{m+1\choose 2} + {n\choose 2}} \Res_{k[z,x]/k[x]} \left[ \frac{\phi(z) \str\big( \partial_{z_1} d_{X}\ldots \partial_{z_m} d_{X} \partial_{x_1} d_{X}\ldots \partial_{x_n} d_{X} \big) \underline{\operatorname{d}\! z}}{\partial_{z_1} V \ldots \partial_{z_m} V} \right] \, , \\
\mathcal D_r(X)(\psi) & = (-1)^{{m\choose 2} + {n+1\choose 2}} \Res_{k[z,x]/k[z]} \left[ \frac{\psi(x) \str\big( \partial_{z_1} d_{X}\ldots \partial_{z_m} d_{X} \partial_{x_1} d_{X}\ldots \partial_{x_n} d_{X} \big) \underline{\operatorname{d}\! x}}{\partial_{x_1} W \ldots \partial_{x_n} W} \right] \, . 
\end{align*}
\end{proposition}

\begin{proof}
We treat the case of $\mathcal D_l(X)$ in detail, the argument for $\mathcal D_r(X)$ works analogously. Since $\End(\Delta_W) = k[x]/(\partial_{x_i}W)$ and $\End(\Delta_V) = k[z]/(\partial_{z_i}V)$ we are free to set $x=y$ and $z=z'$ at appropriate places, cf.~\eqref{defectaction}. Furthermore, $\lambda_X$ will project out all non-zero degree contributions coming from the action of $\lambda_X^{-1}$, so $\lambda_X\circ (\phi \otimes 1_X)\circ \lambda_X^{-1}$ is simply multiplication by the polynomial $\phi(z)$. 

In the lower part of the expression for $\mathcal D_l(X)(\phi)$ in~\eqref{defectaction} we have 
\begin{align}
\widetilde\coev (1) & = \sum_j (-1)^{|e_j|} (\varepsilon\Psi) \left( (-\At_{X})^n (e_j^*\otimes e_j) \right) \nonumber \\
& = \sum_j (-1)^{|e_j| + n} (-1)^{(|e_j| + 1)+\ldots +(|e_j| + n) + n|e_j|} e^*_{j} \otimes e_{k_n} \otimes (\varepsilon\Psi) \left( d(d_{X})_{k_n k_{n-1}} \ldots d(d_{X})_{k_1 j} \right) \nonumber \\
& = \sum_j (-1)^{|e_j|(|e_j| + n) + {n+1\choose 2} + n} e_j^* \otimes e_{k_n}\big( \partial_{[1]} d_{X} \ldots \partial_{[n]} d_{X} \big)_{k_n j} \nonumber \\
& = \sum_j (-1)^{|e_j|(|e_j| + n) + {n\choose 2}} e_j^* \otimes \big( \partial_{x_1} d_{X} \ldots \partial_{x_n} d_{X} \big) (e_{j}) \label{coevtilde1}
\end{align}
which we identify with $(-1)^{{n\choose 2}} \partial_{x_1} d_{X} \ldots \partial_{x_n} d_{X}$ in $\End(X)$. Note that in the last step leading to~\eqref{coevtilde1} we set $\partial_{[i]} d_{X}(x,z) = \partial_{x_i} d_{X}(x,z)$ since $x=y$ in $\End(\Delta_W)$. 

Next we apply the upper part of $\mathcal D_l(X)(\phi)$ in~\eqref{defectaction} to~\eqref{coevtilde1} to get
$$
\mathcal D_l(X)(\phi) = (-1)^{{m+1\choose 2} + {n\choose 2}} \Res_{k[z,x]/k[x]} \left[ \frac{\phi(z) \str\big( \partial_{z_1} d_{X}\ldots \partial_{z_m} d_{X} \partial_{x_1} d_{X}\ldots \partial_{x_n} d_{X} \big) \underline{\operatorname{d}\! z}}{\partial_{z_1} V \ldots \partial_{z_m} V} \right] + \mathcal O(\theta) \, . 
$$
Here we collectively denote the contributions from $\eval_X$ of non-zero degree in the Koszul complex $\Delta_W$ by $\mathcal O(\theta)$. Since we know that $\mathcal D_l(X)(\phi)$ is a morphism in $\End(\Delta_W) = k[x]/(\partial_{x_i}W)$ it follows that $\mathcal O(\theta)$ must be null-homotopic, thus concluding the proof. 
\end{proof}

\begin{corollary}
For any $X\in \hmf(k[z,x], V-W)$ the operators $\mathcal D_l(X)$ and $\mathcal D_r(X)$ are adjoint with respect to the pairings~\eqref{bulktopmet}, i.\,e.~we have
\be\label{Dadjoint}
\big\langle \mathcal D_l(X)(\phi), \psi \big\rangle_W = \big\langle \phi , \mathcal D_r(X)(\psi) \big\rangle_V
\ee
for all $\phi\in \End(\Delta_V)$ and $\psi\in \End(\Delta_W)$. 
\end{corollary}

\begin{remark}
We recall the physical interpretation of the relation~\eqref{Dadjoint}. Both sides of this equation are two-point correlators on the Riemann sphere, with a defect line labelled by~$X$ wrapped around counterclockwise the bulk field~$\phi$, or wrapped around~$\psi$ in clockwise fashion. That both correlators should be equal follows from the fact that the topological defect can be moved around the sphere at no cost: 
$$
\left\langle
\begin{tikzpicture}[baseline=-0.1cm]
\def\R{1.85}
\def\angEl{45}
\filldraw[ball color= white!77!blue,draw=white] (0,0) circle (\R);
\DrawLatitudeCircleU[\R,rotate=130,very thick, blue]{65}
\fill (-0.95,-0.83) circle (1pt) node[above] {{\small$\phi$}}; 
\fill (0.95,-0.83) circle (1pt) node[above] {{\small$\psi$}}; 
\end{tikzpicture}
\right\rangle
=
\left\langle
\begin{tikzpicture}[baseline=-0.1cm]
\def\R{1.85}
\def\angEl{45}
\filldraw[ball color= white!77!blue,draw=white] (0,0) circle (\R);
\DrawLongitudeCircle[\R]{80}
\fill (-0.95,-0.83) circle (1pt) node[above] {{\small$\phi$}}; 
\fill (0.95,-0.83) circle (1pt) node[above] {{\small$\psi$}}; 
\end{tikzpicture}
\right\rangle
=
\left\langle
\begin{tikzpicture}[baseline=-0.1cm]
\def\R{1.85}
\def\angEl{45}
\filldraw[ball color= white!77!blue,draw=white] (0,0) circle (\R);
\DrawLatitudeCircle[\R,rotate=-130, very thick, blue]{65}
\fill (-0.95,-0.83) circle (1pt) node[above] {{\small$\phi$}}; 
\fill (0.95,-0.83) circle (1pt) node[above] {{\small$\psi$}}; 
\end{tikzpicture}\right\rangle .
$$
\end{remark}


\section{Shadows}

The duality structure of $\LG$ affords us the construction of the bicategorical trace in terms of a shadows~\cite{p0807.1471}. We will also see that the shadow functor allows to recover the boundary-bulk and bulk-boundary maps of the two-dimensional topological field theories based on Landau-Ginzburg models. 

\begin{definition}
A bicategory~$\mathcal B$ \textsl{has shadows} if there is a category~$\mathcal C$ together with functors
$$
\langle\!\langle - \rangle\!\rangle : \mathcal B(A,A) \lra \mathcal C 
$$
for every object $A\in \mathcal B$ such that there are natural isomorphisms $\theta : \langle\!\langle X \otimes Y \rangle\!\rangle \lra \langle\!\langle Y \otimes X \rangle\!\rangle$ for every pair of composable 1-morphisms $X,Y$, and the diagrams
$$
\xymatrix{%
\langle\!\langle (X \otimes Y) \otimes Z \rangle\!\rangle \ar[r]^-{\theta} \ar[d]_-{\langle\!\langle \alpha \rangle\!\rangle} & 
\langle\!\langle Z \otimes (X \otimes Y) \rangle\!\rangle \ar[r]^-{\langle\!\langle \alpha^{-1} \rangle\!\rangle} & 
\langle\!\langle (Z \otimes X) \otimes Y \rangle\!\rangle \\
\langle\!\langle X \otimes (Y \otimes Z) \rangle\!\rangle \ar[r]^-{\theta} & 
\langle\!\langle (Y \otimes Z) \otimes X \rangle\!\rangle \ar[r]^-{\langle\!\langle \alpha \rangle\!\rangle} & 
\langle\!\langle Y \otimes (Z \otimes X) \rangle\!\rangle \ar[u]_-{\theta}
}%
$$
and
$$
\xymatrix{%
\langle\!\langle X \otimes 1_A \rangle\!\rangle \ar[r]^-{\theta} \ar[dr]_-{\langle\!\langle \rho \rangle\!\rangle} & 
\langle\!\langle 1_A \otimes X \rangle\!\rangle \ar[r]^-{\theta} \ar[d]^-{\langle\!\langle \lambda \rangle\!\rangle} & 
\langle\!\langle X \otimes 1_A \rangle\!\rangle \ar[dl]^-{\langle\!\langle \rho \rangle\!\rangle}  \\
 & 
\langle\!\langle X \rangle\!\rangle & 
}%
$$
commute whenever they make sense. 
\end{definition}

\begin{proposition}
The bicategory $\LG$ has shadows given by 
\begin{align*}
\langle\!\langle - \rangle\!\rangle : \LG \big( (R,W), (R,W) \big) & \lra \hmf(k,0) \, , \\
Z & \lmt Z\otimes_{\Re} R
\end{align*}
with the isomorphism $\theta: \langle\!\langle X \otimes Y \rangle\!\rangle \lra \langle\!\langle Y\otimes X  \rangle\!\rangle$ induced by the graded twist map $X \otimes Y \lra Y \otimes X$. 
\end{proposition}

The proof is a straightforward check of the axioms, made especially easy by the fact that $\langle\!\langle - \rangle\!\rangle$ is simply defined as tensoring with the actual diagonal~$R$. Note however that this is homotopy equivalent to tensoring with the unit matrix factorisation $\Delta_W$. 

Since $\LG$ is a bicategory with  duals and shadows it is automatically equipped with a 2-categorical trace operation as introduced and discussed at length in~\cite{p0807.1471, ps0910.1306}. We only quote the definition: 

\begin{definition}
Let~$\mathcal B$ be a bicategory with shadows and a dualisable 1-morphism~$Y$. Then the \textsl{trace} of a 2-morphism $\psi: X\otimes Y \lra Y \otimes Z$ is the map 
$$
\xymatrix{%
\langle\!\langle X \rangle\!\rangle \ar[rr]^-{\langle\!\langle1\otimes \coev_Y\rangle\!\rangle} && 
\langle\!\langle X \otimes Y \otimes Y^\dual \rangle\!\rangle \ar[r]^-{\langle\!\langle\psi \otimes 1\rangle\!\rangle} & 
\langle\!\langle Y \otimes Z \otimes Y^\dual \rangle\!\rangle \ar[r]^-{\theta} &
\langle\!\langle Y^\dual \otimes Y \otimes Z \rangle\!\rangle \ar[rr]^-{\langle\!\langle\eval_Y \otimes 1\rangle\!\rangle} && 
\langle\!\langle Z \rangle\!\rangle 
}%
\, . 
$$
\end{definition}

Next we wish to point out a connection between the shadow functor and the structure of two-dimensional open/closed topological field theory (TFT) for Landau-Ginzburg models. Recall that a TFT is the data of a commutative Frobenius algebra~$C$, a Calabi-Yau category~$\mathcal O$, bulk-boundary maps $\beta_A: C \lra \End_{\mathcal O}(A)$, and boundary-bulk maps $\beta^A: \End_{\mathcal O}(A) \lra C$ for all $A\in\mathcal O$. These data are subject to several consistency conditions, see e.\,g.~\cite{Kock}. 

Every Landau-Ginzburg model with potential $W\in R = k[x_1,\ldots,x_n]$ gives rise to a TFT with $C=R/(\partial W)$, $\mathcal O = \hmf(R,W)$ and
$$
\beta_Q : \phi \lmt \phi \cdot 1_Q \, , \qquad 
\beta^Q : \psi \lmt (-1)^{n\choose 2} \str(\psi \, \partial_{x_1} d_Q \ldots \partial_{x_n} d_Q) \, . 
$$
[TODO: give references for this?] We note that these maps can be recovered from the duality and shadow structure of $\LG$ as follows. On the one hand we have 
$$
\langle\!\langle \Delta_W \rangle\!\rangle = R/(\partial W)[n]
$$
since $\langle\!\langle \Delta_W \rangle\!\rangle = \Delta_W \otimes_{\Re} R = (\bigwedge (\bigoplus_{i=1}^n R \theta_i), \sum_{i=1}^n \partial_{x_i}W \theta_i)$ which is homotopy equivalent (and therefore equal in $\hmf(k,0)$) to $R/(\partial W)[n]$. On the other hand $\langle\!\langle Q^\dual \otimes Q \rangle\!\rangle = Q^\vee \otimes_k Q \otimes_{\Re} R = \End_R(Q)$. Thus from the explicit expressions~\eqref{TODO} and~\eqref{TODO} we find that $\beta_Q = \langle\!\langle \widetilde\coev_Q \rangle\!\rangle$ and $\beta^Q = \langle\!\langle \eval_Q \rangle\!\rangle$ (up to a sign, TODO?). 

Motivated by the above this construction can be extended to any 1-morphism in $\LG$: for $X \in \hmf(R_1 \otimes_k R_2, W_1 - W_2)$ we define the \textsl{generalised bulk-boundary} and \textsl{boundary-bulk maps} to be
$$
\langle\!\langle \widetilde\coev_X \rangle\!\rangle : \langle\!\langle \Delta_{W_2} \rangle\!\rangle \lra \langle\!\langle X^\dual \otimes X \rangle\!\rangle
\, , \qquad 
\langle\!\langle \eval_X \rangle\!\rangle : \langle\!\langle X^\dual \otimes X \rangle\!\rangle \lra \langle\!\langle \Delta_{W_2} \rangle\!\rangle \, , 
$$
respectively. 





\newcommand{\etalchar}[1]{$^{#1}$}
\providecommand{\href}[2]{#2}
\begin{thebibliography}{FYH{\etalchar{+}}85}

%\bibitem[AS]{as1105.5117}
%M.~Aganagic and S.~Shakirov, \textsl{Knot {H}omology from {R}efined
%  {C}hern-{S}imons {T}heory},
%  \href{http://arxiv.org/abs/1105.5117}{[arXiv:1105.5117]}.
%
%\bibitem[BN]{bnKhovanov11crossings}
%D.~Bar-Natan, \textsl{Khovanov {H}omology for {K}nots and {L}inks with up to 11
%  {C}rossings}, available at
%  \href{http://www.math.toronto.edu/drorbn/papers/KHTables/KHTables.pdf}{http:%
%//www.math.toronto.edu/drorbn/papers/KHTables/KHTables.pdf}.
%
%\bibitem[Bec]{b1105.0702}
%H.~Becker, \textsl{Khovanov-Rozansky homology via Cohen-Macaulay approximations and Soergel bimodules},
%  \href{http://arxiv.org/abs/1105.0702}{[arXiv:1105.0702]}.
%
%\bibitem[BR07]{br0707.0922}
%I.~Brunner and D.~Roggenkamp, \textsl{B-type defects in {L}andau-{G}inzburg
%  models}, JHEP \textbf{0708} (2007), 093,
%  \href{http://arxiv.org/abs/0707.0922}{[arXiv:0707.0922]}.
%
%\bibitem[CF94]{cf9405183}
%L.~Crane and I.~B. Frenkel, \textsl{Four dimensional topological quantum field
%  theory, {H}opf categories, and the canonical bases}, J. Math. Phys.
%  \textbf{35} (1994), 5136--5154,
%  \href{http://arxiv.org/abs/hep-th/9405183}{[hep-th/9405183]}.
%
%\bibitem[CK08a]{ck0701194}
%S.~Cautis and J.~Kamnitzer, \textsl{Knot homology via derived categories of
%  coherent sheaves {I}, {$sl(2)$} case}, Duke Math. J. \textbf{142} (2008),
%  511--588, \href{http://arxiv.org/abs/math/0701194}{[math.AG/0701194]}.
%
%\bibitem[CK08b]{ck0710.3216}
%S.~Cautis and J.~Kamnitzer, \textsl{Knot homology via derived categories of coherent sheaves {II},
%  {$sl(m)$} case}, Invent. Math. \textbf{174} (2008), 165--232,
%  \href{http://arxiv.org/abs/math/0710.3216}{[math.AG/0710.3216]}.
%
%\bibitem[CM]{cmWebCompileCode}
%N.~Carqueville and D.~Murfet, \textsl{Code to compute {K}hovanov-{R}ozansky
%  homology and defect fusion in {L}andau-{G}inzburg models},
%  \href{http://www.carqueville.net/nils/webCompilations}{http://www.carqueville.net/nils/webCompilations}.

\bibitem[CR10]{cr0909.4381}
N.~Carqueville and I.~Runkel, \textsl{On the monoidal structure of matrix bi-factorisations}, J. Phys.
  A: Math. Theor. \textbf{43} (2010), 275401,
  \href{http://arxiv.org/abs/0909.4381}{[arXiv:0909.4381]}.

\bibitem[CR12]{cr1006.5609}
N.~Carqueville and I.~Runkel, \textsl{Rigidity and defect actions in
  Landau-Ginzburg models}, Comm. Math. Phys. \textbf{310} (2012) 135--179, 
  \href{http://arxiv.org/abs/1006.5609}{[arXiv:1006.5609]}.

%\bibitem[Cra]{c0403266}
%M.~Crainic, \textsl{On the perturbation lemma, and deformations},
%  \href{http://arxiv.org/abs/math/0403266}{[math.AT/0403266]}.
%
%\bibitem[DGR06]{dgr0505662}
%N.~M. Dunfield, S.~Gukov, and J.~Rasmussen, \textsl{The {S}uperpolynomial for
%  {K}not {H}omologies}, Experimental Math. \textbf{15} (2006), 129--159,
%  \href{http://arxiv.org/abs/math/0505662}{[math.GT/0505662]}.
%
%\bibitem[DKR]{dkr1107.0495}
%A.~Davydov, L.~Kong, and I.~Runkel, \textsl{Field theories with defects and the
%  centre functor}, \href{http://arxiv.org/abs/1107.0495}{[arXiv:1107.0495]}.
%
%\bibitem[DBM{\etalchar{+}}11]{dbmmss1106.4305}
%P.~Dunin-Barkowski, A.~Mironov, A.~Morozov, A.~Sleptsov, A.~Smirnov, \textsl{Superpolynomials for toric knots from evolution induced by cut-and-join operators},
%  \href{http://arxiv.org/abs/1106.4305}{[arXiv:1106.4305]}. 
%  
\bibitem[Dyc11]{d0904.4713}
T.~Dyckerhoff, \textsl{Compact generators in categories of matrix factorizations},
  Duke Math. J. \textbf{159} (2011), 223--274,
  \href{http://arxiv.org/abs/0904.4713}{[arXiv:0904.4713]}.

\bibitem[DM]{dm1102.2957}
T.~Dyckerhoff and D.~Murfet, \textsl{Pushing forward matrix factorisations},
  \href{http://arxiv.org/abs/1102.2957}{[arXiv:1102.2957]}.

%\bibitem[FFRS07]{ffrs0607247}
%J.~Fr\"ohlich, J.~Fuchs, I.~Runkel, and C.~Schweigert, \textsl{Duality and
%  defects in rational conformal field theory}, Nucl. Phys. B \textbf{763}
%  (2007), 354--430,
%  \href{http://arxiv.org/abs/hep-th/0607247}{[hep-th/0607247]}.
%
%\bibitem[FYH{\etalchar{+}}85]{Homfly}
%P.~Freyd, D.~Yetter, J.~Hoste, W.~B.~R. Lickorish, K.~Millett, and A.~Ocneanu,
%  \textsl{A new polynomial invariant of knots and links}, Bull. Amer. Math. Soc.
%  \textbf{12} (1985), 239--246.
%
%\bibitem[GIKV10]{gikv0705.1368}
%S.~Gukov, A.~Iqbal, C.~Koz\c{c}az, and C.~Vafa, \textsl{Link {H}omologies and the
%  {R}efined {T}opological {V}ertex}, Comm. Math. Phys. \textbf{298} (2010),
%  757--785, \href{http://arxiv.org/abs/0705.1368}{[arXiv:0705.1368]}.
%
%\bibitem[GSV05]{gsv0412243}
%S.~Gukov, A.~Schwarz, and C.~Vafa, \textsl{Khovanov-{R}ozansky {H}omology and
%  {T}opological {S}trings}, Lett. Math. Phys. \textbf{74} (2005), 53--74,
%  \href{http://arxiv.org/abs/hep-th/0412243}{[hep-th/0412243]}.
%
%\bibitem[GV99]{gv9811131}
%R.~Gopakumar and C.~Vafa, \textsl{On the {G}auge {T}heory/{G}eometry
%  {C}orrespondence}, Adv. Theor. Math. Phys. \textbf{3} (1999), 1415--1443,
%  \href{http://arxiv.org/abs/hep-th/9811131}{[hep-th/9811131]}.
%
%\bibitem[GW]{gw0512298}
%S.~Gukov and J.~Walcher, \textsl{Matrix {F}actorizations and {K}auffman
%  {H}omology}, \href{http://arxiv.org/abs/hep-th/0512298}{[hep-th/0512298]}.
%
%\bibitem[Jae]{j1101.3302}
%T.~C. Jaeger, \textsl{Khovanov-{R}ozansky {H}omology and {C}onway {M}utation},
%  \href{http://arxiv.org/abs/1101.3302}{[arXiv:1101.3302]}.
%
%\bibitem[Jon85]{JonesPolynomialPaper}
%V.~F.~R. Jones, \textsl{A polynomial invariant for knots via von {N}eumann
%  algebras}, Bull. Amer. Math. Soc. \textbf{12} (1985), 103--111.
%
%\bibitem[Kap]{k1004.2307}
%A.~Kapustin, \textsl{Topological {F}ield {T}heory, {H}igher {C}ategories, and
%  {T}heir {A}pplications},
%  \href{http://arxiv.org/abs/1004.2307}{[arXiv:1004.2307]}.
%
%\bibitem[Kaw96]{kawauchibook}
%A.~Kawauchi, \textsl{A {S}urvey of {K}not {T}heory}, Birkh\"auser, 1996.
%
%\bibitem[Kho00]{k9908171}
%M.~Khovanov, \textsl{A categorification of the {J}ones polynomial}, Duke Math. J.
%  \textbf{101} (2000), 359--426,
%  \href{http://arxiv.org/abs/math/9908171}{[math.QA/9908171]}.
%
%\bibitem[Kho07]{k0510265}
%M.~Khovanov, \textsl{Triply-graded link homology and Hochschild homology of Soergel bimodules},
%  Int. Journal of Math. \textbf{18} (2007), 869--885,
%  \href{http://arxiv.org/abs/math/0510265}{[math.GT/0510265]}.
%
%\bibitem[KR07a]{kr0701333}
%M.~Khovanov and L.~Rozansky, \textsl{Virtual crossings, convolutions and a
%  categorification of the {$\operatorname{SO}(2N)$} {K}auffman polynomial},
%  Journal of G\"okova Geometry Topology \textbf{1} (2007), 116--214,
%  \href{http://arxiv.org/abs/math/0701333}{[math.QA/0701333]}.
%
%\bibitem[KR07b]{kr0404189}
%M.~Khovanov and L.~Rozansky, \textsl{Topological Landau-Ginzburg models on the world-sheet foam},
%  Adv. Theor. Math. Phys. \textbf{11} (2007), 233--259,
%  \href{http://arxiv.org/abs/hep-th/0404189}{[hep-th/0404189]}.
%
%\bibitem[KR08a]{kr0401268}
%M.~Khovanov and L.~Rozansky, \textsl{Matrix factorizations and link homology}, Fund. Math.
%  \textbf{199} (2008), 1--91,
%  \href{http://arxiv.org/abs/math/0401268}{[math/0401268]}.
%
%\bibitem[KR08b]{kr0505056}
%M.~Khovanov and L.~Rozansky, \textsl{Matrix factorizations and link homology {II}}, Geometry \&
%  Topology \textbf{12} (2008), 1387--1425,
%  \href{http://arxiv.org/abs/math/0505056}{[math.QA/0505056]}.

\bibitem[Ko]{Kock}
J.~Kock, \textsl{Frobenius Algebras and 2D Topological Quantum Field Theories}, Cambridge University Press, 2003.

%\bibitem[Lam86]{LambekRingsModules}
%J.~Lambek, \textsl{Lectures on rings and modules}, AMS Chelsea Publishing, 1986.

\bibitem[LM]{Calinetal}
C.~I. Lazaroiu and D.~McNamee, unpublished.

\bibitem[Lo]{Loday}
J.-L. Loday, \textsl{Cyclic homology}, Springer, 1997.

%\bibitem[LMnV00]{lmv0010102}
%J.~M.~F. Labastida, M.~Mari\~{n}o, and C.~Vafa, \textsl{Knot {I}nvariants and
%  {T}opological {S}trings}, JHEP \textbf{0011} (2000), 007,
%  \href{http://arxiv.org/abs/hep-th/0010102}{[hep-th/0010102]}.
%
%\bibitem[MSV09]{msv0708.2228}
%M.~Mackaay, M.~Sto\v{s}i\'{c}, and P.~Vaz, \textsl{$\mathfrak{sl}(N)$ link homology ($N\geq 4$) using foams and the Kapustin-Li formula}, Geometry \& Topology \textbf{13} (2009), 1075--1128,
%  \href{http://arxiv.org/abs/0708.2228}{[arXiv:0708.2228]}.
%
%\bibitem[Man07]{m0601629}
%C.~Manolescu, \textsl{Link homology theories from symplectic geometry}, Adv. in
%  Math. \textbf{211} (2007), 363--416,
%  \href{http://www.arxiv.org/abs/math.AG/0601629}{[math.SG/0601629]}.

\bibitem[McN09]{McNameethesis}
D.~McNamee, \textsl{On the mathematical structure of topological defects in
  {L}andau-{G}inzburg models}, MSc Thesis, Trinity College Dublin, 2009.

%\bibitem[Mn05]{marinoknotbook}
%M.~Mari\~{n}o, \textsl{Chern-{S}imons {T}heory, {M}atrix {M}odels, and
%  {T}opological s{t}rings}, Oxford University Press, 2005.
%
%\bibitem[MOY98]{moy1998}
%H.~Murakami, T.~Ohtsuki, and S.~Yamada, \textsl{Homfly polynomial via an
%  invariant of colored plane graphs}, Enseign. Math. \textbf{44} (1998),
%  325--360.
%
%\bibitem[MS]{ms0709.1971}
%V.~Manzorchuck and C.~Stroppel, \textsl{A combinatorial approach to functorial
%  quantum {$\mathfrak{sl}_k$} knot invariants},
%  \href{http://arxiv.org/abs/0709.1971}{[arXiv:0709.1971]}.
%
%\bibitem[OV00]{ov9912123}
%H.~Ooguri and C.~Vafa, \textsl{Knot {I}nvariants and {T}opological {S}trings},
%  Nucl. Phys. B \textbf{577} (2000), 419--438,
%  \href{http://arxiv.org/abs/hep-th/9912123}{[hep-th/9912123]}.

\bibitem[P10]{p0807.1471}
K.~Ponto, \textsl{Shadows and traces in bicategories}, 
Ast\'erisque, (333), 2010, 
\href{http://arxiv.org/abs/0807.1471}{[arXiv:0807.1471]}. 

\bibitem[PS]{ps0910.1306}
K.~Ponto and M.~Shulman, \textsl{Shadows and traces in bicategories}, 
\href{http://arxiv.org/abs/0910.1306}{[arXiv:0910.1306]}. 

%\bibitem[PT87]{HomflyPT}
%J.~Przytycki and P.~Traczyk, \textsl{Conway algebras and skein equivalence of
%  links}, Proc. Amer. Math. Soc. \textbf{100} (1987), 744--748.
%
%\bibitem[Ras]{r0607544}
%J.~Rasmussen, \textsl{Some differentials on {K}hovanov-{R}ozansky homology},
%  \href{http://arxiv.org/abs/math/0607544}{[math.GT/0607544]}.
%
%\bibitem[Ras07]{r0508510}
%J.~Rasmussen, \textsl{Khovanov-{R}ozansky homology of two-bridge knots and links},
%  Duke Math. J. \textbf{136} (2007), 551--583,
%  \href{http://arxiv.org/abs/math/0508510}{[math.GT/0508510]}.
%  
%\bibitem[Ric94]{rickard}
%J.~Rickard, \textsl{Translation functors and equivalences of derived categories for blocks of algebraic groups}, in “Finite dimensional algebras and related topics”, Kluwer (1994), 255-–264.
%
%\bibitem[Rou06]{RouquierMexico}
%R.~Rouquier, \textsl{Categorification of {$\mathfrak{sl}_{2}$} and braid groups},
%  Trends in representation theory of algebras and related topics (2006),
%  137--167.
%
%\bibitem[RT90]{RT1990}
%N.~Reshetikhin and V.~Turaev, \textsl{Ribbon graphs and their invariants derived
%  from quantum groups}, Comm. Math. Phys. \textbf{127} (190), 1--26.
%
%\bibitem[RT91]{RT1991}
%N.~Reshetikhin and V.~Turaev, \textsl{Invariants of 3-manifolds via link polynomials and quantum
%  groups}, Invent. Math. \textbf{103} (1991), 547--597.
%
%\bibitem[SS06]{ss0405089}
%P.~Seidel and I.~Smith, \textsl{A link invariant from the symplectic geometry of
%  nilpotent slices}, Duke Math. J. \textbf{134} (2006), 453--514,
%  \href{http://arxiv.org/abs/math/0405089}{[math.SG/0405089]}.

\bibitem[SW11]{sw0911.0917}
C.~V.~Shepler and S.~Witherspoon, \textsl{Quantum differentiation and chain maps of bimodule complexes}, Algebra and Number Theory \textbf{5}-3 (2011), 339--360, 
\href{http://arxiv.org/abs/0911.0917}{[arXiv:0911.0917]}. 

%\bibitem[Str05]{sCatTLcTCpf}
%C.~Stroppel, \textsl{Categorification of the {T}emperley-{L}ieb category,
%  tangles, and cobordisms via projective functors}, Duke Math. J. \textbf{126}
%  (2005), 547--596.
%
%\bibitem[Sus]{s0701045}
%J.~Sussan, \textsl{Category {$\mathcal O$} and {$\mathfrak{sl}_k$} link
%  invariants}, \href{http://arxiv.org/abs/math/0701045}{[math.QA/0701045]}.
%
%\bibitem[Tur88]{t1988YB}
%V.~Turaev, \textsl{The {Y}ang-{B}axter equation and invariants of links}, Invent.
%  Math. \textbf{92} (1988), 527--553.
%
%\bibitem[Tur10]{turaevbook}
%V.~Turaev, \textsl{Quantum invariants of knots and 3-manifolds}, de Gruyter, 2010,
%  2nd edition.
%
%\bibitem[Weba]{w0610650}
%B.~Webster, \textsl{Khovanov-Rozansky homology via a canopolis formalism},
%  \href{http://arxiv.org/abs/math/0610650}{[math.GT/0610650]}.
%
%\bibitem[Webb]{w1005.4559}
%B.~Webster, \textsl{Knot invariants and higher representation theory {II}: the
%  categorification of quantum knot invariants},
%  \href{http://arxiv.org/abs/1005.4559}{[arXiv:1005.4559]}.
%
%\bibitem[Wit]{w1101.3216}
%E.~Witten, \textsl{Fivebranes and {K}nots},
%  \href{http://arxiv.org/abs/1101.3216}{[arXiv:1101.3216]}.
%
%\bibitem[Wit89]{wittenjones}
%E.~Witten, \textsl{Quantum field theory and the {J}ones polynomial}, Comm. Math.
%  Phys. \textbf{121} (1989), 351--399.
%
%\bibitem[Wit95]{w9207094}
%E.~Witten, \textsl{Chern-{S}imons {G}auge {T}heory {A}s {A} {S}tring {T}heory},
%  Prog. Math. \textbf{133} (1995), 637--678,
%  \href{http://arxiv.org/abs/hep-th/9207094}{[hep-th/9207094]}.
%
%\bibitem[Wu]{w0907.0695}
%H.~Wu, \textsl{A colored {$\mathfrak{sl}(N)$}-homology for links in {$S^3$}},
%  \href{http://arxiv.org/abs/0907.0695}{[arXiv:0907.0695]}.
%
%\bibitem[Wu08]{w0508064}
%H.~Wu, \textsl{Braids, {T}ransversal links and the {K}hovanov-{R}ozansky {T}heory},
%  \href{http://arxiv.org/abs/math/0508064}{[math.GT/0508064]}, Trans. Amer. Math. Soc. \textbf{360}
%  (2008), 3365--3389.
%
%\bibitem[Yon]{y0906.0220}
%Y.~Yonezawa, \textsl{Quantum {$(\mathfrak{sl}_n, \wedge V_n)$} link invariant and
%  matrix factorizations},
%  \href{http://arxiv.org/abs/0906.0220}{[arXiv:0906.0220]}.

\end{thebibliography}

\end{document}
