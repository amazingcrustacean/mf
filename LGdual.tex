% Authors:  Nils Carqueville, Daniel Murfet
 
\documentclass{compositio}
\usepackage{stmaryrd}
\usepackage{amsmath, amscd, amssymb, mathrsfs, accents, amsfonts}
\usepackage{url}
\usepackage[all]{xy}
\usepackage{longtable}
\usepackage{dsfont}
\usepackage{tikz}
\def\nicecolourscheme{\shadedraw[top color=blue!22, bottom color=blue!22, draw=white]}
\def\nicenocolourscheme{\shadedraw[top color=gray!2, bottom color=gray!25, draw=white]}
\def\nicereallynocolourscheme{\shadedraw[top color=white!2, bottom color=white!25, draw=white]}
\definecolor{Myblue}{rgb}{0,0,0.6}
\usepackage[a4paper,colorlinks,citecolor=Myblue,linkcolor=Myblue,urlcolor=Myblue,pdfpagemode=None]{hyperref}

\SelectTips{cm}{}

\newtheorem{theorem}{Theorem}[section]
\newtheorem{proposition}[theorem]{Proposition}
\newtheorem{lemma}[theorem]{Lemma}
\newtheorem{corollary}[theorem]{Corollary}
\newtheorem*{theoremn}{Theorem}

\theoremstyle{definition}
\newtheorem{definition}[theorem]{Definition}
\newtheorem{example}[theorem]{Example}
\newtheorem{remark}[theorem]{Remark}
\newtheorem{s}[theorem]{}
\newtheorem*{setup}{Setup}
\newtheorem*{propositionn}{Proposition}

\numberwithin{equation}{section}

% Operators
\def\eval{\operatorname{ev}}
\def\coev{\operatorname{coev}}
\def\res{\operatorname{Res}}
\def\sg{\operatorname{sg}}
\def\Inj{\operatorname{Inj}}
\def\inc{\operatorname{inc}}
\def\Proj{\operatorname{Proj}}
\def\Coker{\operatorname{Coker}}
\def\Ker{\operatorname{Ker}}
\def\Im{\operatorname{Im}}
\def\free{\operatorname{free}}
\def\can{\operatorname{can}}
\def\ac{\operatorname{ac}}
\def\HH{\operatorname{HH}}
\def\K{\mathbf{K}}
\def\D{\mathbf{D}}
\def\N{\mathbf{N}}
\def\sing{\operatorname{Sg}}
\def\Hom{\operatorname{Hom}}
\def\uHom{\underline{\Hom}}
\def\modd{\operatorname{mod}}
\def\Modd{\operatorname{Mod}}
\def\Grmodd{\operatorname{GrMod}}
\def\CM{\operatorname{CM}}
\def\Ker{\operatorname{Ker}}
\def\Spec{\operatorname{Spec}}
\def\straightK{\operatorname{K}}
\def\straightC{\operatorname{C}}
\def\holim{\operatorname{hocolim}}
\DeclareMathOperator{\Ext}{Ext}
\DeclareMathOperator{\coh}{coh}
\DeclareMathOperator{\serre}{S}
\DeclareMathOperator{\Flat}{Flat}
\DeclareMathOperator{\qc}{qc}
\DeclareMathOperator{\Perf}{Perf}
\DeclareMathOperator{\Map}{Map}
\DeclareMathOperator{\Qco}{Qco}
\DeclareMathOperator{\Tr}{Tr}
\DeclareMathOperator{\End}{End}
\DeclareMathOperator{\rank}{rank}
\DeclareMathOperator{\tot}{Tot}
\DeclareMathOperator{\skos}{K}
\DeclareMathOperator{\hht}{ht}
\DeclareMathOperator{\depth}{depth}
\DeclareMathOperator{\STr}{STr}
\DeclareMathOperator{\tr}{tr}
\DeclareMathOperator{\ch}{ch}
\DeclareMathOperator{\str}{str}
\DeclareMathOperator{\hmf}{hmf}
\DeclareMathOperator{\HMF}{HMF}
\DeclareMathOperator{\HF}{HF}
\DeclareMathOperator{\pr}{pr}
\DeclareMathOperator{\At}{At}
\DeclareMathOperator{\mff}{mf}
\DeclareMathOperator{\MF}{MF}
\DeclareMathOperator{\Sh}{Sh}

\begin{document}

% Commands
\def\Res{\res\!}
\newcommand{\cat}[1]{\mathcal{#1}}
\newcommand{\lto}{\longrightarrow}
\newcommand{\xlto}[1]{\stackrel{#1}\lto}
\newcommand{\mf}[1]{\mathfrak{#1}}
\newcommand{\md}[1]{\mathscr{#1}}
\newcommand{\intvar}{\bs{x}_{\textup{int}}}
\newcommand{\extvar}{\bs{x}_{\textup{ext}}}
\newcommand{\qderu}[2]{\mathbf{D}^{#1}(#2)}
\newcommand{\ud}{\mathrm{d}}
\def\l{\,|\,}
\def\cf{\boldsymbol{cf}}
\def\bx{\boldsymbol{x}}
\def\by{\boldsymbol{y}}
\def\ba{\boldsymbol{a}}
\def\bb{\boldsymbol{b}}
\def\totimes{\otimes}
\def\di{Q}
\newcommand{\cotimes}[1]{\,\widehat{\otimes}_{#1}\,}
\def\QQ{\mathds{Q}}
\def\krc{C}
\def\diffm{d}
\def\diffh{d_{\chi}}
\def\redh{\overline{H}}
\def\ZZ{\mathds{Z}}
\def\bs{\boldsymbol}
\def\Ztwo{\mathds{Z}_2}
\def\mdual{^{\vee}}
\def\KR{\operatorname{KR}}
\def\I{\!\operatorname{i}\!}
\def\E{\operatorname{e}\!}
\def\sln{\mathfrak{sl}(N)}
\def\nN{\mathds{N}}
\def\nZ{\mathds{Z}}
\def\nQ{\mathds{Q}}
\def\nR{\mathds{R}}
\def\nC{\mathds{C}}
\def\Bar{\mathds{B}}
\def\cBar{\widehat{\mathds{B}}}
\def\Re{R^{\operatorname{e}}}
\def\Ae{A^{\operatorname{e}}}
\def\Be{B^{\operatorname{e}}}
\def\Aop{A^{\operatorname{op}}}
\def\Rop{R^{\operatorname{op}}}
\def\lra{\longrightarrow}
\def\lmt{\longmapsto}
\def\LG{\mathcal{LG}_k}
\def\dual{\dagger}
\def\dlangle{\big\langle\!\big\langle}
\def\drangle{\big\rangle\!\big\rangle}
\def\bigdlangle{\Big\langle\!\!\Big\langle}
\def\bigdrangle{\Big\rangle\!\!\Big\rangle}
\newcommand{\Ress}[1]{\res_{#1}\!}
\newcommand{\be}{\begin{equation}}
\newcommand{\ee}{\end{equation}}
\def\Xcirc{%
\begin{tikzpicture}[inner sep=0mm]
\node (X) at (0,0) {$X$};
\node (0) at (0,0) [circle,inner sep=0.99pt, thin,draw=black,fill= white] {};
\end{tikzpicture}%
}
\def\Xbul{%
\begin{tikzpicture}[inner sep=0mm]
\node (X) at (0,0) {$X$};
\node (0) at (0,0) [circle,inner sep=0.99pt, thin,draw=black,fill= black] {};
\end{tikzpicture}%
}

\allowdisplaybreaks

\usetikzlibrary{arrows,calc,decorations.pathreplacing,decorations.markings,shapes.geometric,shadows}
\tikzset{
    string/.style={draw=#1, postaction={decorate}, decoration={markings,mark=at position .51 with {\arrow[draw=#1]{>}}}},
    costring/.style={draw=#1, postaction={decorate}, decoration={markings,mark=at position .51 with {\arrow[draw=#1]{<}}}},
    ostring/.style={draw=#1, postaction={decorate}, decoration={markings,mark=at position .47 with {\arrow[draw=#1]{>}}}},
    ustring/.style={draw=#1, postaction={decorate}, decoration={markings,mark=at position .56 with {\arrow[draw=#1]{>}}}},
    oostring/.style={draw=#1, postaction={decorate}, decoration={markings,mark=at position .43 with {\arrow[draw=#1]{>}}}},
    uustring/.style={draw=#1, postaction={decorate}, decoration={markings,mark=at position .59 with {\arrow[draw=#1]{>}}}},
    directed/.style={string=blue!50!black}, 
    odirected/.style={ostring=blue!50!black}, 
    udirected/.style={ustring=blue!50!black}, 
    oodirected/.style={oostring=blue!50!black}, 
    uudirected/.style={uustring=blue!50!black},     
    redirected/.style={costring= blue!50!black},
}

\usetikzlibrary{fadings,decorations.pathreplacing}

\newcommand\pgfmathsinandcos[3]{%
  \pgfmathsetmacro#1{sin(#3)}%
  \pgfmathsetmacro#2{cos(#3)}%
}
\newcommand\LongitudePlane[3][current plane]{%
  \pgfmathsinandcos\sinEl\cosEl{#2} % elevation
  \pgfmathsinandcos\sint\cost{#3} % azimuth
  \tikzset{#1/.estyle={cm={\cost,\sint*\sinEl,0,\cosEl,(0,0)}}}
}
\newcommand\LatitudePlane[3][current plane]{%
  \pgfmathsinandcos\sinEl\cosEl{#2} % elevation
  \pgfmathsinandcos\sint\cost{#3} % latitude
  \pgfmathsetmacro\yshift{\cosEl*\sint}
  \tikzset{#1/.estyle={cm={\cost,0,0,\cost*\sinEl,(0,\yshift)}}} %
}
\newcommand\DrawLongitudeCircle[2][1]{
  \LongitudePlane{\angEl}{#2}
  \tikzset{current plane/.prefix style={scale=#1}}
  \pgfmathsetmacro\angVis{atan(sin(#2)*cos(\angEl)/sin(\angEl))} %
  \draw[redirected,current plane,color=blue!50!black, very thick] (\angVis:1) arc (\angVis:\angVis+180:1);
  \draw[current plane,dotted,color=blue!50!gray, very thick] (\angVis-180:1) arc (\angVis-180:\angVis:1);
}
\newcommand\DrawLatitudeCircle[2][1]{
  \LatitudePlane{\angEl}{#2}
  \tikzset{current plane/.prefix style={scale=#1}}
  \pgfmathsetmacro\sinVis{sin(#2)/cos(#2)*sin(\angEl)/cos(\angEl)}
  \pgfmathsetmacro\angVis{asin(min(1,max(\sinVis,-1)))}
  \draw[directed,current plane, color=blue!50!black] (\angVis:1) arc (\angVis:-\angVis-180:1);
  \draw[current plane,dashed, color=blue!50!gray] (180-\angVis:1) arc (180-\angVis:\angVis:1);
}
\newcommand\DrawLatitudeCircleU[2][1]{
  \LatitudePlane{\angEl}{#2}
  \tikzset{current plane/.prefix style={scale=#1}}
  \pgfmathsetmacro\sinVis{sin(#2)/cos(#2)*sin(\angEl)/cos(\angEl)}
  \pgfmathsetmacro\angVis{asin(min(1,max(\sinVis,-1)))}
  \draw[redirected,current plane, color=blue!50!black] (\angVis:1) arc (\angVis:-\angVis-180:1);
  \draw[current plane,dashed, color=blue!50!gray] (180-\angVis:1) arc (180-\angVis:\angVis:1);
}





\title{Rise of the Planet of the Coevaluations}
\author{Nils Carqueville}
\email{nils.carqueville@physik.uni-muenchen.de}
\address{Arnold Sommerfeld Center for Theoretical Physics, LMU M\"unchen \& Excellence Cluster Universe}

\author{Daniel Murfet}
\email{daniel.murfet@math.ucla.edu}
\address{Department of Mathematics, UCLA}

\classification{TODO}

\begin{abstract}
They take over. 
\end{abstract}

\maketitle


\section{Background}\label{sec:Background}

\subsection{Bicategory of Landau-Ginzburg models}\label{subsec:bicatLG}

Let us fix the ring $R=k[x_1,\ldots,x_n]$, where~$k$ is a commutative noetherian $\nQ$-algebra; relevant examples are $k=\nC$ and $k=\nC[t_1,\ldots,t_d]$. Typically we will simply write $k[x]$ for $k[x_1,\ldots,x_n]$. We call an element $W\in R$ a \textsl{potential} if $R/(\partial_{x_i}W)$ is finitely generated projective over~$k$ and the $\partial_{x_i}W$ form a regular sequence. In this case $k=\nC$ this simply means that the Jacobi ring $R/(\partial_{x_i}W)$ is a finite-dimensional vector space. 

A \textsl{linear factorisation} of a potential $W\in R$ is a $\nZ_2$-graded $R$-module $X=X^0\oplus X^1$ together with an odd $R$-linear endomorphism~$d_X$ such that $d_X^2=W\cdot 1_X$. If~$X$ is a free $R$-module then the pair $(X,d_X)$ is called a \textsl{matrix factorisation}, and we often refer to it simply by~$X$ without explicitly mentioning the \textsl{differential} $d_X$. 

A \textsl{morphism} of linear factorisations $(X,d_X)$ and $(Y,d_Y)$ is an even $R$-linear map $\varphi: X \longrightarrow Y$ such that $d_Y \varphi = \varphi d_X$. Two morphisms $\varphi, \psi: X\lra Y$ are \textsl{homotopic} if there exists an odd $R$-linear map $\lambda:X\lra Y$ such that $d_Y\lambda + \lambda d_X = \psi-\varphi$. Equality up to homotopy is an equivalence relation. 

The \textsl{(homotopy) category of linear factorisations} $\HF(R,W)$ is the category of linear factorisations of $W\in R$ modulo homotopy relations. We denote by $\HMF(R,W)$ its full subcategory of matrix factorisations, and we write $\hmf(R,W)$ for the full subcategory of matrix factorisations whose objects are isomorphic to matrix factorisations with finitely-generated underlying $R$-modules. These three categories have standard triangulated structures whose shift functor we denote as~$[1]$. 

Given a potential $W\in R=k[x]$ there is always the \textsl{unit matrix factorisation} $\Delta_W \in \hmf(\Re, \widetilde W)$ where $\Re = R\otimes_k R$ and $\widetilde W = W\otimes 1 - 1\otimes W$. Introducing~$n$ formal symbols $\theta_i$ as a convenient notational device, we have 
$$
\Delta_W = \bigwedge \Big( \bigoplus_{i=1}^n \Re \theta_i \Big)
$$
as an $\Re$-module whose $\nZ_2$-grading is given by $\theta$-degree modulo~2. We write 
$$
\pi: \Delta_W \lra R
$$
for the projection $\Delta_W \lra \Re$ to the $\theta$-degree~$0$ component composed with multiplication $\Re \lra R$. Typically we will omit the wedge product and write  e.\,g.~$\theta_i\wedge \theta_j$ simply as $\theta_i \theta_j$. To describe the differential $d_{\Delta_W}$ we further need the variable-changing map
$$
{}^{t_i}(-): k[x,y] \lra k[x,y] \, , \qquad f \lmt f\big|_{x_i\lmt y_i}
$$
in terms of which we can define difference quotient operators
$$
\partial_{[i]}: k[x,y] \lra k[x,y] \, , \qquad f \lmt \frac{{}^{t_1\ldots t_{i-1}}f - {}^{t_1\ldots t_i}f}{x_i-y_i} \, . 
$$
It is easy to check that the $\partial_{[i]}$ satisfy the following kind of Leibniz rule: 
\begin{lemma}\label{lem:LeibnizForDQO}
TODO: Leibniz rule for $\partial_{[i]}$, don't give proof
\end{lemma}

The differential on $\Delta_W$ is given by
\be\label{DeltaW}
d_{\Delta_W} = \delta_+ + \delta_- \, , \qquad \delta_+ = \sum_{i=1}^n \partial_{[i]}W\cdot \theta_i\, , \qquad \delta_- =  \sum_{i=1}^n (x_i-y_i) \cdot \theta_i^*  \,. 
\ee
We call $\Delta_W$ the unit matrix factorisation as it is the unit with respect to the tensor product of matrix factorisations. For our purposes it will be sufficient to consider matrix factorisations $X\in \HMF(R_1\otimes_k R_2, W_2-W_1)$ and $Y\in \HMF(R_2\otimes_k R_3, W_3-W_2)$ (where of course $R_i = k$ is a possibility for any~$i$). Then the \textsl{tensor product matrix factorisation} $Y\otimes X \in \HMF(R_1\otimes_k R_3, W_3-W_1)$ is the module 
$$
Y\otimes X = \Big( (Y^0\otimes_{R_2} X^0) \oplus (Y^1\otimes_{R_2} X^1) \Big) \oplus \Big( (Y^0\otimes_{R_2} X^1) \oplus (Y^1\otimes_{R_2} X^0) \Big) 
$$
together with the differential
$$
d_{Y\otimes X} = d_Y \otimes 1 + 1 \otimes d_X
$$
where the second term comes with the usual Koszul signs when applied to elements. We  observe that e.\,g.~by the argument of~\cite[Section~12]{dm1102.2957} the tensor product does not lead out of the categories $\hmf(R_i \otimes_k R_j, W_j - W_i)$. Also note that there are obvious natural isomorphisms $\alpha_{X,Y,Z}: (X\otimes Y)\otimes Z \lra X\otimes (Y \otimes Z)$. 

Now we can be more specific about the tensor action of the unit matrix factorisation. For $X\in \hmf(R_1 \otimes_k R_2,W_2-W_1)$ there are natural maps
\be\label{lambdarho}
\lambda_X = \pi \otimes 1_X: \Delta_{W_2} \otimes X \lra X \, , \qquad \rho_X = 1_X \otimes \pi : X \otimes \Delta_{W_1} \lra X
\ee
which are isomorphisms in $\hmf(R_1 \otimes_k R_2,W_2-W_1)$. Later in Section~\ref{TODO} we will give a description of their explicit homotopy inverses. 

\begin{definition}
The \textsl{bicategory of Landau-Ginzburg models} $\LG$ consists of the following data: 
\begin{itemize}
\item Objects are pairs $(R,W)$ with $W\in R=k[x]$ a potential. 
\item 1- and 2-morphisms are the objects and morphisms of the categories $\hmf(R_1 \otimes_k R_2,W_2-W_1)$, respectively. 
\item The unit 1-morphisms are $\Delta_W \in \hmf(\Re,\widetilde W)$. 
\item The composition functor is the tensor product
$$
\otimes : \hmf(R_2 \otimes_k R_3,W_3-W_2) \times \hmf(R_1 \otimes_k R_2,W_2-W_1) \lra \hmf(R_1 \otimes_k R_3,W_3-W_1) \, .
$$
\item There are natural 2-isomorphisms $\alpha, \lambda, \rho$ as above. 
\end{itemize}�
\end{definition}

\begin{proposition}[\cite{McNameethesis, Calinetal, cr0909.4381}] 
$\LG$ really is a bicategory, i.\,e.~$\alpha, \lambda, \rho$ are natural isomorphisms up to homotopy, and they satisfy the coherence axioms for bicategories. 
\end{proposition}

One of the main results of the present paper is that $\LG$ is endowed with an additional duality structure. Let us recall the relevant notion. 

\begin{definition}\label{def:bicatduals}
We say a bicategory~$\mathcal B$ \textsl{has adjoints} if for each 1-morphism $X\in \mathcal B(A,B)$ there exists a 1-morphism $X^\dual \in \mathcal B(B,A)$ together with 2-morphisms 
\be\label{evcoev}
\eval_X : X^\dual \otimes X \lra 1_A \, , \qquad \coev_X : 1_B \lra X \otimes X^\dual
\ee
satisfying 
\begin{align}
\rho_X \circ (1_X \otimes \eval_X) \circ \alpha_{X,X^\dual,X} \circ (\coev_X \otimes 1_X) \lambda_X^{-1} & = 1_X \, , \label{uglyZorro1}\\
\lambda_{X^\dual} \circ (\eval_X \otimes 1_{X^\dual}) \circ \alpha^{-1}_{X^\dual,X,X^\dual} \circ (1_{X^\dual} \otimes \coev_X) \circ \rho_{X^\dual}^{-1} & = 1_{X^\dual} \, . \label{uglyZorro2}
\end{align}
\end{definition}

It is convenient (and also suggested by the motivation discussed in the Introduction) to denote identities in bicategories with duals in string diagram notation, see e.\,g.~\cite{TODO} for more details. In this language the evaluation and coevaluation maps~\eqref{evcoev} are written as
$$
\eval_{X} = 
\begin{tikzpicture}[very thick,scale=1.0,color=blue!50!black, baseline=.6cm]
\draw[line width=0pt] 
(2.5,1.6) node[line width=0pt] (I) {{\small$1_A$}}
(3,0) node[line width=0pt] (D) {{\small $X\vphantom{X^\dual}$}}
(2,0) node[line width=0pt] (s) {\small{$X^\dual$}}; 
\draw[directed] (D) .. controls +(0,1) and +(0,1) .. (s);
\draw[dashed] (2.5,0.81) -- (I);
\end{tikzpicture}
\equiv
\begin{tikzpicture}[very thick,scale=1.0,color=blue!50!black, baseline=.6cm]
\draw[line width=0pt] 
(3,0) node[line width=0pt] (D) {{\small$X\vphantom{X^\dual}$}}
(2,0) node[line width=0pt] (s) {{\small$X^\dual$}}; 
\draw[directed] (D) .. controls +(0,1) and +(0,1) .. (s);
\end{tikzpicture} , 
\qquad
\coev_{X} = 
\begin{tikzpicture}[very thick,scale=1.0,color=blue!50!black, baseline=-.6cm,rotate=180]
\draw[line width=0pt] 
(2.5,1.6) node[line width=0pt] (I) {{\small$1_B$}}
(3,0) node[line width=0pt] (D) {{\small$X\vphantom{X^\dual}$}}
(2,0) node[line width=0pt] (s) {{\small$X^\dual$}}; 
\draw[redirected] (D) .. controls +(0,1) and +(0,1) .. (s);
\draw[dashed] (2.5,0.81) -- (I);
\end{tikzpicture}
\equiv
\begin{tikzpicture}[very thick,scale=1.0,color=blue!50!black, baseline=-.6cm,rotate=180]
\draw[line width=0pt] 
(3,0) node[line width=0pt] (D) {{\small$X\vphantom{A^\dual}$}}
(2,0) node[line width=0pt] (s) {{\small$X^\dual$}}; 
\draw[redirected] (D) .. controls +(0,1) and +(0,1) .. (s);
\end{tikzpicture} \, . 
$$
Note that such diagrams are always to be read from bottom to top. Then the defining relations~\eqref{uglyZorro1} and~\eqref{uglyZorro2} translate into the \textsl{Zorro moves}
\be\label{Zorros}
\begin{tikzpicture}[very thick,scale=1.0,color=blue!50!black, baseline=0cm]
\draw[line width=0] 
(-1,1.25) node[line width=0pt] (A) {{\small $X$}}
(1,-1.25) node[line width=0pt] (A2) {{\small $X$}}; 
\draw[directed] (0,0) .. controls +(0,-1) and +(0,-1) .. (-1,0);
\draw[directed] (1,0) .. controls +(0,1) and +(0,1) .. (0,0);
\draw (-1,0) -- (A); 
\draw (1,0) -- (A2); 
\end{tikzpicture}
=
\begin{tikzpicture}[very thick,scale=1.0,color=blue!50!black, baseline=0cm]
\draw[line width=0] 
(0,1.25) node[line width=0pt] (A) {{\small $X$}}
(0,-1.25) node[line width=0pt] (A2) {{\small $X$}}; 
\draw (A2) -- (A); 
\end{tikzpicture}
\, , \qquad
\begin{tikzpicture}[very thick,scale=1.0,color=blue!50!black, baseline=0cm]
\draw[line width=0] 
(1,1.25) node[line width=0pt] (A) {{\small $X^\dual$}}
(-1,-1.25) node[line width=0pt] (A2) {{\small $X^\dual$}}; 
\draw[directed] (0,0) .. controls +(0,1) and +(0,1) .. (-1,0);
\draw[directed] (1,0) .. controls +(0,-1) and +(0,-1) .. (0,0);
\draw (-1,0) -- (A2); 
\draw (1,0) -- (A); 
\end{tikzpicture}
=
\begin{tikzpicture}[very thick,scale=1.0,color=blue!50!black, baseline=0cm]
\draw[line width=0] 
(0,1.25) node[line width=0pt] (A) {{\small $X^\dual$}}
(0,-1.25) node[line width=0pt] (A2) {{\small $X^\dual$}}; 
\draw (A2) -- (A); 
\end{tikzpicture} \, .
\ee

The adjoint~$X^\dual$ of~$X$ as described in Definition~\ref{def:bicatduals} is more precisely called \textsl{right adjoint} to~$X$, and one says that~$X^\dual$ is \textsl{left adjoint} to~$X$ if there are maps 
$$
\widetilde\eval_{X} = 
\begin{tikzpicture}[very thick,scale=1.0,color=blue!50!black, baseline=.6cm]
\draw[line width=0pt] 
(2.5,1.6) node[line width=0pt] (I) {{\small$1_B$}}
(3,0) node[line width=0pt] (D) {{\small $X^\dual\vphantom{X^\dual}$}}
(2,0) node[line width=0pt] (s) {\small{$X\vphantom{X^\dual}$}}; 
\draw[redirected] (D) .. controls +(0,1) and +(0,1) .. (s);
\draw[dashed] (2.5,0.81) -- (I);
\end{tikzpicture}
\equiv
\begin{tikzpicture}[very thick,scale=1.0,color=blue!50!black, baseline=.6cm]
\draw[line width=0pt] 
(3,0) node[line width=0pt] (D) {{\small$X^\dual$}}
(2,0) node[line width=0pt] (s) {{\small$X\vphantom{X^\dual}$}}; 
\draw[redirected] (D) .. controls +(0,1) and +(0,1) .. (s);
\end{tikzpicture} , 
\qquad
\widetilde\coev_{X} = 
\begin{tikzpicture}[very thick,scale=1.0,color=blue!50!black, baseline=-.6cm,rotate=180]
\draw[line width=0pt] 
(2.5,1.6) node[line width=0pt] (I) {{\small$1_A$}}
(3,0) node[line width=0pt] (D) {{\small$X^\dual$}}
(2,0) node[line width=0pt] (s) {{\small$X\vphantom{X^\dual}$}}; 
\draw[directed] (D) .. controls +(0,1) and +(0,1) .. (s);
\draw[dashed] (2.5,0.81) -- (I);
\end{tikzpicture}
\equiv
\begin{tikzpicture}[very thick,scale=1.0,color=blue!50!black, baseline=-.6cm,rotate=180]
\draw[line width=0pt] 
(3,0) node[line width=0pt] (D) {{\small$X^\dual$}}
(2,0) node[line width=0pt] (s) {{\small$X\vphantom{X^\dual}$}}; 
\draw[directed] (D) .. controls +(0,1) and +(0,1) .. (s);
\end{tikzpicture}
$$
satisfying the associated Zorro moves
\be\label{otherZorros}
\begin{tikzpicture}[very thick,scale=1.0,color=blue!50!black, baseline=0cm]
\draw[line width=0] 
(1,1.25) node[line width=0pt] (A) {{\small $X$}}
(-1,-1.25) node[line width=0pt] (A2) {{\small $X$}}; 
\draw[redirected] (0,0) .. controls +(0,1) and +(0,1) .. (-1,0);
\draw[redirected] (1,0) .. controls +(0,-1) and +(0,-1) .. (0,0);
\draw (-1,0) -- (A2); 
\draw (1,0) -- (A); 
\end{tikzpicture}
=
\begin{tikzpicture}[very thick,scale=1.0,color=blue!50!black, baseline=0cm]
\draw[line width=0] 
(0,1.25) node[line width=0pt] (A) {{\small $X$}}
(0,-1.25) node[line width=0pt] (A2) {{\small $X$}}; 
\draw (A2) -- (A); 
\end{tikzpicture}
\, , \qquad
\begin{tikzpicture}[very thick,scale=1.0,color=blue!50!black, baseline=0cm]
\draw[line width=0] 
(-1,1.25) node[line width=0pt] (A) {{\small $X^\dual$}}
(1,-1.25) node[line width=0pt] (A2) {{\small $X^\dual$}}; 
\draw[redirected] (0,0) .. controls +(0,-1) and +(0,-1) .. (-1,0);
\draw[redirected] (1,0) .. controls +(0,1) and +(0,1) .. (0,0);
\draw (-1,0) -- (A); 
\draw (1,0) -- (A2); 
\end{tikzpicture}
=
\begin{tikzpicture}[very thick,scale=1.0,color=blue!50!black, baseline=0cm]
\draw[line width=0] 
(0,1.25) node[line width=0pt] (A) {{\small $X^\dual$}}
(0,-1.25) node[line width=0pt] (A2) {{\small $X^\dual$}}; 
\draw (A2) -- (A); 
\end{tikzpicture} \, .
\ee
We will mostly continue to simply speak of adjoints without a further attribute as we will see that in $\LG$ left and right adjoints coincide. 

Checking that the Zorro moves hold in $\LG$ with the evaluation and coevaluation maps~\eqref{TODO} and duals $X^\dual = X^\vee[n]$ (all to be discussed in detail in Section~\ref{sec:derivcoeval}) is no easy task. Instead of tackling it directly we will in an intermediate step first prove the Zorro moves using a different model for the unit matrix factorisation, namely the completed bar complex of the ring~$R$. As preparation we shall next discuss some background on non-commutative forms. 

TODO: graded stuff

\subsection{Bar complex}\label{subsec:Bar}

Let us for the moment be slightly more general and consider an arbitrary unital associative $k$-algebra~$A$. In our applications to matrix factorisations we will set $A=R=k[x]$. 

\textsl{Non-commutative $n$-forms over~$A$} are elements in 
$$
\Omega^n A = A\otimes \bar A^{\otimes n} 
$$
where $\bar A = A/k$ and in this section by ``$\otimes$'' we mean ``$\otimes_k$''. We denote the projection of $a_0\otimes a_1 \otimes \ldots \otimes a_n \in A^{\otimes n}$ to $\Omega^n A$ as $(a_0,a_1,\ldots,a_n)$. The direct sum 
$$
\Omega A = \bigoplus_{n\geq 0} \Omega^n A
$$
is a differential graded algebra $(\Omega A, d, \cdot)$ with multiplication given by
$$
(a_0,\ldots,a_m) \cdot (a_{m+1},\ldots, a_{m+n}) = \sum_{i=0}^m (-1)^{m-i}(a_0,\ldots,a_{i-1},a_i a_{i+1},a_{i+2},\ldots, a_{m+n}) 
$$
and differential
$$
d: (a_0,\ldots,a_n) \lmt (1,a_0,\ldots,a_n)
$$
where $a_i\in A$. We will write $(a_0,a_1,\ldots,a_n)$ also as $a_0da_1\ldots da_n$. 

More generally one can consider relative non-commutative forms: for a subalgebra $B\subset A$ they are elements in $\Omega_B A = \bigoplus_{n\geq 0} A\otimes_B (A/B)^{\otimes_B n}$, which has a differential graded structure analogous to $\Omega A = \Omega_k A$. We refer to the book~\cite{Loday} for further details. 

A central role is played by the \textsl{(normalised) bar complex} 
$$
\Bar = \bigoplus_{n\geq 0} \Bar_n \, , \qquad \Bar_n = \Omega^n A \otimes A \, .
$$
It is an $(A^{\text{op}} \otimes A)$-module via $(a\otimes a').(a_0da_1\ldots da_n\otimes a_{n+1}) = aa_0da_1\ldots da_n\otimes a_{n+1}a'$. Together with the differential $d\otimes 1_A$ (which by standard abuse of notation we usually simply write~$d$) and the product induced from $\Omega A$ and~$A$, the bar complex~$\Bar$ is a differential graded algebra $(\Bar,d,\cdot)$. 

There is a second differential graded structure on~$\Bar$ if the algebra~$A$ is commutative. To describe it let us first recall that (still for arbitrary~$A$) the bar complex is the standard resolution 
$$
\xymatrix{%
\cdots \ar[r]^-{b'} & A \otimes \bar A^{\otimes 2} \otimes A \ar[r]^-{b'} & A \otimes \bar A \otimes A \ar[r]^-{b'} & A\otimes A \ar[r]^-{b'} & A \ar[r] & 0
}%
$$
of~$A$, where the degree-lowering differential~$b'$ is the $A$-bilinear map
$$
b': (a_0,\ldots,a_n)\otimes a_{n+1} \lmt \sum_{i=0}^{n-1} (-1)^i (a_0,\ldots,a_i a_{i+1},\ldots, a_n) \otimes a_{n+1} + (-1)^n (a_0,\ldots,a_{n-1}) \otimes a_n a_{n+1} \, . 
$$
%Equivalently, in differential form notation~$b'$ acts as
%$$
%b': da_0\ldots da_n\otimes a_{n+1} \lmt (-1)^{n-1} a_0da_1\ldots da_{n-1} \cdot a_n \otimes a_{n+1} + (-1)^n a_0 da_1\ldots da_{n-1} \otimes a_n a_{n+1} \, . 
%$$
From this it is straightforward to check that we have the identity
\be\label{b'd+db'}
b'd+db'=1_{\Bar} \, .
\ee

From now on we assume that~$A$ is commutative. Recall that $(m,n)$-shuffles are permutations in
$$
\operatorname{Sh}(m,n) = \big\{ \sigma\in S_{m+n} \,|\, \sigma(1)<\sigma(2)<\ldots<\sigma(m), \, \sigma(m+1)<\sigma(m+2)<\ldots<\sigma(m+n) \big\} \, . 
$$
We use them to define the $A$-bilinear \textsl{shuffle product}~$\times$ on~$\Bar$ as
\begin{align*}
& (a_0da_1\ldots da_m \otimes a_{m+1}) \times (b_0db_1\ldots db_n \otimes b_{n+1}) \\
& \qquad 
= \sum_{\sigma_{\operatorname{Sh}(m,n)}} (-1)^{|\sigma|} a_0 b_0 \, \sigma_\bullet (da_1\ldots da_m db_1 \ldots db_n) \otimes a_{m+1} b_{n+1}
\end{align*}
where $\sigma_\bullet(da_1\ldots da_j) = da_{\sigma^{-1}(1)}\ldots da_{\sigma^{-1}(j)}$. One finds that $(\Bar,b',\times)$ is a graded-commutative differential graded algebra. Note that for $\omega\in A\otimes A=\Bar_0$ it follows immediately that $\omega\times(-) = \omega\cdot(-)$. 

We now return to the $k$-algebra $A=R=k[x_1,\ldots,x_n]$. Earlier we set $\widetilde W = W\otimes 1 - 1\otimes W \in \Re$ for a potential $W\in R$, in terms of which we now define the bar complex endomorphism
$$
d_{\Bar} = b' + d\widetilde W \times (-) \, . 
$$
\begin{lemma}\label{lemma:barisafactorisation}
$(\Bar,d_{\Bar})$ is a linear factorisation of $\widetilde W\in \Re$. 
\end{lemma}

\begin{proof}
The bar complex $\Bar=\Bar^0 \oplus \Bar^1$ is $\nZ_2$-graded with $\Bar^i = \bigoplus_{n\in 2\nN+i}\Bar_n$. Since~$b'$ and~$\times$ are both $R$-bilinear, $d_{\Bar}$ is indeed $\Re$-linear. Furthermore, we have $b'^2=0$ and $d\widetilde W \times d\widetilde W = (dW\otimes 1) \times (dW\otimes 1) = dWdW\otimes 1 - dWdW \otimes 1 = 0$, so that for $\omega\in\Bar$ we find 
\begin{align*}
d_{\Bar}^2 (\omega) & = b'(d\widetilde W \times \omega) + d\widetilde W \times b'(\omega) \\
& = b'(d\widetilde W) \times \omega - d\widetilde W \times b'(\omega) + d\widetilde W \times b'(\omega) \\
& = \widetilde W \times \omega \\
& = \widetilde W \cdot \omega
\end{align*}
where in the second last step we used~\eqref{b'd+db'} together with $b'(\widetilde W)=0$. 
\end{proof}

If we use~$\pi$ also to denote the projection $\Bar�\lra \Bar_0 = \Re$ composed with multiplication $\Re \lra R$, then $\pi \otimes 1_X : \Bar \otimes X \lra X$ and $1_X \otimes \pi: X \otimes \Bar \lra X$ give left and right actions of~$\Bar$ as in~\eqref{lambdarho}. These maps have homotopy inverses too and we will construct them in Section~\ref{TODO}. For the moment we take the fact that~$\Bar$ is another model for the unit action on matrix factorisations as motivation to discuss its relation to the Koszul matrix factorisation~$\Delta_W$. Before we do this on the level of linear factorisations we shall consider the special case $W=0$. We write $\Delta = \Delta_0 = \bigwedge( \bigoplus_{i=1}^n \Re \theta_i)$ and observe that now $(\Delta_W, d_{\Delta_W})$ reduces to the ordinary Koszul complex $(\Delta, \delta_-)$, see~\eqref{DeltaW}

There are two $\Re$-linear maps between~$\Bar$ and~$\Delta$ which will be important to us: 
\begin{align}
\Phi & : \Delta \lra \Bar \, , \qquad \theta_{i_1}\ldots \theta_{i_p} \lmt \sum_{\sigma\in S_p} (-1)^{|\sigma|} dx_{i_{\sigma(1)}} \ldots dx_{i_{\sigma(p)}} \otimes 1 \, , \label{intro_phi}\\
\Psi & : \Bar \lra \Delta \, , \qquad df_1\ldots df_p \otimes 1 \lmt \sum_{1\leq i_1<\ldots<i_p\leq n} \Big( \prod_{k=1}^p \partial_{[i_k]} f_k \Big) \, \theta_{i_1} \ldots \theta_{i_p} \, \label{intro_psi}.
\end{align}
These maps were studied in~\cite{sw0911.0917}, we only rephrase the presentation of~$\Psi$ in terms of the difference quotient operators~$\partial_{[i]}$ suitable for our setting. One easily verifies that $\Psi\Phi = 1_\Delta$. 

\begin{lemma}\label{PhiPsiDG}
Both~$\Phi$ and~$\Psi$ are maps of differential graded algebras between $(\Delta, \delta_-, \wedge)$ and $(\Bar, b', \times)$. 
\end{lemma}

\begin{proof}
We refer to~\cite{sw0911.0917} for the case of~$\Phi$; since our expression for~$\Psi$ is not manifestly the same as in loc.~cit.~we spell out the proof. Let us first show that~$\Psi$ is compatible with the differentials. On the one hand we compute $(\delta_- \Psi) (df_1\ldots df_p \otimes 1)$ to be 
\begin{align}
& \delta_ -\Big( \sum_{i_1<\ldots <i_p} (\partial_{[i_1]} f_1) \ldots (\partial_{[i_p]} f_p) \, \theta_{i_1} \ldots \theta_{i_p} \Big) \nonumber \\
& =  \sum_{j=1}^n \sum_{i_1<\ldots <i_p} (\partial_{[i_1]} f_1) \ldots (\partial_{[i_p]} f_p) \cdot (x_j - y_j) \sum_{k=1}^p (-1)^{k+1} \delta_{j i_k} \theta_{i_1} \ldots \widehat{\theta_{i_k}} \ldots \theta_{i_p}  \nonumber \\
& = \sum_{k=1}^p (-1)^{k+1}\sum_{i_1<\ldots <i_p} (\partial_{[i_1]} f_1) \ldots ({}^{t_1\ldots t_{i_{k-1}}} f_k - {}^{t_1\ldots t_{i_{k}}} f_k) \ldots (\partial_{[i_p]} f_p) \, \theta_{i_1} \ldots \widehat{\theta_{i_k}} \ldots \theta_{i_p}  \nonumber \\
& = \sum_{2\leq t_2<\ldots< i_p} (f_1 - {}^{t_1\ldots t_{i_{2}-1}} f_1) (\partial_{[i_2]} f_2) \ldots (\partial_{[i_p]} f_p) \, \theta_{i_2} \ldots \theta_{i_p} \nonumber \\
& \qquad + \sum_{k=2}^{p-1} (-1)^{k+1}\sum_{i_1,\ldots,i_p} (\partial_{[i_1]} f_1) \ldots ({}^{t_1\ldots t_{i_{k-1}}} f_k - {}^{t_1\ldots t_{i_{k+1}-1}} f_k) \ldots (\partial_{[i_p]} f_p) \, \theta_{i_1} \ldots \widehat{\theta_{i_k}} \ldots \theta_{i_p}  \nonumber \\
& \qquad + (-1)^{p+1} \sum_{i_1<\ldots< i_{p-1}\leq n-1} (\partial_{[i_1]} f_1) \ldots (\partial_{[i_{p-1}]} f_{p-1}) ({}^{t_1\ldots t_{i_{p-1}}} f_p - {}^{t_1\ldots t_n} f_p)  \, \theta_{i_1} \ldots \theta_{i_{p-1}} \nonumber \\
& = \sum_{2\leq t_2<\ldots <i_p} f_1 (\partial_{[i_2]} f_2) \ldots (\partial_{[i_p]} f_p) \, \theta_{i_2} \ldots \theta_{i_p} \nonumber \\
& \qquad + (-1)^{p} \sum_{i_1<\ldots< i_{p-1}\leq n-1} (\partial_{[i_1]} f_1) \ldots (\partial_{[i_{p-1}]} f_{p-1}) \, {}^{t_1\ldots t_n} f_p  \, \theta_{i_1} \ldots \theta_{i_{p-1}}  \label{delPsi} 
\end{align}
while on the other hand $(\Psi b') (df_1\ldots df_p \otimes 1)$ equals
\begin{align*}
& \Psi \Big( f_1 df_2 \ldots df_p \otimes 1 + \sum_{k=1}^{p-1} (-1)^k df_1 \ldots d(f_k f_{k+1}) \ldots df_p \otimes 1 + (-1)^p df_1 \ldots df_{p-1} \otimes f_p \Big) \\
& = \sum_{i_1<\ldots< i_{p-1}} f_1 (\partial_{[i_1]} f_2) \ldots (\partial_{[i_{p-1}]} f_p) \, \theta_{i_1} \ldots \theta_{i_{p-1}} \\
& \qquad + \sum_{k=1}^{p-1} (-1)^k \sum_{i_1<\ldots< i_{p-1}} (\partial_{[i_1]} f_1) \ldots (\partial_{[i_k]} (f_k f_{k+1}) )\ldots (\partial_{[i_{p-1}]} f_p) \, \theta_{i_1} \ldots \theta_{i_{p-1}} \\
& \qquad + (-1)^p \sum_{i_1<\ldots< i_{p-1}} (\partial_{[i_1]} f_1) \ldots (\partial_{[i_{p-1}]} f_{p-1}) \, {}^{t_1\ldots t_n} f_p \, \theta_{i_1} \ldots \theta_{i_{p-1}} \\
& = \sum_{2\leq i_1<\ldots< i_{p-1}} f_1 (\partial_{[i_1]} f_2) \ldots (\partial_{[i_{p-1}]} f_p) \, \theta_{i_1} \ldots \theta_{i_{p-1}} \\ 
& \qquad + \sum_{2\leq i_2 \ldots< i_{p-1}} f_1 (\partial_{[1]} f_2) (\partial_{[i_2]} f_3) \ldots (\partial_{[i_{p-1}]} f_p) \, \theta_{i_1} \ldots \theta_{i_{p-1}} \\ 
& \qquad + \sum_{k=1}^{p-1} (-1)^k \sum_{i_1<\ldots< i_{p-1}} (\partial_{[i_1]} f_1) \ldots \big\{ ({}^{t_1\ldots t_{i_k -1}}f_k)(\partial_{[i_k]} f_{k+1}) \\
& \qquad\quad + (\partial_{[i_k]} f_k) ({}^{t_1\ldots t_{i_k}} f_{k+1}) \big\} \ldots (\partial_{[i_{p-1}]} f_p) \, \theta_{i_1} \ldots \theta_{i_{p-1}} \\
& \qquad + (-1)^p \sum_{i_1<\ldots< i_{p-1}\leq n-1} (\partial_{[i_1]} f_1) \ldots (\partial_{[i_{p-1}]} f_{p-1}) \, {}^{t_1\ldots t_n} f_p \, \theta_{i_1} \ldots \theta_{i_{p-1}} \\
& \qquad\quad + (-1)^p \sum_{i_1<\ldots< i_{p-2}\leq n-1} (\partial_{[i_1]} f_1) \ldots (\partial_{[i_{p-1}]} f_{p-1}) \, {}^{t_1\ldots t_n} f_p \, \theta_{i_1} \ldots \theta_{i_{p-1}} \\
& = \sum_{2\leq i_1<\ldots< i_{p-1}} f_1 (\partial_{[i_1]} f_2) \ldots (\partial_{[i_{p-1}]} f_p) \, \theta_{i_1} \ldots \theta_{i_{p-1}} \\ 
& \qquad + (-1)^p \sum_{i_1<\ldots< i_{p-1}\leq n-1} (\partial_{[i_1]} f_1) \ldots (\partial_{[i_{p-1}]} f_{p-1}) \, {}^{t_1\ldots t_n} f_p \, \theta_{i_1} \ldots \theta_{i_{p-1}} 
\end{align*}
which agrees with~\eqref{delPsi}. 

To establish compatibility with the products we compute 
\begin{align*}
& \Psi( df_1\ldots df_p \otimes 1) \wedge \Psi( df_{p+1}\ldots df_{p+q} \otimes 1) \\ 
=\, & \Big\{ \sum_{i_1<\ldots< i_p} \Big( \prod_{k=1}^p \partial_{[i_k]} f_k \Big) \theta_{i_1} \ldots \theta_{i_p}\Big\} 
\wedge 
\Big\{ \sum_{i_{p+1}<\ldots< i_{p+q}} \Big( \prod_{k=p+1}^{p+q} \partial_{[i_k]} f_k \Big) \theta_{i_{p+1}} \ldots \theta_{i_{p+q}} \Big\} \\
= \, & \sum_{i_1<\ldots< i_p} \sum_{i_{p+1}<\ldots< i_{p+q}} \Big( \prod_{k=1}^{p+q} \partial_{[i_k]} f_k \Big) \theta_{i_{1}} \ldots \theta_{i_{p+q}} \\
= \, & \sum_{i_1<\ldots< i_{p+q}} \sum_{\sigma\in\operatorname{Sh}(p,q)} (-1)^{|\sigma|} \Big( \prod_{k=1}^{p+q} \partial_{[i_{\sigma(k)}]} f_k \Big) \theta_{i_{1}} \ldots \theta_{i_{p+q}} \\
= \, & \Psi \big((df_1\ldots df_p \otimes 1) \times( df_{p+1}\ldots df_{p+q} \otimes 1) \big)
\end{align*}
where in the third step the anti-commutativity of the~$\theta_i$ allowed us to sum over the longer sequences $i_1<\ldots< i_{p+q}$ by introducing an additional sum over shuffles. 
\end{proof}

Now we come back to consider any $W\in R$. The map~$\Psi$ continues to be a good map on the level of linear factorisations: 
\begin{lemma}\label{PsiHF}
$\Psi: (\Bar, d_\Bar) \lra (\Delta_W, d_{\Delta_W})$ is a morphism in $\HF(\Re,\widetilde W)$. 
\end{lemma}

\begin{proof}
We need to show $d_{\Delta_W} \Psi = \Psi d_\Bar$. But since $d_{\Delta_W} = \delta_+ + \delta_-$ and $d_\Bar = b' + d\widetilde W \times (-)$ by Lemma~\eqref{PhiPsiDG} what remains to be checked is $\delta_+ \Psi = \Psi (d\widetilde W\times (-))$. This can be done: 
$$
\delta_+ \Psi = \Big( \sum_{i=1}^n \partial_{[i]} W \cdot \theta_i^* \Big) \wedge \Psi(-) = \Psi(dW\otimes 1) \wedge \Psi(-) = \Psi (d\widetilde W\times (-)) \, . 
$$
\end{proof}


TODO: Say something about $\Phi$ here too


%\begin{remark}
%Instead of the bar complex $\Bar = \bigoplus_{n\geq 0} \Bar_n$ one can also consider its completed version
%$$
%\cBar = \prod_{n\geq 0} \Bar_n \, .
%$$
%It has differential graded structures $(\cBar,d,\cdot)$ and $(\cBar,b',\times)$ analogous to~$\Bar$, and the maps~$\Phi$ and~$\Psi$ lift to maps to and from the completed bar complex~$\cBar$ with the same properties as in Lemmas~\ref{PhiPsiDG} and~\ref{PsiHF}. 
%\end{remark}

\subsection{Perturbation}

A crucial role will be played by the homological perturbation lemma, which we will use to promote homotopy equivalences of complexes (arising from the bar and Koszul resolutions of the diagonal) to homotopy equivalences of associated matrix factorisations. More importantly, the perturbation lemma provides explicit homotopy inverses in terms of Atiyah classes.

Let $R$ be a ring and $W \in R$. An $R$-linear \textsl{deformation retract datum} is a diagram
\begin{equation}\label{eq:perturbeddiagr1}
\xymatrix@C+2pc{
(X,d_X) \ar@<-1ex>[r]_-{\sigma} & (Y,d_Y) \ar@<-1ex>[l]_-{\pi}
}%
\!\!\!\xymatrix{%
{}\ar@(ur,dr)[]^{h}
}%
\end{equation}
in which $(X,d_X)$ and $(Y,d_Y)$ are linear factorisations of $W$, $\pi, \sigma$ are morphisms of linear factorisations and $h: Y \lto Y$ is a degree one $R$-linear map such that
$$
\pi \sigma = 1 \, , \qquad
\sigma \pi = 1 + d_Yh + hd_Y \, .
$$
A degree one morphism $\delta: Y \lto Y$ is a \textsl{small perturbation} of the deformation retract datum if $1_M - \delta h$ is an isomorphism of $R$-modules. In this case we define
\[
\tau = (1- \delta h)^{-1} \delta
\]
and consider the new ``perturbed'' diagram
\begin{equation}\label{eq:perturbeddiagr}
\xymatrix@C+2pc{
(X,d_{X,\infty}) \ar@<-1ex>[r]_-{\sigma_\infty} & (Y,d_Y+\delta) \ar@<-1ex>[l]_-{\pi_\infty}
}%
\!\!\!\xymatrix{%
{}\ar@(ur,dr)[]^{h_\infty}
}%
\end{equation}
where
\begin{align*}
\sigma_\infty &= \sigma + h\tau\sigma \,, & h_\infty &= h + h \tau h\,,\\
\pi_\infty &= \pi + \pi \tau h \, , & d_{X,\infty} &= d_X + \pi \tau \sigma\,.
\end{align*}

\begin{proposition}\label{prop:pertlemma} Suppose that $h \sigma = 0, \pi h = 0$ and $h^2 = 0$. If $\delta$ is a small perturbation of (\ref{eq:perturbeddiagr1}) such that $(d_Y + \delta)^2 = W' \cdot 1_M$ for some $W' \in R$ then (\ref{eq:perturbeddiagr}) is a deformation retract datum of linear factorisations of $W'$ over $R$.
\end{proposition}
\begin{proof}
Just reference Crainic.
\end{proof}

In the cases of interest to us the sum $\sum_{m \ge 0} (\delta h)^m$ converges, so that $\tau = \sum_{m \ge 0} (\delta h)^m \delta$ and
\[
\sigma_\infty = \sigma + \sum_{m \ge 0} h(\delta h)^m \delta \sigma = \sum_{m \ge 0} (h \delta)^m \sigma\,.
\]

\section{Inverting the unit actions}\label{section:pertandhtpy}

As part of the structure of the bicategory $\LG$ we have specified, for any finite rank matrix factorisation $X \in \hmf(k[x,z], V - W)$, a pair of $k[z]$-$k[x]$-bilinear homotopy equivalences
\begin{equation}\label{eq:pertandhtpy1}
\lambda: \Delta_V \otimes X \lto X, \qquad \rho: X \otimes \Delta_W \lto X
\end{equation}
which we call the \emph{unit actions}. A representing chain map for the inverses of these morphisms is not specified as part of the defining data of the bicategory, and finding such representatives is nontrivial. Indeed, the fundamental technical result in this paper is the calculation of explicit homotopy inverses for $\lambda, \rho$ in terms of associative Atiyah classes. For this we use the commutative diagram
\begin{equation}\label{eq:pertdia1}
\xymatrix{
X \,\widehat{\otimes}\, \Bar \ar[dr]_-{\rho'}\ar[rr]^{1 \otimes \Psi} & & X \otimes \Delta_W \ar[dl]^-{\rho}\\
& X
}
\end{equation}
where $\Psi$ is the canonical map given in (\ref{intro_psi}) and $\rho'$ is explained below. Rather than trying to invert $\rho$ directly, it turns out to be advantageous to invert $\rho'$ and then compose with $\Psi$. Roughly speaking the inverse of $\rho'$ is the geometric series in powers of the associative Atiyah class of $X$ and we make use of a completion of $X \otimes \Bar$ in order to guarantee that this geometric series converges.

The bulk of the argument is given in the following general setting: $k$ is an arbitrary commutative ring, $R, S$ are $k$-algebras and
\[
X \in \HMF(S \otimes_k \Re, V)
\]
is given together with a homogeneous basis $\{ e_i \}_{i \in I}$. The reader should keep in mind the special case where $S = k[z], R = k[x]$ and $X$ the extension of scalars from $S \otimes_k R$ to $S \otimes_k \Re$ of some factorisation of $V - W$ over $k[x,z]$. Let $D$ denote the differential on $X$ and unless specified otherwise $\otimes$ denotes the tensor product $\otimes_{\Re}$. Consider the $S \otimes_k \Re$-module
\[
X \,\widehat{\otimes}\, \Bar := \prod_{m \ge 0} X \otimes \Bar_m
\]
with the $\mathbb{Z}_2$-grading
\[
\big( X \,\widehat{\otimes}\, \Bar \big)^i = \left( \prod_{m \in \mathbb{Z}} X^i \otimes \Bar_{2m} \right) \oplus \left( \prod_{m \in \mathbb{Z}} X^{i+1} \otimes \Bar_{2m+1} \right) . 
\]
As a $\mathbb{Z}_2$-graded $S \otimes_k \Re$-module, $X \,\widehat{\otimes}\, \Bar$ is the inverse limit of the system
\[
\cdots \lto X \otimes \Bar/\Bar_{\ge 2} \lto X \otimes \Bar/\Bar_{\ge 1}
\]
where $\Bar_{\ge m} = \bigoplus_{i \ge m} \Bar_i \subseteq \Bar$ and the maps are the obvious quotients $\Bar/\Bar_{\ge m+1} \lto \Bar/\Bar_{\ge m}$.

Let $W \in A$ be arbitrary and set $\widetilde{W} = W \otimes 1 - 1 \otimes W \in \Re$. For convenience in distinguishing between the left and right copy of $R$ in $\Re = R \otimes_k \Rop$ we write $\Rop$ even though $R$ is commutative. Our first observation is that this graded module can be equipped as a linear factorisation of $V + \widetilde{W}$ over $S \otimes_k \Re$. One checks that there are well-defined operators
\begin{align*}
D(x_0, x_1, \ldots) &= (D(x_0), D(x_1), \ldots) \, , \\
b'(x_0, x_1, \ldots) &= (b'(x_1),b'(x_2),\ldots) \, , \\
d(x_0,x_1,\ldots) &= (0, d(x_0), d(x_1),\ldots) \, , \\
s(x_0,x_1,\ldots) &= (0, s(x_0), s(x_1), \ldots) \, , \\
d \widetilde{W} \times (x_0,x_1,\ldots) &= (0, d \widetilde{W} \times x_0, d \widetilde{W} \times x_1, \ldots)
\end{align*}
on $X \,\widehat{\otimes}\, \Bar$, where $d$ is extended to an $S \otimes_k \Rop$-linear operator on $X \otimes \Bar$ by the rule
\[
d( e_i \otimes \omega ) = (-1)^{|e_i|} e_i \otimes d(\omega)\,,
\]
and $s$ extends to an $S \otimes_k R$-linear operator by
\[
s( e_i \otimes \omega ) = (-1)^{|e_i|} e_i \otimes s(\omega)\,.
\]
Assume that $V + \widetilde{W}$ belongs to the image of either $S \otimes_k R$ or $S \otimes_k \Rop$ in $S \otimes_k \Re$, so that it makes sense to talk about linear factorisations of $V + \widetilde{W}$ over the appropriate one of these two rings.

\begin{lemma} $( X \, \widehat{\otimes}_R\, \Bar, D + b' + d\widetilde{W} \times (-) )$ is a linear factorisation of $V + \widetilde{W}$.
\end{lemma}
\begin{proof}
This can be checked on the inverse system, where it follows from Lemma \ref{lemma:barisafactorisation}.
\end{proof}

There is a morphism of linear factorisations
\[
\pi: X \, \widehat{\otimes}\, \Bar = \prod_{m \ge 0} X \otimes \Bar_m \lto X \otimes \Bar_0 \lto X \otimes R
\]
where the first map is the projection and the second is the product $\Re \lto R$. Next we show using the perturbation lemma that this map is a homotopy equivalence, and we give an explicit homotopy inverse in terms of Atiyah classes. In fact we give two homotopy inverses, one which is right $R$-linear and the other left $R$-linear. We begin with the deformation retract arising from the fact that $s$ and $d$ are contracting homotopies for the differential $b'$.

\begin{lemma}\label{lemma:firstdefo} There are $S$-linear deformation retract datums of $\mathbb{Z}_2$-graded complexes
\begin{equation}\label{eq:firstdefo1}
\xymatrix@C+2pc{
(X \otimes R, 0) \ar@<-1ex>[r]_-{\sigma_d} & (X \, \widehat{\otimes} \, \Bar, b') \ar@<-1ex>[l]_-{\pi}
}%
\!\!\!\xymatrix{%
{}\ar@(ur,dr)[]^{-d}
}%
\end{equation}
and
\begin{equation}\label{eq:firstdefo2}
\xymatrix@C+2pc{
(X \otimes R, 0) \ar@<-1ex>[r]_-{\sigma_s} & (X \, \widehat{\otimes} \, \Bar, b') \ar@<-1ex>[l]_-{\pi}
}%
\!\!\!\xymatrix{%
{}\ar@(ur,dr)[]^{-s}
}%
\end{equation}
where $\sigma_d(e_i \otimes a ) = e_i \otimes (1 \otimes a)$ and $\sigma_s(e_i \otimes a ) = e_i \otimes (a \otimes 1)$. In \eqref{eq:firstdefo1} every map is right $R$-linear and in \eqref{eq:firstdefo2} every map is left $R$-linear.
\end{lemma}
\begin{proof}
The required identities $b' d + d b' = 1 - \sigma \pi$ and $b' s + s b' = 1 - \sigma \pi$ are (\ref{b'd+db'}) and \eqref{??}.
\end{proof}

\begin{lemma}\label{lemma:smallpertde} The perturbation $\delta = D + d\widetilde{W} \times (-)$ is small on $X\,\widehat{\otimes} \, \Bar$ with respect to both of the above deformation retract datums. That is, both $1 + \delta d$ and $1 + \delta s$ are invertible.
\end{lemma}
\begin{proof}
Let $h$ be $-d$ or $-s$. Because we are working with $\prod_{m \ge 0} X \otimes \Bar_m$ it is clear that the sum $\sum_{m \ge 0} (\delta h)^m$ converges as an operator on $X \, \widehat{\otimes}\, \Bar$ and this gives the desired inverse to $1 - \delta h$.
\end{proof}

Next we give the right $R$-linear inverse to $\pi$ in terms of the Atiyah class $\At_X = [d,D]$ on $X \otimes \Bar$.

\begin{proposition}\label{prop:finalpertdefo} There is an $(S \otimes_k \Rop)$-linear deformation retract datum
\[
\xymatrix@C+2pc{
(X \otimes R, D \otimes 1) \ar@<-1ex>[r]_-{\sigma_\infty} & (X \, \widehat{\otimes} \, \Bar, D + b' + d\widetilde{W} \times (-)) \ar@<-1ex>[l]_-{\pi}
}
\]
where
\[
\sigma_\infty = \sum_{m \ge 0} (-1)^m \big[d, D \big]^m \sigma\,.
\]
In particular, $\pi$ is a homotopy equivalence with inverse $\sigma_\infty$.
\end{proposition}
\begin{proof}
It follows from the perturbation lemma (Propostiion \ref{prop:pertlemma}) with Lemmas~\ref{lemma:firstdefo} and~\ref{lemma:smallpertde} that
\[
\xymatrix@C+2pc{
(X \otimes A, b_\infty) \ar@<-1ex>[r]_-{\sigma_\infty} & (X \, \widehat{\otimes} \, \Bar, D + b' + d\widetilde{W} \times (-)) \ar@<-1ex>[l]_-{\pi_\infty}
}
\]
is a deformation retract datum, where with $\tau = \sum_{m \ge 0} (-1)^m (\delta d) \delta$ we have
\begin{align*}
\sigma_\infty = \sum_{m \ge 0} (-1)^m (d \delta)^m \sigma \, , \qquad
\pi_\infty = \pi + \pi \tau h\, , \qquad
b_\infty = \pi \tau \sigma\,.
\end{align*}
Clearly $\pi \delta = \pi D$ and $\pi$ vanishes on $\delta d$, so $\pi A = \pi D$. It follows that $b_\infty = D \otimes 1$ and $\pi_\infty = \pi - \pi D d = \pi$. So there is a deformation retract datum of the stated form, where we may use $d^2 = 0$ and $d \sigma = 0$ to write
\[
\sigma_\infty = \sum_{m \ge 0} (-1)^m \big[ d, \delta \big]^m \sigma = \sum_{m \ge 0} (-1)^m [d, D + d\widetilde{W} \times (-)]^m \sigma\,.
\]
Expanding this yields $\sum_{m \ge 0} [d, D]^m \sigma$ plus terms involving factors of the form
\begin{equation}\label{eq:finalpertdefo2}
[d, d\widetilde{W} \times (-)] [d,D]^i \sigma
\end{equation}
for some $i \ge 0$. Applying $[d,D]^i \sigma$ to an element of $X \otimes_R A$ leaves a tensor whose form component is of the type $da_0 \ldots da_n \otimes a_{n+1}$. By Lemma \ref{lemma:shortobs} below applying $[d, d\widetilde{W} \times (-)]$ to such a tensor yields zero, so all terms (\ref{eq:finalpertdefo2}) must vanish and $\sigma_\infty$ has the form given in the statement of the proposition.
\end{proof}

\begin{lemma}\label{lemma:shortobs} On $\Bar$ we have $[d, d\widetilde{W} \times (-)] = d \widetilde{W} \cdot d(-)$.
\end{lemma}
\begin{proof}
\textbf{todo}
\end{proof}

And the left $R$-linear inverse to $\pi$ in terms of the Atiyah class $\underline{\At}_X = [s,D]$ on $X \otimes \Bar$.

\begin{proposition}\label{prop:finalpertdefo} There is an $(S \otimes_k R)$-linear deformation retract datum
\[
\xymatrix@C+2pc{
(X \otimes R, D \otimes 1) \ar@<-1ex>[r]_-{\sigma_\infty} & (X \, \widehat{\otimes} \, \Bar, D + b' + d\widetilde{W} \times (-)) \ar@<-1ex>[l]_-{\pi}
}
\]
where
\[
\sigma_\infty = \sum_{m \ge 0} (-1)^m \big[s, D \big]^m \sigma\,.
\]
In particular, $\pi$ is a homotopy equivalence with inverse $\sigma_\infty$.
\end{proposition}
\begin{proof}
\textbf{todo}.
\end{proof}

Now we turn to the Koszul stabilisation of the diagonal. Assume that $R = k[x_1,\ldots,x_n]$ with $k$ noetherian and let $(\Delta, d_{\Delta})$ be the finite rank matrix factorisation of $\widetilde{W}$ given in Section~\ref{subsec:bicatLG} where $d_{\Delta} = \delta_{+} + \delta_{-}$. There is a morphism of linear factorisations
\begin{equation}\label{eq:koszulpi}
\pi: X \otimes \Delta \lto X \otimes \Re \lto X \otimes R
\end{equation}
where the first map is the projection of $\Delta$ onto $\mathbb{Z}$-degree zero, and the second map is the product. This is compatible with the map $\pi$ defined for the bar resolution of the diagonal in the sense that there is a commutative diagram
\[
\xymatrix{
X \,\widehat{\otimes}\, \Bar \ar[dr]_-{\pi}\ar[rr]^{1 \otimes \Psi} & & X \otimes \Delta \ar[dl]^-{\pi}\\
& X \otimes R
}
\]
in which every side is a homotopy equivalence: 

\begin{lemma}\label{lemma:koszulstab1} The map $\pi$ of (\ref{eq:koszulpi}) is both an $(S \otimes_k \Rop)$-linear homotopy equivalence and an $(S \otimes_k R)$-linear homotopy equivalence.
\end{lemma}
\begin{proof}
There is an exact sequence of free right (left) $R$-modules
\[
0 \lto \Delta_n \lto \cdots \lto \Delta_0 \lto R \lto 0
\]
which splits, and provides the necessary right (left) $R$-linear $\sigma: R \lto \Delta$ and homotopy $h$ on $\Delta$ to make
\[
\xymatrix@C+2pc{
(X \otimes R, 0) \ar@<-1ex>[r]_-{\sigma} & (X \otimes \Delta, 1 \otimes \delta_{-}) \ar@<-1ex>[l]_-{\pi}
}%
\!\!\!\xymatrix{%
{}\ar@(ur,dr)[]^{-h}
}%
\]
into an $(S \otimes_k \Rop)$-linear deformation retract datum (resp. $S \otimes_k R$-linear). Moreover $\delta = D \otimes 1 + 1 \otimes \delta_{+}$ is a small perturbation and the perturbation lemma applies to show that $\pi$ is a homotopy equivalence.
\end{proof}

\begin{corollary}\label{corollary:koszulstab2} An $(S \otimes_k \Rop)$-linear homotopy inverse to $\pi$ of (\ref{eq:koszulpi}) is given by
\[
\pi^{-1}_d = \sum_{m \ge 0} (-1)^m \Psi \big[d, D\big]^m \sigma_d\,,
\]
while an $(S \otimes_k R)$-linear homotopy inverse is given by
\[
\pi^{-1}_s = \sum_{m \ge 0} (-1)^m \Psi \big[s, D\big]^m \sigma_s\,.
\]
\end{corollary}

Since all tensor products are over $\Re$ everything that we have said above can be restated with $\Bar$ on the left, for example with $\Bar \,\widehat{\otimes} \, X = \prod_{m \ge 0} \Bar_m \,\widehat{\otimes}\, X$.

For future reference, let us conclude by stating the special case of inverses for the unit actions in \eqref{eq:pertandhtpy1}. So we take $S = k[z]$ and $R = k[x]$, $X$ a finite rank matrix factorisation of $V - W$ over $k[x,z]$. By applying the above to $X \otimes_R \Re$ or $\Re \otimes_R X$ we deduce that

\begin{proposition} A $k[z]$-$k[x]$-bilinear homotopy inverse for $\rho$ is given by
\[
\rho^{-1} = \sum_{m \ge 0} (-1)^m \Psi [d, D]^m \sigma\,.
\]
and a $k[z]$-$k[x]$-bilinear homotopy inverse for $\lambda$ is
\[
\lambda^{-1} = \sum_{m \ge 0} (-1)^m \Psi \big[s, D \big]^m \sigma\,.
\]
\end{proposition}

\textbf{todo} more on $s$ here.

\section{Derivation of the evaluation and coevaluation}\label{sec:derivcoeval}
% TODO independence of choice of lambdas and ordering

In this section we derive formulas for the evaluation and coevaluation $2$-morphisms in the bicategory $\LG$ of Landau-Ginzburg models over a noetherian $\mathbb{Q}$-algebra $k$. We fix two potentials $W\in k[x] = k[x_1,\ldots,x_n]$ and $V\in k[z] = k[z_1,\ldots,z_m]$ and a matrix factorisation $X\in \hmf(k[x,z], V-W)$. We also fix a homogeneous basis $\{ e_i \}_{i \in I}$ for $X$ with dual basis $\{ e_i^* \}_{i \in I}$. 

We set $X^\dual = X^\vee[n]$ where $(-)^\vee$ denotes taking the dual module. As recalled in Section~\ref{subsec:bicatLG} there are two kinds of evaluation and coevaluation $2$-morphisms associated with~$X$. The derivation is very similar in both cases, so we give the full details for 
\[
\widetilde\coev_X: \Delta_W \lto X^\dual \otimes_{k[z]} X
\, , \qquad
\widetilde\eval_X: X \otimes_{k[x]} X^\dual \lto \Delta_V
\]
while simply stating the formulas for $\coev_X$ and $\eval_X$, with the details left to the reader. Recall that $\Delta_W$ and $X^\dual \otimes X$ are $k[x]$-bimodules and the coevaluation is $k[x]$-bilinear, while $\Delta_V$ and $X \otimes X^\dual$ are $k[z]$-bimodules and the evaluation is $k[z]$-bilinear. The differentials on the matrix factorisations $\Delta_V, \Delta_W$ is given by \eqref{DeltaW}.

At this point the evaluation and coevaluation morphisms we construct would be more accurately described as a \textsl{candidates}, as it is only later in Section~\ref{sec:Zorro} that we show that our candidates are actually the counit and unit of an adjunction by proving the Zorro moves.

\subsection{Coevaluation}\label{subsec:derivcoeval}

In this section $\Delta$ denotes $\Delta_W$.

\begin{proposition}\label{prop:isogivescoev} There is a canonical homotopy equivalence of $\mathbb{Z}_2$-graded $k$-complexes
\begin{equation}
\Hom_{k[x]\textup{-Bimod}}( \Delta, X^\dual \otimes_{k[z]} X ) \lto \Hom_{k[x,z]}(X,X)\,. \label{eq:isogivescoev}
\end{equation}
\end{proposition}
\begin{proof}
Let $\Delta'$ denote the matrix factorisation of $- \widetilde{W}$ with the same underlying graded free module as $\Delta$ but with the modified differential $d_{\Delta'} = - \delta_{+} + \delta_{-}$. This approximates the diagonal as a matrix factorisation of $- \widetilde{W}$ in just the same way that $\Delta$ approximates it as a factorisation of $\widetilde{W}$. Also let~$K$ denote the $\mathbb{Z}_2$-graded complex with the same underlying graded module as $\Delta$, but the differential~$\delta_{-}$, so~$K$ is the usual Koszul complex of $x_1 - y_1, \ldots, x_n - y_n$.

The product in the exterior algebra gives rise to a morphism of $\mathbb{Z}_2$-graded complexes $(R = k[x])$
\[
\Delta' \otimes_{\Re} \Delta \lto K\,.
\]
Composing with the natural morphism $\varepsilon: K \lto \Re[n]$ and taking the adjoint, we obtain the map
\[
\zeta: \Delta' \lto \Delta^\vee [n]\, , \qquad \omega \lmt \varepsilon( \omega \wedge - )
\]
which is an isomorphism of matrix factorisations. Combining this with various canonical maps and the morphism $\pi: \Delta' \lto \Delta'_0 = \Re \lto R$ we have the following diagram of \textsl{right} $R$-linear morphisms of complexes: 
\begin{equation}
\xymatrix{
\Hom_{\Re}( \Delta, X^\vee[n] \otimes_{k[z]} X)\\
\Delta^\vee \otimes_{\Re} \left( X^\vee[n] \otimes_{k[z]} X \right) \ar[u]^-{\cong}_-{\xi}\\
\left( X^\vee[n] \otimes_{k[z]} X \right) \otimes_{\Re} \Delta^\vee \ar[u]^-{\cong}_-{\textup{swap}}\\
\left( X^\vee \otimes_{k[z]} X \right) \otimes_{\Re} \Delta^\vee[n] \ar[u]^-{\cong}\\
\left( X^\vee \otimes_{k[z]} X \right) \otimes_{\Re} \Delta' \ar[u]^-{\cong}_-{\zeta} \ar[d]^-{\pi}\\
\left( X^\vee \otimes_{k[z]} X \right) \otimes_{\Re} R \ar[d]_-{\cong}\ar[d]^-{\kappa}\\
\Hom_{k[x,z]}(X,X)
}\label{eq:isogivescoev2}
\end{equation}
where for $\eta \in \Delta^\vee, f \in X^\vee \otimes_{k[z]} X$ and $\tau \in \Delta$ we have
\[
\xi( \eta \otimes x )( f ) = (-1)^{|\eta||x|} \eta(f) \cdot x\,,
\]
and for $f \in X^\vee$ and $g,g' \in X$: 
\[
\kappa( f \otimes g )(g') = (-1)^{|f||g|} f(g') \cdot g\,.
\]
So to complete the proof we need only to show that the step marked $\pi$ is a right $R$-linear homotopy equivalence. But this is a consequence of Lemma \ref{lemma:koszulstab1} with $A = k[x], R = \Ae$ and $\theta$ the identity.
\end{proof}

\begin{definition} The \textsl{coevaluation} is the morphism $\widetilde\coev_X: \Delta \lto X^\dual \otimes X$ in the homotopy category $\hmf(\Re, \widetilde{W})$ mapping to $1_X$ under (\ref{eq:isogivescoev})\,.
\end{definition}

To give a representative chain map for the coevaluation we begin with $1_X$ at the bottom of (\ref{eq:isogivescoev2}) and apply the various maps in turn; the only nontrivial step will be to apply the homotopy inverse for $\pi$ found in Section \ref{section:pertandhtpy}. The two different choices of inverse provided there will lead to two different (but homotopic) chain level representatives for the coevaluation.

On the tensor product
\[
\Bar \otimes_{\Re} ( X^\dual \otimes_{k[z]} X)\,
\]
we consider the operator $\At_{X^\dual \otimes X} = [d, d_{X^\dual \otimes X}]$. Here $d$ is extended to the tensor product by
\[
d( \omega \otimes e_i^* \otimes e_j ) = d(\omega) \otimes e_i^* \otimes e_j\,.
\]
For the purposes of forming tensor products we view $\Bar$ as a right $\Re$-module in the usual way for a module over a commutative ring, that is, $\Re$ acts on the right via \textsl{left} form multiplication (there is possibility for confusion since \textsl{right} form multiplication equips $\Bar$ with a different right $\Re$-module structure). With this convention, we may interchangeably write $\Bar$ to the left or right of $X^\dual \otimes_{k[z]} X$ (as it appears in Section \ref{section:pertandhtpy}) provided we keep track of the relevant signs.

Since $d$ is right $R$-linear, $[d, d_{X^\dual \otimes X}] = [d, d_{X^\dual}\otimes 1_X]$ and so we abuse notation and write $\At_{X^\dual}$ for $\At_{X^\dual \otimes X}$. If we apply $\Psi: \Bar \lto K$ after a power of the Atiyah class, we end up with maps
\[
\xymatrix@C+1pc{
X^\dual \otimes_{k[z]} X \ar[r]^-{\At^i_{X^\dual}} & \Bar \otimes_{\Re} ( X^\dual \otimes_{k[z]} X ) \ar[r]^-{\Psi} & K \otimes_{\Re} (X^\dual \otimes_{k[z]} X)
}
\]
which we can apply to the tensor $\iota_X = \sum_{j} (-1)^{|e_j|} e_j^* \otimes e_j$.

\begin{theorem}\label{theorem:coev1} A representative for the coevaluation morphism is the chain map
\begin{equation}\label{eq:coev10}
\widetilde\coev^L_X(\gamma) = \varepsilon \left( \gamma \wedge \sum_{i \ge 0} (-1)^i \Psi \At^i_{X^\dual}( \iota_X ) \right)\
\end{equation}
where the Atiyah class sends forms to the left, as explained above.
\end{theorem}
\begin{proof}
Let us chase $1_X$ upwards through (\ref{eq:isogivescoev2}), noting that $\iota_X = \kappa^{-1}(1_X)$. With $A = k[x], R = \Ae$ and $\theta$ the identity we may apply Corollary \ref{corollary:koszulstab2} to describe a homotopy inverse to $\pi$ in terms of the Atiyah class $[d, d_{X^\dual \otimes X}]$. In light of the observations preceeding the statement of the theorem we have
\begin{equation}\label{eq:coev11}
\pi^{-1} \kappa^{-1}(1_X) = \sum_{m \ge 0} (-1)^m \Psi \big[d, d_{X^\dual}\big]^m( \iota_X )\,.
\end{equation}
A summand $f \otimes \theta_{\bs{i}}$ in (\ref{eq:coev11}), with $f \in X^\dual \otimes X$ and $\theta_{\bs{i}} = \theta_{i_1} \ldots \theta_{i_p}$ for $i_1 < \ldots < i_p$, maps under $\zeta$ to
\[
f \otimes \zeta( \theta_{i_1} \ldots \theta_{i_p} ) = f \otimes \varepsilon\big( \theta_{\bs{i}} \wedge - \big)
\]
and then to the function $\Delta \lto X^\dagger \otimes X$ defined by
\[
\gamma \lmt (-1)^{|\omega||f|} \varepsilon\big( \gamma \wedge \omega \otimes f \big)\,,
\]
and so (\ref{eq:coev10}) follows from (\ref{eq:coev11}).
\end{proof}

There is an alternative presentation which we will also need. On the tensor product
\[
( X^\dual \otimes X) \otimes_{k[x,y]} \Bar
\]
we consider the operator $\At_{X^\dual \otimes X, k[x,y]/k[x]} = [s, d_{X^\dual \otimes X}]$ and its powers. Since $s$ is $k[x]$-linear, this is actually $[s, d_{X}]$ and for this reason we abuse notation and write $\At_{X, k[x,y]/k[x]}$ for $\At_{X^\dual \otimes X, k[x,y]/k[x]}$. If we apply $\Psi: \Bar \lto K$ after a power of the Atiyah class, we end up with maps
\[
\xymatrix{
X^\dual \otimes X \ar[r]^-{\At^i_{X, k[x,y]/k[x]}} & ( X^\dual \otimes X ) \otimes_{k[x,y]} \Bar \ar[r]^-{\Psi} & (X^\dual \otimes X)\otimes_{k[x,y]} K\,.
}
\]

\begin{theorem} A representative for the coevaluation morphism is the chain map
\[
\widetilde\coev^R_X(\gamma) = \varepsilon \left( \sum_{i \ge 0} (-1)^i \Psi \At^i_{X,k[x,y]/k[x]}( \iota_X ) \wedge \gamma \right)\,.
\]
\end{theorem}

\begin{example} Give an example for say $n = 3$ to explain how the coevaluation picks out the ``curvature'' in the orthogonal to infinitesimal element $\gamma$.
\end{example}

\newpage

\subsection{Evaluation}\label{subsec:eval}

In this section we write $B = k[z]$ and $\Be = B \otimes_k B$ so that $\widetilde\eval_X$ is a morphism of (infinite rank) matrix factorisations of $\widetilde{V} = V \otimes 1 - 1 \otimes V$ over $\Be$. The polynomials $\partial_{x_i} W$ act null-homotopically on $X$ and we let $\lambda_i \in \Hom_{k[x,z]}(X,X)$ denote a degree-one map with $[d_X, \lambda_i] = f_i \cdot 1_X$. For example $\lambda_i = \partial_{x_i} d_X$ would do, but it will be important to allow other choices.

The evaluation map will be produced by a perturbation whose initial data is the morphism $\eval_0$ of the next proposition.

\begin{proposition} There is a morphism of linear factorisations of $\widetilde{V}$ over $\Be$, 
\[
\eval_0: X \otimes_{k[x]} X^\dual \lto B\,,
\]
defined by
\[
\eval_0( f \otimes \nu ) = (-1)^{\binom{n}{2} + n|f|} \Res_{k[x,z]/k[z]} \left[ \frac{ \str( \lambda_1 \cdots \lambda_n \circ f \circ \nu) \underline{\operatorname{d}\!x}}{\partial_{x_1}W \ldots \partial_{x_n} W} \right]
\]
\end{proposition}
\begin{proof}
To describe this map we set
\[
\Hom_{k[x,z]}(\bar{X}, \bar{X}) = \Hom_{k[x,z]}(X, X) \otimes_{k[x]} k[x]/(\partial_{x_1}W, \ldots, \partial_{x_n}W)\,.
\]
Then for each $1 \le i \le n$ we have a closed, odd $k[x,z]$-linear operator $\lambda_i^\bullet$ on $\Hom_{k[x,z]}(\bar{X}, \bar{X})$
\[
\lambda_i^\bullet(\alpha) = \lambda_i \circ \alpha\,.
\]
There is a chain of $\Be$-linear morphisms of linear factorisations of $\widetilde{V}$
\begin{equation}
\xymatrix{
X \otimes_{k[x]} (X^\vee [n]) \ar[d]_-{\cong}\\
(X \otimes_{k[x]} X^\vee)[n] \ar[d]^-{\textup{quotient}} \ar[d]_-{\cong}\\
\big( B \otimes_{B^e} (X \otimes_{k[x]} X^\vee)\big)[n] \ar[d]^-{\kappa} \ar[d]_-{\cong}\\
\Hom_{k[x,z]}(X,X)[n] \ar[d]^-{\lambda_1^\bullet \cdots \lambda_n^\bullet}\\
\Hom_{k[x,z]}(\bar{X},\bar{X})[n]
}
\end{equation}
which composed with the supertrace and residue gives the desired morphism. Here $\kappa$ is defined by
\[
\kappa(1 \otimes f \otimes \nu)(s) = \nu(s) \cdot t\,.
\]
\end{proof}

\newpage

\section{Zorro moves}\label{sec:Zorro}

In this section we will show that the bicategory $\LG$ of Landau-Ginzburg models has adjoints for any noetherian $\mathbb{Q}$-algebra $k$. Let us fix two arbitrary potentials $W\in k[x] = k[x_1,\ldots,x_n]$ and $V\in k[z] = k[z_1,\ldots,z_m]$. Then we want to prove that for any matrix factorisation $X\in \hmf(k[x,z], V-W)$ and its adjoint $X^\dual = X^\vee[n]\in \hmf(k[x,z], W-V)$ the Zorro moves~\eqref{Zorros} and~\eqref{otherZorros} are satisfied. Let $\{ e_i \}_{i \in I}$ denote a homogeneous $k[x,z]$-basis of $X$ and $\{ e_i^* \}_{i \in I}$ the dual basis of $X^\vee$.

Let us consider the first identity of~\eqref{otherZorros} in more detail: 
\be\label{Zorro1detail}
\begin{tikzpicture}[very thick,scale=1.0,color=blue!50!black, baseline=0cm]

\fill (0.2,1.6) circle (0pt) node {{\small $\Delta_V$}};
\fill (-0.2,-1.6) circle (0pt) node {{\small $\Delta_W$}};

\fill (1,1.8) circle (2.5pt) node[right] {{\small $\lambda$}};
\fill (-1,-1.8) circle (2.5pt) node[left] {{\small $\rho^{-1}$}};

\fill (-1.25,-2.25) circle (0pt) node {{\footnotesize $z$}};
\fill (-0.75,-2.25) circle (0pt) node {{\footnotesize $x$}};

\fill (-1.5,0) circle (0pt) node {{\footnotesize $z\vphantom{z}$}};
\fill (-0.5,0) circle (0pt) node {{\footnotesize $x$}};
\fill (0.5,0) circle (0pt) node {{\footnotesize $z$}};
\fill (1.5,0) circle (0pt) node {{\footnotesize $x\vphantom{z}$}};

\fill (1.25,2.25) circle (0pt) node {{\footnotesize $x$}};
\fill (0.75,2.25) circle (0pt) node {{\footnotesize $z\vphantom{x}$}};

\draw[dashed] (-0.5,0.75) .. controls +(0,0.75) and +(-0.25,-0.75) .. (1,1.8);
\draw[dashed] (0.5,-0.75) .. controls +(0,-0.75) and +(0.25,0.75) .. (-1,-1.8);

\draw[line width=0] 
(1,2.7) node[line width=0pt] (A) {{\small $X$}}
(-1,-2.7) node[line width=0pt] (A2) {{\small $X$}}; 
\draw[redirected] (0,0) .. controls +(0,1) and +(0,1) .. (-1,0);
\draw[redirected] (1,0) .. controls +(0,-1) and +(0,-1) .. (0,0);
\draw (-1,0) -- (A2); 
\draw (1,0) -- (A); 
\end{tikzpicture}
=
\begin{tikzpicture}[very thick,scale=1.0,color=blue!50!black, baseline=0cm]
\draw[line width=0] 
(0,2.7) node[line width=0pt] (A) {{\small $X$}}
(0,-2.7) node[line width=0pt] (A2) {{\small $X$}}; 
\draw (A2) -- (A); 
\end{tikzpicture}
\ee
Here we also label the two domains to the left and right of the 1-morphism~$X$ by the variables pertaining to the objects $(k[x],W), (k[z],V) \in \LG$. We call the left-hand side of (\ref{Zorro1detail}) the \textsl{Zorro map} and denote it $\mathcal Z$. It is the composite
\be\label{eq:zorro1a1}
\xymatrix@C+2pc{
X \ar[r]^-{\rho^{-1}} & X \otimes \Delta_W \ar[r]^-{1 \otimes\,\widetilde{\coev}} & X \otimes X^{\dagger} \otimes X \ar[r]^-{\widetilde{\eval}\, \otimes 1} & \Delta_V \otimes X \ar[r]^-{\lambda} & X\,,
}
\ee
which we prove is homotopic to the identity on $X$. Note that we continue to discard subscripts that are clear from the context, e.\,g.~$\widetilde\eval = \widetilde\eval_X$. 

We keep in mind that there are two distinct $\Delta$'s which appear: if we write $R = k[x]$ and $S = k[z]$ then one of the $\Delta$'s is an $R$-$R$-bimodule, and the other is an $S$-$S$-bimodule. It is sometimes convenient to identify $\Re$ with $k[x,y]$ so that the right $R$-action is via polynomials in the $y$-variables.

Let us introduce the notation
$$
\langle\!\langle-\rangle\!\rangle = (-1)^{n+1\choose 2} \Res_{k[x,y,z]/k[y,z]} \left[ \frac{ \varepsilon\Psi(-) \underline{\operatorname{d}\!x}}{\partial_{x_1}W \ldots \partial_{x_n} W} \right]
$$
for the map $S \otimes_k \Bar \lto S \otimes_k R = k[y,z]$ defined using the map $\varepsilon\Psi: \Bar \lto S \otimes_k R$ of Section \ref{??} and $\underline{\operatorname{d}\!x}=\operatorname{d}\!x_1\ldots \operatorname{d}\!x_n$. Here $\Bar$ is the bar complex of $R$ over $k$, and the two copies of $R$ play very different roles in $\langle\!\langle - \rangle\!\rangle$. The left copy (i.\,e.~$k[x]$) is integrated out, and the map is linear with respect to the right action, so that overall $\langle\!\langle - \rangle\!\rangle$ is $k[y,z]$-linear.

Let $\At$ denote the Atiyah class of $\End(X)$, which is an operator on $\End(X) \otimes_{k[x]} \Bar$. Let $\lambda_i$ be a null-homotopy on $X$ for the action of $\partial_{x_i} W$, for example $\lambda_i=\partial_{x_i}d_X(x,z)$, and set $\Lambda=\lambda_1\ldots \lambda_n$. Our first order of business is to establish the following expression for the Zorro map.

\begin{lemma}\label{lemma:Zorrointermediate} We have
\be\label{Zorrointermediate}
\mathcal Z = \sum_j (-1)^{|e_j| + n} \big\langle\!\big\langle \str \big( \Lambda \circ \At^n (-\circ e_j^*) \big) \big\rangle\!\big\rangle \cdot e_j
\ee
where for $f \in X$, the endomorphism $f \circ e_j^*$ sends $g$ to $e_j^*(g) \cdot f$.
\end{lemma}

Before giving the proof of the lemma, let us sketch the argument that this morphism is homotopic to the identity. In outline, the Atiyah class is ``anti-self-adjoint'' with respect to $\langle\!\langle \str( - ) \rangle\!\rangle$: 
\begin{equation}\label{eq:trueproofeq}
\big\langle\!\big\langle \str \big( \Lambda \circ \At^n (-)\big) \big\rangle\!\big\rangle \simeq (-1)^n \big\langle\!\big\langle \str \big( \At^n( \Lambda ) \circ -\big) \big\rangle\!\big\rangle\,.
\end{equation}
We will see that $\At^n( \Lambda )$ is a transition determinant in the sense of the calculus of residues, so that the right-hand side of~(\ref{eq:trueproofeq}) can be identified with $\str(-)$. Hence
\begin{equation}\label{eq:trueproofeq2}
\mathcal{Z} \simeq \sum_j (-1)^{|e_j|} \str(-\circ e_j^*) \cdot e_j = \sum_j \str(e_j^* \circ -) \cdot e_j = 1_X\,.
\end{equation}
The reader will appreciate the parallel with the argument for duality in the monoidal category of vector spaces, with the new feature here being the Atiyah class and its anti-self-adjointness.

There are two subtleties which complicate the proof. The first is that we must deal throughout with associative Atiyah classes and thus non-commutative differential forms (compare with the proof of non-degeneracy in Section~\ref{section:dualityadjointop} where only ordinary Atiyah classes are involved). The second is that the identites above only hold, in general, up to homotopy ``$\simeq$''.

For the proof of Lemma \ref{lemma:Zorrointermediate} let us recall the explicit formulas for the maps involved: 
\begin{align*}
\rho^{-1}: X \lto X \otimes \Delta \, , \qquad \rho^{-1}(a) &= \sum_{i \ge 0}(-1)^i \Psi\! \At^i_X(a)\,,\\
\widetilde{\coev}: \Delta \lto X^{\dagger} \otimes X \, , \qquad \widetilde{\coev}(\gamma) &= \varepsilon\Big( \gamma \wedge \sum_{i \ge 0}(-1)^i (\Psi\!\At^i_{X^{\dagger}} \otimes 1)(\iota_X) \Big)\,,\\
&= \sum_{i,j} (-1)^{i+|e_j|} \varepsilon \Big( \gamma \wedge \Psi\!\At^i_{X^{\dagger}}( e_j^* ) \otimes e_j \Big)\\
\widetilde{\eval}: X \otimes X^{\dagger} \lto \Delta \, , \qquad \widetilde{\eval}(\alpha) &= (-1)^{\binom{n+1}{2}}\Res_{k[x,y,z]/k[y,z]}\! \left[ \frac{\str ( \alpha
%|_{z=z'}
 \Lambda) \underline{\operatorname{d}\! x}}{\partial_{x_1}W \ldots \partial_{x_n} W} \right] + \mathcal{O}(\theta)\,.
\end{align*}
%Note that tensoring with the diagonal defines a map $(-)|_{z=z'}: X \otimes X^{\dagger} \lto \Hom_{k[x,z]}(X,X)$ (\textbf{more} there is no longer a $z'$ -- so there's no need to mention it!) 
Here $\mathcal{O}(\theta)$ denotes higher order terms which do not survive the map $\lambda$ in~\eqref{Zorro1detail} and therefore may be ignored for present purposes. Also define $\nabla$ and Atiyah classes (\textbf{todo}).

\begin{proof}[Proof of Lemma \ref{lemma:Zorrointermediate}]
Consider the image of a basis element $e_q$ under the first two maps of (\ref{eq:zorro1a1}): 
\begin{align}
(1 \otimes \widetilde{\coev}) \circ \rho^{-1}( e_q ) &= \sum_{i,k,j} (-1)^{i+k} \varepsilon\Big( \Psi\! \At^i_X( e_q ) \wedge \Psi \!\At^k_{X^{\dagger}}( e_j^* ) \otimes e_j \Big) \nonumber\\
&= \sum_{j} \sum_{i+k = n} (-1)^n \varepsilon\Psi\Big( \At^i_X( e_q ) \times \At^k_{X^{\dagger}}( e_j^* ) \otimes e_j \Big) \nonumber\\
&= \sum_{j} (-1)^n \varepsilon\Psi\Big( \At^n_{X \otimes X^{\dagger}}( e_q \otimes e_j^* ) \otimes e_j \Big)\,. \label{firstzorromaps}
\end{align} 
In the first step we use that $\Psi$ intertwines the shuffle product on $\Bar$ with the exterior product on~$\Delta$, and in the last step we use Lemma \ref{lemma:shuffleatiyah} below, whose proof is delayed to the end of the section. Applying $\widetilde\eval$ in the first two components of $X \otimes X^{\dagger} \otimes X$ to the expression in (\ref{firstzorromaps}) has the effect of 
%identifying $z$ with $z'$ (TODO: not necessary). This changes 
changing the associative Atiyah class of $X \otimes X^{\dagger}$ to the associative Atiyah class of $\End(X)$ and so
\begin{align}
\mathcal{Z}(e_q) &= \lambda \circ (\eval \otimes 1) \circ (1 \otimes \coev) \circ \rho^{-1}( e_q )\\
&= \sum_j (-1)^{|e_j|+n+\binom{n+1}{2}}\Res_{k[x]/k}\! \left[ \frac{ \varepsilon\Psi \str\Big( \At^n( e_q \circ e_j^* ) \Lambda \Big) \underline{\operatorname{d}\! x} }{\partial_{x_1}W \ldots \partial_{x_n} W} \right] \cdot e_j
\end{align}
Since the Zorro map~$\mathcal Z$ is $k[x,z]$-linear it is fixed by its action on basis elements~$e_q$, which proves that $\mathcal{Z}$ is given by (\ref{Zorrointermediate}).
\end{proof}

\begin{lemma}\label{lemma:shuffleatiyah} 
TODO: $\At \times \At = \At$.
\end{lemma}

% TODO: use the below for the proof... 
%\begin{lemma}
%$\Gamma(e_q) = \sum_{n\geq 0} \sum_j (-1)^{|e_j| + n} (\At_{X\otimes X^\dual})^n(e_q \otimes e_j^* \otimes e_j)$. 
%\end{lemma}
%
%\begin{proof}
%We compute $\Gamma(e_q)$, using $\At_X(e_q) = (-1)^{|e_q|+1} e_k \otimes d(d_X)_{kq}$ and paying attention to Koszul signs: 
%\begin{align}
%e_q & \stackrel{\rho_X^{-1}}{\longmapsto} \sum_{a\geq 0} (-1)^a \At_X^a(e_q) \nonumber \\
%& \qquad\; = \sum_{a\geq 0} (-1)^{a + (|e_q|+1)+\ldots+(|e_q|+a)} e_{k_a} \otimes d(d_X)_{k_a k_{a-1}} \ldots d(d_X)_{k_1 q} \nonumber \\
%& \stackrel{\gamma_1}{\longmapsto} \Big( \sum_{a\geq 0} (-1)^{a + a|e_q| + {a+1\choose 2}} e_{k_a} \otimes d(d_X)_{k_a k_{a-1}} \ldots d(d_X)_{k_1 q} \Big) \nonumber \\
%& \qquad\; \otimes \Big( \sum_{b\geq 0} \sum_j (-1)^{|e_j|+b} (-1)^{(|e_j|+1)+\ldots+(|e_j|+b) + b|e_j|} e^*_{l_b} \otimes e_j \otimes d(d_{X^\dual})_{l_b l_{b-1}} \ldots d(d_{X^\dual})_{l_1 j} \Big) \nonumber \\
%& \stackrel{\gamma_3 \circ \gamma_2}{\longmapsto} \sum_{n\geq 0} \sum_j (-1)^{|e_j|+n} \sum_{a=0}^n(-1)^{a|e_q| + {a+1\choose 2} + {n-a+1\choose 2}} \sum_{\sigma \in \Sh(n-a,a)} (-1)^{|\sigma|} e_{k_a} \otimes e^*_{l_{n-a}} \otimes e_j \nonumber \\
%& \qquad\;   \otimes \sigma_\bullet \left( d(d_{X^\dual})_{l_{n-a} l_{n-a-1}} \ldots d(d_{X^\dual})_{l_1 j} d(d_X)_{k_a k_{a-1}} \ldots d(d_X)_{k_1 q} \right) \nonumber \\
%& \qquad\; =  \sum_{n\geq 0} \sum_j (-1)^{|e_j|+n} \sum_{a=0}^n (-1)^{a|e_q| + {a+1\choose 2} + {n-a+1\choose 2} + a(n-a)} \sum_{\sigma \in \Sh(a,n-a)} (-1)^{|\sigma|}  \label{31} \\
%& \qquad\qquad \cdot e_{k_a} \otimes e^*_{l_{n-a}} \otimes e_j \otimes \sigma_\bullet \left( d(d_X)_{k_a k_{a-1}} \ldots d(d_X)_{k_1 q} d(d_{X^\dual})_{l_{n-a} l_{n-a-1}} \ldots d(d_{X^\dual})_{l_1 j} \right) \nonumber \\
%& \qquad\; =  \sum_{n\geq 0} \sum_j (-1)^{|e_j|+n} ( \At_X + \At_{X^\dual} )^n (e_q \otimes e^*_j \otimes e_j) \label{32} \\
%& \qquad\; =  \sum_{n\geq 0} \sum_j (-1)^{|e_j|+n} (\At_{X\otimes X^\dual})^n (e_q \otimes e^*_j \otimes e_j) \, . \nonumber
%\end{align}
%To understand the penultimate step we note that the sign $(-1)^{a|e_q| + {a+1\choose 2} + {n-a+1\choose 2} + a(n-a)}$ with $\sigma=\operatorname{id}$ in~\eqref{31} is precisely that of the contribution to~\eqref{32} where $\At_{X^\dual}$ first acts $n-a$ times on $e_q\otimes e^*_j\otimes e_j$, followed by $\At_X^a$. The sign $(-1)^{|\sigma|}$ appears in~\eqref{32} if some $\At_X$ acts before some of the $\At_{X^\dual}$. 
%\end{proof}


The form of self-adjointness of the Atiyah class that we will use is contained in the next result. The $p = 0$ case is the statement that for $\varphi \in \End(X) \otimes_{k[x]} \Bar$ we have
\[
\big\langle\!\big\langle \str( \lambda_1 \ldots \lambda_n \circ \At(\varphi) ) \big\rangle\!\big\rangle = \sum_{i=1}^n (-1)^{i-1} \Big\langle\!\Big\langle \nabla \left\{ f_i \str( \lambda_1 \ldots \widehat{\lambda_i} \ldots \lambda_n \circ \varphi ) \right\} \Big\rangle\!\Big\rangle + h_0( D(\varphi))
\]
for some function $h_0$, where as usual $\widehat{\lambda_i}$ means that $\lambda_i$ is omitted. Applying the lemma inductively will interpolate between the left- and right-hand sides of (\ref{eq:trueproofeq}). For the general statement we will need some notation: given a sequence $\bs{i} = (i_1,\ldots,i_p)$ in $\{1, \ldots, n\}$ (not necessarily in ascending order) let
\[
\ell(\bs{i}) = p, \qquad w(\bs{i}) = i_1 + \ldots + i_p, \qquad |\bs{i}| = \sum_{1 \le a < b \le n} \delta_{i_a > i_b}\, \qquad \gamma(\bs{i}) = w(\bs{i}) + |\bs{i}| + \binom{p+1}{2}\,.
\]
The empty sequence $\emptyset$ is the unique sequence of length $\ell(\emptyset) = 0$. We write $\Lambda_{i_1,\ldots,i_p}$ for the product $\Lambda = \lambda_1 \cdots \lambda_n$ with the $\lambda_{i_1},\ldots,\lambda_{i_p}$ omitted. We deliberately confuse an endomorphism of $X$ with the operator on $\End(X) \otimes_{k[x]} \Bar$ which acts by post-composition with the endomorphism. 
%Let $D$ denote the differential on $\End(X)$, i.\,e.~the commutator with~$d_X$.

\begin{lemma}\label{lemma:zorromain} For $\varphi \in \End(X) \otimes_{k[x]} \Bar$ and $0 \le p < n$ we have
\begin{align*}
\sum_{\ell(\bs{i}) = p} (-1)^{\gamma(\bs{i}) + p} & \Big\langle\!\Big\langle \nabla f_{i_1} \nabla f_{i_2} \cdots \nabla f_{i_p} \str\big( \Lambda_{i_1,\ldots,i_p} \circ \At(\varphi) \big) \Big\rangle\!\Big\rangle\\
&= \sum_{\ell(\bs{i}) = p+1} (-1)^{\gamma(\bs{i}) + p+1} \Big\langle\!\Big\langle \nabla f_{i_1} \nabla f_{i_2} \cdots \nabla f_{i_{p+1}} \str\big( \Lambda_{i_1,\ldots,i_p,i_{p+1}} \circ \varphi \big) \Big\rangle\!\Big\rangle + h_p( D( \varphi ) )
\end{align*}
where $f_i = \partial_{x_i} W$, $D=[d_X,-]$ is the differential on $\End(X)$, and the sums are over all sequences of length $p, p+1$, respectively, and
\[
h_p( \psi ) = \sum_{\ell(\bs{i}) = p} (-1)^{\gamma(\bs{i}) +p} \Big\langle\!\Big\langle \nabla f_{i_1} \cdots \nabla f_{i_p} \str\big( [ \Lambda_{i_1,\ldots,i_p}, \nabla](\psi) \big) \Big\rangle\!\Big\rangle\,.
\]
\end{lemma}
\begin{proof}
Fix a sequence $\bs{i}$ of length $p$. The super Jacobi identity for operators on $\End(X) \otimes \Bar$ gives
\be\label{superJacobi}
\big[ \Lambda_{i_1,\ldots,i_p}, \At \big] = (-1)^{n-p} \big[ D, \big[ \Lambda_{i_1,\ldots,i_p}, \nabla \big] \big] - \big[ \nabla, \big[ D, \Lambda_{i_1,\ldots,i_p} \big] \big]\,.
\ee
It is easy to see that $\str \circ \nabla = \nabla \circ \str$ and hence $\str \circ \At = 0$ so if we apply~\eqref{superJacobi} to~$\varphi$ and then apply $\dlangle \eta \str( - ) \drangle$ with $\eta = \nabla \circ f_{i_1} \circ \cdots \circ \nabla \circ f_{i_p}$, we have
\begin{align}
\dlangle \eta \str\big( \Lambda_{i_1,\ldots,i_p} \circ \At(\varphi) \big) \drangle &= (-1)^{n-p} \dlangle \eta \str\big( \big[ D, \big[ \Lambda_{i_1,\ldots,i_p}, \nabla \big] \big] \varphi \big) \drangle \nonumber\\
&\qquad - \dlangle \eta \str\big( \big[ \nabla, \big[ D, \Lambda_{i_1,\ldots,i_p} \big] \big] \varphi \big) \drangle \nonumber\\
&= \dlangle \eta \str\big( \big[ \Lambda_{i_1,\ldots,i_p}, \nabla \big] ( D\varphi ) \big) \drangle \label{eq:hutch1}\\
&\qquad- \dlangle \eta \nabla \str\big( \big[ D, \Lambda_{i_1,\ldots,i_p} \big] \varphi \big) \drangle \label{eq:hutch2}\\
&\qquad+ (-1)^{n-p+1} \bigdlangle \eta \str\big( \big[ D, \Lambda_{i_1,\ldots,i_p} \big]( \nabla \varphi ) \big) \drangle \, . \label{eq:hutch3}
\end{align}
Since $[d_X, \lambda_i] = f_i$ we have
\[
\big[ D, \Lambda_{i_1,\ldots,i_p} \big] = \sum_{a \notin \{ i_1, \ldots, i_p \} } (-1)^{a-1 + \#\{ b \l i_b < a \}} f_{a} \cdot \Lambda_{i_1,\ldots,i_p,a}
\]
where the sign counts the number of $\lambda$'s to the left of $\lambda_a$ in $\Lambda_{i_1,\ldots,i_p}$. Substituting, the last summand (\ref{eq:hutch3}) in the above is, up to a sign, 
\[
\sum_{a \notin \{ i_1,\ldots, i_p \}} (-1)^{a-1 + \#\{ b \l i_b < a \}} \bigdlangle \nabla f_{i_1} \cdots \nabla f_{i_p} f_a \str\big( \big[ D, \Lambda_{i_1,\ldots,i_p,a} \big]( \nabla \varphi ) \big) \bigdrangle \, .
\]
If we sum over all sequences $\bs{i}$ with the signs given in the statement of the lemma it is straightforward to check that the above term vanishes, because $f_{i_p} f_a$ and $f_a f_{i_p}$ appear in the sum with opposite signs. This completes the proof, since the sum over $\bs{i}$ of (\ref{eq:hutch1}) and (\ref{eq:hutch2}) gives us the right-hand side of the equation in the statement of the lemma.
\end{proof}

\begin{lemma}\label{lemma:langlecalc1} For $\varphi \in \End(X)$ we have
\[
\big\langle\!\big\langle \str \big( \Lambda \At^n (\varphi)\big) \big\rangle\!\big\rangle = \str(\varphi)\big|_{x \lmt y} + \sum_{p=0}^{n-1} h_i( D \At^{n-p-1}(\varphi) )\,.
\]
\end{lemma}
\begin{proof}
Applying Lemma \ref{lemma:zorromain} repeatedly yields
\begin{align*}
\big\langle\!\big\langle \str \big( \Lambda \At^n (\varphi)\big) \big\rangle\!\big\rangle = (-1)^n \sum_{\sigma \in S_n} (-1)^{|\sigma|} \big\langle\!\big\langle \nabla f_{\sigma(1)} \cdots \nabla f_{\sigma(n)} \str(\varphi) \big\rangle\!\big\rangle + \sum_{p=0}^{n-1} h_i( D \At^{n-p-1}(\varphi) )\,.
\end{align*}
%where 
%$$
%\delta = 
%\det 
%\begin{pmatrix}
%df_1 & df_2 & \cdots & df_n \\
%\vdots & \vdots & & \vdots \\
%df_1 & df_2 & \cdots & df_n 
%\end{pmatrix} .
%$$
%This concludes the proof that the Zorro map~\eqref{Zorro1superdetail} is homotopic to the identity. 
But by a small argument involving transition determinants (\textbf{todo}) this equals 
\[
= \str( \varphi )|_{x \lmt y} + \sum_{p=0}^{n-1} h_i( D \At^{n-p-1}(\varphi) ) \, .
\]
\end{proof}

The upshot is that, by the previous lemma and Lemma \ref{lemma:Zorrointermediate},
\begin{align*}
\mathcal{Z} &= \sum_j (-1)^{|e_j| + n} \big\langle\!\big\langle \str \big( \Lambda \circ \At^n (-\circ e_j^*) \big) \big\rangle\!\big\rangle \cdot e_j\\
&= \sum_j (-1)^{|e_j|} \str(-\circ e_j^*) \cdot e_j + H\\
&= \sum_j \str(e_j^* \circ -) \cdot e_j + H\\
&= 1_X + H
\end{align*}
where
\[
H = \sum_j \sum_{p=0}^{n-1} (-1)^{|e_j|} h_i( D \At^{n-p-1}(- \circ e_j^*) ) \cdot e_j\,.
\]
As the difference of two closed maps $H = \mathcal{Z} - 1_X$ is a closed endomorphism of $X$, and to complete the proof that $\mathcal{Z} \simeq 1_X$ it remains to prove that $H$ is null-homotopic. 

The idea is to use the non-degenerate pairing on the morphism spaces of $\hmf(k[x,z], V-W)$. If~$k$ is a field this is the Kapustin-Li pairing of \cite{??,??} but in general we use the contents of Section \ref{??} (\textbf{more}). The precise statement (Theorem \ref{??}) is that there is a homotopy equivalence of $\mathbb{Z}_2$-graded complexes over $k$
\be
\Hom_{k[x,z]}(X,X) \lto \Hom_k( \Hom_{k[x,z]}(X,X), k )[m+n]\,, \qquad \varphi \lmt \langle \varphi, - \rangle_X\,.
\ee
induced by the pairing
\begin{equation}\label{eq:bracketkl}
\langle -, - \rangle_X = \frac{1}{n!} \sum_{\sigma \in S_n} (-1)^{|\sigma|} \Res_{k[y,z]/k} \left[ \frac{ \str( - \circ - \circ \Lambda^{(x)} \Lambda^{(z)}) \underline{\operatorname{d}\!y} \underline{\operatorname{d}\!z}}{\partial_{x_1}W \ldots \partial_{x_n} W \partial_{z_1} V \ldots \partial_{z_m} V} \right]
\end{equation}
where $=\Lambda=\Lambda^{(x)}$ is the product of homotopies for the action of $\partial_{x_i}W$ introduced earlier, and $\Lambda^{(z)}$ is the product of homotopies for the action of $\partial_{z_j}V$ on~$X$. To prove that $\mathcal{Z} \simeq 1_X$ it is therefore enough to prove that the functionals $\langle \mathcal{Z}, - \rangle_X$ and $\langle 1_X, - \rangle_X$ on $\Hom_{k[x,z]}(X,X)$ are homotopic, or what is the same, that $\langle H, \varphi \rangle_X$ can be written as a function of $D(\varphi)=[d_X,\varphi]$. This will follow from a careful study of the map $\str( H \circ - )$.

We compute for a homogeneous element $\varphi \in \End(X)$ that
\begin{align}
\str(H \varphi) &= \str( \mathcal{Z} \varphi ) - \str(\varphi) \nonumber\\
&= \sum_i (-1)^{|e_i|} e^*_i \big( \mathcal Z(\varphi(e_i)) \big) - \str(\varphi)\nonumber \\
& = \sum_{i,j} (-1)^{|e_i| + |e_j| + n} e^*_i \Big( \big\langle\!\big\langle \str( \Lambda \At^n (\varphi(e_i)\otimes e^*_j \otimes e_j)) \big\rangle\!\big\rangle \Big) - \str(\varphi)\nonumber \\
& = \sum_{i,j} (-1)^{|e_i| + |e_j| + n + n|e_j|} e^*_i \Big( \big\langle\!\big\langle \str( \Lambda \At^n (\varphi(e_i)\otimes e^*_j )) \big\rangle\!\big\rangle e_j \Big) - \str(\varphi)\nonumber \\
& = \big\langle\!\big\langle \str \big( \Lambda \At^n \big(\sum_i \varphi(e_i)\otimes e^*_j \big)\big) \big\rangle\!\big\rangle - \str(\varphi)\nonumber \\
& = \big\langle\!\big\langle \str \big( \Lambda \At^n (\varphi)\big) \big\rangle\!\big\rangle - \str(\varphi) \nonumber\\
&= \sum_{p=0}^{n-1} h_i( D \At^{n-p-1}(\varphi) ) \label{strlambdaAtn}
\end{align}
where in the last step we use Lemma \ref{lemma:langlecalc1}. For the final steps in the argument we will need the following lemma.

\begin{lemma}\label{lem:PsiToTheRight}
Let $\omega\in\Omega(k[x_1,\ldots,x_n])$ be and $n$-form. Then $\Psi(\omega f)=\Psi(\omega) f(y)$ for any $f\in k[x]$. 
\end{lemma}

\begin{proof}
We may assume that $\omega$ is of the form $da_1\ldots da_n$ for some $a_i\in k[x]$. Then
\begin{align*}
\Psi(\omega f) & = \Psi \Big( \sum_{i=1}^n (-1)^{n-i} da_1\ldots da_{i-1} d(a_i a_{i+1}) da_{i+2}\ldots da_n df + (-1)^n a_1 da_2\ldots da_n df \Big) \\
& = \sum_{i=1}^n (-1)^{n-i} \partial_{[1]} a_1\ldots \partial_{[i-1]} a_{i-1} \partial_{[i]} (a_i a_{i+1}) \partial_{[i+1]} a_{i+2} \ldots \partial_{[n-1]} a_n \partial_{[n]} f \\
& \qquad + (-1)^n a_1 \partial_{[1]} a_2 \ldots \partial_{[n-1]} a_n \partial_{[n]} f \\
& = \sum_{i=1}^n (-1)^{n-i} \partial_{[1]} a_1\ldots \partial_{[i-1]} a_{i-1} \Big(\partial_{[i]} a_i {}	^{t_1\ldots t_i}a_{i+1} + {}^{t_1\ldots t_{i-1}} a_i \partial_{[i]} a_{i+1} \Big) \\
& \qquad \cdot  \partial_{[i+1]} a_{i+2} \ldots \partial_{[n-1]} a_n \partial_{[n]} f + (-1)^n a_1 \partial_{[1]} a_2 \ldots \partial_{[n-1]} a_n \partial_{[n]} f \\
& = \partial_{[1]} a_1 \ldots \partial_{[n]} a_n f(y) \\
& = \Psi(\omega) f(y) \, ,
\end{align*}
where we used Lemma~\ref{lem:LeibnizForDQO} in the third step. 
\end{proof}

\begin{theorem}\label{theorem:mainzorro} The Zorro map $\mathcal{Z}$ is homotopic to $1_X$.
\end{theorem}
\begin{proof}
As we have already mentioned, it suffices by non-degeneracy of \eqref{eq:bracketkl} to prove that $\langle H, - \rangle_X$ is a null-homotopic function $\Hom_{k[x,z]}(X,X) \lto k$. For $\varphi' \in \End(X)$ the numerator in the residue which computes $\langle H, \varphi' \rangle_X$ is, using \eqref{strlambdaAtn}, a differential form multiplied by
\[
\str( H \varphi' \Lambda^{(x)} \Lambda^{(z)}) = \sum_{p=0}^{n-1} h_p( D \At^{n-p-1}(\varphi' \Lambda^{(x)} \Lambda^{(z)}) )
\]
But given $\varphi' \in \End(X)$ if we set $\varphi = \varphi' \circ \Lambda^{(x)} \Lambda^{(z)}$ and substitute in (\ref{??}) we find that the two functionals $\langle \mathcal{Z}, - \rangle_X$ and $\langle 1_X, - \rangle_X$ differ by a term which is an expression in residues of
\begin{equation}\label{eq:mainzorro1}
h_p( D \At^{n-p-1}(\varphi) ) = h_p( D \At^{n-p-1}(\varphi' \Lambda^{(x)} \Lambda^{(z)}) )\,.
\end{equation}
But $\At$ is closed, so 
\begin{align}
h_p( D \At^{n-p-1}(\varphi' \Lambda^{(x)} \Lambda^{(z)}) ) &= h_p \At^{n-p-1}D(\varphi' \Lambda^{(x)} \Lambda^{(z)}) ) \nonumber\\
&= h_p \At^{n-p-1}\big( D(\varphi') \Lambda^{(x)} \Lambda^{(z)} )\\
&\qquad + (-1)^{|\phi'|} h_p \At^{n-p-1}\big( \varphi' D(\Lambda^{(x)}) \Lambda^{(z)} ) \label{eq:mainzorro4}\\
&\qquad + (-1)^{|\phi'| + n} h_p \At^{n-p-1}\big( \varphi' \Lambda^{(x)} D(\Lambda^{(z)}) ) \, . \label{eq:mainzorro5}
\end{align}
Now $D(\Lambda^{(z)})$ is a linear combination of terms divisible by the $\partial_{z_j} V$, and since $h_p$ and $\At$ are $k[z]$-linear these coefficients pass through to annihilate with the denominator in the residue in  (\ref{eq:bracketkl}). 

Similarly $\varphi' D(\Lambda^{(x)}) \Lambda^{(z)}$ is a sum of terms of the form $f_j \alpha$ for various $j$. But the Atiyah class and the commutator $[\Lambda_{i_1,\ldots,i_p}, \nabla]$ are right linear (TODO: explain right action on $\Bar$?) so
\begin{align*}
h_p \At^{n-p-1}\big( \alpha f_j ) &= h_p\left( \At^{n-p-1}(\alpha) \cdot f_j \right)\\
&= \sum_{\ell(\bs{i}) = p} (-1)^{\gamma(\bs{i}) +p} \Big\langle\!\Big\langle \nabla \left\{ f_{i_1} \cdots \nabla f_{i_p} \str\big( [ \Lambda_{i_1,\ldots,i_p}, \nabla]\At^{n-p-1}(\alpha) \big) \cdot f_j \right\} \Big\rangle\!\Big\rangle\,.
\end{align*}
But $\nabla( f_{i_1} \beta f_j ) = f_{i_1} \nabla( \beta f_j ) + df_{i_1} \beta f_j$. The first summand vanishes in the residue for obvious reasons, and the second vanishes since by Lemma~\ref{lem:PsiToTheRight}
\[
\dlangle - \cdot f_j \drangle = \dlangle - \drangle \cdot f_j(y)
\]
which vanishes in the residue.

The upshot is that (\ref{eq:mainzorro4}) and (\ref{eq:mainzorro5}) do not contribute under the residue, so that
\[
\langle H, - \rangle_X = 
\frac{1}{n!} \sum_{\sigma \in S_n}\sum_{p=0}^{n-1} (-1)^{|\sigma|} \Res_{k[y,z]/k} \left[ \frac{ h_p \At^{n-p-1}( D(-) \Lambda^{(x)} \Lambda^{(z)} ) \underline{\operatorname{d}\!y} \underline{\operatorname{d}\!z}}{\partial_{x_1}W \ldots \partial_{x_n} W \partial_{z_1} V \ldots \partial_{z_m} V} \right]\,.
\]
This proves that $\langle \mathcal{Z}, - \rangle_X$ and $\langle 1_X, - \rangle_X$ are homotopic functionals, and so by the non-degeneracy of Theorem \ref{??} we conclude that $\mathcal{Z}$ and $1_X$ are homotopic.
\end{proof}

The three other Zorro moves are proven analogously (\textbf{todo}).

\begin{theorem}
TODO: $\LG$ has adjoints\ldots
\end{theorem}


[TODO: Show that canonical definition of dual 2-morphism gives the same as naive definition; that's the same as proving the morphism-migration property (say for evaluation, here it follows directly from pre-/post-composition etc.)]

TODO: show that $\LG$ is pivotal in the right sense

\section{Defect action on bulk fields}\label{sec:defectaction}

In any bicategory with adjoints there are natural maps between the endomorphism spaces of unit 1-morphisms. Roughly, these maps are constructed by capturing a 2-morphism of a unit 1-morphism inside a loop labelled by an arbitrary 1-morphism (and its adjoint). Below we present the details for the case of the bicategory $\LG$. We will also give the interpretation in terms of defect actions on bulk fields in Landau-Ginzburg models. 

Let $X\in \hmf(k[x,z], V-W)$ as before. In this section when we write $\End$ we mean the spaces of 2-endomorphisms in $\LG$. We define maps 
$$
\mathcal D_l(X): \End(\Delta_V) \longrightarrow \End(\Delta_W) \, , \qquad
\mathcal D_r(X): \End(\Delta_W) \longrightarrow \End(\Delta_V)
$$
in terms of the morphisms encoding the monoidal and adjunction structures as follows. For $\phi\in \End(\Delta_V)$ and $\psi\in \End(\Delta_W)$ we set
\begin{align*}
\mathcal D_l(X)(\phi) & = \eval_X \circ (1_{X^\dual}\otimes (\lambda_X \circ (\phi\otimes 1_X)\circ \lambda_X^{-1})) \circ \widetilde\coev_X \, , \\ 
\mathcal D_r(X)(\psi) & = \widetilde\eval_X \circ (1_{X}\otimes (\lambda_{X^\dual} \circ (\phi\otimes 1_{X^\dual})\circ \lambda_{X^\dual}^{-1})) \circ \coev_X \, .
\end{align*}
In the special case $\phi=1$ and $\psi=1$ we obtain the left and right \textsl{quantum dimensions} $\operatorname{qdim}_l(X) = \mathcal D_l(X)(1)$ and $\operatorname{qdim}_r(X) = \mathcal D_r(X)(1)$. 

Diagrammatically these definitions read 
\be\label{defectaction}
\mathcal D_l(X)(\phi) = 
\begin{tikzpicture}[very thick,scale=0.8,color=blue!50!black, baseline]

\fill (1.5,1.3) circle (2.5pt) node[right] {{\small $\lambda_X$}};
\fill (1.5,-1.3) circle (2.5pt) node[right] {{\small $\lambda^{-1}_X$}};
\fill (0,0) circle (2.5pt) node[left] {{\small $\phi$}};

\fill (0.6,1) circle (0pt) node {{\small $\Delta_V$}};
\fill (0.6,-0.95) circle (0pt) node {{\small $\Delta_V$}};

\draw[directed] (1.5,1.3) .. controls +(0,1.5) and +(0,1.5) .. (-1.5,1.3);
\draw[directed] (-1.5,-1.3) .. controls +(0,-1.5) and +(0,-1.5) .. (1.5,-1.3);
\draw (1.5,-1.3) -- (1.5,1.3)
node[midway,left] {{{\footnotesize$z'\vphantom{y}$}}}
node[midway,right] {{{\footnotesize$y\vphantom{yz'}$}}};
\draw (-1.5,-1.3) -- (-1.5,1.3)
node[midway,left] {{{\footnotesize$x\vphantom{yz'}$}}}
node[midway,right] {{{\footnotesize$z\vphantom{yz'}$}}};
\draw[dashed] (0,0) .. controls +(0,1) and +(-0.5,-1) .. (1.5,1.3);
\draw[dashed] (0,0) .. controls +(0,-1) and +(-0.5,1) .. (1.5,-1.3);
\draw[dashed] (0,-2.5) -- (0,-3.5)
node[near end,right] {{{\small$\Delta_W$}}};
\draw[dashed] (0,2.47) -- (0,3.5)
node[near end,right] {{{\small$\Delta_W$}}};
\end{tikzpicture}
\, , \qquad 
\mathcal D_r(X)(\psi) = 
\begin{tikzpicture}[very thick,scale=0.8,color=blue!50!black, baseline]

\fill (1.5,1.3) circle (2.5pt) node[right] {{\small $\lambda_{X^\dual}$}};
\fill (1.5,-1.3) circle (2.5pt) node[right] {{\small $\lambda^{-1}_{X^\dual}$}};
\fill (0,0) circle (2.5pt) node[left] {{\small $\psi$}};

\fill (0.6,1) circle (0pt) node {{\small $\Delta_W$}};
\fill (0.6,-0.95) circle (0pt) node {{\small $\Delta_W$}};

\draw[redirected] (1.5,1.3) .. controls +(0,1.5) and +(0,1.5) .. (-1.5,1.3);
\draw[redirected] (-1.5,-1.3) .. controls +(0,-1.5) and +(0,-1.5) .. (1.5,-1.3);
\draw (1.5,-1.3) -- (1.5,1.3)
node[midway,left] {{{\footnotesize$y\vphantom{yz'}$}}}
node[midway,right] {{{\footnotesize$z'\vphantom{yz'}$}}};
\draw (-1.5,-1.3) -- (-1.5,1.3)
node[midway,left] {{{\footnotesize$z\vphantom{yz'}$}}}
node[midway,right] {{{\footnotesize$x\vphantom{yz'}$}}};
\draw[dashed] (0,0) .. controls +(0,1) and +(-0.5,-1) .. (1.5,1.3);
\draw[dashed] (0,0) .. controls +(0,-1) and +(-0.5,1) .. (1.5,-1.3);
\draw[dashed] (0,-2.5) -- (0,-3.5)
node[near end,right] {{{\small$\Delta_V$}}};
\draw[dashed] (0,2.47) -- (0,3.5)
node[near end,right] {{{\small$\Delta_V$}}};
\end{tikzpicture}
\ee
where again we indicated our choice of variable names in the four domains. 

\begin{remark}
$\End(\Delta_W) = k[x]/(\partial_{x_i}W)$ is the Hochschild cohomology of $\hmf(k[x], W)$~\cite{d0904.4713}. This space also precisely describes bulk fields of Landau-Ginzburg models with potential~$W$, it is a commutative Frobenius algebra whose non-degenerate pairing
\be\label{bulktopmet}
\langle \phi, \psi \rangle_W = (-1)^n \Res_{k[x]/k} \left[ \frac{\phi \psi \, \underline{\operatorname{d}\!x}}{\partial_{x_1}W\ldots\partial_{x_n} W}\right]
\ee
describes two-point correlators. Furthermore, matrix factorisations of $V-W$ describe defect conditions between different Landau-Ginzburg models. Hence the maps~\eqref{defectaction} have the natural interpretations in terms of defect operators on bulk fields: for example, a bulk field~$\phi$ in the theory with potential~$V$ is mapped to the bulk field $\mathcal D_l(X)(\phi)$ in the theory with potential~$W$ by wrapping around its insertion on the worldsheet a defect line labelled by~$X$, and then collapsing this loop onto the insertion point. This limiting process is non-singular as the bicategory $\LG$ describes the purely topological sector of Landau-Ginzburg models. 
\end{remark}

Using the ``folding trick'' (which relates defects to boundary conditions in a product theory) one can argue for explicit expressions for $\mathcal D_l(X)$ and $\mathcal D_r(X)$. This was done in~\cite{cr1006.5609} for the case $V=W$. Here we use our adjunction formulas to directly prove it for the general case: 

\begin{proposition}
For any $X\in \hmf(k[z_1,\ldots,z_m,x_1,\ldots,x_n], V-W)$, $\phi\in \End(\Delta_V)$ and $\psi\in \End(\Delta_W)$ we have
\begin{align*}
\mathcal D_l(X)(\phi) & = (-1)^{{m+1\choose 2} + {n\choose 2}} \Res_{k[x,z]/k[x]} \left[ \frac{\phi(z) \str\big( \partial_{z_1} d_{X}\ldots \partial_{z_m} d_{X} \partial_{x_1} d_{X}\ldots \partial_{x_n} d_{X} \big) \underline{\operatorname{d}\! z}}{\partial_{z_1} V \ldots \partial_{z_m} V} \right] \, , \\
\mathcal D_r(X)(\psi) & = (-1)^{{m\choose 2} + {n+1\choose 2}} \Res_{k[x,z]/k[z]} \left[ \frac{\psi(x) \str\big( \partial_{z_1} d_{X}\ldots \partial_{z_m} d_{X} \partial_{x_1} d_{X}\ldots \partial_{x_n} d_{X} \big) \underline{\operatorname{d}\! x}}{\partial_{x_1} W \ldots \partial_{x_n} W} \right] \, . 
\end{align*}
\end{proposition}

\begin{proof}
We treat the case of $\mathcal D_l(X)$ in detail, the argument for $\mathcal D_r(X)$ works analogously. Since $\End(\Delta_W) = k[x]/(\partial_{x_i}W)$ and $\End(\Delta_V) = k[z]/(\partial_{z_i}V)$ we are free to set $x=y$ and $z=z'$ at appropriate places, cf.~\eqref{defectaction}. Furthermore, $\lambda_X$ will project out all non-zero degree contributions coming from the action of $\lambda_X^{-1}$, so $\lambda_X\circ (\phi \otimes 1_X)\circ \lambda_X^{-1}$ is simply multiplication by the polynomial $\phi(z)$. 

In the lower part of the expression for $\mathcal D_l(X)(\phi)$ in~\eqref{defectaction} we have 
\begin{align}
\widetilde\coev (1) & = \sum_j (-1)^{|e_j|} (\varepsilon\Psi) \left( (-\At_{X})^n (e_j^*\otimes e_j) \right) \nonumber \\
& = \sum_j (-1)^{|e_j| + n} (-1)^{(|e_j| + 1)+\ldots +(|e_j| + n) + n|e_j|} e^*_{j} \otimes e_{k_n} \otimes (\varepsilon\Psi) \left( d(d_{X})_{k_n k_{n-1}} \ldots d(d_{X})_{k_1 j} \right) \nonumber \\
& = \sum_j (-1)^{|e_j|(|e_j| + n) + {n+1\choose 2} + n} e_j^* \otimes e_{k_n}\big( \partial_{[1]} d_{X} \ldots \partial_{[n]} d_{X} \big)_{k_n j} \nonumber \\
& = \sum_j (-1)^{|e_j|(|e_j| + n) + {n\choose 2}} e_j^* \otimes \big( \partial_{x_1} d_{X} \ldots \partial_{x_n} d_{X} \big) (e_{j}) \label{coevtilde1}
\end{align}
which we identify with $(-1)^{{n\choose 2}} \partial_{x_1} d_{X} \ldots \partial_{x_n} d_{X}$ in $\End(X)$. Note that in the last step leading to~\eqref{coevtilde1} we set $\partial_{[i]} d_{X}(x,z) = \partial_{x_i} d_{X}(x,z)$ since $x=y$ in $\End(\Delta_W)$. 

Next we apply the upper part of $\mathcal D_l(X)(\phi)$ in~\eqref{defectaction} to~\eqref{coevtilde1} to get
$$
\mathcal D_l(X)(\phi) = (-1)^{{m+1\choose 2} + {n\choose 2}} \Res_{k[x,z]/k[x]} \left[ \frac{\phi(z) \str\big( \partial_{z_1} d_{X}\ldots \partial_{z_m} d_{X} \partial_{x_1} d_{X}\ldots \partial_{x_n} d_{X} \big) \underline{\operatorname{d}\! z}}{\partial_{z_1} V \ldots \partial_{z_m} V} \right] + \mathcal O(\theta) \, . 
$$
Here we collectively denote the contributions from $\eval_X$ of non-zero degree in the Koszul complex $\Delta_W$ by $\mathcal O(\theta)$. Since we know that $\mathcal D_l(X)(\phi)$ is a morphism in $\End(\Delta_W) = k[x]/(\partial_{x_i}W)$ it follows that $\mathcal O(\theta)$ must be null-homotopic, thus concluding the proof. 
\end{proof}

\begin{corollary}
For any $X\in \hmf(k[x,z], V-W)$ the operators $\mathcal D_l(X)$ and $\mathcal D_r(X)$ are adjoint with respect to the pairings~\eqref{bulktopmet}, i.\,e.~we have
\be\label{Dadjoint}
\big\langle \mathcal D_l(X)(\phi), \psi \big\rangle_W = \big\langle \phi , \mathcal D_r(X)(\psi) \big\rangle_V
\ee
for all $\phi\in \End(\Delta_V)$ and $\psi\in \End(\Delta_W)$. 
\end{corollary}

\begin{remark}
We recall the physical interpretation of the relation~\eqref{Dadjoint}. Both sides of this equation are two-point correlators on the Riemann sphere, with a defect line labelled by~$X$ wrapped around counterclockwise the bulk field~$\phi$, or wrapped around~$\psi$ in clockwise fashion. That both correlators should be equal follows from the fact that the topological defect can be moved around the sphere at no cost: 
$$
\left\langle
\begin{tikzpicture}[baseline=-0.1cm]
\def\R{1.85}
\def\angEl{45}
\filldraw[ball color= white!77!blue,draw=white] (0,0) circle (\R);
\DrawLatitudeCircleU[\R,rotate=130,very thick, blue]{65}
\fill (-0.95,-0.83) circle (1pt) node[above] {{\small$\phi$}}; 
\fill (0.95,-0.83) circle (1pt) node[above] {{\small$\psi$}}; 
\end{tikzpicture}
\right\rangle
=
\left\langle
\begin{tikzpicture}[baseline=-0.1cm]
\def\R{1.85}
\def\angEl{45}
\filldraw[ball color= white!77!blue,draw=white] (0,0) circle (\R);
\DrawLongitudeCircle[\R]{80}
\fill (-0.95,-0.83) circle (1pt) node[above] {{\small$\phi$}}; 
\fill (0.95,-0.83) circle (1pt) node[above] {{\small$\psi$}}; 
\end{tikzpicture}
\right\rangle
=
\left\langle
\begin{tikzpicture}[baseline=-0.1cm]
\def\R{1.85}
\def\angEl{45}
\filldraw[ball color= white!77!blue,draw=white] (0,0) circle (\R);
\DrawLatitudeCircle[\R,rotate=-130, very thick, blue]{65}
\fill (-0.95,-0.83) circle (1pt) node[above] {{\small$\phi$}}; 
\fill (0.95,-0.83) circle (1pt) node[above] {{\small$\psi$}}; 
\end{tikzpicture}\right\rangle .
$$
\end{remark}


\section{Shadows}\label{sec:shadows}

The adjunctions in $\LG$ afford us the construction of a bicategorical trace in terms of shadow functors~\cite{p0807.1471}. We will also see that shadows allow to recover and generalise the boundary-bulk and bulk-boundary maps of the two-dimensional topological field theories based on Landau-Ginzburg models. 

\begin{definition}
A bicategory~$\mathcal B$ \textsl{has shadows} if there is a category~$\mathcal C$ together with functors
$$
\langle\!\langle - \rangle\!\rangle : \mathcal B(A,A) \lra \mathcal C 
$$
for every object $A\in \mathcal B$ such that there are natural isomorphisms $\theta : \langle\!\langle X \otimes Y \rangle\!\rangle \lra \langle\!\langle Y \otimes X \rangle\!\rangle$ for every pair of composable 1-morphisms $X,Y$, and the diagrams
$$
\xymatrix{%
\langle\!\langle (X \otimes Y) \otimes Z \rangle\!\rangle \ar[r]^-{\theta} \ar[d]_-{\langle\!\langle \alpha \rangle\!\rangle} & 
\langle\!\langle Z \otimes (X \otimes Y) \rangle\!\rangle \ar[r]^-{\langle\!\langle \alpha^{-1} \rangle\!\rangle} & 
\langle\!\langle (Z \otimes X) \otimes Y \rangle\!\rangle \\
\langle\!\langle X \otimes (Y \otimes Z) \rangle\!\rangle \ar[r]^-{\theta} & 
\langle\!\langle (Y \otimes Z) \otimes X \rangle\!\rangle \ar[r]^-{\langle\!\langle \alpha \rangle\!\rangle} & 
\langle\!\langle Y \otimes (Z \otimes X) \rangle\!\rangle \ar[u]_-{\theta}
}%
$$
and
$$
\xymatrix{%
\langle\!\langle X \otimes 1_A \rangle\!\rangle \ar[r]^-{\theta} \ar[dr]_-{\langle\!\langle \rho \rangle\!\rangle} & 
\langle\!\langle 1_A \otimes X \rangle\!\rangle \ar[r]^-{\theta} \ar[d]^-{\langle\!\langle \lambda \rangle\!\rangle} & 
\langle\!\langle X \otimes 1_A \rangle\!\rangle \ar[dl]^-{\langle\!\langle \rho \rangle\!\rangle}  \\
 & 
\langle\!\langle X \rangle\!\rangle & 
}%
$$
commute whenever they make sense. 
\end{definition}

\begin{proposition}
The bicategory $\LG$ has shadows given by 
\begin{align*}
\langle\!\langle - \rangle\!\rangle : \LG \big( (R,W), (R,W) \big) & \lra \hmf(k,0) \, , \\
Z & \lmt Z\otimes_{\Re} R
\end{align*}
with the isomorphism $\theta: \langle\!\langle X \otimes Y \rangle\!\rangle \lra \langle\!\langle Y\otimes X  \rangle\!\rangle$ induced by the graded swap map $X \otimes Y \lra Y \otimes X$. 
\end{proposition}

The proof is a straightforward check of the axioms, made especially easy by the fact that $\langle\!\langle - \rangle\!\rangle$ is simply defined as tensoring with the actual diagonal~$R$. Note however that this is homotopy equivalent to tensoring with the unit matrix factorisation $\Delta_W$. 

Since $\LG$ is a bicategory with adjoints and shadows it is automatically equipped with a 2-categorical trace operation as introduced and discussed at length in~\cite{p0807.1471, ps0910.1306}. We only quote the definition: 

\begin{definition}
Let~$\mathcal B$ be a bicategory with shadows and 1-morphism~$Y$ with adjoint~$Y^\dual$. Then the \textsl{trace} of a 2-morphism $\psi: X\otimes Y \lra Y \otimes Z$ is the map 
$$
\xymatrix{%
\langle\!\langle X \rangle\!\rangle \ar[rr]^-{\langle\!\langle1\otimes \coev_Y\rangle\!\rangle} && 
\langle\!\langle X \otimes Y \otimes Y^\dual \rangle\!\rangle \ar[r]^-{\langle\!\langle\psi \otimes 1\rangle\!\rangle} & 
\langle\!\langle Y \otimes Z \otimes Y^\dual \rangle\!\rangle \ar[r]^-{\theta} &
\langle\!\langle Y^\dual \otimes Y \otimes Z \rangle\!\rangle \ar[rr]^-{\langle\!\langle\eval_Y \otimes 1\rangle\!\rangle} && 
\langle\!\langle Z \rangle\!\rangle 
}%
\, . 
$$
\end{definition}

Next we wish to point out a connection between shadows and the structure of two-dimensional open/closed topological field theory (TFT) for Landau-Ginzburg models. Recall that a TFT is the data of a commutative Frobenius algebra~$C$, a Calabi-Yau category~$\mathcal O$, bulk-boundary maps $\beta_A: C \lra \End_{\mathcal O}(A)$, and boundary-bulk maps $\beta^A: \End_{\mathcal O}(A) \lra C$ for all $A\in\mathcal O$. These data are subject to several consistency conditions, see e.\,g.~\cite{TODO: Lazaroiu and Moore-Segal}. 

Every Landau-Ginzburg model with potential $W\in R = k[x_1,\ldots,x_n]$ gives rise to a TFT with $C=R/(\partial W)$, $\mathcal O = \hmf(R,W)$ and
\be\label{betaQ}
\beta_Q : \phi \lmt \phi \cdot 1_Q \, , \qquad 
\beta^Q : \psi \lmt (-1)^{n\choose 2} \str(\psi \, \partial_{x_1} d_Q \ldots \partial_{x_n} d_Q) \, . 
\ee
The hardest part in establishing this result is to prove the non-degeneracy of the Kapustin-Li pairing and that the Cardy condition holds; this was first done in~\cite{m0912.1629} and~\cite{pv1002.2116}, respectively, in the case where~$k$ is a field. In Sections~\ref{section:dualityadjointop} and~\ref{sec:CardyCondition} we will give new proofs for the general case. 

We note that the maps~\eqref{betaQ} can be recovered from the adjunction and shadow structure of $\LG$ as follows. On the one hand we have 
$$
\langle\!\langle \Delta_W \rangle\!\rangle = R/(\partial W)[n]
$$
since $\langle\!\langle \Delta_W \rangle\!\rangle = \Delta_W \otimes_{\Re} R = (\bigwedge (\bigoplus_{i=1}^n R \theta_i), \sum_{i=1}^n \partial_{x_i}W \theta_i)$ which is homotopy equivalent (and therefore equal in $\hmf(k,0)$) to $R/(\partial W)[n]$. On the other hand $\langle\!\langle Q^\dual \otimes Q \rangle\!\rangle = Q^\vee \otimes_k Q \otimes_{\Re} R = \End_R(Q)$. Thus from the explicit expressions~\eqref{TODO} and~\eqref{TODO} we find that $\beta_Q = \langle\!\langle \widetilde\coev_Q \rangle\!\rangle$ and $\beta^Q = \langle\!\langle \eval_Q \rangle\!\rangle$ (up to a sign, TODO?). 

Motivated by the above this construction can be extended to any 1-morphism in $\LG$: for $X \in \hmf(k[x_1,\ldots,x_n,z_1,\ldots,z_m], V-W)$ we define the \textsl{generalised bulk-boundary} and \textsl{boundary-bulk maps} to be
$$
\beta_X= \langle\!\langle \widetilde\coev_X \rangle\!\rangle : \langle\!\langle \Delta_{W} \rangle\!\rangle \lra \langle\!\langle X^\dual \otimes X \rangle\!\rangle
\, , \qquad 
\beta^X= \langle\!\langle \eval_X \rangle\!\rangle : \langle\!\langle X^\dual \otimes X \rangle\!\rangle \lra \langle\!\langle \Delta_{W} \rangle\!\rangle \, , 
$$
respectively. Substituting the expressions~\eqref{TODO} and~\eqref{TODO} we find that the form of $\beta_X$ stays the same while for $m\neq 0$ the generalised boundary-bulk map $\beta^X$  involves a residue and additional derivatives $\partial_{z_i} d_X$: 
\begin{align*}
\beta_X : k[x]/(\partial V)[n] & \lra \End_{k[x,z]}(X)[m] \, , \\
\phi & \lmt \phi \cdot 1_{X}[m] \, , \\
\beta^X : \End_{k[x,z]}(X)[m] & \lra k[x]/(\partial V)[n] \, , \\
\psi & \lmt (-1)^{m+1\choose 2} \Res_{k[x,z]/k[x]} \left[ \frac{\str\left( \psi\, \partial_{z_1} d_X \ldots \partial_{z_m} d_X \partial_{x_1} d_X \ldots \partial_{x_n} d_X \right) \underline{\operatorname{d}\!z}}{\partial_{z_1} W \ldots \partial_{z_m} W} \right] \, . 
\end{align*}

[TODO: comment on relation to defect action??]


\section{Cardy condition}\label{sec:CardyCondition}

As another application of our explicit expressions for adjunctions in $\LG$ in this section we present a new proof of the Cardy condition. This deep result long eluded rigorous proofs, which were only recently given in~\cite{pv1002.2116, dm1102.2957} using rather heavy machinery. With our expressions for evaluations and coevaluations in $\LG$ it will follow effortlessly for arbitrary commutative noetherian $\nQ$-algebras~$k$, such as $k=\nC[t_1,\ldots,t_d]$, simply from considerations of one single diagram. 

Let us again fix a potential $W\in k[x] = k[x_1,\ldots,x_n]$, two matrix factorisations $X,Y \in \hmf(k[x], W)$, and also two maps $\varphi: X\lra X$, $\psi: Y\lra Y$. We interpret~$X$ and~$Y$ as 1-morphisms $(k,0)\lra (k[x],W)$ in $\LG$, and in this way we obtain the 2-morphism (where here and below we do no longer display dashed lines for unit 1-endomorphisms)
\be\label{annulus}
%%%%%%%%%%%%%%%%%%%%%%%%%%
\begin{tikzpicture}[very thick,scale=0.8,color=blue!50!black, baseline,>=stealth]
\nicecolourscheme (0,0) circle (2);
\nicereallynocolourscheme (0,0) circle (1);
\fill (1.5,0) circle (0pt) node[white] {{\small$W$}};
\draw (0,0) circle (2);
\draw[->, very thick] (-0.100,2) -- (-0.101,2) node[above] {{\small$\eval_Y$}}; 
\draw[->, very thick] (0.100,-2) -- (0.101,-2) node[below] {{\small$\widetilde\coev_Y$}}; 
\fill (45:2) circle (2.5pt) node[right] {{\small$\psi$}};
%
\draw (0,0) circle (1);
\draw[->, very thick] (0.100,1) -- (0.101,1) node[above] {{\small$\widetilde\eval_X$}}; 
\draw[->, very thick] (-0.100,-1) -- (-0.101,-1) node[below] {{\small$\coev_X$}}; 
\fill (135:1) circle (2.5pt) node[left] {{\small$\varphi$}};
\end{tikzpicture} 
%%%%%%%%%%%%%%%%%%%%%%%%%%
\ee
which is a map $k\lra k$ that we call~$C$. Note that we colour-code the domain representing the non-trivial object $(k[x],W)$ in $\LG$. This in not only a visual aid for interpreting the components of the map~$C$, but~\eqref{annulus} can also be interpreted as an annulus correlator with ``boundary conditions'' $X,Y$ and ``field insertions'' $\varphi, \psi$. 

To put ourselves in a position to better understand and manipulate~\eqref{annulus} let us spell out its constituents. From the general expressions~\eqref{TODO} for evaluations and coevaluations we find \begin{align}
& \widetilde\eval_X = \str(-) + \mathcal O(\theta) \, , \qquad \eval_Y = (-1)^{n+1\choose 2} \Res_{k[x]/k} \left[ \frac{\str( \, - \circ \Lambda_Y)}{\partial_1 W\ldots \partial_n W} \right] \, , \nonumber \\
& \coev_X: 1 \lmt \Lambda_X \, , \qquad  \widetilde\coev_Y: 1\lmt 1_Y \label{evcoeveasy}
\end{align}
where $\Lambda_X = \partial_1 d_X\ldots \partial_n d_X$ and $\Lambda_Y = \partial_1 d_Y\ldots \partial_n d_Y$. 

We will now compute the diagram~\eqref{annulus} in two different ways and then compare the results. One possibility is to first collapse the $X$-loop to obtain a ``$(k[x],W)$-bubble'', and then collapse its boundary loop~$Y$ in turn to arrive at an expression for $C:k \lra k$. It follows from the general results of Section~\ref{sec:defectaction} or directly from~\eqref{evcoeveasy} that the first step of collapsing~$X$ produces the 2-morphism $\str(\varphi \Lambda_X)\in\End(\Delta_W)$. Similarly, the second step of collapsing~$Y$ produces a residue: 
$$
C 
= 
\begin{tikzpicture}[very thick,scale=0.6,color=blue!50!black, baseline,>=stealth]
\nicecolourscheme (0,0) circle (2);
\nicereallynocolourscheme (0,0) circle (1);
\fill (1.5,0) circle (0pt) node[white] {{\small$W$}};
\draw (0,0) circle (2);
\draw[->, very thick] (-0.100,2) -- (-0.101,2) node[above] {}; 
\draw[->, very thick] (0.100,-2) -- (0.101,-2) node[below] {}; 
\fill (45:2) circle (2.5pt) node[right] {{\small$\psi$}};
%
\draw (0,0) circle (1);
\draw[->, very thick] (0.100,1) -- (0.101,1) node[above] {}; 
\draw[->, very thick] (-0.100,-1) -- (-0.101,-1) node[below] {}; 
\fill (135:1) circle (2.5pt) node[left] {{\small$\varphi$}};
\end{tikzpicture} 
= 
\begin{tikzpicture}[very thick,scale=0.6,color=blue!50!black, baseline,>=stealth]
\nicecolourscheme (0,0) circle (2);
\fill (1.5,0) circle (0pt) node[white] {{\small$W$}};
\draw (0,0) circle (2);
\draw[->, very thick] (-0.100,2) -- (-0.101,2) node[above] {}; 
\draw[->, very thick] (0.100,-2) -- (0.101,-2) node[below] {}; 
\fill (45:2) circle (2.5pt) node[right] {{\small$\psi$}};
%
\fill (45:0) circle (2.5pt) node[below] {{\small$\str(\varphi\Lambda_X)$}};
\end{tikzpicture} 
= 
(-1)^{n+1\choose 2} \Res_{k[k]/k} \left[ \frac{\str \big( \str( \varphi \Lambda_X) \psi \Lambda_Y \big)}{\partial_1 W \ldots \partial_n W} \right] \, .
$$
Using the expressions $\beta^X(\varphi) = (-1)^{n\choose 2} \str(\varphi \Lambda_X)$ and $\beta^Y(\psi) = (-1)^{n\choose 2} \str(\psi \Lambda_Y)$ for the boundary-bulk maps from Section~\ref{sec:shadows} we thus find from the above that
\be\label{C1}
C = 
(-1)^{n+1\choose 2} \Res_{k[x]/k} \left[ \frac{\beta^X(\varphi) \, \beta^Y (\psi)}{\partial_1 W \ldots \partial_n W} \right] \, .
\ee

On the other hand, we may compute~$C$ by first fusing the two loops in~\eqref{annulus} together to obtain a single loop labelled by $X^\dual\otimes Y = \Hom(X,Y) \in \End_{\LG}( (k,0) )$. Then we use
$$
\eval_{X^\dual \otimes Y} = \str(-) \, , \qquad \widetilde\coev_{X^\dual \otimes Y} : 1 \lmt 1_{X^\dual \otimes Y}
$$
to arrive at
\be\label{C2}
C 
= 
\begin{tikzpicture}[very thick,scale=0.6,color=blue!50!black, baseline,>=stealth]
\nicecolourscheme (0,0) circle (2);
\nicereallynocolourscheme (0,0) circle (1);
\fill (1.5,0) circle (0pt) node[white] {{\small$W$}};
\draw (0,0) circle (2);
\draw[->, very thick] (-0.100,2) -- (-0.101,2) node[above] {}; 
\draw[->, very thick] (0.100,-2) -- (0.101,-2) node[below] {}; 
\fill (45:2) circle (2.5pt) node[right] {{\small$\psi$}};
%
\draw (0,0) circle (1);
\draw[->, very thick] (0.100,1) -- (0.101,1) node[above] {}; 
\draw[->, very thick] (-0.100,-1) -- (-0.101,-1) node[below] {}; 
\fill (135:1) circle (2.5pt) node[left] {{\small$\varphi$}};
\end{tikzpicture} 
= 
\begin{tikzpicture}[very thick,scale=0.6,color=blue!50!black, baseline,>=stealth]
\draw (0,0) circle (2);
\draw[->, very thick] (-0.100,2) -- (-0.101,2) node[above] {}; 
\draw[->, very thick] (0.100,-2) -- (0.101,-2) node[below] {}; 
\fill (45:2) circle (2.5pt) node[right] {{\small$1\otimes\psi$}};
\fill (135:2) circle (2.5pt) node[left] {{\small$1\otimes\varphi$}};
\end{tikzpicture} 
= 
\str (\psi \circ (-) \circ \varphi ) \, .
\ee
Here the second step follows from the general calculus of bicategories with duals; see e.\,g.~\cite[Lemma~3.1(iii)]{cr1006.5609} for a detailed proof [TODO: only if we have pivotality! Need to prove that first!]. Comparing~\eqref{C1} and~\eqref{C2} we thus observe: 

\begin{theorem}
The Cardy condition holds in $\LG$: for 1-morphisms $X,Y \in \hmf( k[x_1,\ldots,x_n], W)$ and 2-morphisms $\varphi: X\lra X$, $\psi: Y\lra Y$ we have
$$
\str (\psi \circ (-) \circ \varphi )
= 
(-1)^{n+1\choose 2} \Res_{k[k]/k} \left[ \frac{\beta^X(\varphi) \, \beta^Y (\psi)}{\partial_1 W \ldots \partial_n W} \right] \, .
$$
\end{theorem}

TODO: The sign is wrong, it should only be $(-1)^n$. 

TODO: Mention HRR as special example, more discussion needed? 



\section{Duality via adjoint operators}\label{section:dualityadjointop}
% All rings are commutative by default unless they are called "associative"
% TODO match generality to dviat5

Let $k$ be a noetherian $\mathbb{Q}$-algebra and $R = k[x_1,\ldots,x_n]$. If $k$ is a field then we know from \cite{??,??} that for any polynomial $W$ with an isolated singularity, the Kapustin-Li formula gives a non-degenerate pairing on the mapping spaces of the triangulated category of matrix factorisations of $W$ over $R$. In this section we prove the analogue of this result for an arbitrary base $k$. In particular, the mapping spaces of the triangulated category will not be flat over $k$, so we work instead with dg-categories.

Given $W \in R$ and two finite rank matrix factorisations $X,Y$ of $W$ let $\Hom(X,Y) = \Hom_R(X,Y)$ denote the $\mathbb{Z}_2$-graded Hom-complex. We define a natural $k$-bilinear pairing
\[
\langle -, - \rangle: \Hom(X,Y) \otimes_k \Hom(Y,X) \lto k[n]
\]
by the following formula, where $\lambda_i = \partial_{x_i}d_Y, \omega = \ud x_1 \ldots \ud x_n$ and $f_i = \partial_{x_i} W$: 
\begin{equation}\label{eq:generalklpairing}
\langle \varphi, \psi \rangle = \frac{1}{n!} \sum_{\sigma \in S_n} (-1)^{|\sigma|} \Ress{R/k}\Bigg[ \frac{ \str( \varphi \circ \psi \circ \lambda_{\sigma(1)} \circ \cdots \circ \lambda_{\sigma(n)} )\, \omega }{ f_1,\ldots,f_n } \Bigg]\,.
\end{equation}
Such a pairing is \textsl{homotopically non-degenerate} if the adjoint morphism of $\mathbb{Z}_2$-graded complexes
\begin{align*}
\Hom(X,Y) & \lto \Hom_k( \Hom(Y,X), k )[n]\,,\\
\varphi & \lmt \langle \varphi, - \rangle
\end{align*}
is a homotopy equivalence over $k$. The main theorem here is that under some reasonable hypotheses the pairing $\langle -, - \rangle$ is homotopically non-degenerate. This non-degeneracy is used in Section~\ref{sec:Zorro} to prove that specified evaluation and coevaluation morphisms give rise to adjoint $1$-morphisms in the bicategory of Landau-Ginzburg models over $k$.

\begin{theorem}\label{theorem:generalkl} Suppose that the following conditions are satisfied
\begin{itemize}
\item[(H1)] The partial derivatives $f_i = \partial_{x_i} W$ form a regular sequence $f = \{f_1,\ldots,f_n\}$ in $R$.
\item[(H2)] $\bar{R} = R/(f_1,\ldots,f_n)R$ is a finitely generated projective $k$-module.
\item[(H3)] The $R$-linear map
\begin{align*}
\bar{R} & \lto \Hom_k( \bar{R}, k)\, , \\
r & \lmt \Ress{R/k}\Bigg[ \frac{r \cdot (-) \, \ud \underline{x}}{ f_1, \ldots, f_n} \Bigg]
\end{align*}
is an isomorphism.
\end{itemize}
Then the pairing $\langle -, - \rangle$ is homotopically non-degenerate.
\end{theorem}

There is also version when $X,Y$ are graded matrix factorisations; see Section \ref{section:gradedduality}. The proof of the theorem takes up the rest of this section. We argue that $\Hom(X,Y)[n]$ and $\Hom_k(\Hom(Y,X),k)$ are homotopy equivalent by showing that they split idempotents on
\[
\Hom(\bar{X},\bar{Y}) = \Hom(X,Y) \otimes_R \bar{R}
\]
and
\[
\Hom_k( \Hom(\bar{Y}, \bar{X}), k) = \Hom_k( \Hom(Y,X) \otimes_R \bar{R}, k )
\]
respectively, in the homotopy category of $\mathbb{Z}_2$-graded $k$-complexes. There is an isomorphism
\begin{equation}\label{eq:frobnondeg}
\Hom(\bar{X}, \bar{Y}) \cong \Hom_k( \Hom(\bar{Y}, \bar{X}), k)
\end{equation}
of complexes which identify these idempotents, and from this the theorem follows. The existence of the isomorphism (\ref{eq:frobnondeg}) is an easy consequence of the fact that $\bar{R}$ is Frobenius over $k$. The hard work lies in showing that this isomorphism identifies the two idempotents, and for this we need to carefully study adjointness between operators on dg-categories. Thus the first part of the proof, in Section \ref{section:adjointopdg}, consists in formalising this kind of adjointness.

In fact it will be convenient to prove the theorem slightly more generally, since in order to equip~$R$ with a connection we may need to pass to the $fR$-adic completion. Consequently, for the rest of this section $R$ is an arbitrary noetherian $k$-algebra, $W \in R$, and our actual hypotheses are:
\begin{itemize}
\item[(H1$'$)] $f = \{ f_1,\ldots,f_n \}$ is any regular sequence in $R$ with the property that multiplication by $f_i$ is a null-homotopic endomorphism of $X,Y \in \hmf(R,W)$ for $1 \le i \le n$.
\item[(H2$'$)] $\bar{R} = R/(f_1,\ldots,f_n)R$ is a finitely generated projective $k$-module.
\item[(H3$'$)] $R$ admits a flat $k$-linear connection
\[
\nabla: R \lto R \otimes_{k[f]} \Omega^1_{k[f]/k}
\]
as a $k[f]$-module, which is standard in the sense of \cite{??}.
\item[(H4$'$)] There is an $n$-form $\omega \in \Omega^n_{R/k}$ such that the $R$-linear map
\begin{align*}
\bar{R} & \lto \Hom_k( \bar{R}, k)\,\\
r & \lmt \Ress{R/k}\Bigg[ \frac{r \cdot (-) \cdot \omega}{ f_1, \ldots, f_n} \Bigg]
\end{align*}
is an isomorphism.
\end{itemize}
In this situation one can still write down the pairing (\ref{eq:generalklpairing}) and we prove that this is homotopically non-degenerate (Proposition \ref{prop:almostkltheorem}). The theorem follows by passing from $k[x_1,\ldots,x_n]$ to a completion, which satisfies (H3$'$) above even if the polynomial ring does not.

\subsection{Adjoint operators on dg-categories}\label{section:adjointopdg}
% TODO: Point out that these Atiyah classes are from connections over k[f], not k like in the associative case.

In this section $\otimes$ denotes $\otimes_k$. Let $\cat{C}$ be a $\mathbb{Z}_2$-graded dg-category over $k$ equipped with the data of a $k$-linear morphism of complexes 
\[
c_{XY}: \cat{C}(X,Y) \otimes \cat{C}(Y,X) \lto k
\]
for each pair of objects $X,Y$ in $\cat{C}$. 
When it is convenient we write $\langle \alpha, \beta \rangle$ for $c_{XY}(\alpha \otimes \beta)$. Throughout the differentials on the $\cat{C}(X,Y)$ and their tensor products are denoted $D$, with additional subscripts if necessary.

\begin{definition}\label{defn:nondegpair} We say that the pairing $\{ c_{XY} \}_{X,Y \in \cat{C}}$ is
\begin{itemize}
\item[(i)] \textsl{cyclic} if for all $X,Y$ the diagram
\[
\xymatrix{
\cat{C}(X,Y) \otimes \cat{C}(Y,X) \ar[dr]_{c_{XY}}\ar[rr]^\tau & & \cat{C}(Y,X) \otimes \cat{C}(X,Y) \ar[dl]^{c_{YX}}\\
& k
}
\]
commutes, where $\tau$ is the graded twist $\tau( \varphi \otimes \psi) = (-1)^{|\varphi||\psi|} \psi \circ \varphi$; 
\item[(ii)] \textsl{non-degenerate} if for all $X,Y$ the morphism
\begin{align*}
\zeta_{XY}: \cat{C}(X,Y) & \lto \Hom_k( \cat{C}(Y,X), k)\,,\\
\varphi & \lmt c_{XY}( \varphi \otimes - )
\end{align*}
is an isomorphism of complexes.
\end{itemize}
\end{definition}

%\begin{remark} \textbf{todo}. Probably this all works for a weaker form of non-degeneracy, where $\zeta$ is just a homotopy equivalence, but we don't need this. In what generality should we state things, then? Maybe we also only need cyclicity up to homotopy.
%\end{remark}

From now on we assume $\cat{C}$ is equipped with a cyclic non-degenerate pairing. In this section an \textsl{operator} is a closed homogeneous $k$-linear operator on some mapping complex $\cat{C}(X,Y)$ in $\cat{C}$, that is, a closed homogeneous element of the complex $\Hom_k( \cat{C}(X,Y), \cat{C}(X,Y) )$. We are interested in linear operators on the $\cat{C}(X,Y)$ and adjunctions between them, with respect to the pairing.

Recall that if $\Psi$ is homogeneous then $1 \otimes \Psi$ acts on tensors with Koszul signs.

\begin{definition}\label{defn:adjointop} An operator $\Phi$ on $\cat{C}(X,Y)$ is \textsl{adjoint} to an operator $\Psi$ on $\cat{C}(Y,X)$ if both~$\Phi$ and~$\Psi$ are homogeneous of the same degree and the diagram
\[
\xymatrix@C+1pc@R+0.5pc{
\cat{C}(X,Y) \otimes \cat{C}(Y,X) \ar[d]_{\Phi \otimes 1} \ar[r]^{1 \otimes \Psi} & \cat{C}(X,Y) \otimes \cat{C}(Y,X) \ar[d]^{c_{XY}}\\
\cat{C}(X,Y) \otimes \cat{C}(Y,X) \ar[r]_-{c_{XY}} & k
}
\]
commutes up to homotopy. Equivalently, there is a $k$-linear degree $|\Phi|+1$ morphism
\[
\mu: \cat{C}(X,Y) \otimes \cat{C}(Y,X) \lto k
\]
with the property that $[D, \mu] = c_{XY} \circ (1 \otimes \Psi) - c_{XY} \circ ( \Phi \otimes 1)$. Evaluated on homogeneous morphisms $\alpha, \beta$ this identity reads
\begin{equation}\label{eq:adjointopeq}
(-1)^{|\Phi|} \mu D(\alpha \otimes \beta) = (-1)^{|\alpha||\Psi|}\langle \alpha, \Psi(\beta) \rangle - \langle \Phi(\alpha) , \beta \rangle \, .
\end{equation}
\end{definition}

We will show in a moment that this type of adjointness is symmetric in $\Phi, \Psi$ so there is no need to distinguish between left and right adjoints. If $\Psi$ is a homogeneous operator on $\cat{C}(Y,X)$ then $\Psi^*$ is the operator on $\Hom_k(\cat{C}(Y,X),k)$ defined by $\Psi^*(f) = (-1)^{|f||\Psi|} f \circ \Psi$. 
%It is easy to see that: 

\begin{lemma} An operator $\Phi$ is adjoint to $\Psi$ if and only if the diagram
\begin{equation}\label{eq:lemadjointdia}
\xymatrix@C+2pc{
\cat{C}(X,Y) \ar[d]_\Phi \ar[r]^-{\zeta} & \Hom_k( \cat{C}(Y,X), k ) \ar[d]^{\Psi^*}\\
\cat{C}(X,Y) \ar[r]_-{\zeta} & \Hom_k( \cat{C}(Y,X), k)
}
\end{equation}
commutes up to homotopy.
\end{lemma}

\begin{lemma} An operator $\Phi$ is adjoint to $\Psi$ if and only if $\Psi$ is adjoint to $\Phi$.
\end{lemma}
\begin{proof}
By naturality of the graded twist $\tau$ and the cyclicity axiom for $c_{XY}$.
\end{proof}

\begin{lemma} Any operator $\Phi$ admits an adjoint, which is unique up to homotopy. 
\end{lemma}
\begin{proof}
This follows from commutativity of (\ref{eq:lemadjointdia}).
%If $\Phi$ is adjoint to $\Psi$ then from the previous lemma and (\ref{eq:lemadjointdia}) we deduce that $\Psi$ is homotopic to $\zeta^{-1} \circ \Phi^* \circ \zeta$ and from this both claims are clear.
\end{proof}

\begin{definition} Given an operator $\Phi$ we denote by $\Phi^{\dagger}$ the adjoint of $\Phi$.
\end{definition}

Since the adjoint is only well-defined up to homotopy it is implicit that in any identities involving the dagger notation we are working in the homotopy category of $\mathbb{Z}_2$-graded complexes. The following basic properties of adjoints are easily checked.

\begin{lemma} 
\begin{enumerate}
\item Let $\Phi_1, \Phi_2$ be operators on $\cat{C}(X,Y)$ and $\cat{C}(Y,Z)$ respectively. Then
\[
(\Phi_2 \circ \Phi_1)^{\dagger} = (-1)^{|\Phi_1||\Phi_2|} \Phi_1^{\dagger} \circ \Phi_2^{\dagger}\,.
\]
\item If $\Phi_1, \Phi_2$ are operators on $\cat{C}(X,Y)$ of the same degree, then $(\Phi_1 + \Phi_2)^{\dagger} = \Phi_1^{\dagger} + \Phi_2^{\dagger}$.
\end{enumerate}
\end{lemma}

So much for the general theory; from now on $\cat{C}$ denotes the dg-category whose objects are finite rank matrix factorisations of $W$ and whose mapping complexes are given by the quotients
\[
\cat{C}(X,Y) = \Hom(\bar{X}, \bar{Y}) = \Hom(X,Y) \otimes_R \bar{R}\,.
\]
While the ordinary dg-category of matrix factorisations only admits a pairing which is homotopically non-degenerate, this (linear) quotient $\cat{C}$ admits a cyclic non-degenerate pairing in the stronger sense explained above. To define it, let
\[
\langle - \rangle: R \lto k
\]
denote the $k$-linear map
\[
\langle r \rangle = \Ress{R/k}\Bigg[ \frac{r \cdot \omega}{ f_1, \ldots, f_n} \Bigg]\,.
\]
This annihilates the ideal $(f_1,\ldots,f_n)R$ and therefore factors via a $k$-linear map $\bar{R} \lto k$.

\begin{proposition}\label{prop:barnondeg} The pairing on $\cat{C}$ defined by
\begin{align*}
\langle -, - \rangle: \Hom(\bar{X}, \bar{Y}) \otimes \Hom(\bar{Y}, \bar{X}) \lto k\,, \qquad
\langle \varphi, \psi \rangle = \langle \str( \varphi \circ \psi ) \rangle
\end{align*}
is cyclic and non-degenerate in the sense of Definition \ref{defn:nondegpair}.
\end{proposition}
\begin{proof}
The pairing factors as
\begin{equation}\label{eq:barnondeg}
\xymatrix@C+1pc{
\Hom(\bar{X}, \bar{Y}) \otimes \Hom(\bar{Y}, \bar{X}) \ar[r]^-{-\circ-} & \Hom(\bar{Y}, \bar{Y}) \ar[r]^-{\str} & \bar{R} \ar[r]^-{\langle - \rangle} & k
}
\end{equation}
so it is clear that it is a closed $k$-linear map, and moreover cyclicity follows from the cyclicity of the supertrace. Non-degeneracy follows from hypothesis (H4$'$) and the following calculation, in which the first step is adjoint (\textbf{todo} not really!) to the composite of the first two maps in (\ref{eq:barnondeg})
\begin{align*}
\Hom_R(\bar{X}, \bar{Y}) &\cong \Hom_R( \Hom(\bar{Y}, \bar{X}), \bar{R} )\,\\
&\cong \Hom_R( \Hom(\bar{Y}, \bar{X}), \Hom_k( \bar{R}, k) )\\
&\cong \Hom_k( \Hom(\bar{Y}, \bar{X}), k )\,.
\end{align*}
\end{proof}

We have in mind two special classes of operators on the dg-category $\cat{C}$. The first arises because the partial derivatives $f_i=\partial_{x_i}W$ act as zero on the cohomology of $\cat{C}$ but via nonzero maps on the dg-level. In what follows let $X,Y$ denote finite rank matrix factorisations of $W$.

\begin{definition} Given a null-homotopy $\lambda_i$ for the action of $f_i$ on $Y$, that is, a map $\lambda_i \in \Hom(Y,Y)$ of degree one with $[D, \lambda_i] = f_i \cdot 1_Y$, we define the odd operator $\lambda_i^\bullet$ on $\Hom(\bar{X},\bar{Y})$ by
\[
\lambda_i^\bullet(\varphi) = \lambda_i \circ \varphi\,
\]
and the odd operator ${\lambda_i}_\bullet$ on $\Hom(\bar{Y}, \bar{X})$ by
\[
{\lambda_i}_\bullet(\varphi) = (-1)^{|\varphi|} \varphi \circ \lambda_i \,.
\]
\end{definition}

Observe that composition with $\lambda_i$ is not a closed map on $\Hom(X,Y)$ but is closed as an operator on $\Hom(\bar{X}, \bar{Y})$. These operators give the simplest example of an adjoint pair: 

\begin{lemma} The operator $\lambda_i^\bullet$ is adjoint to ${\lambda_i}_\bullet$.
\end{lemma}
\begin{proof}
The identity (\ref{eq:adjointopeq}) holds with $\mu = 0$, since
\begin{align*}
\langle {\lambda_i}_\bullet(\alpha), \beta \rangle &= \langle \str( {\lambda_i}_\bullet(\alpha) \circ \beta ) \rangle\\
&= (-1)^{|\varphi|} \langle \str( \alpha \circ \lambda_i \circ \beta ) \rangle\\
&= (-1)^{|\varphi|} \langle \alpha, \lambda_i^\bullet(\beta) \rangle\,.
\end{align*}
\end{proof}

The second class of operators are the components of the Atiyah class. Recall that by hypothesis (H2$'$) the ring $R$ admits a flat $k$-linear connection $\nabla$ as a $k[f]$-module. The $n$ components of this connection define $k$-linear operators $\partial_{f_i} = (\ud f_i)^* \circ \nabla$ on $R$ with the property that $[\partial_{f_i}, f_j] = \delta_{ij}$.

Any free $R$-module admits a $k$-linear connection over $k[f]$. For convenience choose homogeneous $R$-bases $\{ e_i \}_{i \in I}$ for $X$ and $\{ e_j \}_{j \in J}$ for $Y$ respectively. Then the maps $e_{ji} = e_j \circ e_i^*$ form an $R$-basis for $\Hom(X,Y)$ and the induced $k$-linear connection over $k[f]$ is defined by
\begin{align*}
\nabla = \nabla_{XY}: \Hom(X,Y) & \lto \Hom(X,Y) \otimes_{k[f]} \Omega^1_{k[f]/k}\,,\\
r e_{ji} & \lmt e_{ji} \otimes \nabla(r)\,.
\end{align*}
This connection has components $\partial_{f_i}$ defined by $\partial_{f_i}( r e_{ji} ) = \partial_{f_i}(r) \cdot e_{ji}$ which are $k$-linear operators on $\Hom(X,Y)$. The \textsl{Atiyah class} of $\Hom(X,Y)$ is the commutator
\[
\At = \At_{XY} = [D, \nabla] = D \circ \nabla - \nabla \circ D: \Hom(X,Y) \lto \Hom(X,Y) \otimes_{k[f]} \Omega^1_{k[f]/k}\,.
\]
The Atiyah class is $k[f]$-linear and is easily seen to be a morphism of complexes from $\Hom(X,Y)$ to the shift $\Hom(X,Y) \otimes_{k[f]} \Omega^1_{k[f]/k}[1]$. Moreover the homotopy class of this morphism is independent of the choice of connection, and in this sense choosing a basis for $X,Y$ is harmless. In terms of the components this means: 

\begin{definition} For $1 \le i \le n$ the components $\At_i = [D, \partial_{f_i}]$ of the Atiyah class define $k$-linear closed operators of degree one on $\Hom(\bar{X}, \bar{Y})$.
\end{definition}

For each pair of objects $X,Y$ in $\cat{C}$ we have defined a family of operators $\At_i$ on $\cat{C}(X,Y)$, defined using a choice of basis but independent of this choice up to homotopy. Conceptually, duality in the dg-category of matrix factorisations rests on the fact that these operators are anti-self-adjoint. To prove this we need the following Leibniz rule for Atiyah classes.

\begin{lemma}\label{lemma:leibnizatiyah} For homogeneous $\alpha \in \Hom(Y,Z)$ and $\beta \in \Hom(X,Y)$
\begin{equation}\label{eq:leibnizcompos}
\At_i( \alpha \circ \beta ) = \At_i(\alpha) \circ \beta + (-1)^{|\alpha|} \alpha \circ \At_i(\beta) + [g,D]( \alpha \otimes \beta)
\end{equation}
where $g( \alpha \otimes \beta ) = \partial_i(\alpha) \circ \beta + \alpha \circ \partial_i(\beta) - \partial_i( \alpha \circ \beta )$.
\end{lemma}
\begin{proof}
If $\nabla_{YZ}$ denotes the connection on $\Hom(Y,Z)$ and $\nabla_{XY}$ the connection on $\Hom(X,Y)$ then
\[
\nabla_{YZ,XY}(\alpha \otimes \beta) = \nabla_{YZ}(\alpha) \otimes \beta + \alpha \otimes \nabla_{XY}(\beta)
\]
defines a connection on $\Hom(Y,Z) \otimes_{k[f]} \Hom(X,Y)$. With $\At_{YZ,XY} = [D, \nabla_{YZ,XY}]$ it follows from naturality of the Atiyah class \cite{??} that the diagram
\[
\xymatrix{
\Hom(Y,Z) \otimes_{k[f]} \Hom(X,Y) \ar[d]_{\At_{YZ,XY}}\ar[r]^-{\kappa} & \Hom(X,Z) \ar[d]^{\At_{XZ}}\\
\Hom(Y,Z) \otimes_{k[f]} \Hom(X,Y) \otimes_{k[f]} \Omega^1_{k[f]/k} \ar[r]_-{\kappa} & \Hom(X,Z) \otimes_{k[f]} \Omega^1_{k[f]/k}
}
\]
commutes up to homotopy, where $\kappa$ is the composition map. Specifically, if $g = [\kappa, \nabla] = \kappa \circ \nabla_{YZ,XY} - \nabla_{XZ} \circ \kappa$ then a simple calculation using the graded Jacobi identity shows that
\[
\At_{XZ} \circ \kappa - \kappa \circ \At_{YZ,XY} = [g,D]\,.
\]
Applying both sides to $\alpha \otimes \beta$ and projecting to the $\ud f_i$ component yields (\ref{eq:leibnizcompos}).
\end{proof}

\begin{lemma}\label{lemma:stratiyahzero} For any $\alpha \in \Hom(X,X)$ we have $\str( \At_i( \alpha ) ) = 0$ in $R$.
\end{lemma}
\begin{proof}
Consider the diagram
\[
\xymatrix@C+1pc{
\Hom(X,X) \ar[d]_{\At_{XX}} \ar[r]^-{\str} & R \ar[d]^{\At = 0}\\
\Hom(X,X) \otimes_{k[f]} \Omega^1_{k[f]/k} \ar[r]_-{\str} & R \otimes_{k[f]} \Omega^1_{k[f]/k}\,.
}
\]
By the graded Jacobi identity $\str \circ \At_{XX} = [\str, \At] = [g,D]$ where $g: \Hom(X,X) \lto R \otimes_{k[f]} \Omega^1_{k[f]/k}$ is $g = [\str, \nabla]$. But from the way we have defined our connections, it is clear that $\str$ and $\nabla$ commute, so $g = 0$ and $\str \circ \At_{XX} = 0$.
\end{proof}

\begin{proposition} The operator $\At_i$ on $\Hom(\bar{X}, \bar{Y})$ is adjoint to $-\At_i$ on $\Hom(\bar{Y}, \bar{X})$.
\end{proposition}
\begin{proof}
If we apply $\langle \str( - ) \rangle$ to both sides of (\ref{eq:leibnizcompos}) and use Lemma \ref{lemma:stratiyahzero} we find that
\begin{align*}
\langle \At_i(\alpha), \beta \rangle = -(-1)^{|\alpha|} \langle \alpha,  \At_i(\beta) \rangle - \langle \str( gD(\alpha \otimes \beta) ) \rangle\,.
\end{align*}
So $\mu( \alpha \otimes \beta ) = \langle \str( g( \alpha \otimes \beta ) ) \rangle$ is a homotopy expressing $\At_i$ as adjoint to $-\At_i$.
\end{proof}

\subsection{Idempotents}\label{section:idempkl}

In the previous section we constructed operators $\lambda_i^\bullet, {\lambda_i}_\bullet$ and $\At_i$ on the dg-category $\cat{C}$. Now we use these operators to define idempotent endomorphisms of the complex
\[
\cat{C}(X,Y) = \Hom(\bar{X}, \bar{Y})
\]
which split in the homotopy category of $\mathbb{Z}_2$-graded $k$-complexes to give $\Hom(X,Y)$. In this way the dg-category of matrix factorisations can be recovered from the quotient $\cat{C}$ and the non-degenerate pairing defined above induces the homotopically non-degenerate pairing $\langle -, - \rangle$.

The main result of \cite{??} is that if $V$ is the free $k$-module on the basis $\theta_1,\ldots,\theta_n$ then there is a homotopy equivalence
\[
\Hom(\bar{X}, \bar{Y}) \cong \Hom(X,Y) \otimes_k \bigwedge V\,.
\]
There are consequently $2^n$ ways to embed $\Hom(X,Y)$ in the homotopy category of $k$-complexes as a direct summand in $\Hom(\bar{X}, \bar{Y})$. The ``top degree'' embedding, corresponding to the form $\theta_1 \ldots \theta_n$, is determined by the following idempotent endomorphism of $\Hom(\bar{X}, \bar{Y})$: 
\[
e = \frac{1}{(n!)^2} (-1)^{\binom{n}{2}}\sum_{\sigma,\tau \in S_n} (-1)^{|\sigma\tau|} \cdot \lambda_{\sigma(1)}^\bullet \cdots \lambda_{\sigma(n)}^\bullet \At_{\tau(1)} \cdots \At_{\tau(n)}\,.
\]
The embedding corresponding to the $0$-form $1$ in $\bigwedge V$ was not discussed in \cite{??} but we give the details in Appendix \ref{??}. The upshot is that this embedding is determined by the idempotent
\[
e' = \frac{1}{(n!)^2} (-1)^{\binom{n+1}{2}}\sum_{\sigma, \tau \in S_n} (-1)^{|\sigma\tau|} \cdot \At_{\tau(1)} \cdots \At_{\tau(n)} {\lambda_{\sigma(1)}}_\bullet \cdots {\lambda_{\sigma(n)}}_\bullet\,.
\]
To be precise, there is a diagram of degree zero $k$-linear morphisms of complexes
\[
\xymatrix@C+2pc{
\Hom(\bar{X}, \bar{Y}) \ar@<-0.5ex>[r]_-{\psi} & \Hom(X,Y)[n] \ar@<-0.5ex>[l]_-{\vartheta}
}
\]
with $\psi \circ \vartheta = 1$ and $\vartheta \circ \psi = e$ (equalities meaning equal up to $k$-linear homotopy) and a diagram
\[
\xymatrix@C+2pc{
\Hom(\bar{Y}, \bar{X}) \ar@<-0.5ex>[r]_-{\psi'} & \Hom(Y,X) \ar@<-0.5ex>[l]_-{\vartheta'}
}
\]
with $\psi' \circ \vartheta' = 1$ and $\vartheta' \circ \psi' = e'$. A concrete description of $\psi, \psi'$ is not important for us, but we will need to know that $\vartheta'$ is simply the quotient map, and that
\[
\vartheta = \frac{1}{n!} \sum_{\sigma \in S_n} (-1)^{|\sigma|} \lambda_{\sigma(1)}^\bullet \cdots \lambda_{\sigma(n)}^{\bullet}\,.
\]

\begin{proposition}\label{prop:eadjointeprime} The idempotent $e$ is adjoint to $(-1)^n e'$. Equivalently, the diagram
\begin{equation}\label{eq:eadjointeprime}
\xymatrix@C+2pc{
\Hom(\bar{X}, \bar{Y}) \ar[d]_-{e} \ar[r]^-{\zeta}_-{\cong} & \Hom_k(\Hom(\bar{Y},\bar{X}),k) \ar[d]^-{(-1)^n (e')^*}\\
\Hom(\bar{X}, \bar{Y}) \ar[r]_-{\zeta}^-{\cong} & \Hom_k( \Hom(\bar{Y}, \bar{X}), k )
}
\end{equation}
commutes up to homotopy.
\end{proposition}
\begin{proof} 
Observe that by the (anti-)self-adjointness established in the previous section
\begin{align*}
\left( \lambda_{\sigma(1)}^\bullet \cdots \lambda_{\sigma(n)}^\bullet \At_{\tau(1)} \cdots \At_{\tau(n)} \right)^{\dagger} &= (-1)^n \At^\dagger_{\tau(n)} \cdots \At^\dagger_{\tau(1)} (\lambda_{\sigma(n)}^\bullet)^\dagger \cdots (\lambda_{\sigma(1)}^\bullet)^{\dagger}\\
&= \At_{\tau(n)} \cdots \At_{\tau(1)} {\lambda_{\sigma(n)}}_{\bullet} \cdots {\lambda_{\sigma(1)}}_{\bullet}\,,
\end{align*}
from which it is immediate that $e^{\dagger} = (-1)^n e'$.
\end{proof}

\begin{proposition}\label{prop:almostkltheorem} The pairing (\ref{eq:generalklpairing}) is homotopically non-degenerate.
\end{proposition}
\begin{proof} Consider the diagram
\[
\xymatrix@C+2pc{
\Hom(\bar{X},\bar{Y}) \ar[d]_-{\zeta}^-{\cong} \ar@<-0.5ex>[r]_-{\psi} & \Hom(X,Y)[n] \ar@<-0.5ex>[l]_-{\vartheta} \ar@{.>}[d]^{\xi}\\
\Hom_k(\Hom(\bar{Y}, \bar{X},k) \ar@<-0.5ex>[r]_-{(\vartheta')^*} & \Hom_k( \Hom(Y,X), k) \ar@<-0.5ex>[l]_-{(\psi')^*}
}
\]
where $\xi = (\vartheta')^* \circ \zeta \circ \vartheta$. It is immediate from commutativity of~\eqref{eq:eadjointeprime} up to homotopy that $\xi$ is a homotopy equivalence with inverse $(-1)^n \psi \circ \zeta^{-1} \circ (\psi')^*$. To prove the theorem it only remains to observe that $\xi(\alpha)$ is the functional $\langle \alpha, - \rangle$. But
\begin{align*}
\xi(\alpha) &= (\vartheta')^* \zeta \vartheta( \alpha )\\
&= (\vartheta')^* \zeta\left( \frac{1}{n!} \sum_{\sigma \in S_n} (-1)^{|\sigma|} \lambda_{\sigma(1)} \circ \cdots \circ \lambda_{\sigma(n)} \circ \alpha \right)\\
&= \frac{1}{n!} (-1)^n \sum_{\sigma \in S_n} (-1)^{|\sigma|} \langle \str( \lambda_{\sigma(1)} \circ \cdots \circ \lambda_{\sigma(n)} \circ \alpha \circ (-) ) \rangle\\
&= \frac{1}{n!} \sum_{\sigma \in S_n} (-1)^{|\sigma|} \langle \str(  \alpha \circ (-) \circ \lambda_{\sigma(1)} \circ \cdots \circ \lambda_{\sigma(n)} ) \rangle\\
&= \langle \alpha, - \rangle
\end{align*}
which completes the proof.
\end{proof}

Let us now abandon the general setting, and return to the situation of the main theorem where $R = k[x_1,\ldots,x_n]$, and (H1)--(H3) are satisfied for $f_i = \partial_{x_i} W$ and $\omega = \ud x_1 \ldots \ud x_n$.

\begin{proof}[Proof of Theorem \ref{theorem:generalkl}] Set $I = (f_1,\ldots,f_n)R$ and let $\widehat{R}$ denote the $I$-adic completion of $R$. Then by \cite{??} this algebra admits a flat standard connection as a $k[f]$-module, and the other axioms among (H1$'$)--(H4$'$) follow for $\widehat{R}$ from our hypotheses about $R$ with $\omega = \ud x_1 \cdots \ud x_n$ (here we use that residues for $R/k$ and $\widehat{R}/k$ agree). Note that in (H1$'$) we only need that the $f_i$ act null-homotopically on the matrix factorisations
\[
\widehat{X} = X \otimes_R \widehat{R} \, , \qquad \widehat{Y} = Y \otimes_R \widehat{R}
\]
extended from $R$. Consider the diagram
\[
\xymatrix@C+2pc{
\Hom_R(X,Y) \ar[d]_{\can} \ar[r] & \Hom_k( \Hom_R(Y,X), k )[n]\\
\Hom_{\widehat{R}}(\widehat{X}, \widehat{Y}) \ar[r] & \Hom_k( \Hom_{\widehat{R}}(\widehat{X}, \widehat{Y}), k)[n] \ar[u]_{\can}
}
\]
where the columns are the canonical maps, and the rows are adjoint to the two versions of the pairing (\ref{eq:generalklpairing}). It is easy to see that this diagram commutes, and as the bottom row is a homotopy equivalence by Proposition \ref{prop:almostkltheorem}, to prove the theorem it is enough to argue that the canonical map
\[
\Hom_R(X,Y) \lto \Hom_{\widehat{R}}(\widehat{X}, \widehat{Y})
\]
is a homotopy equivalence of $\mathbb{Z}_2$-graded complexes over $k$. But this is a consequence of the general fact that the ``pushforward commutes with flat base change'', see \cite[Remark 7.7]{??}.
\end{proof}

\subsection{Graded duality}\label{section:gradedduality}

In this section we collect together the modifications necessary in order to make Theorem \ref{theorem:generalkl} compatible with an additional $\mathbb{Z}$-grading on matrix factorisations. Our conventions for graded modules and matrix factorisations are contained in Section \ref{??}.

Let $k$ be a noetherian $\mathbb{Q}$-algebra which is graded, and let $R = k[x_1,\ldots,x_n]$ be a graded ring in such a way that the structural map $k \lto R$ has degree zero. Let $W \in R$ be given, homogeneous of degree $|W| = 2c$ and let $X,Y$ be finite rank graded matrix factorisations of $W$ over $R$. Define
\[
a = \sum_{i=1}^n |x_i| - nc \, . 
\]
Then the graded analogue of the earlier theorem is:

\begin{theorem}\label{theorem:generalklgraded} 
Under the conditions (H1)--(H3) of Theorem \ref{theorem:generalkl} the pairing $\langle -, - \rangle$ is adjoint to a homotopy equivalence of $\mathbb{Z} \times \mathbb{Z}_2$-graded complexes over $k$
\[
\Hom(X,Y) \lto \Hom^{\operatorname{gr}}_k( \Hom(Y,X), k )[n](a)\,.
\]
\end{theorem}

Note that $\Hom(Y,X)$ is unlikely to be finitely generated over $k$, so there is a need to distinguish between $\Hom_k$ and $\Hom_k^{\operatorname{gr}}$. For the proof we need the following fact about residues.

\begin{lemma}\label{lemma:residuesarehomog} If $f_1,\ldots,f_n$ is a regular sequence in $R$ of homogeneous elements, then the map
\[
\Ress{R/k}\Bigg[ \frac{(-) \, \ud \underline{x}}{ f_1, \ldots, f_n} \Bigg]: R \lto k
\]
is homogeneous of degree $\sum_{i=1}^n( |x_i| - |f_i| )$.
\end{lemma}
\begin{proof}
(\textbf{todo}) Follows from the determinantal formula for residues in Lipman's book.
\end{proof}

\begin{proof}[Proof of Theorem \ref{theorem:generalklgraded}]
Consider the following diagram of $\mathbb{Z}_2$-graded complexes
\[
\xymatrix@C+2pc{
\Hom(X,Y) \ar@{.>}[dr]_-{\zeta_{\operatorname{gr}}} \ar[r]^-{\xi} & \Hom(\Hom(Y,X),k)[n]\\
& \Hom^{gr}_k( \Hom(Y,X), k)[n](a) \ar[u]_{\iota = \inc}
}
\]
where $\xi$ is adjoint to (\ref{eq:generalklpairing}), and is therefore a homotopy equivalence by the earlier theorem. It is clear from the explicit formula for the pairing and Lemma \ref{lemma:residuesarehomog} that the image of $\xi$ lies in the subspace of homogeneous maps, and we let $\zeta_{\operatorname{gr}}$ denote this factorisation. The value of $a$ is chosen such that $\zeta_{gr}$ is homogeneous of degree zero. It is enough to prove that $\zeta_{\operatorname{gr}}$ is a homotopy equivalence of $\mathbb{Z}_2$-graded complexes (forgetting the $\mathbb{Z}$-grading) since then by an elementary argument we can find a homotopy inverse which is degree zero, and graded homotopies.

It suffices to prove that the inclusion $\iota$ is a homotopy equivalence of $\mathbb{Z}_2$-graded complexes. This would be a tautology if $\Hom(Y,X)$ were a finitely generated $k$-module, and in general we use the fact that $\Hom(Y,X)$ embeds up to homotopy in a finitely generated complex. That is: by \cite[Section ??]{blah} we have a diagram
\begin{equation}\label{eq:generalklgraded}
\xymatrix@C+2pc{
\Hom(\bar{Y}, \bar{X}) \ar@<-0.5ex>[r]_-{\psi} & \Hom(Y,X)[n] \ar@<-0.5ex>[l]_-{\vartheta}
}
\end{equation}
of $\mathbb{Z}_2$-graded complexes over $k$ with $\psi \circ \vartheta \simeq 1$. But if one examines the explicit expressions for $\vartheta, \psi$ given in loc.\,cit.~it is clear that $\vartheta$ has degree $-a$, $\psi$ has degree $a$, and the homotopy between $\psi \circ \vartheta$ and the identity is homogeneous. If we apply both $\Hom_k(-,k)$ and $\Hom_k^{\operatorname{gr}}(-,k)$ to (\ref{eq:generalklgraded}) we obtain a diagram
\[
\xymatrix@C+2pc{
\Hom_k( \Hom(\bar{Y}, \bar{X}), k ) \ar@<-0.5ex>[r]_-{\vartheta^*} & \Hom_k(\Hom(Y,X),k)[n] \ar@<-0.5ex>[l]_-{\psi^*}\\
\Hom_k^{\operatorname{gr}}(\Hom(\bar{Y}, \bar{X}), k) \ar[u]^{=} \ar@<-0.5ex>[r]_-{\vartheta^*} & \Hom_k^{\operatorname{gr}}(\Hom(Y,X),k)[n](a) \ar@<-0.5ex>[l]_-{\psi^*} \ar[u]_{\iota}
}
\]
in which both of the implict squares commute up to homotopy, and from this we deduce that $\iota$ is a homotopy equivalence with inverse $\vartheta^* \circ \psi^*$.
\end{proof}


\appendix

\section{Symmetrised idempotents}\label{section:symidem}

In Section \ref{section:dualityadjointop} we used the idempotent pushforward construction of \cite{dm1102.2957} to establish a relative form of duality for matrix factorisations. We needed slightly more than what is stated in \textsl{loc.cit.} and in this appendix we provide the necessary additions. With later applications in mind we work in the same generality as the original construction, but the reader should keep in mind that for Section \ref{section:dualityadjointop} we only need the case $Z = \Hom_R(X,Y)$ and $W = 0$ of the following.

Let $k$ be a commutative $\mathbb{Q}$-algebra and $R$ a $k$-algebra with a quasi-regular sequence $\{ f_1, \ldots, f_n \}$ such that $\bar{R} = R/(f_1,\ldots,f_n)R$ is a finitely generated projective $k$-module. Let $W \in k$ be given and let $Z$ be a finite rank matrix factorisation of $W$ over $R$ with each $f_i$ acting null-homotopically on $Z$, and moreover let $\lambda_i \in \Hom_R(Z,Z)$ be odd maps with $[D, \lambda_i] = f_i \cdot 1_Z$ for each $1 \le i \le n$ where $D = d_Z$. Unless specified otherwise, $\otimes = \otimes_R$.

The aim is to write $Z$, up to $k$-linear homotopy equivalence, as the splitting of an idempotent on the quotient $\bar{Z} = Z \otimes \bar{R}$. Suppose that $R$ admits a flat $k$-linear connection $\nabla$ as a $k[f]$-module which is standard in the sense of \cite{??}. If $K(f) = K(f_1,\ldots,f_n)$ denotes the usual Koszul complex then by hypothesis the map $K(f) \lto \bar{R}$ is a quasi-isomorphism and by \cite[Section 10]{dm1102.2957} the projection
\begin{equation}\label{eq:symidem1}
\pi: Z \otimes K(f) \lto Z \otimes \bar{R} = \bar{Z}
\end{equation}
is a homotopy equivalence over $k$. 

As a graded $R$-algebra $K(f)$ is the exterior algebra $\bigwedge F$ where $F$ is $R$-free on symbols $\theta_1,\ldots,\theta_n$ placed in degree $-1$. Since the $f_i$ all act null-homotopically on $Z$ it is easy to see that $Z \otimes K(f)$ is homotopy equivalent over $R$ to the tensor product $Z \otimes \bigwedge F$, with no differential placed on the exterior component. This is just a direct sum of $2^n$ copies of $Z$ and $Z[1]$. In light of (\ref{eq:symidem1}) this means that in the homotopy category of linear factorisations of $W$ over $k$, there is an isomorphism
\[
\bar{Z} \cong Z \otimes_k \bigwedge (k\theta_1 \oplus \cdots \oplus k\theta_n)
\]
between $\bar{Z}$ and a direct sum of shifted copies of $Z$, and to this direct sum decomposition corresponds $2^n$ orthogonal $k$-linear idempotents on $\bar{Z}$. Our aim is to describe the idempotents corresponding to the top degree summand $Z \theta_1 \cdots \theta_n$ and the bottom degree summand $Z \cdot 1$.

To this end consider the case of a single $f_i$, so there is an exact sequence
\[
\xymatrix{
0 \ar[r] & Z \ar[r]^-{\kappa_i} & Z \otimes K(f_i) \ar[r]^-{\varepsilon_i} & Z\theta_i \ar[r] & 0
}
\]
where $Z\theta_i = Z[1]$, $\kappa_i(x) = x \otimes 1$ is the inclusion and $\varepsilon_i( x \theta_i ) = (-1)^{|x|} x$. This sequence is split exact; choosing a splitting is equivalent to choosing a null-homotopy on $X$ for the action of $f_i$. The maps
\[
\rho_i(x + y \theta_i) = x + (-1)^{|y|} \lambda_i(y), \qquad \vartheta_i(x) = (-1)^{|x|} x \theta_i - \lambda_i(x)
\]
satisfy $\kappa_i \circ \rho_i + \vartheta_i \circ \varepsilon_i = 1$ and $\rho_i \circ \kappa_i = 1, \varepsilon_i \circ \theta_i = 1$ and $\rho_i \circ \vartheta_i = 0$. That is, they equip $Z \otimes K(f_i)$ with the structure of a biproduct $Z \oplus Z\theta_i$. It will be convenient to have a more compact definition of $\vartheta_i$. Let us agree that left multiplication by $\theta_i$ on $X \otimes K(f)$ comes with signs $\theta_i \cdot (x \otimes \eta) = (-1)^{|x|} x \otimes \theta_i \eta$ so that multiplication by $\theta_i$ defines a map $\theta_i: X \lto X \otimes K(f_i)$ and $\vartheta_i = \theta_i - \lambda_i$.

To construct idempotents on $Z \otimes K(f)$ we can inductively apply either of the projections $\varepsilon$ or $\rho$ until we reach form degree zero, e.g. in the first step we can choose to ``keep'' $\theta_n$
\[
Z \otimes K(f_1,\ldots,f_n) \cong (Z \otimes K(f_1,\ldots,f_{n-1})) \otimes K(f_n) \xlto{\varepsilon} Z \otimes K(f_1,\ldots,f_{n-1})\theta_n\,,
\]
or we can choose to project out $\theta_n$ by applying $\rho$ instead. If after keeping $\theta_n$ we proceed to apply the $\varepsilon$ projections for the rest of the $\theta_i$, we obtain a split epimorphism $\varepsilon: Z \otimes K(f) \lto Z[n]$ defined by $\varepsilon(x \theta_1 \cdots \theta_n) = (-1)^{n|x|} x$ with left inverse $\vartheta$ defined by concatenating all the $\vartheta_i$. This left inverse depends on the order in which we project out the $\theta_i$, but if we symmetrise over all permutations we obtain the left inverse
\[
\vartheta = \frac{1}{n!} \sum_{\sigma \in S_n} (-1)^{|\sigma|} (\theta_{\sigma(1)} - \lambda_{\sigma(1)}) \circ \cdots \circ (\theta_{\sigma(n)} - \lambda_{\sigma(n)})
\]
to $\varepsilon$. One checks directly that $\varepsilon \circ \vartheta = 1$, so $\vartheta \circ \varepsilon$ is an idempotent on $Z \otimes K(f)$ splitting to $Z[n]$.

At the other extreme we can choose at each stage to project out the $\theta_i$ using $\rho$ to obtain a split epimorphism $Z \otimes K(f) \lto Z$ with left inverse the inclusion $\kappa(x) = x \otimes 1$. The precise formula for this epimorphism depends on the order in which we choose to project out the $\theta_i$. If we symmetrise over all permutations we arrive at the morphism
\[
\rho: Z \otimes K(f) \lto Z
\]
defined for $i_1 < \cdots < i_p$ by
\[
\rho( x \theta_{i_1} \cdots \theta_{i_p} ) = \frac{1}{n!} \sum_{\sigma \in S_n} (-1)^{p|x| + \binom{p}{2} + |\tau|} 
\]

So we have produced the two idempotents on $Z \otimes K(f)$ corresponding to top form degree and bottom form degree. Let us now shift these to idempotents on $\bar{Z}$. $e$ is taken care of in \cite{??} so we worry only about $e'$.
\[
\xymatrix{
\bar{Z} \ar@<-0.5ex>[r]_-{\sigma_\infty} & Z \otimes_R K(f) \ar@<-0.5ex>[l]_-{\pi} \ar@<-0.5ex>[r]_-{\rho} & Z \ar@<-0.5ex>[l]_-{\kappa}
}
\]
Here $\sigma_\infty$ is described by

\begin{lemma} We have
\[
\sigma_\infty = \sum_{m \ge 0} \sum_{\sigma \in S_{m,n}} (-1)^{\binom{m+1}{2}} \frac{1}{m!} \At_{\sigma(1)} \cdots \At_{\sigma(m)} \theta_{\sigma(1)} \cdots \theta_{\sigma(m)}
\]
\end{lemma}
\begin{proof}
By construction modulo the $f_i$ we have ($\nabla_{gr}$ has degree one for commutators)
\begin{align*}
\sigma_\infty &= \sum_{m \ge 0} (-1)^m (\tau^{-1} \nabla_{gr} D) \cdots (\tau^{-1} \nabla_{gr} D) \,\\
&= \sum_{m \ge 0} (-1)^m \frac{1}{m!} [ D, \nabla_{gr} ]^m \\
&= \sum_{m \ge 0} \sum_{\sigma \in S_{m,n}} (-1)^{\binom{m+1}{2}} \frac{1}{m!} \At_{\tau(1)} \cdots \At_{\tau(m)} \theta_{\sigma(1)} \cdots \theta_{\sigma(m)}
\end{align*}
\end{proof}

It follows that
\begin{align*}
e' &= \pi \circ \kappa \circ \rho \circ \sigma_\infty\\
&= \sum_{m \ge 0} \sum_{\sigma \in S_{m,n}} (-1)^{\binom{m+1}{2}} \rho \At_{\sigma(1)} \cdots \At_{\sigma(m)} \theta_{\sigma(1)} \cdots \theta_{\sigma(m)}
&= the good formula
\end{align*}



\newcommand{\etalchar}[1]{$^{#1}$}
\providecommand{\href}[2]{#2}
\begin{thebibliography}{FYH{\etalchar{+}}85}

%\bibitem[AS]{as1105.5117}
%M.~Aganagic and S.~Shakirov, \textsl{Knot {H}omology from {R}efined
%  {C}hern-{S}imons {T}heory},
%  \href{http://arxiv.org/abs/1105.5117}{[arXiv:1105.5117]}.
%
%\bibitem[BN]{bnKhovanov11crossings}
%D.~Bar-Natan, \textsl{Khovanov {H}omology for {K}nots and {L}inks with up to 11
%  {C}rossings}, available at
%  \href{http://www.math.toronto.edu/drorbn/papers/KHTables/KHTables.pdf}{http:%
%//www.math.toronto.edu/drorbn/papers/KHTables/KHTables.pdf}.
%
%\bibitem[Bec]{b1105.0702}
%H.~Becker, \textsl{Khovanov-Rozansky homology via Cohen-Macaulay approximations and Soergel bimodules},
%  \href{http://arxiv.org/abs/1105.0702}{[arXiv:1105.0702]}.
%
%\bibitem[BR07]{br0707.0922}
%I.~Brunner and D.~Roggenkamp, \textsl{B-type defects in {L}andau-{G}inzburg
%  models}, JHEP \textbf{0708} (2007), 093,
%  \href{http://arxiv.org/abs/0707.0922}{[arXiv:0707.0922]}.
%
%\bibitem[CF94]{cf9405183}
%L.~Crane and I.~B. Frenkel, \textsl{Four dimensional topological quantum field
%  theory, {H}opf categories, and the canonical bases}, J. Math. Phys.
%  \textbf{35} (1994), 5136--5154,
%  \href{http://arxiv.org/abs/hep-th/9405183}{[hep-th/9405183]}.
%
%\bibitem[CK08a]{ck0701194}
%S.~Cautis and J.~Kamnitzer, \textsl{Knot homology via derived categories of
%  coherent sheaves {I}, {$sl(2)$} case}, Duke Math. J. \textbf{142} (2008),
%  511--588, \href{http://arxiv.org/abs/math/0701194}{[math.AG/0701194]}.
%
%\bibitem[CK08b]{ck0710.3216}
%S.~Cautis and J.~Kamnitzer, \textsl{Knot homology via derived categories of coherent sheaves {II},
%  {$sl(m)$} case}, Invent. Math. \textbf{174} (2008), 165--232,
%  \href{http://arxiv.org/abs/math/0710.3216}{[math.AG/0710.3216]}.
%
%\bibitem[CM]{cmWebCompileCode}
%N.~Carqueville and D.~Murfet, \textsl{Code to compute {K}hovanov-{R}ozansky
%  homology and defect fusion in {L}andau-{G}inzburg models},
%  \href{http://www.carqueville.net/nils/webCompilations}{http://www.carqueville.net/nils/webCompilations}.

\bibitem[CR10]{cr0909.4381}
N.~Carqueville and I.~Runkel, \textsl{On the monoidal structure of matrix bi-factorisations}, J. Phys.
  A: Math. Theor. \textbf{43} (2010), 275401,
  \href{http://arxiv.org/abs/0909.4381}{[arXiv:0909.4381]}.

\bibitem[CR12]{cr1006.5609}
N.~Carqueville and I.~Runkel, \textsl{Rigidity and defect actions in
  Landau-Ginzburg models}, Comm. Math. Phys. \textbf{310} (2012) 135--179, 
  \href{http://arxiv.org/abs/1006.5609}{[arXiv:1006.5609]}.

%\bibitem[Cra]{c0403266}
%M.~Crainic, \textsl{On the perturbation lemma, and deformations},
%  \href{http://arxiv.org/abs/math/0403266}{[math.AT/0403266]}.
%
%\bibitem[DGR06]{dgr0505662}
%N.~M. Dunfield, S.~Gukov, and J.~Rasmussen, \textsl{The {S}uperpolynomial for
%  {K}not {H}omologies}, Experimental Math. \textbf{15} (2006), 129--159,
%  \href{http://arxiv.org/abs/math/0505662}{[math.GT/0505662]}.
%
%\bibitem[DKR]{dkr1107.0495}
%A.~Davydov, L.~Kong, and I.~Runkel, \textsl{Field theories with defects and the
%  centre functor}, \href{http://arxiv.org/abs/1107.0495}{[arXiv:1107.0495]}.
%
%\bibitem[DBM{\etalchar{+}}11]{dbmmss1106.4305}
%P.~Dunin-Barkowski, A.~Mironov, A.~Morozov, A.~Sleptsov, A.~Smirnov, \textsl{Superpolynomials for toric knots from evolution induced by cut-and-join operators},
%  \href{http://arxiv.org/abs/1106.4305}{[arXiv:1106.4305]}. 
%  
\bibitem[Dyc11]{d0904.4713}
T.~Dyckerhoff, \textsl{Compact generators in categories of matrix factorizations},
  Duke Math. J. \textbf{159} (2011), 223--274,
  \href{http://arxiv.org/abs/0904.4713}{[arXiv:0904.4713]}.

\bibitem[DM]{dm1102.2957}
T.~Dyckerhoff and D.~Murfet, \textsl{Pushing forward matrix factorisations},
  \href{http://arxiv.org/abs/1102.2957}{[arXiv:1102.2957]}.

%\bibitem[FFRS07]{ffrs0607247}
%J.~Fr\"ohlich, J.~Fuchs, I.~Runkel, and C.~Schweigert, \textsl{Duality and
%  defects in rational conformal field theory}, Nucl. Phys. B \textbf{763}
%  (2007), 354--430,
%  \href{http://arxiv.org/abs/hep-th/0607247}{[hep-th/0607247]}.
%
%\bibitem[FYH{\etalchar{+}}85]{Homfly}
%P.~Freyd, D.~Yetter, J.~Hoste, W.~B.~R. Lickorish, K.~Millett, and A.~Ocneanu,
%  \textsl{A new polynomial invariant of knots and links}, Bull. Amer. Math. Soc.
%  \textbf{12} (1985), 239--246.
%
%\bibitem[GIKV10]{gikv0705.1368}
%S.~Gukov, A.~Iqbal, C.~Koz\c{c}az, and C.~Vafa, \textsl{Link {H}omologies and the
%  {R}efined {T}opological {V}ertex}, Comm. Math. Phys. \textbf{298} (2010),
%  757--785, \href{http://arxiv.org/abs/0705.1368}{[arXiv:0705.1368]}.
%
%\bibitem[GSV05]{gsv0412243}
%S.~Gukov, A.~Schwarz, and C.~Vafa, \textsl{Khovanov-{R}ozansky {H}omology and
%  {T}opological {S}trings}, Lett. Math. Phys. \textbf{74} (2005), 53--74,
%  \href{http://arxiv.org/abs/hep-th/0412243}{[hep-th/0412243]}.
%
%\bibitem[GV99]{gv9811131}
%R.~Gopakumar and C.~Vafa, \textsl{On the {G}auge {T}heory/{G}eometry
%  {C}orrespondence}, Adv. Theor. Math. Phys. \textbf{3} (1999), 1415--1443,
%  \href{http://arxiv.org/abs/hep-th/9811131}{[hep-th/9811131]}.
%
%\bibitem[GW]{gw0512298}
%S.~Gukov and J.~Walcher, \textsl{Matrix {F}actorizations and {K}auffman
%  {H}omology}, \href{http://arxiv.org/abs/hep-th/0512298}{[hep-th/0512298]}.
%
%\bibitem[Jae]{j1101.3302}
%T.~C. Jaeger, \textsl{Khovanov-{R}ozansky {H}omology and {C}onway {M}utation},
%  \href{http://arxiv.org/abs/1101.3302}{[arXiv:1101.3302]}.
%
%\bibitem[Jon85]{JonesPolynomialPaper}
%V.~F.~R. Jones, \textsl{A polynomial invariant for knots via von {N}eumann
%  algebras}, Bull. Amer. Math. Soc. \textbf{12} (1985), 103--111.
%
%\bibitem[Kap]{k1004.2307}
%A.~Kapustin, \textsl{Topological {F}ield {T}heory, {H}igher {C}ategories, and
%  {T}heir {A}pplications},
%  \href{http://arxiv.org/abs/1004.2307}{[arXiv:1004.2307]}.
%
%\bibitem[Kaw96]{kawauchibook}
%A.~Kawauchi, \textsl{A {S}urvey of {K}not {T}heory}, Birkh\"auser, 1996.
%
%\bibitem[Kho00]{k9908171}
%M.~Khovanov, \textsl{A categorification of the {J}ones polynomial}, Duke Math. J.
%  \textbf{101} (2000), 359--426,
%  \href{http://arxiv.org/abs/math/9908171}{[math.QA/9908171]}.
%
%\bibitem[Kho07]{k0510265}
%M.~Khovanov, \textsl{Triply-graded link homology and Hochschild homology of Soergel bimodules},
%  Int. Journal of Math. \textbf{18} (2007), 869--885,
%  \href{http://arxiv.org/abs/math/0510265}{[math.GT/0510265]}.
%
%\bibitem[KR07a]{kr0701333}
%M.~Khovanov and L.~Rozansky, \textsl{Virtual crossings, convolutions and a
%  categorification of the {$\operatorname{SO}(2N)$} {K}auffman polynomial},
%  Journal of G\"okova Geometry Topology \textbf{1} (2007), 116--214,
%  \href{http://arxiv.org/abs/math/0701333}{[math.QA/0701333]}.
%
%\bibitem[KR07b]{kr0404189}
%M.~Khovanov and L.~Rozansky, \textsl{Topological Landau-Ginzburg models on the world-sheet foam},
%  Adv. Theor. Math. Phys. \textbf{11} (2007), 233--259,
%  \href{http://arxiv.org/abs/hep-th/0404189}{[hep-th/0404189]}.
%
%\bibitem[KR08a]{kr0401268}
%M.~Khovanov and L.~Rozansky, \textsl{Matrix factorizations and link homology}, Fund. Math.
%  \textbf{199} (2008), 1--91,
%  \href{http://arxiv.org/abs/math/0401268}{[math/0401268]}.
%
%\bibitem[KR08b]{kr0505056}
%M.~Khovanov and L.~Rozansky, \textsl{Matrix factorizations and link homology {II}}, Geometry \&
%  Topology \textbf{12} (2008), 1387--1425,
%  \href{http://arxiv.org/abs/math/0505056}{[math.QA/0505056]}.
%
%\bibitem[Ko]{Kock}
%J.~Kock, \textsl{Frobenius Algebras and 2D Topological Quantum Field Theories}, Cambridge University Press, 2003.
%
%\bibitem[Lam86]{LambekRingsModules}
%J.~Lambek, \textsl{Lectures on rings and modules}, AMS Chelsea Publishing, 1986.

\bibitem[LM]{Calinetal}
C.~I. Lazaroiu and D.~McNamee, unpublished.

\bibitem[Lo]{Loday}
J.-L. Loday, \textsl{Cyclic homology}, Springer, 1997.

%\bibitem[LMnV00]{lmv0010102}
%J.~M.~F. Labastida, M.~Mari\~{n}o, and C.~Vafa, \textsl{Knot {I}nvariants and
%  {T}opological {S}trings}, JHEP \textbf{0011} (2000), 007,
%  \href{http://arxiv.org/abs/hep-th/0010102}{[hep-th/0010102]}.
%
%\bibitem[MSV09]{msv0708.2228}
%M.~Mackaay, M.~Sto\v{s}i\'{c}, and P.~Vaz, \textsl{$\mathfrak{sl}(N)$ link homology ($N\geq 4$) using foams and the Kapustin-Li formula}, Geometry \& Topology \textbf{13} (2009), 1075--1128,
%  \href{http://arxiv.org/abs/0708.2228}{[arXiv:0708.2228]}.
%
%\bibitem[Man07]{m0601629}
%C.~Manolescu, \textsl{Link homology theories from symplectic geometry}, Adv. in
%  Math. \textbf{211} (2007), 363--416,
%  \href{http://www.arxiv.org/abs/math.AG/0601629}{[math.SG/0601629]}.

\bibitem[McN09]{McNameethesis}
D.~McNamee, \textsl{On the mathematical structure of topological defects in
  {L}andau-{G}inzburg models}, MSc Thesis, Trinity College Dublin, 2009.
  
\bibitem[Mu]{m0912.1629}
D.~Murfet, \textsl{Residues and duality for singularity categories of isolated {G}orenstein singularities},
  \href{http://arxiv.org/abs/0912.1629}{[arXiv:0912.1629]}.  

%\bibitem[Mn05]{marinoknotbook}
%M.~Mari\~{n}o, \textsl{Chern-{S}imons {T}heory, {M}atrix {M}odels, and
%  {T}opological s{t}rings}, Oxford University Press, 2005.
%
%\bibitem[MOY98]{moy1998}
%H.~Murakami, T.~Ohtsuki, and S.~Yamada, \textsl{Homfly polynomial via an
%  invariant of colored plane graphs}, Enseign. Math. \textbf{44} (1998),
%  325--360.
%
%\bibitem[MS]{ms0709.1971}
%V.~Manzorchuck and C.~Stroppel, \textsl{A combinatorial approach to functorial
%  quantum {$\mathfrak{sl}_k$} knot invariants},
%  \href{http://arxiv.org/abs/0709.1971}{[arXiv:0709.1971]}.
%
%\bibitem[OV00]{ov9912123}
%H.~Ooguri and C.~Vafa, \textsl{Knot {I}nvariants and {T}opological {S}trings},
%  Nucl. Phys. B \textbf{577} (2000), 419--438,
%  \href{http://arxiv.org/abs/hep-th/9912123}{[hep-th/9912123]}.

\bibitem[P10]{p0807.1471}
K.~Ponto, \textsl{Shadows and traces in bicategories}, 
Ast\'erisque, (333), 2010, 
\href{http://arxiv.org/abs/0807.1471}{[arXiv:0807.1471]}. 

\bibitem[PS]{ps0910.1306}
K.~Ponto and M.~Shulman, \textsl{Shadows and traces in bicategories}, 
\href{http://arxiv.org/abs/0910.1306}{[arXiv:0910.1306]}. 

\bibitem[PV]{pv1002.2116}
A.~Polishchuk and A.~Vaintrob, \textsl{Chern characters and {H}irzebruch-{R}iemann-{R}och formula for matrix factorizations}, 
\href{http://arxiv.org/abs/1002.2116}{[arXiv:1002.2116]}. 


%\bibitem[PT87]{HomflyPT}
%J.~Przytycki and P.~Traczyk, \textsl{Conway algebras and skein equivalence of
%  links}, Proc. Amer. Math. Soc. \textbf{100} (1987), 744--748.
%
%\bibitem[Ras]{r0607544}
%J.~Rasmussen, \textsl{Some differentials on {K}hovanov-{R}ozansky homology},
%  \href{http://arxiv.org/abs/math/0607544}{[math.GT/0607544]}.
%
%\bibitem[Ras07]{r0508510}
%J.~Rasmussen, \textsl{Khovanov-{R}ozansky homology of two-bridge knots and links},
%  Duke Math. J. \textbf{136} (2007), 551--583,
%  \href{http://arxiv.org/abs/math/0508510}{[math.GT/0508510]}.
%  
%\bibitem[Ric94]{rickard}
%J.~Rickard, \textsl{Translation functors and equivalences of derived categories for blocks of algebraic groups}, in “Finite dimensional algebras and related topics”, Kluwer (1994), 255-–264.
%
%\bibitem[Rou06]{RouquierMexico}
%R.~Rouquier, \textsl{Categorification of {$\mathfrak{sl}_{2}$} and braid groups},
%  Trends in representation theory of algebras and related topics (2006),
%  137--167.
%
%\bibitem[RT90]{RT1990}
%N.~Reshetikhin and V.~Turaev, \textsl{Ribbon graphs and their invariants derived
%  from quantum groups}, Comm. Math. Phys. \textbf{127} (190), 1--26.
%
%\bibitem[RT91]{RT1991}
%N.~Reshetikhin and V.~Turaev, \textsl{Invariants of 3-manifolds via link polynomials and quantum
%  groups}, Invent. Math. \textbf{103} (1991), 547--597.
%
%\bibitem[SS06]{ss0405089}
%P.~Seidel and I.~Smith, \textsl{A link invariant from the symplectic geometry of
%  nilpotent slices}, Duke Math. J. \textbf{134} (2006), 453--514,
%  \href{http://arxiv.org/abs/math/0405089}{[math.SG/0405089]}.

\bibitem[SW11]{sw0911.0917}
C.~V.~Shepler and S.~Witherspoon, \textsl{Quantum differentiation and chain maps of bimodule complexes}, Algebra and Number Theory \textbf{5}-3 (2011), 339--360, 
\href{http://arxiv.org/abs/0911.0917}{[arXiv:0911.0917]}. 

%\bibitem[Str05]{sCatTLcTCpf}
%C.~Stroppel, \textsl{Categorification of the {T}emperley-{L}ieb category,
%  tangles, and cobordisms via projective functors}, Duke Math. J. \textbf{126}
%  (2005), 547--596.
%
%\bibitem[Sus]{s0701045}
%J.~Sussan, \textsl{Category {$\mathcal O$} and {$\mathfrak{sl}_k$} link
%  invariants}, \href{http://arxiv.org/abs/math/0701045}{[math.QA/0701045]}.
%
%\bibitem[Tur88]{t1988YB}
%V.~Turaev, \textsl{The {Y}ang-{B}axter equation and invariants of links}, Invent.
%  Math. \textbf{92} (1988), 527--553.
%
%\bibitem[Tur10]{turaevbook}
%V.~Turaev, \textsl{Quantum invariants of knots and 3-manifolds}, de Gruyter, 2010,
%  2nd edition.
%
%\bibitem[Weba]{w0610650}
%B.~Webster, \textsl{Khovanov-Rozansky homology via a canopolis formalism},
%  \href{http://arxiv.org/abs/math/0610650}{[math.GT/0610650]}.
%
%\bibitem[Webb]{w1005.4559}
%B.~Webster, \textsl{Knot invariants and higher representation theory {II}: the
%  categorification of quantum knot invariants},
%  \href{http://arxiv.org/abs/1005.4559}{[arXiv:1005.4559]}.
%
%\bibitem[Wit]{w1101.3216}
%E.~Witten, \textsl{Fivebranes and {K}nots},
%  \href{http://arxiv.org/abs/1101.3216}{[arXiv:1101.3216]}.
%
%\bibitem[Wit89]{wittenjones}
%E.~Witten, \textsl{Quantum field theory and the {J}ones polynomial}, Comm. Math.
%  Phys. \textbf{121} (1989), 351--399.
%
%\bibitem[Wit95]{w9207094}
%E.~Witten, \textsl{Chern-{S}imons {G}auge {T}heory {A}s {A} {S}tring {T}heory},
%  Prog. Math. \textbf{133} (1995), 637--678,
%  \href{http://arxiv.org/abs/hep-th/9207094}{[hep-th/9207094]}.
%
%\bibitem[Wu]{w0907.0695}
%H.~Wu, \textsl{A colored {$\mathfrak{sl}(N)$}-homology for links in {$S^3$}},
%  \href{http://arxiv.org/abs/0907.0695}{[arXiv:0907.0695]}.
%
%\bibitem[Wu08]{w0508064}
%H.~Wu, \textsl{Braids, {T}ransversal links and the {K}hovanov-{R}ozansky {T}heory},
%  \href{http://arxiv.org/abs/math/0508064}{[math.GT/0508064]}, Trans. Amer. Math. Soc. \textbf{360}
%  (2008), 3365--3389.
%
%\bibitem[Yon]{y0906.0220}
%Y.~Yonezawa, \textsl{Quantum {$(\mathfrak{sl}_n, \wedge V_n)$} link invariant and
%  matrix factorizations},
%  \href{http://arxiv.org/abs/0906.0220}{[arXiv:0906.0220]}.

\end{thebibliography}

\end{document}
