% Authors:  Nils Carqueville, Daniel Murfet
 
\documentclass{compositio}
\usepackage{stmaryrd}
\usepackage{amsmath, amscd, amssymb, mathrsfs, accents, amsfonts}
\usepackage{url}
\usepackage[all]{xy}
\usepackage{longtable}
\usepackage{dsfont}
\usepackage{tikz}
\usetikzlibrary{decorations.pathmorphing}
\usetikzlibrary{calc}
\usetikzlibrary{decorations.markings}
\def\nicedashedcolourscheme{\shadedraw[top color=blue!22, bottom color=blue!22, draw=gray, dashed]}
\def\nicecolourscheme{\shadedraw[top color=blue!22, bottom color=blue!22, draw=white]}
\def\nicepalecolourscheme{\shadedraw[top color=blue!12, bottom color=blue!12, draw=white]}
\def\nicenocolourscheme{\shadedraw[top color=gray!2, bottom color=gray!25, draw=white]}
\def\nicereallynocolourscheme{\shadedraw[top color=white!2, bottom color=white!25, draw=white]}
\definecolor{Myblue}{rgb}{0,0,0.6}
\usepackage[a4paper,colorlinks,citecolor=Myblue,linkcolor=Myblue,urlcolor=Myblue,pdfpagemode=None]{hyperref}

\SelectTips{cm}{}

\newtheorem{theorem}{Theorem}[section]
\newtheorem{proposition}[theorem]{Proposition}
\newtheorem{lemma}[theorem]{Lemma}
\newtheorem{corollary}[theorem]{Corollary}
\newtheorem*{theoremn}{Theorem}

\theoremstyle{definition}
\newtheorem{definition}[theorem]{Definition}
\newtheorem{example}[theorem]{Example}
\newtheorem{remark}[theorem]{Remark}
\newtheorem{s}[theorem]{}
\newtheorem*{setup}{Setup}
\newtheorem*{propositionn}{Proposition}

\numberwithin{equation}{section}

% Operators
\def\eval{\operatorname{ev}}
\def\coev{\operatorname{coev}}
\def\res{\operatorname{Res}}
\def\sg{\operatorname{sg}}
\def\Inj{\operatorname{Inj}}
\def\inc{\operatorname{inc}}
\def\Proj{\operatorname{Proj}}
\def\Coker{\operatorname{Coker}}
\def\Ker{\operatorname{Ker}}
\def\Im{\operatorname{Im}}
\def\free{\operatorname{free}}
\def\can{\operatorname{can}}
\def\ac{\operatorname{ac}}
\def\HH{\operatorname{HH}}
\def\K{\mathbf{K}}
\def\D{\mathbf{D}}
\def\N{\mathbf{N}}
\def\sing{\operatorname{Sg}}
\def\Hom{\operatorname{Hom}}
\def\uHom{\underline{\Hom}}
\def\modd{\operatorname{mod}}
\def\Modd{\operatorname{Mod}}
\def\Grmodd{\operatorname{GrMod}}
\def\CM{\operatorname{CM}}
\def\Ker{\operatorname{Ker}}
\def\Spec{\operatorname{Spec}}
\def\straightK{\operatorname{K}}
\def\straightC{\operatorname{C}}
\def\holim{\operatorname{hocolim}}
\DeclareMathOperator{\Ext}{Ext}
\DeclareMathOperator{\coh}{coh}
\DeclareMathOperator{\serre}{S}
\DeclareMathOperator{\Flat}{Flat}
\DeclareMathOperator{\qc}{qc}
\DeclareMathOperator{\Perf}{Perf}
\DeclareMathOperator{\Map}{Map}
\DeclareMathOperator{\Qco}{Qco}
\DeclareMathOperator{\Tr}{Tr}
\DeclareMathOperator{\End}{End}
\DeclareMathOperator{\rank}{rank}
\DeclareMathOperator{\tot}{Tot}
\DeclareMathOperator{\skos}{K}
\DeclareMathOperator{\hht}{ht}
\DeclareMathOperator{\depth}{depth}
\DeclareMathOperator{\STr}{STr}
\DeclareMathOperator{\tr}{tr}
\DeclareMathOperator{\ch}{ch}
\DeclareMathOperator{\str}{str}
\DeclareMathOperator{\hmf}{hmf}
\DeclareMathOperator{\HMF}{HMF}
\DeclareMathOperator{\HF}{HF}
\DeclareMathOperator{\pr}{pr}
\DeclareMathOperator{\lAt}{lAt}
\DeclareMathOperator{\At}{At}
\DeclareMathOperator{\mff}{mf}
\DeclareMathOperator{\MF}{MF}
\DeclareMathOperator{\Sh}{Sh}
\DeclareMathOperator{\Der}{Der}

\begin{document}

% Commands
\def\globalsigneval{\binom{n}{2}}
\def\Atlarrow{\overset{\leftarrow}{\At}}
\def\Res{\res\!}
\newcommand{\cat}[1]{\mathcal{#1}}
\newcommand{\lto}{\longrightarrow}
\newcommand{\xlto}[1]{\stackrel{#1}\lto}
\newcommand{\mf}[1]{\mathfrak{#1}}
\newcommand{\md}[1]{\mathscr{#1}}
\newcommand{\intvar}{\bs{x}_{\textup{int}}}
\newcommand{\extvar}{\bs{x}_{\textup{ext}}}
\newcommand{\qderu}[2]{\mathbf{D}^{#1}(#2)}
\newcommand{\ud}{\mathrm{d}}
\def\l{\,|\,}
\def\cf{\boldsymbol{cf}}
\def\bx{\boldsymbol{x}}
\def\by{\boldsymbol{y}}
\def\ba{\boldsymbol{a}}
\def\bb{\boldsymbol{b}}
\def\totimes{\otimes}
\def\di{Q}
\newcommand{\cotimes}[1]{\,\widehat{\otimes}_{#1}\,}
\def\QQ{\mathds{Q}}
\def\krc{C}
\def\diffm{d}
\def\diffh{d_{\chi}}
\def\redh{\overline{H}}
\def\ZZ{\mathds{Z}}
\def\bs{\boldsymbol}
\def\Ztwo{\mathds{Z}_2}
\def\mdual{^{\vee}}
\def\KR{\operatorname{KR}}
\def\I{\!\operatorname{i}\!}
\def\E{\operatorname{e}\!}
\def\sln{\mathfrak{sl}(N)}
\def\nN{\mathds{N}}
\def\nZ{\mathds{Z}}
\def\nQ{\mathds{Q}}
\def\nR{\mathds{R}}
\def\nC{\mathds{C}}
\def\Bar{\mathds{B}}
\def\cBar{\widehat{\mathds{B}}}
\def\Se{S^{\operatorname{e}}}
\def\Re{R^{\operatorname{e}}}
\def\Ae{A^{\operatorname{e}}}
\def\Be{B^{\operatorname{e}}}
\def\Aop{A^{\operatorname{op}}}
\def\Rop{R^{\operatorname{op}}}
\def\lra{\longrightarrow}
\def\lmt{\longmapsto}
\def\LG{\mathcal{LG}_k}
\def\dual{\dagger}
\def\dlangle{\big\langle\!\big\langle}
\def\drangle{\big\rangle\!\big\rangle}
\def\bigdlangle{\Big\langle\!\!\Big\langle}
\def\bigdrangle{\Big\rangle\!\!\Big\rangle}
\def\reprod{\gamma}
\newcommand{\Ress}[1]{\res_{#1}\!}
\newcommand{\be}{\begin{equation}}
\newcommand{\ee}{\end{equation}}
\def\Xcirc{%
\begin{tikzpicture}[inner sep=0mm]
\node (X) at (0,0) {$X$};
\node (0) at (0,0) [circle,inner sep=0.99pt, thin,draw=black,fill= white] {};
\end{tikzpicture}%
}
\def\Xbul{%
\begin{tikzpicture}[inner sep=0mm]
\node (X) at (0,0) {$X$};
\node (0) at (0,0) [circle,inner sep=0.99pt, thin,draw=black,fill= black] {};
\end{tikzpicture}%
}

\renewcommand{\labelenumi}{(\roman{enumi})}

\allowdisplaybreaks

\usetikzlibrary{arrows,calc,decorations.pathreplacing,decorations.markings,shapes.geometric,shadows}
\tikzset{
    string/.style={draw=#1, postaction={decorate}, decoration={markings,mark=at position .51 with {\arrow[draw=#1]{>}}}},
    costring/.style={draw=#1, postaction={decorate}, decoration={markings,mark=at position .51 with {\arrow[draw=#1]{<}}}},
    ostring/.style={draw=#1, postaction={decorate}, decoration={markings,mark=at position .47 with {\arrow[draw=#1]{>}}}},
    ustring/.style={draw=#1, postaction={decorate}, decoration={markings,mark=at position .56 with {\arrow[draw=#1]{>}}}},
    oostring/.style={draw=#1, postaction={decorate}, decoration={markings,mark=at position .43 with {\arrow[draw=#1]{>}}}},
    uustring/.style={draw=#1, postaction={decorate}, decoration={markings,mark=at position .59 with {\arrow[draw=#1]{>}}}},
    directed/.style={string=blue!50!black}, 
    odirected/.style={ostring=blue!50!black}, 
    udirected/.style={ustring=blue!50!black}, 
    oodirected/.style={oostring=blue!50!black}, 
    uudirected/.style={uustring=blue!50!black},     
    redirected/.style={costring= blue!50!black},
}

\usetikzlibrary{fadings,decorations.pathreplacing}

\newcommand\pgfmathsinandcos[3]{%
  \pgfmathsetmacro#1{sin(#3)}%
  \pgfmathsetmacro#2{cos(#3)}%
}
\newcommand\LongitudePlane[3][current plane]{%
  \pgfmathsinandcos\sinEl\cosEl{#2} % elevation
  \pgfmathsinandcos\sint\cost{#3} % azimuth
  \tikzset{#1/.estyle={cm={\cost,\sint*\sinEl,0,\cosEl,(0,0)}}}
}
\newcommand\LatitudePlane[3][current plane]{%
  \pgfmathsinandcos\sinEl\cosEl{#2} % elevation
  \pgfmathsinandcos\sint\cost{#3} % latitude
  \pgfmathsetmacro\yshift{\cosEl*\sint}
  \tikzset{#1/.estyle={cm={\cost,0,0,\cost*\sinEl,(0,\yshift)}}} %
}
\newcommand\DrawLongitudeCircle[2][1]{
  \LongitudePlane{\angEl}{#2}
  \tikzset{current plane/.prefix style={scale=#1}}
  \pgfmathsetmacro\angVis{atan(sin(#2)*cos(\angEl)/sin(\angEl))} %
  \draw[redirected,current plane,color=blue!50!black, very thick] (\angVis:1) arc (\angVis:\angVis+180:1);
  \draw[current plane,dotted,color=blue!50!gray, very thick] (\angVis-180:1) arc (\angVis-180:\angVis:1);
}
\newcommand\DrawLatitudeCircle[2][1]{
  \LatitudePlane{\angEl}{#2}
  \tikzset{current plane/.prefix style={scale=#1}}
  \pgfmathsetmacro\sinVis{sin(#2)/cos(#2)*sin(\angEl)/cos(\angEl)}
  \pgfmathsetmacro\angVis{asin(min(1,max(\sinVis,-1)))}
  \draw[directed,current plane, color=blue!50!black] (\angVis:1) arc (\angVis:-\angVis-180:1);
  \draw[current plane,dashed, color=blue!50!gray] (180-\angVis:1) arc (180-\angVis:\angVis:1);
}
\newcommand\DrawLatitudeCircleU[2][1]{
  \LatitudePlane{\angEl}{#2}
  \tikzset{current plane/.prefix style={scale=#1}}
  \pgfmathsetmacro\sinVis{sin(#2)/cos(#2)*sin(\angEl)/cos(\angEl)}
  \pgfmathsetmacro\angVis{asin(min(1,max(\sinVis,-1)))}
  \draw[redirected,current plane, color=blue!50!black] (\angVis:1) arc (\angVis:-\angVis-180:1);
  \draw[current plane,dashed, color=blue!50!gray] (180-\angVis:1) arc (180-\angVis:\angVis:1);
}





\title{On signs}%
% Adjunctions and defects in Landau-Ginzburg models
%
% Other titles: 
% Rise of the Planet of the Coevaluations
% Adjoint defects in Landau-Ginzburg models
% Adjunction and defects in Landau-Ginzburg models
% Adjunctions and defects in Landau-Ginzburg models
% On adjunctions and defects in Landau-Ginzburg models

\author{Nils Carqueville}
\email{nils.carqueville@physik.uni-muenchen.de}
\address{Arnold Sommerfeld Center for Theoretical Physics, LMU M\"unchen \& Excellence Cluster Universe}

\author{Daniel Murfet}
\email{daniel.murfet@math.ucla.edu}
\address{Department of Mathematics, UCLA}

\classification{TODO}

\begin{abstract}
TODO: abstract
\end{abstract}

\maketitle

\section{Introduction}\label{sec:Introduction}

Landau-Ginzburg models play an important role in many areas of mathematical physics and pure mathematics. Among them are diverse fields such as singularity theory, representation theory, (homological) mirror symmetry, knot invariants, and conformal or topological field theory. These areas are intimately interrelated, and this richness is one of the aesthetic motivations for studying Landau-Ginzburg models. Another general motivation is their dual nature of affording insight into deep results while at the same time being concrete enough to allow for hands-on computations. 

In this paper we will show how this dichotomy manifests itself in the context of two-dimensional topological field theory (TFT) with defects. In short, we explain how Landau-Ginzburg models give rise to a bicategory with adjoints and we show that there is a simple description of the structure maps in this bicategory in terms of basic invariants called Atiyah classes. These structure maps are explicit enough that one can do nontrivial computations, but also conceptually elegant in the sense that working with diagrams in the bicategory reduces to a calculus of Atiyah classes.

To set the stage and provide some background, we recall a few details about TFTs with defects; for more detailed accounts we refer to~\cite{k1004.2307} and~\cite[Section~2]{dkr1107.0495}. We imagine the bulk sector theories~$T_I$ to ``live'' on a two-dimensional surface called the \textsl{worldsheet}. More precisely, the worldsheet may be partitioned into various domains to which the (not necessarily distinct) theories~$T_I$ are associated, and which are separated by one-dimensional oriented \textsl{defect lines} $D_\alpha$. A sketch of a typical such partitioned worldsheet looks as follows: 
\be\label{worldsheetwithdefects}
%%%%%%%%%%%%%%%%%%%%%%
\begin{tikzpicture}[very thick,scale=0.7,color=blue!50!black, baseline,>=stealth]
\clip (0,0) ellipse (6cm and 3cm);
\nicedashedcolourscheme (0,0) ellipse (6cm and 3cm);

\draw (-4.15,1.3) [white] node {{\scriptsize $T_1$}};
\draw (-5.1,-0.3) [white] node {{\scriptsize $T_2$}};
\draw (-1.3,2.2) [white] node {{\scriptsize $T_3$}};
\draw (2.2,2.0) [white] node {{\scriptsize $T_4$}};
\draw (1,0) [white] node {{\scriptsize $T_5$}};

\draw (-4.9,-1) node {{\scriptsize $D_1$}};
\draw (-4.7,0.8) node {{\scriptsize $D_2$}};
\draw (-2.5,1.3) node {{\scriptsize $D_3$}};
\draw (-3.6,-1.1) node {{\scriptsize $D_4$}};
\draw (-2.1,0.6) node {{\scriptsize $D_5$}};
\draw (3.5,1.8) node {{\scriptsize $D_6$}};
\draw (-0.4,0) node {{\scriptsize $D_7$}};
\draw (1.7,-1.7) node {{\scriptsize $D_8$}};

\draw (-4,0.45) node {{\scriptsize $\phi_1$}};
\draw (-2.1,-2) node {{\scriptsize $\phi_2$}};
\draw (-1.45,-2.15) node {{\scriptsize $\phi_3$}};
\draw (-0.62,-2.2) node {{\scriptsize $\phi_4$}};
\draw (4.5,0.84) node {{\scriptsize $\phi_5$}};
\draw (0.85,2.32) node {{\scriptsize $\phi_6$}};

\draw[->, very thick, out=60, in=260] (-5,-2) to (-4.5,-1);
\draw[very thick, out=80, in=220] (-4.5,-1) to (-4,0);
\filldraw (-4,0) circle (2.5pt);
\draw[->,very thick, out=150, in=350] (-4,0) to (-5,0.5);
\draw[very thick, out=170, in=350] (-5,0.5) to (-6,0.75);

\draw[->,very thick, out=-30, in=110] (-4,0) to (-3.2,-1);
\draw[very thick, out=-70, in=170] (-3.2,-1) to (-2.5,-2);
\filldraw (-2.5,-2) circle (2.5pt);

\draw[->,very thick, out=70, in=230] (-2.5,-2) to (-1.5,0.5);
\draw[very thick, out=50, in=200] (-1.5,0.5) to (1,2);
\filldraw (1,2) circle (2.5pt);

\draw[very thick, out=-30, in=180] (1,2) to (2,1.5);
\draw[->,very thick, out=0, in=270] (2,1.5) to (3,2);
\draw[very thick, out=90, in=0] (3,2) to (2,2.5);
\draw[very thick, out=180, in=30] (2,2.5) to (1,2);

\draw[very thick] (1,0) circle (1);
\draw[->,very thick, out=90, in=270] (0,0) to (0,0);

\filldraw (-1.5,-2.5) circle (2.5pt);

\filldraw (-0.5,-2.6) circle (2.5pt);
\draw[->,very thick, out=0, in=220] (-0.5,-2.6) to (2,-2);
\draw[very thick, out=40, in=260] (2,-2) to (4.5,0.5);
\filldraw (4.5,0.5) circle (2.5pt);

\draw[->,very thick, out=40, in=260] (-4,0) to (-3,1.5);
\draw[very thick, out=80, in=270] (-3,1.5) to (-2.5,3);
\end{tikzpicture}
%%%%%%%%%%%%%%%%%%%%%%
\ee
In addition to the labels $T_I$ for the two-dimensional domains and $D_\alpha$ for the one-dimensional defect lines, we also include labels~$\phi_i$ for zero-dimensional points. These labels are interpreted as describing ``degrees of freedom'', or \textsl{fields}, inserted at the points on the worldsheet. Note that the fields can also be placed at junctions of multiple defect lines. 

A TFT is a functor that assigns a number called the \textsl{correlator} to a labelled worldsheet like~\eqref{worldsheetwithdefects}. Its topological nature implies that the value of any correlator does not depend on the precise position of the $D_\alpha$'s and $\phi_i$'s, but only on their isotopy class. 

It is natural to try and make sense of this situation in terms of a bicategory: its objects are the theories $T_I$, 1-morphisms are labelled defect lines $D_\alpha$, and 2-morphisms are the fields $\phi_i$. The composition of 2-morphisms is the operator product of the fields, which is strictly associative because we only consider \textsl{topological} field theories. The composition of 1-morphisms comes about as follows: since the exact locus of the defect lines does not matter, two (or more) adjacent defect lines can be brought together arbitrarily close, and the limit of this \textsl{fusion} is well-defined and non-singular. The unit of the fusion product is the \textsl{invisible defect} which by definition leaves any other defect line invariant under fusion (from both left and right). 

Thus any TFT with defects is expected to be associated with a bicategory. Standard examples of bicategories include the bicategory of small categories (with objects, 1- and 2-morphisms given by categories, functors and natural transformations, respectively) or rings (rings, bimodules, bimodule maps), but these are too generic to describe TFTs. A rich and interesting example is the bicategory of algebraic varieties with Fourier-Mukai kernels as 1-morphisms~\cite{ct1007.2679} which describes B-twisted sigma models, see~e.\,g.~\cite{MB2}. 

The subject of this paper is the bicategory $\LG$ of Landau-Ginzburg models over a base ring $k$.\footnote{We will allow any commutative noetherian $\nQ$-algebra~$k$ for the coefficients of the polynomial ring, e.\,g.~$k=\nC$ or $k=\nC[t_1,\ldots,t_d]$ where the~$t_i$ may be interpreted as deformation parameters.} Roughly, its objects (or bulk theories) are given by elements (or \textsl{potentials})~$W$ in a polynomial ring $R=k[x_1,\ldots,x_n]$ which have isolated singularities, and its 1- and 2-morphisms (or defects and fields) are described by the triangulated categories of matrix factorisations of potential differences $V-W$. The fusion of 1-morphisms is the tensor product~\cite{br0707.0922}, and the invisible defect is the stabilised diagonal which we present as the Koszul factorisation $\Delta = \bigwedge (\bigoplus_{i=1}^n (R\otimes_k R)\cdot \theta_i)$ (see Section~\ref{sec:Background} for a detailed description). That $\LG$ is indeed a bicategory has been worked out in~\cite{McNameethesis, Calinetal, cr0909.4381}. 

On general grounds it is expected that the bicategorical description of TFTs with defects involves additional structure. For example, the fact that we consider \textsl{oriented} defect lines should give rise to adjunctions on the level of 1-morphisms: any defect line should be adjoint to the ``same'' defect line with reversed orientation. In the case of Landau-Ginzburg models this means that for every matrix factorisation~$X$ of $V-W$, viewed as a $1$-morphism between the bulk theories described by a potential $W \in k[x_1,\ldots,x_n]$ and $V \in k[z_1,\ldots,z_m]$, there should be a matrix factorisation~$X^\dual$ of $W-V$ together with evaluation and coevaluation maps
\be\label{evcoevintro}
\widetilde\eval_X: X \otimes_{k[x]} X^\dual \lra \Delta 
\, , \qquad
\widetilde\coev_X: \Delta \lra X^\dual \otimes_{k[z]} X
\ee
defining an adjunction between~$X$ and~$X^\dual$. Similarly, one expects a left adjoint~${}^\dual X$ of~$X$. The special case where~$V$ and~$W$ are the same polynomial and depend on only one variable was worked out in~\cite{cr1006.5609}; we refer to that paper's introductory section for further motivation from physics.

In the present paper we prove that every $1$-morphism $X$ in $\LG$ has left and right adjoints given by ${}^\dual X = X^{\vee}[m]$ and $X^\dual = X^{\vee}[n]$ where $X^{\vee} = \Hom_{k[x,z]}(X, k[x,z])$ is the dual factorisation. Our focus is on providing explicit expressions for the above evaluation and coevaluation maps, with the aim of making it simple and convenient to do algebra in the bicategory $\LG$, and in particular to evaluate correlators such as \eqref{worldsheetwithdefects}. 

To give a flavour of how explicit the formulas are: if $\{ e_i \}_i$ is a basis for the free $k[x,z]$-module $X$ with dual basis $\{ e_i^* \}_i$ then the coevaluation is defined by
\be\label{eq:formula_intro_coev}
\widetilde\coev_X( \gamma ) = \sum_{i,j} (-1)^{(r+1)|e_j| + s} \big\{ \partial_{[b_r]}(d_X) \ldots \partial_{[b_1]}(d_X) \big\}_{ji} \cdot e_i^* \otimes e_j\,.
\ee
Here $\gamma$ is an element of the exterior algebra $\Delta$ and $\gamma' = \theta_{b_1} \ldots \theta_{b_r}$ is the ``complement'' of $\gamma$, i.e. $b_1 < \cdots < b_l$ is the unique sequence with $\gamma \wedge \gamma' = (-1)^{s} \theta_1 \ldots \theta_n$ a scalar multiple of the top degree form, and $\partial_{[i]}$ is a divided difference operator, which applied to a polynomial $f(x, z)$ yields a polynomial
\[
\partial_{[i]} f(x,z) = \frac{ f - f|_{x_i \mapsto x'_i}}{x_i - x'_i}
\]
in the ring $k[x,z,x']$ with an additional set of variables $x' = x'_1, \ldots, x_n'$. To make sense of \eqref{eq:formula_intro_coev} we interpret the $x_i$ as acting on the left of the $k[x]$-bimodule $X^\dual \otimes_{k[z]} X$ and the $x'_i$ as acting on the right.

The other evaluation and coevaluation maps have similar elementary presentations, see Section \ref{??}, but for most computations a more sophisticated formula in terms of associative Atiyah classes is the most convenient.

We do so by developing a conceptual framework from which we obtain explicit expressions for the above adjunction maps, opening the door for many applications. The two key ingredients for our description are \textsl{homological perturbation} and the introduction of \textsl{associative Atiyah classes} to the world of matrix factorisations, following a suggestion by Buchweitz. In this way the adjunction structures can be organised in a rather clean and elegant way, the basic intuition behind which is roughly as follows: [TODO: some words on Atiyah classes, curvature, and all that] 

\cite{superquillen} and \cite[Appendix $1$]{arnold}. The most important invariants of a Riemannian manifold $M$ are the skew-symmetric curvature operators $F_{ij}$.

\begin{itemize}
\item What is a superconnection?
\item Curvature of a superconnection
\item 
\end{itemize}

Next we consider some examples of calculations which can be done with our explicit formulas.

Recall that a bulk field in a Landau-Ginzburg model with potential $W\in k[x]$ is described as an endomorphism of the stabilised diagonal $\Delta_W$, so the field is an element in the Jacobi algebra $\operatorname{Jac}(W) = k[x]/(\partial_{x_i} W)$. Given a matrix factorisation~$X$ of $V-W$, i.\,e.~a defect between the theories~$W$ and $V\in k[z_1,\ldots,z_m]$, we obtain an operator $\mathcal D_r(X)$ between the spaces of bulk fields by sending $\psi\in\operatorname{Jac}(W)$ to an element in $\operatorname{Jac}(V)$ obtained by ``wrapping the defect line labelled by~$X$ around~$\psi$''. We can make rigorous sense of this \textsl{defect action on bulk fields} in terms of string diagrams in the bicategory $\LG$ as 
\begin{align}\label{defectactionIntro}
\mathcal D_r(X)(\psi) = 
%%%%%%%%%%%%%%%%%%%%%%%%%%%%
\begin{tikzpicture}[very thick,scale=0.6,color=blue!50!black, baseline,>=stealth]
\nicepalecolourscheme (0,0) circle (3.5);
\fill (2.2,-2.2) circle (0pt) node[white] {{\small$V$}};
\nicecolourscheme (0,0) circle (2);
\fill (1.1,-1.1) circle (0pt) node[white] {{\small$W$}};
%
\draw (0,0) circle (2);
\draw[<-, very thick] (0.100,2) -- (-0.101,2) node[above] {}; 
\draw[<-, very thick] (-0.100,-2) -- (0.101,-2) node[below] {}; 
\fill (135:0) circle (3.3pt) node[right] {{\small$\psi$}};
\fill (130:2) circle (3.3pt) node[left] {{\small$\rho_X$}};
\fill (230:2) circle (3.3pt) node[left] {{\small$\rho^{-1}_{X}$}};
\draw[dashed] (135:0) .. controls +(0,1) and +(0.5,-1) .. (130:2);
\fill (-0.1,1.15) circle (0pt) node {{\small $\Delta_W$}};
\fill (-0.1,-1.15) circle (0pt) node {{\small $\Delta_W$}};
\draw[dashed] (135:0) .. controls +(0,-1) and +(0.5,1) .. (230:2);
\draw[dashed] (270:2) -- (270:3.3)
node[near end,right] {{{\small$\Delta_V$}}};
\draw[dashed] (90:2) -- (90:3.3)
node[near end,right] {{{\small$\Delta_V$}}};
\end{tikzpicture} 
%%%%%%%%%%%%%%%%%%%%%%%%%%%%
\equiv\;
%%%%%%%%%%%%%%%%%%%%%%%%%%%%
\begin{tikzpicture}[very thick,scale=0.6,color=blue!50!black, baseline,>=stealth]
\nicepalecolourscheme (0,0) circle (3.5);
\fill (2.2,-2.2) circle (0pt) node[white] {{\small$V$}};
\nicecolourscheme (0,0) circle (2);
\fill (1.1,-1.1) circle (0pt) node[white] {{\small$W$}};
%
\fill (180:2) circle (0pt) node[left] {{\small$X$}};
\draw (0,0) circle (2);
\draw[<-, very thick] (0.100,2) -- (-0.101,2) node[above] {}; 
\draw[<-, very thick] (-0.100,-2) -- (0.101,-2) node[below] {}; 
\fill (135:0) circle (3.3pt) node[left] {{\small$\psi$}};
\end{tikzpicture} 
%%%%%%%%%%%%%%%%%%%%%%%%%%%% 
\; ,
\end{align}
where as usual (and explained in more detail in Section~\ref{subsec:bicatLG}) evaluation and coevaluation maps are denoted as caps and cups, respectively, $\rho_X$ is the right action of $\Delta_W$ on~$X$, and we always read diagrams like the above from bottom to top. Thus~\eqref{defectactionIntro} equals $\widetilde\eval_X \circ (1_{X}\otimes (\rho_{X} \circ (1_X \otimes \psi)\circ \rho_{X}^{-1})) \circ \coev_X$, from which in Section~\ref{sec:defectaction} we will prove the general formula 
\be\label{DrXpsi}
\mathcal D_r(X)(\psi)  = (-1)^{{m\choose 2} + {n\choose 2}} \Res_{k[x,z]/k[z]} \left[ \frac{\psi \str\big(  \partial_{x_1} D\ldots \partial_{x_n} D \, \partial_{z_1} D\ldots \partial_{z_m} D \big) \underline{\operatorname{d}\! x}}{\partial_{x_1} W \ldots \partial_{x_n} W} \right]
\ee
as well as various properties such as $\mathcal D_r(X\otimes Y) = \mathcal D_r(X) \circ \mathcal D_r(Y)$. 

One may also consider the situation in~\eqref{defectactionIntro} with an additional defect field $\Phi\in\End(X)$ inserted on the $X$-loop. We will see that this simply amounts to the insertion of~$\Phi$ as a factor inside the supertrace in~\eqref{DrXpsi}. As the two special cases $V=0$ and $W=0$ we thus obtain
$$
\!\!
%%%%%%%%%%%%%%%%%%%%%%%%%%%%
\begin{tikzpicture}[very thick,scale=0.4,color=blue!50!black, baseline,>=stealth]
\nicecolourscheme (0,0) circle (2);
\fill (-1.0,-1.0) circle (0pt) node[white] {{\small$W$}};
%
\fill (180:1.8) circle (0pt) node[left] {{\small$X$}};
\fill (130:2) circle (3.3pt) node[left] {{\small$\Phi$}};
\draw (0,0) circle (2);
\draw[<-, very thick] (0.100,2) -- (-0.101,2) node[above] {}; 
\draw[<-, very thick] (-0.100,-2) -- (0.101,-2) node[below] {}; 
\fill (135:0) circle (3.3pt) node[left] {{\small$\psi$}};
\end{tikzpicture} 
%%%%%%%%%%%%%%%%%%%%%%%%%%%%
= 
(-1)^{{n\choose 2}} \Res_{k[x]/k} \!\!\left[ \frac{\psi \str\big( \Phi \Lambda_X^{(x)} \big) \underline{\operatorname{d}\! x}}{\partial_{x_1} W \ldots \partial_{x_n} W} \right] , 
\qquad
%%%%%%%%%%%%%%%%%%%%%%%%%%%%
\begin{tikzpicture}[very thick,scale=0.4,color=blue!50!black, baseline,>=stealth]
\nicepalecolourscheme (0,0) circle (3.5);
\fill (-2.1,-2.1) circle (0pt) node[white] {{\small$V$}};
\shadedraw[top color=white, bottom color=white, draw=white] (0,0) circle (2);
%
\fill (180:1.8) circle (0pt) node[left] {{\small$X$}};
\fill (130:2) circle (3.3pt) node[left] {{\small$\Phi$}};
\draw (0,0) circle (2);
\draw[<-, very thick] (0.100,2) -- (-0.101,2) node[above] {}; 
\draw[<-, very thick] (-0.100,-2) -- (0.101,-2) node[below] {}; 
\end{tikzpicture} 
%%%%%%%%%%%%%%%%%%%%%%%%%%%%
 = 
(-1)^{{m\choose 2}} \str\big( \Phi \Lambda_X^{(z)} \big)
$$
where $\Lambda_X^{(x)} = \partial_{x_1} D\ldots \partial_{x_n} D$ and $\Lambda_X^{(z)} = \partial_{z_1} D\ldots \partial_{z_m} D$. 
In this way we respectively recover the \textsl{Kapustin-Li disk correlator} and the \textsl{boundary-bulk map} (which reduces to the \textsl{Chern character} $(-1)^{{m\choose 2}} \str( \partial_{z_1} D\ldots \partial_{z_m} D)$ for $\Phi=1$). 

Another application of our construction of adjunctions in $\LG$ is a new proof of the \textsl{Cardy condition} (see Section~\ref{sec:ocTFT} for its precise statement). This generalisation of the Hirzebruch-Riemann-Roch theorem is the most ``quantum'' among the axioms for open/closed TFTs (as it stems from a one-loop diagram) and may accordingly be viewed as a particularly deep structure. In the case of Landau-Ginzburg models it was proved only recently in~\cite{pv1002.2116} and~\cite{dm1102.2957} for $k=\nC$ using rather heavy or technical machinery. Or proof works for any commutative noetherian $\nQ$-algebra~$k$ and simply follows from the fact that the 2-morphism in $\LG$ to be read off from the diagram 
$$
%%%%%%%%%%%%%%%%%%%%%%%%%%%%
\begin{tikzpicture}[very thick,scale=0.8,color=blue!50!black, baseline,>=stealth]
\nicecolourscheme (0,0) circle (2);
\nicereallynocolourscheme (0,0) circle (1);
\fill (1.5,0) circle (0pt) node[white] {{\small$W$}};
\draw (0,0) circle (2);
\draw[->, very thick] (-0.100,2) -- (-0.101,2) node[above] {}; %{{\small$\eval_Y$}}; 
\draw[->, very thick] (0.100,-2) -- (0.101,-2) node[below] {}; % {{\small$\widetilde\coev_Y$}}; 
\fill (45:2) circle (2.5pt) node[right] {{\small$\psi$}};
%
\draw (0,0) circle (1);
\draw[->, very thick] (0.100,1) -- (0.101,1) node[above] {}; % {{\small$\widetilde\eval_X$}}; 
\draw[->, very thick] (-0.100,-1) -- (-0.101,-1) node[below] {}; % {{\small$\coev_X$}}; 
\fill (135:1) circle (2.5pt) node[left] {{\small$\varphi$}};
\fill (180:0.9) circle (0pt) node[left] {{\small$X$}};
\fill (0:2.7) circle (0pt) node[left] {{\small$Y$}};
\end{tikzpicture} 
%%%%%%%%%%%%%%%%%%%%%%%%%%%%
$$
(which is to be identified with an annulus correlator) can be evaluated in two ways: either by first contracting the inner $X$-loop and then contracting the outer $Y$-loop, or by first fusing~$X$ with~$Y$ and then contracting the fused $(X^\dual \otimes Y)$-loop. Applying special cases of our expressions~\eqref{evcoevexplicit} then immediately produces the Cardy condition, see Theorem~\ref{thm:CardyCondition}. 

[TODO: mention new, more general Kapustin-Li proof here or somewhere else]

Let us conclude this introductory section by naming some further motivations for and future applications of the results presented in this paper. One of the most intriguing properties of Landau-Ginzburg models is that they are on one of the two sides of the \textsl{CFT/LG correspondence}. It roughly states that many aspects of a large class of conformal field theories (CFTs) can be described in terms of (non-conformal) Landau-Ginzburg models. CFTs have an extremely rich structure, and the subclass of \textsl{rational} CFTs are particularly well-understood. Starting with~\cite{tft1} their description in terms of a beautiful categorical theory has been developed, and one may wonder which structures encountered in rational CFT can also be found in the theory of Landau-Ginzburg models -- whether or not they correspond to a rational CFT. The complete answer to this question is not known, but at least all the formal properties of $\LG$ that we establish in this paper also hold for the bicategory of topological defects in rational conformal field theory. Of course this by no means exhausts the richness of rational CFT, and there remain many properties of Landau-Ginzburg models to be uncovered, inspired by the CFT/LG correspondence. 

One example is the \textsl{generalised orbifold} procedure of~\cite{ffrs0909.5013} which constructs all rational CFTs of fixed central charge and with identical left and right chiral algebras from any given single such CFT. Carried over to Landau-Ginzburg models this leads to the following picture: under the right circumstances a Landau-Ginzburg model with potential~$V$ can be obtained from a model with completely different potential~$W$ by identifying a 1-endomorphism~$F$ in $\LG(W,W)$ that can be equipped with the structure of a special symmetric Frobenius algebra (see e.\,g.~\cite[Section~3]{tft1}). Then the category of matrix factorisations of~$V$ is equivalent to the category of $F$-modules. The results of the present paper facilitate the construction of suitable algebras as follows: for any defect $X\in \LG(W,V)$ with invertible \textsl{quantum dimension} $\mathcal D_r(X)(1)$ we obtain a special symmetric Frobenius algebra $F = X^\dual \otimes X$ that allows to recast everything about theory~$V$ in terms of theory~$W$. Also note that with our explicit formula~\eqref{DrXpsi} the condition of invertibility of the quantum dimension is very easy to check in practice. The details of this construction will be described along with examples in the forthcoming paper~\cite{genorb}. 

[TODO: 2-representations]

The rest of the present paper is organised as follows. In Section~\ref{sec:Background} we collect necessary background material on matrix factorisations, bicategories with adjoints, noncommutative forms, residues, and homological perturbation theory. Section~\ref{section:atiyahclasses} introduces various notions of associative Atiyah classes, which together with homological perturbation allow us to invert or lift up to homotopy certain maps pertaining to the stabilised diagonal in Section~\ref{section:pertandhtpy}. Using these results we construct explicit evaluation and coevaluation maps in Section~\ref{sec:derivcoeval} and prove that they indeed endow the bicategory $\LG$ with left and right adjoints in Section~\ref{sec:Zorro}, and we discuss their relation in Section~\ref{sec:pivotality}. The next three sections discuss applications of this main result: defect action on bulk fields in Section~\ref{sec:defectaction}, open/closed TFT and in particular the Cardy condition in Section~\ref{sec:ocTFT}, and a bicategorical trace in Section~\ref{sec:shadows}. Finally, in Section~\ref{section:dualityadjointop} we provide a proof, similar in spirit to the construction of Section~\ref{sec:Zorro}, of the nondegeneracy of the Kapustin-Li pairing, for the general base~$k$ over which we work also in the bulk of the paper. 


\begin{acknowledgements}
TODO: especially Ragnar Buchweitz, also Ilka Brunner, Daniel Plencner, Ingo Runkel and who else? 
\end{acknowledgements}


\section{Background}\label{sec:Background}

Throughout rings are commutative and $k$ is a noetherian $\mathbb{Q}$-algebra, unless specified otherwise.

\subsection{Bicategories and adjunction}\label{subsec:bicat}

In this section we recall the basic theory of bicategories, with \cite{bor94} as our main reference. In the next section we introduce the main example of interest, the bicategory of Landau-Ginzburg models. The basic references for bicategories are \cite{benabou, gray, kellystreet} and for a useful survey see \cite{lack}. 

\begin{definition} A bicategory $\cat{B}$ consists of the following data:
\begin{itemize}
\item A class $|\cat{B}|$ of \textsl{objects}.
\item For each pair $A,B$ of objects a small category $\cat{B}(A,B)$ whose objects we call \textsl{$1$-morphisms} and whose arrows we call \textsl{$2$-morphisms}. Composition of $2$-morphisms is denoted $\gamma \circ \delta$.
\item For each triple $A,B,C$ of objects a functor
\[
c_{ABC}: \cat{B}(A,B) \times \cat{B}(B,C) \lto \cat{B}(A,C)\,.
\]
Given $1$-morphisms $f: A \lto B$ and $g: B \lto C$ we write $g \otimes f$ for their \textsl{composite} $c_{ABC}( f, g )$, and given $2$-morphisms $\gamma: f \lto f'$ and $\delta: g \lto g'$ we write $\delta \otimes \gamma$ for $c_{ABC}(\gamma, \delta)$.
\item For each object $A$ an identity $1$-morphism $\Delta_A: A \lto A$.
\item For each triple of composable $1$-morphisms $h, g, f$ a $2$-isomorphism
\[
\alpha_{f,g,h}: (h \otimes g) \otimes f \lto h \otimes (g \otimes f)
\]
natural with respect to $2$-morphisms in all three variables.
\item For each $1$-morphism $f: A \lto B$ a pair of $2$-isomorphisms
\begin{align*}
\lambda_f&: \Delta_B \otimes f \lto f\,,\\
\rho_f&: f \otimes \Delta_A \lto f
\end{align*}
natural with respect to $2$-morphisms in the variable $f$.
\end{itemize}
This data is subject to two coherence axioms, one involving the associator $\alpha$ and the other the left and right unit actions $\lambda, \rho$, see diagrams $(7.18)$ and $(7.19)$ of \cite{bor94}.
\end{definition}

The identity $2$-endomorphism of a $1$-morphism $f: A \lto B$ is denoted $1_f$ and the identity $2$-endomorphism of $\Delta_A$ is denoted $1_A$. For the remainder of this section, $\cat{B}$ denotes a bicategory.

We are interested in the relation of adjointness between $1$-morphisms in $\cat{B}$. In the bicategory of categories, functors and natural transformations, adjunction between functors can be expressed either in terms of a natural isomorphism of mapping spaces, or as an equation involving the units and counits of adjunction. In a general bicategory we define adjunction in terms of an equation; the equivalent description in terms of natural isomorphisms will not be used in the body of the text, but see Appendix \ref{app:adjoints}.

Our references for adjunction in bicategories are \cite[Chapter $6$]{gray} and \cite{kellystreet}.

\begin{definition}\label{def:adjointbicats} Two $1$-morphisms $f: A \lto B$ and $g: B \lto A$ are part of an \textsl{adjoint pair} when there exist $2$-morphisms
\be\label{evcoev}
\eval : g \otimes f \lra \Delta_A \, , \qquad \coev : \Delta_B \lra f \otimes g
\ee
satisfying 
\begin{align}
\rho_f \circ (1_f \otimes \eval) \circ \alpha_{f,g,f} \circ (\coev \otimes 1_f) \circ \lambda_f^{-1} & = 1_f \, , \label{uglyZorro1}\\
\lambda_{g} \circ (\eval \otimes 1_{g}) \circ \alpha^{-1}_{g,f,g} \circ (1_{g} \otimes \coev) \circ \rho_{g}^{-1} & = 1_{g} \, . \label{uglyZorro2}
\end{align}
Equivalently, the following two composites evaluate to the identity:
\begin{align}
\xymatrix@C+1.5pc
{
f \ar[r]^-{\lambda_f^{-1}} & \Delta_B \otimes f \ar[r]^-{\coev \otimes 1_f} & ( f \otimes g ) \otimes f \ar[r]^-{\alpha_{f,g,f}} & f \otimes ( g \otimes f ) \ar[r]^-{1_f \otimes \eval} & f \otimes \Delta_A \ar[r]^-{\rho_f} & f
}\,,\\
\xymatrix@C+1.5pc
{
g \ar[r]^-{\rho_g^{-1}} & g \otimes \Delta_B \ar[r]^-{1_g \otimes \coev} & g \otimes ( f \otimes g ) \ar[r]^-{\alpha^{-1}_{g,f,g}} & (g \otimes f) \otimes g \ar[r]^-{\eval \otimes 1_g} & \Delta_A \otimes g \ar[r]^-{\lambda_g} & g
}\,.
\end{align}
In this case we say that $g$ is \textsl{left adjoint} to $f$ and that $f$ is \textsl{right adjoint} to $g$, and we write $g \dashv f$. The $2$-morphisms $\eval$ and $\coev$ are referred to as the \textsl{evaluation} and \textsl{coevaluation} maps.
\end{definition}

\begin{definition}\label{def:bicatduals}
$\mathcal B$ \textsl{has left adjoints} (resp. \textsl{has right adjoints}) if every 1-morphism in $\cat{B}$ admits a left adjoint (resp. admits a right adjoint). If they exist these adjoints are unique up to isomorphism, and the unique left and right adjoints of $f$ are denoted by ${}^\dual f$ and $f^\dual$ respectively.
\end{definition}

If a $1$-morphism $f: A \lto B$ has both a left and right adjoint then we write the evaluation and coevaluation maps for the adjunction ${}^\dual f \dashv f$ with $f$ as a subscript, that is
\be\label{evcoev2}
\eval_f : {}^\dual f \otimes f \lra \Delta_A \, , \qquad \coev_f : \Delta_B \lra f \otimes {}^\dual f\,.
\ee
For the adjunction $f \dashv f^\dual$ we write the evaluation and coevaluation maps as
\be\label{evcoev3}
\widetilde\eval_f : f \otimes f^\dual \lra \Delta_B \, , \qquad \widetilde\coev_f : \Delta_A \lra f^\dual \otimes f\,.
\ee
It is convenient to denote identities in bicategories with adjoints in \textsl{string diagram} notation. We will explain a refined version of this notation in Section \ref{??}, but for the moment string diagrams have the meaning given in \cite{JSGoTCI,JSGoTCII} and also explained very clearly in \cite{ladia,khovdia}. In this language the evaluation and coevaluation maps~\eqref{evcoev2} are written as
$$
\eval_{f} = 
%%%%%%%%%%%%%%%%%%%%%%
\begin{tikzpicture}[very thick,scale=1.0,color=blue!50!black, baseline=.6cm]
\draw[line width=0pt] 
(2.5,1.6) node[line width=0pt] (I) {{\small$\Delta_A$}}
(3,0) node[line width=0pt] (D) {{\small $f\vphantom{f^\dual}$}}
(2,0) node[line width=0pt] (s) {\small{${}^\dual f$}}; 
\draw[directed] (D) .. controls +(0,1) and +(0,1) .. (s);
\draw[dashed] (2.5,0.81) -- (I);
\end{tikzpicture}
\equiv
\begin{tikzpicture}[very thick,scale=1.0,color=blue!50!black, baseline=.6cm]
\draw[line width=0pt] 
(3,0) node[line width=0pt] (D) {{\small$f\vphantom{f^\dual}$}}
(2,0) node[line width=0pt] (s) {{\small${}^\dual f$}}; 
\draw[directed] (D) .. controls +(0,1) and +(0,1) .. (s);
\end{tikzpicture}
%%%%%%%%%%%%%%%%%%%%%%
, 
\qquad
\coev_{f} = 
%%%%%%%%%%%%%%%%%%%%%%
\begin{tikzpicture}[very thick,scale=1.0,color=blue!50!black, baseline=-.6cm,rotate=180]
\draw[line width=0pt] 
(2.5,1.6) node[line width=0pt] (I) {{\small$\Delta_B$}}
(3,0) node[line width=0pt] (D) {{\small$f\vphantom{{}^\dual f}$}}
(2,0) node[line width=0pt] (s) {{\small${}^\dual f$}}; 
\draw[redirected] (D) .. controls +(0,1) and +(0,1) .. (s);
\draw[dashed] (2.5,0.81) -- (I);
\end{tikzpicture}
\equiv
\begin{tikzpicture}[very thick,scale=1.0,color=blue!50!black, baseline=-.6cm,rotate=180]
\draw[line width=0pt] 
(3,0) node[line width=0pt] (D) {{\small$f\vphantom{A^\dual}$}}
(2,0) node[line width=0pt] (s) {{\small${}^\dual f$}}; 
\draw[redirected] (D) .. controls +(0,1) and +(0,1) .. (s);
\end{tikzpicture}
%%%%%%%%%%%%%%%%%%%%%%
 \, . 
$$
Note that such diagrams are always to be read from bottom to top. Then the defining relations~\eqref{uglyZorro1} and~\eqref{uglyZorro2} translate into the \textsl{Zorro moves}
\be\label{Zorros}
\begin{tikzpicture}[very thick,scale=1.0,color=blue!50!black, baseline=0cm]
\draw[line width=0] 
(-1,1.25) node[line width=0pt] (A) {{\small $f$}}
(1,-1.25) node[line width=0pt] (A2) {{\small $f$}}; 
\draw[directed] (0,0) .. controls +(0,-1) and +(0,-1) .. (-1,0);
\draw[directed] (1,0) .. controls +(0,1) and +(0,1) .. (0,0);
\draw (-1,0) -- (A); 
\draw (1,0) -- (A2); 
\end{tikzpicture}
=
\begin{tikzpicture}[very thick,scale=1.0,color=blue!50!black, baseline=0cm]
\draw[line width=0] 
(0,1.25) node[line width=0pt] (A) {{\small $f$}}
(0,-1.25) node[line width=0pt] (A2) {{\small $f$}}; 
\draw (A2) -- (A); 
\end{tikzpicture}
\, , \qquad
\begin{tikzpicture}[very thick,scale=1.0,color=blue!50!black, baseline=0cm]
\draw[line width=0] 
(1,1.25) node[line width=0pt] (A) {{\small ${}^\dual f$}}
(-1,-1.25) node[line width=0pt] (A2) {{\small ${}^\dual f$}}; 
\draw[directed] (0,0) .. controls +(0,1) and +(0,1) .. (-1,0);
\draw[directed] (1,0) .. controls +(0,-1) and +(0,-1) .. (0,0);
\draw (-1,0) -- (A2); 
\draw (1,0) -- (A); 
\end{tikzpicture}
=
\begin{tikzpicture}[very thick,scale=1.0,color=blue!50!black, baseline=0cm]
\draw[line width=0] 
(0,1.25) node[line width=0pt] (A) {{\small ${}^\dual f$}}
(0,-1.25) node[line width=0pt] (A2) {{\small ${}^\dual f$}}; 
\draw (A2) -- (A); 
\end{tikzpicture} \, .
\ee

Similarly the evaluation and coevaluation maps \eqref{evcoev3} are written
$$
\widetilde\eval_{f} = 
%%%%%%%%%%%%%%%%%%%%%%
\begin{tikzpicture}[very thick,scale=1.0,color=blue!50!black, baseline=.6cm]
\draw[line width=0pt] 
(2.5,1.6) node[line width=0pt] (I) {{\small$1_B$}}
(3,0) node[line width=0pt] (D) {{\small $f^\dual\vphantom{f^\dual}$}}
(2,0) node[line width=0pt] (s) {\small{$f\vphantom{f^\dual}$}}; 
\draw[redirected] (D) .. controls +(0,1) and +(0,1) .. (s);
\draw[dashed] (2.5,0.81) -- (I);
\end{tikzpicture}
%%%%%%%%%%%%%%%%%%%%%%
\equiv
%%%%%%%%%%%%%%%%%%%%%%
\begin{tikzpicture}[very thick,scale=1.0,color=blue!50!black, baseline=.6cm]
\draw[line width=0pt] 
(3,0) node[line width=0pt] (D) {{\small$f^\dual$}}
(2,0) node[line width=0pt] (s) {{\small$f\vphantom{f^\dual}$}}; 
\draw[redirected] (D) .. controls +(0,1) and +(0,1) .. (s);
\end{tikzpicture}
%%%%%%%%%%%%%%%%%%%%%%
, 
\qquad
\widetilde\coev_{f} = 
%%%%%%%%%%%%%%%%%%%%%%
\begin{tikzpicture}[very thick,scale=1.0,color=blue!50!black, baseline=-.6cm,rotate=180]
\draw[line width=0pt] 
(2.5,1.6) node[line width=0pt] (I) {{\small$1_A$}}
(3,0) node[line width=0pt] (D) {{\small$f^\dual$}}
(2,0) node[line width=0pt] (s) {{\small$f\vphantom{f^\dual}$}}; 
\draw[directed] (D) .. controls +(0,1) and +(0,1) .. (s);
\draw[dashed] (2.5,0.81) -- (I);
\end{tikzpicture}
%%%%%%%%%%%%%%%%%%%%%%
\equiv
%%%%%%%%%%%%%%%%%%%%%%
\begin{tikzpicture}[very thick,scale=1.0,color=blue!50!black, baseline=-.6cm,rotate=180]
\draw[line width=0pt] 
(3,0) node[line width=0pt] (D) {{\small$f^\dual$}}
(2,0) node[line width=0pt] (s) {{\small$f\vphantom{f^\dual}$}}; 
\draw[directed] (D) .. controls +(0,1) and +(0,1) .. (s);
\end{tikzpicture}
%%%%%%%%%%%%%%%%%%%%%%
$$
satisfying the associated Zorro moves
\be\label{otherZorros}
%%%%%%%%%%%%%%%%%%%%%%
\begin{tikzpicture}[very thick,scale=1.0,color=blue!50!black, baseline=0cm]
\draw[line width=0] 
(1,1.25) node[line width=0pt] (A) {{\small $f$}}
(-1,-1.25) node[line width=0pt] (A2) {{\small $f$}}; 
\draw[redirected] (0,0) .. controls +(0,1) and +(0,1) .. (-1,0);
\draw[redirected] (1,0) .. controls +(0,-1) and +(0,-1) .. (0,0);
\draw (-1,0) -- (A2); 
\draw (1,0) -- (A); 
\end{tikzpicture}
%%%%%%%%%%%%%%%%%%%%%%
=
%%%%%%%%%%%%%%%%%%%%%%
\begin{tikzpicture}[very thick,scale=1.0,color=blue!50!black, baseline=0cm]
\draw[line width=0] 
(0,1.25) node[line width=0pt] (A) {{\small $f$}}
(0,-1.25) node[line width=0pt] (A2) {{\small $f$}}; 
\draw (A2) -- (A); 
\end{tikzpicture}
%%%%%%%%%%%%%%%%%%%%%%
\, , \qquad
%%%%%%%%%%%%%%%%%%%%%%
\begin{tikzpicture}[very thick,scale=1.0,color=blue!50!black, baseline=0cm]
\draw[line width=0] 
(-1,1.25) node[line width=0pt] (A) {{\small $f^\dual$}}
(1,-1.25) node[line width=0pt] (A2) {{\small $f^\dual$}}; 
\draw[redirected] (0,0) .. controls +(0,-1) and +(0,-1) .. (-1,0);
\draw[redirected] (1,0) .. controls +(0,1) and +(0,1) .. (0,0);
\draw (-1,0) -- (A); 
\draw (1,0) -- (A2); 
\end{tikzpicture}
%%%%%%%%%%%%%%%%%%%%%%
=
%%%%%%%%%%%%%%%%%%%%%%
\begin{tikzpicture}[very thick,scale=1.0,color=blue!50!black, baseline=0cm]
\draw[line width=0] 
(0,1.25) node[line width=0pt] (A) {{\small $f^\dual$}}
(0,-1.25) node[line width=0pt] (A2) {{\small $f^\dual$}}; 
\draw (A2) -- (A); 
\end{tikzpicture} \, .
%%%%%%%%%%%%%%%%%%%%%%
\ee
Note that in general there is no reason for the left and right adjoints to coincide.

\subsection{Bicategory of Landau-Ginzburg models}\label{subsec:bicatLG}

Next we define the bicategory $\LG$ of Landau-Ginzburg models over the base ring $k$. Recall that $k$ is a noetherian $\nQ$-algebra, with relevant examples being $k=\nC$ and $k=\nC[t_1,\ldots,t_d]$.

Objects of $\LG$ are pairs $(x, W)$ where $x = (x_1,\ldots,x_n)$ is an ordered sequence of variables and $W \in R = k[x_1,\ldots,x_n]$ is a \textsl{potential}, by which we mean that the Jacobi ring $R/(\partial_{x_1}W,\ldots,\partial_{x_n} W)$ is a finitely generated free $k$-module and the partial derivatives $\partial_{x_i}W$ form a regular sequence in $R$. If $k=\nC$ this simply means that the Jacobi ring is a finite-dimensional vector space. Typically we will simply write $k[x]$ for $k[x_1,\ldots,x_n]$. We will usually leave the chosen variable ordering implicit, and refer to objects of $\LG$ as pairs $(R, W)$ or even just a potential $W$ if this will not cause confusion.

The category $\LG(W, V)$ is defined in terms of matrix factorisations, which we now recall. Let $R = k[x_1,\ldots,x_n]$. A \textsl{linear factorisation} of $W\in R$ is a $\nZ_2$-graded $R$-module $X=X^0\oplus X^1$ together with an odd $R$-linear endomorphism~$d_X$ such that $d_X^2=W\cdot 1_X$. If~$X$ is a free $R$-module then the pair $(X,d_X)$ is called a \textsl{matrix factorisation}, and we often refer to it by~$X$ without explicitly mentioning the \textsl{differential} $d_X$; given a basis for~$X$ we sometimes identify the latter with the associated matrix: 
$$
d_{X} = \begin{pmatrix} 0 & d_X^1 \\ d_X^0 & 0\end{pmatrix} .
$$
Given two linear factorisations $X,Y$ of $W$, $\Hom_R(X,Y)$ is a $\mathbb{Z}_2$-graded complex with differential
\[
d( \varphi ) = d_Y \circ \varphi - (-1)^{|\varphi|} \varphi \circ d_X\,.
\]
A \textsl{morphism} of linear factorisations $(X,d_X)$ and $(Y,d_Y)$ is an even $R$-linear map $\varphi: X \longrightarrow Y$ such that $d_Y \varphi = \varphi d_X$. Two morphisms $\varphi, \psi: X\lra Y$ are \textsl{homotopic} if there exists an odd $R$-linear map $\lambda:X\lra Y$ such that $d_Y\lambda + \lambda d_X = \psi-\varphi$. Equality up to homotopy is an equivalence relation.

Given a linear factorisation $X$ of $W$ the \textsl{dual factorisation} $X^\vee = \Hom_R(X, R)$ is a linear factorisation of $-W$ with $d_{X^\vee}( \nu ) = -(-1)^{|\nu|} \nu \circ d_X$. In terms of matrices
\be\label{eq:differentials_adjoints}
d_{X^\vee} = \begin{pmatrix} 0 & (d_X^0)^\vee \\ -(d_X^1)^\vee & 0\end{pmatrix}\,.
\ee

The \textsl{(homotopy) category of linear factorisations} $\HF(R,W)$ is the category of linear factorisations of $W\in R$ modulo homotopy relations. We denote by $\HMF(R,W)$ its full subcategory of matrix factorisations, and we write $\hmf(R,W)$ for the full subcategory of \emph{finite rank matrix factorisations}, i.e. the matrix factorisations $X$ whose underlying $R$-module is free of finite rank. These three categories have standard triangulated structures whose shift functor we denote as~$[1]$. 

Since we work with polynomials rather than power series, $\hmf(R,W)$ is not necessarily idempotent complete; for a simple example of this phenomenon see \cite{??}. However $\HMF(R,W)$ has arbitrary coproducts and is therefore idempotent complete, and we denote by $\hmf(R,W)^\omega$ the idempotent closure of $\hmf(R,W)$ in this larger triangulated category. More concretely, this idempotent closure is the full subcategory of $\HMF(R,W)$ whose objects are those matrix factorisations $Y$ which are direct summands of finite rank matrix factorisations (in the homotopy category). This is an idempotent complete triangulated category. 

For objects $(R = k[x], W)$ and $(S = k[z], V)$ of $\LG$ we define
\be
\LG( (R, W), (S,V) ) = \hmf( R \otimes_k S, V - W )^\omega = \hmf( k[x,z], V - W)^\omega\,.
\ee
The reason for taking idempotent completions is that, as we will see in a moment, when composing $1$-morphisms we will find that the resulting matrix factorisation is not \emph{a priori} finite rank, only a summand in the homotopy category of something finite rank. There seem to be two natural ways to resolve this: work throughout with power series rings and completed tensor products, or work with idempotent completions. The latter seems less technical, and has the advantage that the formalism also applies to graded rings and matrix factorisations.

In $\LG$ we compose $1$-morphisms using tensor products. For our purposes it will be sufficient to consider matrix factorisations $X\in \HMF(R_1\otimes_k R_2, W_2-W_1)$ and $Y\in \HMF(R_2\otimes_k R_3, W_3-W_2)$ (where $W_i \in R_i$ and of course $R_i = k$ is a possibility for any~$i$). Then the \textsl{tensor product matrix factorisation} $Y\otimes_{R_2} X \in \HMF(R_1\otimes_k R_3, W_3-W_1)$ is the $\nZ_2$-graded module 
$$
Y\otimes_{R_2} X = \Big( (Y^0\otimes_{R_2} X^0) \oplus (Y^1\otimes_{R_2} X^1) \Big) \oplus \Big( (Y^0\otimes_{R_2} X^1) \oplus (Y^1\otimes_{R_2} X^0) \Big) 
$$
together with the differential
$$
d_{Y\otimes X} = d_Y \otimes 1 + 1 \otimes d_X
$$
where the second term comes with the usual Koszul signs when applied to elements. Notice that even if $X,Y$ are finite rank over $R_1 \otimes_k R_2$ and $R_2 \otimes_k R_3$ respectively, the tensor product $Y \otimes_{R_2} X$ is of \emph{infinite} rank over $R_1 \otimes_k R_3$ whenever $R_2 \neq k$. We observe however that e.\,g.~by the argument of~\cite[Section~12]{dm1102.2957} the tensor product $Y \otimes_{R_2} X$ is still a direct summand in the homotopy category of something finite rank, i.e. we may define
\[
Y \circ X = Y \otimes_{R_2} X \in \hmf( R_1 \otimes_k R_3, W_3 - W_1 )^\omega = \LG( W_1, W_3 )\,.
\]
Letting the tensor product act in the obvious way on morphisms this defines a functor
\be\label{eq:backgroundcomptensor}
c_{W_1, W_2, W_3}: \LG( W_1, W_2 ) \times \LG( W_2, W_3 ) \lto \LG( W_1, W_3 ), \qquad c( X, Y ) = Y \otimes_{R_2} X\,.
\ee
Also note that there are obvious natural isomorphisms $\alpha_{X,Y,Z}: (X\otimes Y)\otimes Z \lra X\otimes (Y \otimes Z)$. In the rest of the paper we drop the ``$\circ$'' notation for composition and simply write tensor products.

Finally, we discuss the unit $1$-morphisms $\Delta_W$ and the left and right unit actions $\lambda, \rho$. Given $W\in R=k[x]$ there is always the \textsl{unit matrix factorisation} $\Delta_W \in \hmf(\Re, \widetilde W)$ where $\Re = R\otimes_k R$ and $\widetilde W = W\otimes 1 - 1\otimes W$. Introducing~$n$ formal symbols $\theta_i$ as a convenient notational device, we have 
\be\label{DeltaW}
\Delta_W = \bigwedge \Big( \bigoplus_{i=1}^n \Re \theta_i \Big)
\ee
as an $\Re$-module whose $\nZ_2$-grading is given by $\theta$-degree modulo~2. Typically we will omit the wedge product and write  e.\,g.~$\theta_i\wedge \theta_j$ simply as $\theta_i \theta_j$. To describe the differential $d_{\Delta_W}$ we further need the variable-changing map
$$
{}^{t_i}(-): k[x,x'] \lra k[x,x'] \, , \qquad f \lmt f\big|_{x_i\lmt x'_i}
$$
in terms of which we can define difference quotient operators
\be\label{diffquotop}
\partial_{[i]}: k[x,x'] \lra k[x,x'] \, , \qquad f \lmt \frac{{}^{t_1\ldots t_{i-1}}f - {}^{t_1\ldots t_i}f}{x_i-x'_i} \, . 
\ee
It is easy to check that the $\partial_{[i]}$ satisfy the following kind of Leibniz rule: 
\begin{lemma}\label{lem:LeibnizForDQO}
For $f,g \in k[x,x']$ we have $\partial_{[i]}(fg) = (\partial_{[i]}f) ({}^{t_1\ldots t_{i}}g) + ({}^{t_1\ldots t_{i-1}}f) (\partial_{[i]}g)$. 
\end{lemma}

The differential on $\Delta_W$ is then given by (\textbf{todo} contraction)
\be\label{DeltaW}
d_{\Delta_W} = \delta_+ + \delta_- \, , \qquad \delta_+ = \sum_{i=1}^n \partial_{[i]}W\cdot \theta_i\, , \qquad \delta_- =  \sum_{i=1}^n (x_i-x'_i) \cdot \theta_i^*  \,. 
\ee
We call $\Delta_W$ the unit matrix factorisation as it is the unit with respect to the tensor product of matrix factorisations. It is also referred to as the \textsl{stabilised diagonal} \cite{d0904.4713} or \textsl{Koszul model} of the diagonal since the morphism of linear factorisations of $\widetilde W$
\be\label{DeltaWstabmap}
\pi: \Delta_W \lra R\,,
\ee
given by the projection $\Delta_W \lra \Re$ to the $\theta$-degree~$0$ component composed with multiplication $\Re \lra R$, is universal in the homotopy category of linear factorisations among all morphisms from finite rank matrix factorisations to $R$.

By making a choice of how to order the~$\theta_i$ we obtain an $\Re$-linear map 
\be\label{eq:vareps}
\varepsilon: \Delta_W \lra \Re[n] 
\, , \qquad 
\theta_1\ldots \theta_n \lmt 1
\ee
that is non-zero only on elements in top $\theta$-degree.

For $X\in \HMF(R_1 \otimes_k R_2,W_2-W_1)$ there are natural maps
\be\label{lambdarho}
\lambda_X = \pi \otimes 1_X: \Delta_{W_2} \otimes_{R_2} X \lra X \, , \qquad \rho_X = 1_X \otimes \pi : X \otimes_{R_1} \Delta_{W_1} \lra X
\ee
which are isomorphisms in $\HMF(R_1 \otimes_k R_2,W_2-W_1)$. These morphisms give the left and right unit actions in the bicategory of Landau-Ginzburg models. Later in Section~\ref{section:pertandhtpy} we will give a description of their explicit homotopy inverses. 

To summarise:

\begin{definition}
The \textsl{bicategory of Landau-Ginzburg models} $\LG$ consists of the following data: 
\begin{itemize}
\item Objects are pairs $(R,W)$ with $W\in R=k[x]$ a potential. 
\item 1- and 2-morphisms are the objects and morphisms of the categories
\[
\LG((R, W), (S,V)) = \hmf(R \otimes_k S,V-W)^\omega\,.
\]
\item The unit 1-morphisms are $\Delta_W \in \hmf(\Re,\widetilde W)$. 
\item The composition functor is the tensor product \eqref{eq:backgroundcomptensor}.
\item There are natural 2-isomorphisms $\alpha, \lambda, \rho$ as above. 
\end{itemize}
\end{definition}

\begin{proposition}[\cite{McNameethesis, Calinetal, cr0909.4381}] 
$\LG$ really is a bicategory, i.\,e.~$\alpha, \lambda, \rho$ are natural isomorphisms up to homotopy, and they satisfy the coherence axioms for bicategories. 
\end{proposition}

\begin{remark}\label{remark:gradedbicategory}
An analogous statement is also true of the bicategory of \textsl{graded} Landau-Ginzburg models $\LG^{\mathrm{gr}}$. Its objects are pairs $(R,W)$ as above but with the additional requirement that~$R$ is a $\nZ$-graded ring and~$W$ is homogeneus of some degree $2c$. A graded linear factorisation is then a pair $(X,d_X)$ as before, but in addition we ask~$X$ to be a $\nZ$-graded module and~$d_X$ homogeneous of degree~$c$. Morphisms of graded linear factorisations must have $\nZ$-degree zero. As shown in~\cite{cr0909.4381} this is in particular true of the structure maps $\alpha, \lambda, \rho$. 

As a relevant example let us discuss the graded Hom. For graded linear factorisations $X,Y$ of $W,W' \in R_{2c}$, respectively, we can form the $\nZ$-graded $R$-module $\Hom_{\mathrm{gr}}(X,Y) = \bigoplus_{i \in \mathds{Z}} \Hom_R(X,Y)_i$, where $\Hom_R(X,Y)_i$ denotes the group of all $R$-linear maps $X \lto Y$ of $\nZ$-degree~$i$. This is a submodule of $\Hom_R(X,Y)$, and if~$X,Y$ are finitely generated then $\Hom_{\mathrm{gr}}(X,Y) = \Hom_R(X,Y)$. This graded module has a $\nZ_2$-decomposition into even and odd maps, 
\begin{align*}
\Hom_{\mathrm{gr}}(X,Y) &= \Big( \Hom_{\mathrm{gr}}(X^0, Y^0) \oplus \Hom_{\mathrm{gr}}(X^1, Y^1) \Big) \oplus \Big( \Hom_{\mathrm{gr}}(X^0, Y^1) \oplus \Hom_{\mathrm{gr}}(X^1, Y^0) \Big) \, .
\end{align*}
Together with the differential defined on a $\nZ$-homogeneous map $\alpha$ by $\alpha \lmt d_Y \circ \alpha - (-1)^{|\alpha|} \alpha \circ d_X$, the graded module $\Hom_{\mathrm{gr}}(X,Y)$ is a graded linear factorisation of $W' - W$. In particular if $W = W'$ we have a $(\ZZ \times \nZ_2)$-graded complex. This construction is functorial, in the sense that if $\varphi: Y \lra Y'$ is a morphism of graded linear factorisations of degree~$i$, then $\Hom_{\mathrm{gr}}(1_X,\varphi)$ is another morphism of degree~$i$, and similarly in the first variable.
\end{remark}

\subsection{Bar complex}\label{subsec:Bar}

For a polynomial ring the standard resolutions of the diagonal are the Koszul and Bar resolutions. Above we defined the unit $1$-morphisms in $\LG$ using a Koszul model for the diagonal, but in the study of adjoints it will quickly become necessary to use the Bar model for the diagonal as well. For this reason we recall in this section the necessary background on noncommutative forms and the Bar complex from \cite{Loday, cuntzquillen}. We explain how to use the Bar complex to construct a second model for the diagonal, and construct the map $\Psi$ which relates the Koszul and Bar models.

For the moment $R$ is an arbitrary ring. \textsl{Noncommutative $n$-forms over~$R$} are elements in 
$$
\Omega^n R = R\otimes \bar R^{\otimes n} 
$$
where $\bar R = R/k$ and in this section by ``$\otimes$'' we mean ``$\otimes_k$''. We denote the projection of $a_0\otimes a_1 \otimes \ldots \otimes a_n \in R^{\otimes (n+1)}$ to $\Omega^n R$ as $(a_0,a_1,\ldots,a_n)$. The direct sum 
$$
\Omega R = \bigoplus_{n\geq 0} \Omega^n R
$$
is a differential graded algebra $(\Omega R, d, \cdot)$ with multiplication given by
$$
(a_0,\ldots,a_m) \cdot (a_{m+1},\ldots, a_{m+n}) = \sum_{i=0}^m (-1)^{m-i}(a_0,\ldots,a_{i-1},a_i a_{i+1},a_{i+2},\ldots, a_{m+n}) 
$$
and differential
$$
d: (a_0,\ldots,a_n) \lmt (1,a_0,\ldots,a_n)
$$
where $a_i\in R$. We will write $(a_0,a_1,\ldots,a_n)$ also as $a_0da_1\ldots da_n$. 

More generally one can consider relative noncommutative forms: for a subalgebra $B\subset R$ they are elements in $\Omega_B R = \bigoplus_{n\geq 0} R\otimes_B (R/B)^{\otimes_B n}$, which has a differential graded structure analogous to $\Omega R = \Omega_k R$. We refer to the book~\cite{Loday} for further details. 

A central role is played by the \textsl{(normalised) bar complex} 
\be\label{BarComplex}
\Bar = \bigoplus_{n\geq 0} \Bar_n \, , \qquad \Bar_n = \Omega^n R \otimes R \, .
\ee
It is an $(R \otimes R^{\text{op}})$-module via $(a\otimes a').(a_0da_1\ldots da_n\otimes a_{n+1}) = aa_0da_1\ldots da_n\otimes a_{n+1}a'$. Together with the differential $d\otimes 1_R$ (which by standard abuse of notation we usually simply write~$d$) and the product induced from $\Omega R$ and~$R$, the bar complex~$\Bar$ is a differential graded algebra $(\Bar,d,\cdot)$. While $d$ is right $R$-linear, we will also make use of the left $R$-linear operator $s$ on $\Bar$ defined by
% for the operator s see coevc3
\be\label{eq:definition_s}
s( a_0 da_1 \ldots da_n \otimes a_{n+1} ) = (-1)^{n+1} a_0 da_1 \ldots da_n da_{n+1}\,.
\ee
There is a second differential graded structure on~$\Bar$ if the algebra~$R$ is commutative. To describe it let us first recall that (still for arbitrary~$R$) the bar complex is the standard resolution 
$$
\xymatrix{%
\cdots \ar[r]^-{b'} & R \otimes \bar R^{\otimes 2} \otimes R \ar[r]^-{b'} & R \otimes \bar R \otimes R \ar[r]^-{b'} & R\otimes R \ar[r]^-{b'} & R \ar[r] & 0
}%
$$
of~$R$, where the degree-lowering differential~$b'$ is the $R$-bilinear map
$$
b': (a_0,\ldots,a_n)\otimes a_{n+1} \lmt \sum_{i=0}^{n-1} (-1)^i (a_0,\ldots,a_i a_{i+1},\ldots, a_n) \otimes a_{n+1} + (-1)^n (a_0,\ldots,a_{n-1}) \otimes a_n a_{n+1} \, . 
$$
%Equivalently, in differential form notation~$b'$ acts as
%$$
%b': da_0\ldots da_n\otimes a_{n+1} \lmt (-1)^{n-1} a_0da_1\ldots da_{n-1} \cdot a_n \otimes a_{n+1} + (-1)^n a_0 da_1\ldots da_{n-1} \otimes a_n a_{n+1} \, . 
%$$
From this it is straightforward to check that we have the identities
\be\label{b'd+db'} 
b'd+db'=1_{\Bar} \, , \qquad b' s + s b' = 1_{\Bar}\,.
\ee
From now on we assume that~$R$ is commutative. Recall that $(m,n)$-shuffles are permutations in
$$
\operatorname{Sh}(m,n) = \big\{ \sigma\in S_{m+n} \,|\, \sigma(1)<\sigma(2)<\ldots<\sigma(m), \, \sigma(m+1)<\sigma(m+2)<\ldots<\sigma(m+n) \big\} \, . 
$$
We use them to define the $R$-bilinear \textsl{shuffle product}~$\times$ on~$\Bar$ as
\begin{align*}
& (a_0da_1\ldots da_m \otimes a_{m+1}) \times (b_0db_1\ldots db_n \otimes b_{n+1}) \\
& \qquad 
= \sum_{\sigma \in \operatorname{Sh}(m,n)} (-1)^{|\sigma|} a_0 b_0 \, \sigma_\bullet (da_1\ldots da_m db_1 \ldots db_n) \otimes a_{m+1} b_{n+1}
\end{align*}
where $\sigma_\bullet(da_1\ldots da_j) = da_{\sigma^{-1}(1)}\ldots da_{\sigma^{-1}(j)}$. One finds that $(\Bar,b',\times)$ is a graded-commutative differential graded algebra. Note that for $\omega\in R\otimes R=\Bar_0$ it follows immediately that $\omega\times(-) = \omega\cdot(-)$. 

We now return to the $k$-algebra $R=k[x_1,\ldots,x_n]$. Earlier we set $\widetilde W = W\otimes 1 - 1\otimes W \in \Re$ for a potential $W\in R$, in terms of which we now define the bar complex endomorphism
$$
d_{\Bar} = b' + d\widetilde W \times (-) \, . 
$$
(\textbf{todo} Cite other places where this bar model of the diagonal appears)

\begin{lemma}\label{lemma:barisafactorisation}
$(\Bar,d_{\Bar})$ is a linear factorisation of $\widetilde W\in \Re$. 
\end{lemma}

\begin{proof}
The bar complex $\Bar=\Bar^0 \oplus \Bar^1$ is $\nZ_2$-graded with $\Bar^i = \bigoplus_{n\in 2\nN+i}\Bar_n$. Since~$b'$ and~$\times$ are both $R$-bilinear, $d_{\Bar}$ is indeed $\Re$-linear. Furthermore, we have $b'^2=0$ and $d\widetilde W \times d\widetilde W = (dW\otimes 1) \times (dW\otimes 1) = dWdW\otimes 1 - dWdW \otimes 1 = 0$, so that for $\omega\in\Bar$ we find 
\begin{align*}
d_{\Bar}^2 (\omega) & = b'(d\widetilde W \times \omega) + d\widetilde W \times b'(\omega) \\
& = b'(d\widetilde W) \times \omega - d\widetilde W \times b'(\omega) + d\widetilde W \times b'(\omega) \\
& = \widetilde W \times \omega \\
& = \widetilde W \cdot \omega
\end{align*}
where in the second last step we used~\eqref{b'd+db'} together with $b'(\widetilde W)=0$. 
\end{proof}

If we use~$\pi$ also to denote the projection $\Bar�\lra \Bar_0 = \Re$ composed with multiplication $\Re \lra R$, then $\pi \otimes 1_X : \Bar \otimes X \lra X$ and $1_X \otimes \pi: X \otimes \Bar \lra X$ give left and right actions of~$\Bar$ as in~\eqref{lambdarho}. These maps have homotopy inverses too and we will construct them in Section~\ref{section:pertandhtpy}. For the moment we take the fact that~$\Bar$ is another model for the unit action on matrix factorisations as motivation to discuss its relation to the Koszul matrix factorisation~$\Delta_W$. 

Before we do this on the level of linear factorisations we consider the case of $\mathbb{Z}$-graded complexes. We write $\Delta = \bigwedge( \bigoplus_{i=1}^n \Re \theta_i)$ and observe that $(\Delta, \delta_-)$ is the ordinary Koszul complex, see~\eqref{DeltaW}. 

There are two $\Re$-linear maps between~$\Bar$ and~$\Delta$ which will be important to us: 
\begin{align}
\Phi & : \Delta \lra \Bar \, , \qquad \theta_{i_1}\ldots \theta_{i_p} \lmt \sum_{\sigma\in S_p} (-1)^{|\sigma|} dx_{i_{\sigma(1)}} \ldots dx_{i_{\sigma(p)}} \otimes 1 \, , \label{intro_phi}\\
\Psi & : \Bar \lra \Delta \, , \qquad df_1\ldots df_p \otimes 1 \lmt \sum_{1\leq i_1<\ldots<i_p\leq n} \Big( \prod_{k=1}^p \partial_{[i_k]} f_k \Big) \, \theta_{i_1} \ldots \theta_{i_p} \, \label{intro_psi}.
\end{align}
These maps were studied in~\cite{sw0911.0917}, we only rephrase the presentation of~$\Psi$ in terms of the difference quotient operators~$\partial_{[i]}$ suitable for our setting. One easily verifies that $\Psi\Phi = 1_\Delta$.

\begin{lemma}\label{PhiPsiDG}
Both~$\Phi$ and~$\Psi$ are maps of differential graded algebras between $(\Delta, \delta_-, \wedge)$ and $(\Bar, b', \times)$. 
\end{lemma}

\begin{proof}
We refer to~\cite{sw0911.0917} for the case of~$\Phi$; since our expression for~$\Psi$ is not manifestly the same as in loc.~cit.~we spell out the proof. Let us first show that~$\Psi$ is compatible with the differentials. On the one hand we compute $(\delta_- \Psi) (df_1\ldots df_p \otimes 1)$ to be 
\begin{align}
& \delta_ -\Big( \sum_{i_1<\ldots <i_p} (\partial_{[i_1]} f_1) \ldots (\partial_{[i_p]} f_p) \, \theta_{i_1} \ldots \theta_{i_p} \Big) \nonumber \\
& =  \sum_{i_1<\ldots <i_p} (\partial_{[i_1]} f_1) \ldots (\partial_{[i_p]} f_p) \cdot (x_{i_k} - x'_{i_k}) \sum_{k=1}^p (-1)^{k+1} \theta_{i_1} \ldots \widehat{\theta_{i_k}} \ldots \theta_{i_p}  \nonumber \\
& = \sum_{k=1}^p (-1)^{k+1}\sum_{i_1<\ldots <i_p} (\partial_{[i_1]} f_1) \ldots ({}^{t_1\ldots t_{i_{k-1}}} f_k - {}^{t_1\ldots t_{i_{k}}} f_k) \ldots (\partial_{[i_p]} f_p) \, \theta_{i_1} \ldots \widehat{\theta_{i_k}} \ldots \theta_{i_p}  \nonumber \\
& = \sum_{2\leq i_2<\ldots< i_p} (f_1 - {}^{t_1\ldots t_{i_{2}-1}} f_1) (\partial_{[i_2]} f_2) \ldots (\partial_{[i_p]} f_p) \, \theta_{i_2} \ldots \theta_{i_p} \nonumber \\
& \qquad + \sum_{k=2}^{p-1} (-1)^{k+1}\sum_{i_1 < \ldots <i_p} (\partial_{[i_1]} f_1) \ldots ({}^{t_1\ldots t_{i_{k-1}}} f_k - {}^{t_1 \ldots t_{i_{k+1}-1}} f_k) \ldots (\partial_{[i_p]} f_p) \, \theta_{i_1} \ldots \widehat{\theta_{i_k}} \ldots \theta_{i_p}  \nonumber \\
& \qquad + (-1)^{p+1} \sum_{i_1<\ldots< i_{p-1}\leq n-1} (\partial_{[i_1]} f_1) \ldots (\partial_{[i_{p-1}]} f_{p-1}) ({}^{t_1\ldots t_{i_{p-1}}} f_p - {}^{t_1\ldots t_n} f_p)  \, \theta_{i_1} \ldots \theta_{i_{p-1}} \nonumber \\
& = \sum_{2\leq t_2<\ldots <i_p} f_1 (\partial_{[i_2]} f_2) \ldots (\partial_{[i_p]} f_p) \, \theta_{i_2} \ldots \theta_{i_p} \nonumber \\
& \qquad + (-1)^{p} \sum_{i_1<\ldots< i_{p-1}\leq n-1} (\partial_{[i_1]} f_1) \ldots (\partial_{[i_{p-1}]} f_{p-1}) \, {}^{t_1\ldots t_n} f_p  \, \theta_{i_1} \ldots \theta_{i_{p-1}}  \label{delPsi} 
\end{align}
while on the other hand $(\Psi b') (df_1\ldots df_p \otimes 1)$ equals
\begin{align*}
& \Psi \Big( f_1 df_2 \ldots df_p \otimes 1 + \sum_{k=1}^{p-1} (-1)^k df_1 \ldots d(f_k f_{k+1}) \ldots df_p \otimes 1 + (-1)^p df_1 \ldots df_{p-1} \otimes f_p \Big) \\
& = \sum_{i_1<\ldots< i_{p-1}} f_1 (\partial_{[i_1]} f_2) \ldots (\partial_{[i_{p-1}]} f_p) \, \theta_{i_1} \ldots \theta_{i_{p-1}} \\
& \qquad + \sum_{k=1}^{p-1} (-1)^k \sum_{i_1<\ldots< i_{p-1}} (\partial_{[i_1]} f_1) \ldots (\partial_{[i_k]} (f_k f_{k+1}) )\ldots (\partial_{[i_{p-1}]} f_p) \, \theta_{i_1} \ldots \theta_{i_{p-1}} \\
& \qquad + (-1)^p \sum_{i_1<\ldots< i_{p-1}} (\partial_{[i_1]} f_1) \ldots (\partial_{[i_{p-1}]} f_{p-1}) \, {}^{t_1\ldots t_n} f_p \, \theta_{i_1} \ldots \theta_{i_{p-1}} \\
& = \sum_{2\leq i_1<\ldots< i_{p-1}} f_1 (\partial_{[i_1]} f_2) \ldots (\partial_{[i_{p-1}]} f_p) \, \theta_{i_1} \ldots \theta_{i_{p-1}} \\ 
& \qquad + \sum_{2\leq i_2 \ldots< i_{p-1}} f_1 (\partial_{[1]} f_2) (\partial_{[i_2]} f_3) \ldots (\partial_{[i_{p-1}]} f_p) \, \theta_{i_1} \ldots \theta_{i_{p-1}} \\ 
& \qquad + \sum_{k=1}^{p-1} (-1)^k \sum_{i_1<\ldots< i_{p-1}} (\partial_{[i_1]} f_1) \ldots \big\{ ({}^{t_1\ldots t_{i_k -1}}f_k)(\partial_{[i_k]} f_{k+1}) \\
& \qquad\quad + (\partial_{[i_k]} f_k) ({}^{t_1\ldots t_{i_k}} f_{k+1}) \big\} \ldots (\partial_{[i_{p-1}]} f_p) \, \theta_{i_1} \ldots \theta_{i_{p-1}} \\
& \qquad + (-1)^p \sum_{i_1<\ldots< i_{p-1}\leq n-1} (\partial_{[i_1]} f_1) \ldots (\partial_{[i_{p-1}]} f_{p-1}) \, {}^{t_1\ldots t_n} f_p \, \theta_{i_1} \ldots \theta_{i_{p-1}} \\
& \qquad\quad + (-1)^p \sum_{i_1<\ldots< i_{p-2}\leq n-1} (\partial_{[i_1]} f_1) \ldots (\partial_{[i_{p-1}]} f_{p-1}) \, {}^{t_1\ldots t_n} f_p \, \theta_{i_1} \ldots \theta_{i_{p-1}} \\
& = \sum_{2\leq i_1<\ldots< i_{p-1}} f_1 (\partial_{[i_1]} f_2) \ldots (\partial_{[i_{p-1}]} f_p) \, \theta_{i_1} \ldots \theta_{i_{p-1}} \\ 
& \qquad + (-1)^p \sum_{i_1<\ldots< i_{p-1}\leq n-1} (\partial_{[i_1]} f_1) \ldots (\partial_{[i_{p-1}]} f_{p-1}) \, {}^{t_1\ldots t_n} f_p \, \theta_{i_1} \ldots \theta_{i_{p-1}} 
\end{align*}
which agrees with~\eqref{delPsi}. 

To establish compatibility with the products we compute 
\begin{align*}
& \Psi( df_1\ldots df_p \otimes 1) \wedge \Psi( df_{p+1}\ldots df_{p+q} \otimes 1) \\ 
=\, & \Big\{ \sum_{i_1<\ldots< i_p} \Big( \prod_{k=1}^p \partial_{[i_k]} f_k \Big) \theta_{i_1} \ldots \theta_{i_p}\Big\} 
\wedge 
\Big\{ \sum_{i_{p+1}<\ldots< i_{p+q}} \Big( \prod_{k=p+1}^{p+q} \partial_{[i_k]} f_k \Big) \theta_{i_{p+1}} \ldots \theta_{i_{p+q}} \Big\} \\
= \, & \sum_{i_1<\ldots< i_p} \sum_{i_{p+1}<\ldots< i_{p+q}} \Big( \prod_{k=1}^{p+q} \partial_{[i_k]} f_k \Big) \theta_{i_{1}} \ldots \theta_{i_{p+q}} \\
= \, & \sum_{i_1<\ldots< i_{p+q}} \sum_{\sigma\in\operatorname{Sh}(p,q)} (-1)^{|\sigma|} \Big( \prod_{k=1}^{p+q} \partial_{[i_{\sigma(k)}]} f_k \Big) \theta_{i_{1}} \ldots \theta_{i_{p+q}} \\
= \, & \Psi \big((df_1\ldots df_p \otimes 1) \times( df_{p+1}\ldots df_{p+q} \otimes 1) \big)
\end{align*}
where in the third step the anti-commutativity of the~$\theta_i$ allowed us to sum over the longer sequences $i_1<\ldots< i_{p+q}$ by introducing an additional sum over shuffles. 
\end{proof}

On a practical level both the Bar and Koszul complexes resolve the diagonal, so there must be a chain map lifting the identity on the diagonal, and $\Psi$ gives one particular example of such a chain map. For the reader looking for more conceptual insight, we mention that in our constructions the map $\Psi$ will almost always appear as part of the composite $\varepsilon \Psi: \Bar \lto \Re[n]$. This composite has a natural interpretation in Hochschild cohomology as a product of $n$ Hochschild cocycles associated to derivations; we explain this perspective in Appendix \ref{appendix:mapeppsi}.

Now we come back to consider any $W\in R$. The map~$\Psi$ continues to be a good map on the level of linear factorisations: 
\begin{lemma}\label{PsiHF}
$\Psi: (\Bar, d_\Bar) \lra (\Delta_W, d_{\Delta_W})$ is a morphism in $\HF(\Re,\widetilde W)$. 
\end{lemma}

\begin{proof}
We need to show $d_{\Delta_W} \Psi = \Psi d_\Bar$. But since $d_{\Delta_W} = \delta_+ + \delta_-$ and $d_\Bar = b' + d\widetilde W \times (-)$ by Lemma~\eqref{PhiPsiDG} what remains to be checked is $\delta_+ \Psi = \Psi (d\widetilde W\times (-))$. This can be done: 
$$
\delta_+ \Psi = \Big( \sum_{i=1}^n \partial_{[i]} W \cdot \theta_i^* \Big) \wedge \Psi(-) = \Psi(dW\otimes 1) \wedge \Psi(-) = \Psi (d\widetilde W\times (-)) \, . 
$$
\end{proof}


%\begin{remark}
%Instead of the bar complex $\Bar = \bigoplus_{n\geq 0} \Bar_n$ one can also consider its completed version
%$$
%\cBar = \prod_{n\geq 0} \Bar_n \, .
%$$
%It has differential graded structures $(\cBar,d,\cdot)$ and $(\cBar,b',\times)$ analogous to~$\Bar$, and the maps~$\Phi$ and~$\Psi$ lift to maps to and from the completed bar complex~$\cBar$ with the same properties as in Lemmas~\ref{PhiPsiDG} and~\ref{PsiHF}. 
%\end{remark}

\subsection{Residues}\label{section:residuebackground}

Since residues feature prominently in the Kapustin-Li pairing and in the evaluation maps constructed in Section~\ref{sec:derivcoeval} we briefly recall their definition and basic property; see~\cite{TODO} for more details. Given a regular sequence $(f_1,\ldots,f_n)$ in $k[x]$ it is the $k[y]$-linear map that sends a polynomial $g\in k[x,y]$ to the expression
$$
\Res_{k[x,y]/k[y]} \left[ \frac{g \, \underline{\operatorname{d}\! x}}{f_1, \ldots, f_n} \right] 
\in k[y]
$$
where $\underline{\operatorname{d}\!x}=\operatorname{d}\!x_1\ldots \operatorname{d}\!x_n$. In practice the residue is computed by the transformation rule
$$
\Res_{k[x,y]/k[y]} \left[ \frac{g \, \underline{\operatorname{d}\! x}}{f_1, \ldots, f_n} \right] 
= 
\Res_{k[x,y]/k[y]} \left[ \frac{\det(C) g \, \underline{\operatorname{d}\! x}}{f'_1, \ldots, f'_n} \right] 
\, , \qquad
f'_i = \sum_{j=1}^n C_{ij} f_j 
\, , \qquad
C_{ij} \in k[x,y] \, ,
$$
together with the defining property $\Res_{k[x,y]/k[y]} \left[ \underline{\operatorname{d}\! x}/(x_1^{a_1} ,\ldots, x_n^{a_n}) \right] = \delta_{a_1,1}\ldots \delta_{a_n,1}$. Note that the order of the elements of the regular sequence in the denominator plays a role; changing that order produces a permutation sign. 

\subsection{Perturbation}

A crucial role will be played by the homological perturbation lemma, which we will use to promote homotopy equivalences of complexes (arising from the bar and Koszul resolutions of the diagonal) to homotopy equivalences of associated matrix factorisations. More importantly, the perturbation lemma provides explicit homotopy inverses in terms of Atiyah classes.

Let $R$ be a ring and $W \in R$. An $R$-linear \textsl{deformation retract datum} is a diagram
\begin{equation}\label{eq:perturbeddiagr1}
\xymatrix@C+2pc{
(X,d_X) \ar@<-1ex>[r]_-{\sigma} & (Y,d_Y) \ar@<-1ex>[l]_-{\pi}
}%
\!\!\!\xymatrix{%
{}\ar@(ur,dr)[]^{h}
}%
\end{equation}
in which $(X,d_X)$ and $(Y,d_Y)$ are linear factorisations of $W$, $\pi, \sigma$ are morphisms of linear factorisations and $h: Y \lto Y$ is a degree one $R$-linear map such that
$$
\pi \sigma = 1 \, , \qquad
\sigma \pi = 1 + d_Yh + hd_Y \, .
$$
A degree one morphism $\delta: Y \lto Y$ is a \textsl{small perturbation} of the deformation retract datum if $1_Y - \delta h$ is an isomorphism of $R$-modules. In this case we define
\[
\tau = (1- \delta h)^{-1} \delta
\]
and consider the new ``perturbed'' diagram
\begin{equation}\label{eq:perturbeddiagr}
\xymatrix@C+2pc{
(X,d_{X,\infty}) \ar@<-1ex>[r]_-{\sigma_\infty} & (Y,d_Y+\delta) \ar@<-1ex>[l]_-{\pi_\infty}
}%
\!\!\!\xymatrix{%
{}\ar@(ur,dr)[]^{h_\infty}
}%
\end{equation}
where
\begin{align*}
\sigma_\infty &= \sigma + h\tau\sigma \,, & h_\infty &= h + h \tau h\,,\\
\pi_\infty &= \pi + \pi \tau h \, , & d_{X,\infty} &= d_X + \pi \tau \sigma\,.
\end{align*}

\begin{proposition}\label{prop:pertlemma} Suppose that $h \sigma = 0, \pi h = 0$ and $h^2 = 0$. If $\delta$ is a small perturbation of (\ref{eq:perturbeddiagr1}) such that $(d_Y + \delta)^2 = W' \cdot 1_M$ for some $W' \in R$ then (\ref{eq:perturbeddiagr}) is a deformation retract datum of linear factorisations of $W'$ over $R$.
\end{proposition}
\begin{proof}
This follows straightforwardly from the standard results in~\cite{c0403266}. 
\end{proof}

In the cases of interest to us the sum $\sum_{m \ge 0} (\delta h)^m$ converges, so that $\tau = \sum_{m \ge 0} (\delta h)^m \delta$ and
\[
\sigma_\infty = \sigma + \sum_{m \ge 0} h(\delta h)^m \delta \sigma = \sum_{m \ge 0} (h \delta)^m \sigma\,.
\]

\subsection{Canonical morphisms}\label{section:canonicalmaps}
% see NMF and NMF2

We end this section by recalling the definition of several canonical maps. Throughout~$R$ is a ring and all factorisations are over~$R$. Let~$X$ and~$Y$ denote linear factorisations of~$W$ and~$V$, respectively.

\begin{lemma}\label{lemma:iso_tensorhom1} There is a natural morphism of linear factorisations of $V - W$, 
$$
\xi: X^{\vee} \otimes_R Y \lto \Hom_R(X,Y)\, , \qquad
\xi( \nu \otimes y )(x) = (-1)^{|\nu||y|} \nu(x) \cdot y \, ,
$$
which is an isomorphism if~$X$ is a finitely generated projective $R$-module.
\end{lemma}

\begin{lemma} There is a natural isomorphism of linear factorisations of $W + V$, 
$$
\textup{swap}: X \otimes_R Y \lto Y \otimes_R X\,, \qquad
x \otimes y \lmt (-1)^{|x||y|} y \otimes x\,.
$$
\end{lemma}

Precomposing~$\xi$ with this swap isomorphism we have a canonical morphism
\be\label{eq:iso_tensorhom2}
Y \otimes_R X^{\vee} \lto \Hom_R(X,Y)\,, \qquad
y \otimes \nu \lmt \big\{ x \lmt \nu(y) \cdot x \big\}\,.
\ee
Since it is unlikely to cause confusion, we also denote this map by $\xi$.

\begin{lemma}\label{lem:shiftsigns}
There are natural isomorphisms of linear factorisations of $W + V$, 
$$
X[i] \otimes_R Y \lto (X \otimes_R Y)[i]\,, \qquad
x \otimes y \lmt x \otimes y\,,
$$
and
$$
X \otimes_R Y[i] \lto (X \otimes _R Y)[i]\,,\qquad 
x \otimes y \lmt (-1)^{i|x|} x \otimes y\,.
$$
\end{lemma}

\section{Atiyah classes}\label{section:atiyahclasses}
% see assat, the onct series

As explained in the Introduction, the structure of the bicategory $\LG$ can be understood in terms of associative Atiyah classes. In this section we develop the basic theory of these operators, beginning with the simplest definition (which involves a choice of basis) and then afterwards developing the more abstract general theory in the setting of noncommutative form-valued connections. For this paper the simpler definition is sufficient, since we will only use the Atiyah class to construct explicit chain level representatives of morphisms which we already know to be well-defined up to homotopy, and in this situation it is natural to make some auxiliary choices. In any case, we develop the general theory in order to have a more solid foundation for the subject of associative Atiyah classes.

Let $k$ be a ring, $R$ a $k$-algebra, and set $\Re = R_1 \otimes_k R_2$ where $R_i = R$ for $i \in \{1,2\}$. We define  the operators $s,d$ on the bar complex $\Bar = \Bar_{R/k}$ as in Section \ref{subsec:Bar} and recall that $\Bar$ is equipped with an $\Re $-bimodule structure by left and right multiplication. Let $(X,D)$ be a pair consisting of a $\mathbb{Z}_2$-graded free $\Re $-module $X$ and an odd $\Re $-linear operator $D$ on $X$ (for the moment we do not require any condition on $D^2$). By default $\otimes$ means $\otimes_{\Re }$ in this section.

Given a homogeneous basis $\{ e_i \}_{i}$ for~$X$ we can extend~$d$ and~$s$ to operators
\begin{align}
d&: X \otimes \Bar \lto X \otimes \Bar, &\qquad d( e_i \otimes \omega ) = (-1)^{|e_i|} e_i \otimes d \omega\,, \label{eq:extendd}\\
s&: X \otimes \Bar \lto X \otimes \Bar, &\qquad s( e_i \otimes \omega ) = (-1)^{|e_i|} e_i \otimes s \omega\,. \label{eq:extends}
\end{align}
So for example $d( a e_i \otimes \omega ) = (-1)^{|e_i|} e_i \otimes d( a \omega )$ for $a \in \Re $.

\begin{definition}\label{defn:atiyahclassbasis}
The \textsl{associative Atiyah classes} of $(X,D)$ are operators on $X \otimes \Bar$ defined by
\begin{align*}
\At_{2}(X) &= [d, D] = d D + D d\,,\\
\At_{1}(X) &= [s, D] = s D + D s\,.
\end{align*}
\end{definition}

One checks that for $\omega \in \Bar$
\be\label{eq:at2nopower}
\At_{2}(X)( e_i \otimes \omega ) = \sum_j (-1)^{|e_j|} e_j \otimes d( D_{ji} ) \omega\,,
\ee
and more generally for $m \ge 1$,
\be
\At_{2}(X)^m( e_i \otimes \omega ) = (-1)^{m|e_i| + \binom{m+1}{2}} \sum_{j_1,\ldots,j_m} e_{j_m} \otimes d( D_{j_m, j_{m-1}} ) \ldots d(D_{j_2,j_1})d( D_{j_1 i} ) \omega\,\label{eq:powerofatiyah}
\ee
% for signs see for example shati
There are slightly more complicated formulas for $\At_1$, but since this Atiyah class is less important for us than $\At_2$ we leave the details to Remark \ref{remark:at2formulas} below. We call these \textsl{associative} Atiyah classes to distinguish them from the more standard Atiyah classes defined using commutative differential forms, but since the latter do not appear in this paper (with the exception of Appendix \ref{section:dualityadjointop}) we will omit the qualifier ``associative''. Without conditions on $D^2$ there is no interpretation of $\At_1$ and $\At_{2}$ as a cohomology class, so we should perhaps refer to Atiyah \textsl{operators}, but we persist with the above terminology.

To discuss the linearity we need to be clear that $X \otimes \Bar$ is made into a right $\Re $-module via the right action of $\Re $ on $\Bar$ by multiplication. It is then apparent from \eqref{eq:at2nopower} that $\At_{2}(X)$ is right $\Re $-linear. The subscripts on the Atiyah classes arise because we think of $d$ (resp.~$s$) as differentiating in the $R_1$-directions (resp.~$R_2$-directions).

\begin{remark}\label{remark:linearatother} 
Suppose that $S$ is a $k$-algebra and $X$ a $\mathbb{Z}_2$-graded free $(S \otimes_k \Re)$-module with $(S \otimes_k \Re)$-linear odd operator $D$. If we choose a homogeneous $(S \otimes_k \Re)$-basis for~$X$ and extend $d,s$ to $S$-linear operators on $X \otimes \Bar$ via \eqref{eq:extendd}, \eqref{eq:extends} then $\At_2(X) = [d, D]$ and $\At_1(X) = [s, D]$ define operators on $X \otimes \Bar$, and all our remarks apply equally well to this kind of Atiyah class.

Similarly if instead $X$ is a $\mathbb{Z}_2$-graded free $(S \otimes_k R)$-module with $(S \otimes_k R)$-linear odd operator $D$ and we choose a homogeneous basis for $X$ and extend $d,s$ to $S$-linear operators on $X \otimes_R \Bar$ then $\At_2(X) = [d, D]$ and $\At_1(X) = [s,D]$ are operators on $X \otimes_R \Bar = (X \otimes_R \Re) \otimes \Bar$ which agree with the Atiyah classes of the $(S \otimes_k \Re)$-module $X \otimes_R \Re$.
\end{remark}

There are two more variants of the Atiyah class that differ from the above only in the way in which we order the noncommutative differential forms. We can extend $s,d$ to operators on $\Bar \otimes X$ (using the right $\Re $-multiplication on $\Bar$ to form the tensor product) as above, but now without Koszul signs, and define operators on this tensor product by $\lAt_{2}(X) = [d, D]$ and $\lAt_{1}(X) = [s,D]$. 
The difference from the other Atiyah classes appears when we exponentiate: 
\begin{equation}\label{eq:leftatiyahexp}
\lAt_{2}(X)^m( \omega \otimes e_i ) = \sum_{j_1,\ldots,j_m} \omega d( D_{j_1,i} ) d(D_{j_2,j_1}) \ldots d(D_{j_m,j_{m-1}}) \otimes e_{j_m}\,.
\end{equation}
From this it is clear that $\lAt_{2}$ is left $\Re $-linear, using the action of left multiplication on $\Bar \otimes X$.

In addition to these four distinct kinds of Atiyah classes, we will sometimes need purely cosmetic variants where we switch the order of elements of $X$ and $\Bar$ in the tensors. This creates an awkward situation with the notation, since we want elements of $\Re $ traversing the tensor product to act on the same side of $\Bar$ as before. Maybe the clearest way to indicate this is to use the map
\begin{equation}\label{eq:attau}
\reprod: \Re  \otimes_k \Re  \lto \Re  \otimes_k \Re , \qquad \reprod ( a \otimes a' ) = aa' \otimes 1
\end{equation}
so that in $\reprod_*( \Bar ) \otimes X$ we have $\omega \otimes a x$ = $a \omega \otimes x$. If we understand $d$ and $s$ to be extended to operators on $\reprod_*( \Bar ) \otimes X$ as before, then we define operators on this tensor product by $\Atlarrow_{2}(X) = [ d, D ]$ and $\Atlarrow_{1}(X) = [s, D]$. Then there is a commutative diagram
\[
\xymatrix@C+2pc@R+1pc{
X \otimes \Bar \ar[d]_{\At_{2}(X)^m} \ar[r]^-{\cong} & \reprod_*( \Bar ) \otimes X \ar[d]^{\Atlarrow_{2}(X)^m}\\
X \otimes \Bar \ar[r]_-{\cong} & \reprod_*( \Bar ) \otimes X
}
\]
in which the rows are graded twist maps (with Koszul signs). To be explicit,
\be
\Atlarrow_{2}(X)^m( \omega \otimes e_i ) = (-1)^{m|\omega| + \binom{m}{2}} \sum_{j_1,\ldots,j_m} d(D_{j_m,j_{m-1}}) \ldots d(D_{j_2,j_1}) d(D_{j_1,i}) \omega \otimes e_{j_m}\,.
\ee

\begin{remark}\label{remark:atofshift} Consider the pair $(X[1], -D)$, with the same homogeneous basis as $X$. Then $\At_2(X)$ and $\At_2(X[1])$ define the same operator on the underlying module $X \otimes \Bar$. On the other hand, as operators on $\gamma_*(\Bar) \otimes X$ we have $\Atlarrow_2(X[1]) = - \Atlarrow_2(X)$.
\end{remark}

\begin{remark}\label{remark:at2formulas} In the formulas for $\At_1$ it is sufficient for our purposes to only consider $\omega' \in \Bar$ of the form $\omega' = a_0 da_1 \ldots da_n \otimes 1$. Then
\be
\At_1(X)(e_i \otimes \omega' ) = \sum_j (-1)^{|e_j| + n + 1} e_j \otimes D_{ji}^{(1)} \omega' d( D_{ji}^{(2)} )
\ee
where we write $D_{ji} = \sum D_{ji}^{(1)} \otimes D_{ji}^{(2)} \in \Re$. The additional sign compared to \eqref{eq:at2nopower} comes from the sign in the definition \eqref{eq:definition_s} of the operator $s$. More generally for $m \ge 1$,
\be
\At_1(X)^m( e_i \otimes \omega' ) = (-1)^{m|e_i| + mn} \sum_{j_1,\ldots,j_m} e_{j_m} \otimes D_{j_1 i}^{(1)} \ldots D_{j_mj_{m-1}}^{(1)} \omega' d(D_{j_1i}^{(2)}) \ldots d(D_{j_mj_{m-1}}^{(2)})\,.
\ee
We will also need % see shati2
\be
\Atlarrow_1(X)^m( \omega' \otimes e_i ) = (-1)^m \sum_{j_1,\ldots,j_m} D_{j_1 i}^{(1)} \ldots D_{j_mj_{m-1}}^{(1)} \omega' d(D_{j_1i}^{(2)}) \ldots d(D_{j_mj_{m-1}}^{(2)}) \otimes e_{j_m}\,.
\ee
\end{remark}

\subsection{Properties}

With the above notation, let $(X,d_X)$ and $(Y,d_Y)$ be $\mathbb{Z}_2$-graded free $\Re $-modules equipped with odd $\Re $-linear differentials (again, no condition on $d_X^2$ or $d_Y^2$) and respective homogeneous bases $\{ e_i \}_{i}$ and $\{ f_j \}_{j}$. Giving $X \otimes Y$ the differential $d_{X \otimes Y} = d_X \otimes 1 + 1 \otimes d_Y$ and the basis $\{ e_i \otimes f_j \}_{i,j}$ we have
\begin{align*}
\At(X) &\in \End( X \otimes \Bar ) \, ,\\
\Atlarrow(Y) &\in \End( \reprod_*(\Bar) \otimes Y ) \, ,\\
\At(X \otimes Y) &\in \End( X \otimes Y \otimes \Bar )
\end{align*}
where $\At$ denotes either $\At_1$ or $\At_2$. An important property of the commutative Atiyah class is that the class of $X \otimes Y$ can be expressed in terms of the classes of $X$ and $Y$. The analogue for associative Atiyah classes involves the shuffle product, which we make into an $\Re $-linear operation
\begin{equation}\label{eq:shuffle_opat}
\xymatrix@C+2pc{
X \otimes \reprod_*(\Bar) \otimes \reprod_*(\Bar) \otimes Y \ar[r]^-{1 \otimes \times \otimes 1} & X \otimes \reprod_*(\Bar) \otimes Y \ar[r]^-{\cong} & X \otimes Y \otimes \Bar
}
\end{equation}
where the last map is the graded twist on the second two components. %The following lemma is sufficient for our purposes, but it would be desirable to have a natural, basis-free analogue.

\begin{remark} \textbf{todo} Reminder that $\otimes = \otimes_{\Re}$ and notice that later we will apply everything to $R$-modules extended to $\Re$-modules?
\end{remark}

\begin{lemma}\label{lemma:atshufat} For $m \ge 0$ and $u \in \{1,2\}$ we have
\begin{align}
\At_u(X \otimes Y)^m( e_i \otimes f_j ) &= \sum_{p+q = m} \At_u(X)^p(e_i) \times \Atlarrow_u(Y)^q(f_j)\label{eq:atshufat}
\end{align}
where on the right-hand side we use the operation \eqref{eq:shuffle_opat}. 
\end{lemma}
\begin{proof}
We have
\[
\At_2(X \otimes Y)(e_i \otimes f_j) = \sum_{i_1} (-1)^{|e_{i_1}| + |f_j|} e_{i_1} \otimes f_j \otimes d( d_{X,i_1,i}) + \sum_{j_1} (-1)^{|f_{j_1}|} e_i \otimes f_{j_1} \otimes d( d_{Y, j_1,j} )\,.
\]
Iterating, we see that $\At_2(X \otimes Y)^m(e_i \otimes f_j)$ is a sum over all indices of terms
\be\label{eq:proofofatiyahshuff}
e_{i_{m-p}} \otimes f_{j_p} \otimes \sigma_\bullet( \underbrace{d(d_{X,i_{m-p}i_{m-p-1}}), \ldots, d(d_{X,i_1i})}_{m-p}, \underbrace{d(d_{Y,j_pj_{p-1}}), \ldots, d(d_{Y,j_1j})}_p )
\ee
where $\sigma$ is an $(m-p,p)$ shuffle and the sign that is attached to such a term is $(-1)^q$ where
\[
q = |e_{i_{m-p}}| + \ldots + |e_{i_1}| + |f_{j_p}| + \ldots + |f_{j_1}| + (m-p)|f_{j_p}| + |\sigma|\,.
\]
But considering \eqref{eq:powerofatiyah} it is straightforward to check that the right-hand side of~\eqref{eq:atshufat} is a sum over the same collection of terms, and the signs match, so that \eqref{eq:atshufat} holds for $u = 2$. The situation for $\At_1$ is similar (\textbf{todo} additional explanation?).
\end{proof}

%\begin{remark}\label{remark:twistshufat} If in \eqref{eq:shuffle_opat} instead of twisting $\Bar$ to the right we twisted it to the left, so that the operation terminated in $\gamma_*(\Bar) \otimes X \otimes Y$, then \eqref{eq:atshufat} would instead compute $\Atlarrow(X \otimes Y)^m( e_i \otimes f_j )$.
%\end{remark}

\subsection{Connections}\label{section:connections}

It is better to construct the Atiyah class from a choice of connection, rather than a choice of basis, and such is the purpose of this section. However, as we have already explained, these remarks will not be needed in the sequel and are included only for the sake of completeness. For this reason we restrict ourselves to a discussion of only one of the many ``types'' of Atiyah class presented above, the modifications for the other cases being left to the reader.

Let $k$ be a ring and $R$ a $k$-algebra, $\Omega R = \Omega_k R$ the differential graded algebra $(\Omega R, d, \cdot)$ of noncommutative differential forms explained in Section \ref{subsec:Bar}. A \textsl{connection} on a $\mathbb{Z}_2$-graded $R$-module $X$ is a $k$-linear map
\[
\nabla: X \lto X \otimes_R \Omega^1 R\,,
\]
which sends $X^i$ into $X^i \otimes_R \Omega^1 R$ and satisfies the graded Leibniz rule
\[
\nabla( x a ) = \nabla( x ) a + (-1)^{|x|} x da 
\]
for $x \in X, a \in R$. Notice that by left and right multiplication $\Omega R$ is an $R$-bimodule, and so $X \otimes_R \Omega^1 R$ is naturally an $R$-bimodule. Let us write $\Re  = R_1 \otimes_k R_2$ with $R_i = R$ so that the right $R$-module structure on $X \otimes_R \Omega^1 R$ is the action of $R_2$. The next lemma is an exercise in the Leibniz rule.

\begin{lemma} Let $(X,D)$ be a linear factorisation of $W \in R$. If $\nabla$ is a connection on $X$ then
\[
[ \nabla, D ] = \nabla D + D \nabla: X \lto X \otimes_R \Omega^1 R
\]
is $R_2$-linear.
\end{lemma}

The element $[ \nabla, D ]$ of $\Hom_{R_2}(X, X \otimes_R \Omega^1 R)$ is not necessarily closed unless $W$ belongs to the image of the structure morphism $K \lto W$, which we indicate by saying that $W \in k$.

\begin{lemma} If $W \in k$ then $[ \nabla, D ]$ is a morphism of linear factorisations of $W$ over $R_2$.
\end{lemma}

\begin{lemma} If $\nabla'$ is another connection on $X$ then
\[
[ \nabla, D ] - [ \nabla', D] = [ \nabla - \nabla', D]
\]
is $R_2$-linear. In particular if $W \in k$ then the homology class of $[ \nabla, D ]$ in $H^0 \Hom_{R_2}(X, X \otimes_R \Omega^1 R)$ is independent of the choice of connection.
\end{lemma}

\begin{remark} To relate this to the earlier construction, observe that if $X$ is a free $\Re $-module with basis $\{ e_i \}_{i}$ then $\Omega R$ can be identified with $\Bar$ and the operator $d: X \lto X \otimes \Bar_1 = X \otimes_R \Omega^1 R$ of \eqref{eq:extendd} is a connection.
\end{remark}

\section{Perturbation and inverting unit actions}\label{section:pertandhtpy}
% mostly from barmf5, unitlg. The stuff involving s is from coevc3.

The fundamental technical results in this paper are constructions, using the perturbation lemma, of explicit homotopy inverses to morphisms involving the stabilised diagonal. Generically the results are geometric series in the associative Atiyah classes of the previous section. To give a specific example, recall that in defining the bicategory $\LG$ we have specified for any $1$-morphism $X \in \LG(W,V)$ a pair of natural isomorphisms
\begin{equation}\label{eq:pertandhtpy1}
\lambda: \Delta_V \otimes X \lto X \, , \qquad \rho: X \otimes \Delta_W \lto X
\end{equation}
called the \textsl{unit actions}, see \eqref{lambdarho}. Representing chain maps for the inverses of these morphisms are necessary for computing with diagrams in $\LG$, and in particular are needed for proving the Zorro moves in Section \ref{sec:Zorro}, but finding such representatives is nontrivial. 

Instead of inverting $\lambda$ and $\rho$ directly, which is difficult, we proceed by identifying $\rho$ as the shadow of a similar canonical map $\rho'$ on the bar model for the diagonal via a commutative diagram
\begin{equation}\label{eq:pertdia1}
\xymatrix{
X \,\widehat{\otimes}\, \Bar \ar[dr]_-{\rho'}\ar[rr]^{1 \otimes \Psi} & & X \otimes \Delta_W \ar[dl]^-{\rho}\\
& X
}
\end{equation}
where $\Psi$ is the canonical map given in (\ref{intro_psi}). Roughly speaking the inverse of $\rho'$ is the geometric series in powers of the Atiyah class of $X$ and by postcomposing with $\Psi$ we obtain the desired inverse to $\rho$. A similar argument works for inverting $\lambda$. A completion of $X \otimes \Bar$ is used in order to guarantee that the geometric series converges. We return to this example in Section \ref{section:rhoandlambdainverse} below.

Another important example is the following: given an object $(k[x],W) \in \LG$ and a $1$-morphism $X \in \LG(W,W)$ consider the problem of lifting a $k[x]$-bilinear morphism of linear factorisations $X \lto k[x]$ to a morphism $X \lto \Delta_W$ along the stabilisation map $\pi_{\Delta}: \Delta_W \lto k[x]$, 
\[
\xymatrix{
X \ar[dr] \ar@{.>}[rr] & & \Delta_W \ar[dl]^-{\pi_{\Delta}} \\
& k[x]\,.
}
\]
The solution to this lifting problem is given in Section \ref{section:liftingproblem}, and will be used to give an explicit formula for the evaluation maps in Section \ref{sec:derivcoeval}.

In order to address these examples and others at the same time, we work in the following general setting: $k$ is a ring, $R, S$ are $k$-algebras and $\Re  = R_1 \otimes_k R_2$ where $R_i = R$ for $i \in \{1,2\}$. We assume that a matrix factorisation
\be\label{eq:xisamf}
X \in \HMF(S \otimes_k \Re, U)
\ee
is given with homogeneous basis $\{ e_i \}_{i}$ as an $(S \otimes_k \Re)$-module. Let $D$ denote the differential on $X$ and unless specified otherwise in this section $\otimes$ denotes the tensor product $\otimes_{\Re}$. Consider the module
\[
X \,\widehat{\otimes}\, \Bar := \prod_{m \ge 0} X \otimes \Bar_m
\]
with the $\mathbb{Z}_2$-grading
\[
\big( X \,\widehat{\otimes}\, \Bar \big)^i = \left( \prod_{m \in \mathbb{Z}} X^i \otimes \Bar_{2m} \right) \oplus \left( \prod_{m \in \mathbb{Z}} X^{i+1} \otimes \Bar_{2m+1} \right) . 
\]
It is sometimes helpful to view $X \,\widehat{\otimes}\, \Bar$ as the inverse limit of the system
\[
\cdots \lto X \otimes \Bar/\Bar_{\ge 2} \lto X \otimes \Bar/\Bar_{\ge 1}
\]
where $\Bar_{\ge m} = \bigoplus_{i \ge m} \Bar_i \subseteq \Bar$ and the maps are the obvious quotients $\Bar/\Bar_{\ge m+1} \lto \Bar/\Bar_{\ge m}$. Since $\Bar$ is an $\Re $-bimodule via left and right multiplication there are natural left and right $\Re $-actions on $X \,\widehat{\otimes}\, \Bar$. Since the left and right actions of $R_2$ on $\Bar$ agree, this amounts to an $R_2$-action on $X \,\widehat{\otimes}\, \Bar$ together with the structure of an $R_1$-bimodule. There is of course an $S$-module structure as well, which is preserved by all maps and which we will henceforth not emphasise.

Let $W \in R$ be arbitrary and set $\widetilde{W} = W \otimes 1 - 1 \otimes W \in \Re$. Our first observation is that $X \,\widehat{\otimes}\, \Bar$ can be equipped as a linear factorisation of $U + \widetilde{W}$. One checks that there are well-defined operators on $X \,\widehat{\otimes}\, \Bar$ given by
\begin{align*}
D(x_0, x_1, \ldots) &= (D(x_0), D(x_1), \ldots) \, , \\
b'(x_0, x_1, \ldots) &= (b'(x_1),b'(x_2),\ldots) \, , \\
d(x_0,x_1,\ldots) &= (0, d(x_0), d(x_1),\ldots) \, , \\
s(x_0,x_1,\ldots) &= (0, s(x_0), s(x_1), \ldots) \, , \\
d \widetilde{W} \times (x_0,x_1,\ldots) &= (0, d \widetilde{W} \times x_0, d \widetilde{W} \times x_1, \ldots)
\end{align*}
where $d$ and $s$ are extended to $X \otimes \Bar$ as in Section \ref{section:atiyahclasses} using the chosen basis. 

%For the next result assume that $U + \widetilde{W}$ belongs to the image of $S \otimes_k R_i$ in $S \otimes_k \Re $ for some $i \in \{1,2\}$ so that it makes sense to talk about linear factorisations of $U + \widetilde{W}$ over $S \otimes_k R_i$.

\begin{lemma} $( X \, \widehat{\otimes}_R\, \Bar, d_X + b' + d\widetilde{W} \times (-) )$ is a linear factorisation of $U + \widetilde{W}$.
\end{lemma}
\begin{proof}
This can be checked on the inverse system, where it follows from Lemma \ref{lemma:barisafactorisation}.
\end{proof}

There is a morphism of linear factorisations
\begin{equation}\label{eq:pibar}
\pi: X \, \widehat{\otimes}\, \Bar = \prod_{m \ge 0} X \otimes \Bar_m \lto X \otimes \Bar_0 \lto X \otimes R
\end{equation}
where the first map is the projection and the second is the product $\Re \lto R$. Next we show using the perturbation lemma that this map is a homotopy equivalence, and we give an explicit homotopy inverse in terms of Atiyah classes. In fact we give two homotopy inverses, one which is $R_2$-linear and the other left $R_1$-linear. We begin with the deformation retract arising from the fact that~$s$ and~$d$ are contracting homotopies for the differential $b'$.

\begin{remark}\label{remark:specialcaseinvert} The reader should keep in mind the special case where $S = k[z], R = k[x]$ and $X$ is the extension of scalars from $S \otimes_k R$ to $S \otimes_k \Re$ of a matrix factorisation $X'$ of $V - W$ over $k[x,z]$, with $V \in k[z], W \in k[x]$. Then $\pi$ is a morphism $X' \,\widehat{\otimes}_R\, \Bar \lto X'$ of linear factorisations of $V - W$.
\end{remark}

All deformation retract data in the following are at least $S$-linear.

\begin{lemma}\label{lemma:firstdefo} There are deformation retract data of $\mathbb{Z}_2$-graded complexes
\begin{equation}\label{eq:firstdefo1}
\xymatrix@C+2pc{
(X \otimes R, 0) \ar@<-1ex>[r]_-{\sigma_2} & (X \, \widehat{\otimes} \, \Bar, b') \ar@<-1ex>[l]_-{\pi}
}%
\!\!\!\xymatrix{%
{}\ar@(ur,dr)[]^{-d}
}%
\end{equation}
and
\begin{equation}\label{eq:firstdefo2}
\xymatrix@C+2pc{
(X \otimes R, 0) \ar@<-1ex>[r]_-{\sigma_1} & (X \, \widehat{\otimes} \, \Bar, b') \ar@<-1ex>[l]_-{\pi}
}%
\!\!\!\xymatrix{%
{}\ar@(ur,dr)[]^{-s}
}%
\end{equation}
where
\begin{equation}\label{eq:sigmsandd}
\sigma_2(e_i \otimes a ) = e_i \otimes (1 \otimes a) \, , \qquad \sigma_1(e_i \otimes a ) = e_i \otimes (a \otimes 1)\,.
\end{equation}
Moreover, in \eqref{eq:firstdefo1} every map is $R_2$-linear and in \eqref{eq:firstdefo2} every map is left $R_1$-linear.
\end{lemma}
\begin{proof}
The required identities $b' d + d b' = 1 - \sigma \pi$ and $b' s + s b' = 1 - \sigma \pi$ are given in (\ref{b'd+db'}).
\end{proof}

\begin{lemma}\label{lemma:smallpertde} The perturbation $\delta = D + d\widetilde{W} \times (-)$ is small on $X\,\widehat{\otimes} \, \Bar$ with respect to both of the above deformation retract data. That is, both $1 + \delta d$ and $1 + \delta s$ are invertible.
\end{lemma}
\begin{proof}
Let $h$ be $-d$ or $-s$. Because we are working with $\prod_{m \ge 0} X \otimes \Bar_m$ it is clear that the sum $\sum_{m \ge 0} (\delta h)^m$ converges as an operator on $X \, \widehat{\otimes}\, \Bar$ and this gives the desired inverse to $1 - \delta h$.
\end{proof}

In the notation of Section \ref{section:atiyahclasses} we can now describe the homotopy inverses of $\pi$ using Atiyah classes. Note that $X$ is a free $(S \otimes_k \Re)$-module so we are using the convention of Remark \ref{remark:linearatother}.

\begin{proposition}\label{prop:finalpertdefo} The morphism $\pi$ is an $R_2$-linear homotopy equivalence with inverse
\[
\sigma_\infty = \sum_{m \ge 0} (-1)^m \At_{2}(X)^m \sigma_2\,.
\]
More precisely, there is an $R_2$-linear deformation retract datum
\be\label{eq:finalpertdefo001}
\xymatrix@C+2pc{
(X \otimes R, D \otimes 1) \ar@<-1ex>[r]_-{\sigma_\infty} & (X \, \widehat{\otimes} \, \Bar, D + b' + d\widetilde{W} \times (-)) \ar@<-1ex>[l]_-{\pi}
}.
\ee
\end{proposition}
\begin{proof}
It follows from the perturbation lemma (Proposition \ref{prop:pertlemma}) with Lemmas~\ref{lemma:firstdefo} and~\ref{lemma:smallpertde} that
\[
\xymatrix@C+2pc{
(X \otimes R, b_\infty) \ar@<-1ex>[r]_-{\sigma_\infty} & (X \, \widehat{\otimes} \, \Bar, D + b' + d\widetilde{W} \times (-)) \ar@<-1ex>[l]_-{\pi_\infty}
}
\]
is a deformation retract datum, where with $\tau = \sum_{m \ge 0} (-1)^m (\delta d) \delta$ we have
\begin{align*}
\sigma_\infty = \sum_{m \ge 0} (-1)^m (d \delta)^m \sigma_2 \, , \qquad
\pi_\infty = \pi + \pi \tau h\, , \qquad
b_\infty = \pi \tau \sigma_2\,.
\end{align*}
Clearly $\pi \delta = \pi D$ and $\pi$ vanishes on $\delta d$, so $\pi \tau = \pi D$. It follows that $b_\infty = D \otimes 1$ and $\pi_\infty = \pi - \pi D d = \pi$. So there is a deformation retract datum \eqref{eq:finalpertdefo001}, where we may use $d^2 = 0$ and $d \sigma_2 = 0$ to write
\[
\sigma_\infty = \sum_{m \ge 0} (-1)^m \big[ d, \delta \big]^m \sigma_2 = \sum_{m \ge 0} (-1)^m [d, D + d\widetilde{W} \times (-)]^m \sigma_2\,.
\]
Expanding this yields $\sum_{m \ge 0} (-1)^m [d, D]^m \sigma_2$ plus terms involving factors of the form
\begin{equation}\label{eq:finalpertdefo2}
[d, d\widetilde{W} \times (-)] [d,D]^i \sigma_2
\end{equation}
for some $i \ge 0$. Applying $[d,D]^i \sigma_2$ to an element of $X \otimes R$ produces a tensor whose form component is of the type $da_0 \ldots da_n \otimes a_{n+1}$. By Lemma \ref{lemma:shortobs} below applying $[d, d\widetilde{W} \times (-)]$ to such a tensor yields zero, so all terms of the form (\ref{eq:finalpertdefo2}) must vanish and $\sigma_\infty$ is as given in the statement of the proposition.
\end{proof}

\begin{lemma}\label{lemma:shortobs} On $\Bar$ we have $[d, d\widetilde{W} \times (-)] = d \widetilde{W} \cdot d(-)$ and $[s, d\widetilde{W} \times (-)] = s(-) \cdot d\widetilde{W}$.
\end{lemma}
% see p.21 of (onct2), and p.4.1 of (coevc3) for the second statement
\begin{proof}
With $\alpha = d( W \otimes 1 - 1 \otimes W ) = dW \otimes 1$ and $\omega = a_0 da_1 \ldots da_n \otimes a_{n+1}$ we have
\begin{align*}
\alpha \times \omega &= a_0 dW da_1 \ldots da_n \otimes a_{n+1} - a_0 da_1 dW da_2 \ldots da_n \otimes a_{n+1}\\
& \qquad+ \ldots + (-1)^n a_0 da_1 \ldots da_n dW \otimes a_{n+1}\,.
\end{align*}
On the other hand
\begin{align*}
\alpha \times d\omega &= dW da_0 da_1 \ldots da_n \otimes a_{n+1} - da_0 dW da_1 \ldots da_n \otimes a_{n+1}\\
&\qquad+ \ldots + (-1)^{n+1} da_0 da_1 \ldots da_n dW \otimes a_{n+1}\,,
\end{align*}
from which it is clear that $d( \alpha \times \omega ) + \alpha \times d\omega = \alpha \cdot d\omega$. The second statement follows by a similar calculation.
\end{proof}

\begin{proposition}\label{prop:finalpertdefo} The morphism $\pi$ is a left $R_1$-linear homotopy equivalence with inverse
\[
\sigma_\infty = \sum_{m \ge 0} (-1)^m \At_{1}(X)^m \sigma_1\,.
\]
More precisely, there is a left $R_1$-linear deformation retract datum
\[
\xymatrix@C+2pc{
(X \otimes R, D \otimes 1) \ar@<-1ex>[r]_-{\sigma_\infty} & (X \, \widehat{\otimes} \, \Bar, D + b' + d\widetilde{W} \times (-)) \ar@<-1ex>[l]_-{\pi}
}\,.
\]
\end{proposition}
\begin{proof}
Using the second equation of Lemma \ref{lemma:shortobs} the proof is almost identical to that of Proposition \ref{prop:finalpertdefo}, so we omit it.
\end{proof}

%\begin{remark} In the situation of Remark \ref{remark:specialcaseinvert} we have shown that $\pi: X' \,\widehat{\otimes}_R\, \Bar \lto X'$ is an $R$-linear homotopy equivalence for the two possible $R$-actions on $X' \,\widehat{\otimes}_R\, \Bar$, the $R$-action on $X'$ which moves across the tensor to act as $R_1$ on $\Bar$, and the $R_2$-action on $\Bar$.
%\end{remark}

Now we turn to the Koszul stabilisation of the diagonal. Assume that $R = k[x_1,\ldots,x_n]$ and let $(\Delta, d_{\Delta})$ be the finite-rank matrix factorisation of $\widetilde{W}$ given in Section~\ref{subsec:bicatLG} where $d_{\Delta} = \delta_{+} + \delta_{-}$. We continue to assume that $X$ is a matrix factorisation as given in \eqref{eq:xisamf}. 

With $\pi_{\Delta}: \Delta \lto R$ the stabilisation morphism of \eqref{DeltaWstabmap} there is a morphism
\begin{equation}\label{eq:koszulpi}
\pi_{\Delta} = 1 \otimes \pi_{\Delta}: X \otimes \Delta \lto X \otimes R\,.
\end{equation}
This is compatible with the map $\pi$ of \eqref{eq:pibar} in the sense that there is a commutative diagram
\[
\xymatrix{
X \,\widehat{\otimes}\, \Bar \ar[dr]_-{\pi}\ar[rr]^{1 \otimes \Psi} & & X \otimes \Delta \ar[dl]^-{\pi_\Delta}\\
& X \otimes R
}
\]
in which every side is a homotopy equivalence: 

\begin{lemma}\label{lemma:koszulstab1} $\pi_\Delta$ is an $R_2$-linear homotopy equivalence and a left $R_1$-linear homotopy equivalence.
\end{lemma}
\begin{proof}
With $\Delta_j = \bigwedge^j (\bigoplus_{i=1}^n \Re \theta_i)$ there is a split exact sequence
\[
0 \lto \Delta_n \lto \cdots \lto \Delta_0 \lto R \lto 0
\]
and the splittings provide the $R_2$-linear (resp. left $R_1$-linear) $\sigma: R \lto \Delta$ and homotopy $h$ making
\[
\xymatrix@C+2pc{
(X \otimes R, 0) \ar@<-1ex>[r]_-{\sigma} & (X \otimes \Delta, 1 \otimes \delta_{-}) \ar@<-1ex>[l]_-{\pi_\Delta}
}%
\!\!\!\xymatrix{%
{}\ar@(ur,dr)[]^{-h}
}%
\]
into an $R_2$-linear deformation retract datum (resp. left $R_1$-linear). Moreover $\delta = D \otimes 1 + 1 \otimes \delta_{+}$ is a small perturbation and the perturbation lemma shows that $\pi_\Delta$ is a homotopy equivalence.
\end{proof}

\begin{corollary}\label{corollary:koszulstab2} An $R_2$-linear homotopy inverse to $\pi_\Delta$ is given by
\begin{equation}\label{eq:koszulstab2_1}
\pi^{-1}_2 = \sum_{l \ge 0} (-1)^l \Psi \At_{2}(X)^l \sigma_2\,,
\end{equation}
while a left $R_1$-linear homotopy inverse is given by
\begin{equation}\label{eq:koszulstab2_2}
\pi^{-1}_1 = \sum_{l \ge 0} (-1)^l \Psi \At_{1}(X)^l \sigma_1\,.
\end{equation}
where $\sigma_1, \sigma_2$ are as in \eqref{eq:sigmsandd}.
\end{corollary}

\subsection{Inverses of unit actions}\label{section:rhoandlambdainverse}

Let us now return to the problem of inverting the unit actions $\lambda$ and $\rho$, using the setting of Remark \ref{remark:specialcaseinvert}, so that $R = k[x]$, $S = k[z]$, $X$ is a matrix factorisation over $S \otimes_k R$ of $V - W$ and we take $X' = X \otimes_R \Re$ as the relevant matrix factorisation in the above. In this case $\pi_{\Delta}$ is precisely the right unit action $\rho: X \otimes_R \Delta_W \lto X$.

The relevant $S$-$R$-bimodule structure on $X \otimes_R \Delta_W$ is the one coming from the action of $S$ on $X$ and the right action of $R$ on $\Delta_W$ (which we have called $R_2$ above) and with respect to this bimodule structure the homotopy inverse of $\rho$ is given by \eqref{eq:koszulstab2_1}, that is,
\be\label{eq:formula_rhoinverse}
\rho^{-1} = \sum_{l \ge 0} (-1)^l \Psi \At_{2}(X)^l \sigma_2\,.
\ee
Here $\At_2(X)$ denotes the operator $[d, d_{X}]$ on $X \otimes_R \Bar$ with $d$ being extended to $X \otimes_R \Bar$ using a fixed homogeneous $S \otimes_k R$-basis $\{ e_i \}_i$ for $X$ as explained in Remark \ref{remark:linearatother}. The summands in this formula are composites of $S$-$R_2$-bilinear maps
\[
\xymatrix@C+2pc{
X \ar[r]^-{\sigma_2} & X \otimes_R \Re \ar[r]^-{\At_2(X)^l} & X' \otimes_R \Bar
}
\]
where $\sigma_2(e_i) = e_i \otimes (1 \otimes 1)$.

To give a formula for the homotopy inverse of $\lambda$, first let $\Delta_V$ denote the unit matrix factorisation of $\widetilde V$ and $X' = \Se \otimes_S X$. The unit action $\lambda: \Delta_V \otimes_S X \lto X$ can be written as a composite
\[
\xymatrix{
\Delta_V \otimes_S X \cong \Delta_V \otimes_{\Se} X' \ar[r]^-{\pi_{\Delta}} & S \otimes_{\Se} X' \cong X
}\,.
\]
Writing $\Se = S_1 \otimes_k S_2$, we seek an $S_1$-$R$-bilinear inverse to $\pi_{\Delta}$. It would suffice to invert
\be\label{eq:mashedkoalas}
X' \otimes_{\Se} \Delta_V \lto X' \otimes_{\Se} S
\ee
and then compose on both ends with swap isomorphisms, provided we keep in mind that we want an $S_1$-linear, not $S_2$-linear, inverse to \eqref{eq:mashedkoalas}. But switching the role of $R$ and $S$ in the above this is exactly what is provided by \eqref{eq:koszulstab2_2}. Postcomposing with the swap isomorphism has the effect of converting $\At_{1}$ into $\Atlarrow_{1}$, so the conclusion is that an $S_1$-$R$-bilinear homotopy inverse for $\lambda$ is
\be\label{eq:formulalambdainv}
\lambda^{-1} = \sum_{l \ge 0} (-1)^l \Psi \Atlarrow_{1}(X)^l \sigma_1\,.
\ee
Here $\Atlarrow_1(X)$ denotes the operator $[s, d_{X}]$ on $\gamma_*(\Bar) \otimes_S X$ and the summands in this formula are composites of $S_1$-$R$-bilinear maps
\[
\xymatrix@C+2pc{
X \ar[r]^-{\sigma_1} & \Se \otimes_S X \ar[r]^-{\At_1(X)^l} & \gamma_*(\Bar) \otimes_S X
}
\]
where $\sigma_1(e_i) = (1 \otimes 1) \otimes e_i$. Notice that we have shown that $\lambda, \rho$ are homotopy equivalences, and provided explicit inverses, for an arbitrary matrix factorisation $X$ (i.\,e.~not necessarily finite rank).

\subsection{The lifting problem}\label{section:liftingproblem}

The universal property of $\pi_{\Delta}: \Delta_W \lto R$ is that for any matrix factorisation $Y \in \hmf(\Re , \widetilde{W})$ and morphism of linear factorisations $\varphi: Y \lto R$ there is a unique (up to homotopy) morphism $\varphi_{\textup{lift}}$ making the following diagram commute up to homotopy
\be\label{eq:liftingproblem2}
\xymatrix{
Y \ar[dr]_-{\varphi} \ar@{.>}[rr]^-{\varphi_{\textup{lift}}} & & \Delta_W \ar[dl]^-{\pi_\Delta} \\
& R
} .
\ee
In this section we give an explicit formula for $\varphi_{\textup{lift}}$.

Having chosen a homogeneous basis $\{ e_i \}_{i}$ for $Y$, the formula is written in terms of the maps
\[
\xymatrix@C+2pc{
Y \ar[r]^-{\lAt_2(Y)^l} & \Bar \otimes_{\Re } Y \ar[r]^-{1 \otimes \varphi'} & \Bar \otimes_{\Re } \Re  \cong \Bar \ar[r]^-{\Psi} & \Delta_W
} .
\]
where $\varphi': Y \lto \Re $ is the $\Re $-linear map defined by $\varphi'(e_i) = 1 \otimes \varphi(e_i)$.
% see lipert3

\begin{proposition}\label{prop:liftingresult} 
In the above notation
\begin{equation}\label{eq:liftingresult}
\varphi_{\textup{lift}} = \sum_{l \ge 0} (-1)^l \Psi (1 \otimes \varphi') \lAt_{2}(Y)^l
\end{equation}
is a morphism of matrix factorisations making \eqref{eq:liftingproblem2} commute.
\end{proposition}
\begin{proof}
We apply Corollary \ref{corollary:koszulstab2} to see that the composite
\[
\xymatrix@C+2pc{
\Hom_{\Re }(Y, \Delta) \ar[r]^-{\cong}_-{\xi^{-1}} & Y^{\vee} \otimes_{\Re } \Delta \ar[r]^-{1 \otimes \pi_{\Delta}} & Y^{\vee} \otimes_{\Re } R \ar[r]^-{\cong}_-{\xi} & \Hom_{\Re }(Y, R)
}
\]
is a homotopy equivalence, with $\xi$ denoting the canonical isomorphisms (see Section \ref{section:canonicalmaps}). Evaluating the homotopy inverse $\xi \circ \pi_2^{-1} \circ \xi^{-1}$ on the cohomology class of $\varphi$ yields the map $\varphi_{\textup{lift}}$ defined by
\[
\varphi_{\textup{lift}} = \sum_i \sum_{l \ge 0} (-1)^m \xi \Psi \At_2(Y^\vee)^l \sigma_2( e_i^* \otimes \varphi(e_i) )\,.
\]
A straightforward computation yields (with $D = d_Y$)
\[
\varphi_{\textup{lift}}(e_j) = \sum_{l \ge 0}\sum_{i_0, i_1, \ldots, i_{l-1}} (-1)^{l|e_{i_0}| + l} \Psi\left( d( D_{i_{l-1},j}) d( D_{i_{l-2},i_{l-1}} ) \ldots d( D_{i_0, i_1} ) \cdot (1 \otimes \varphi(e_{i_0}) ) \right)
\]
which by \eqref{eq:leftatiyahexp} agrees with the right-hand side of \eqref{eq:liftingresult}.
\end{proof}

%\begin{remark} As the proof makes clear, we could have made use of $\pi^{-1}_1$ instead of $\pi^{-1}_2$ in the construction of $\varphi_{\textup{lift}}$, so that an alternative chain level representative of this homotopy class could be given using $\lAt_{1}$ instead of $\lAt_{2}$.
%\end{remark}

\section{Evaluation and coevaluation}\label{sec:derivcoeval}

Every $1$-morphism in $\LG$ has both a left and right adjoint. Specifically, if a $1$-morphism $W \lto V$ is represented by a finite rank matrix factorisation $X$ of $V - W$ over $k[x,z]$, where $W\in k[x] = k[x_1,\ldots,x_n]$ and $V\in k[z] = k[z_1,\ldots,z_m]$ are potentials, then the $1$-morphisms
\[
X^\dual = X^\vee[n], \qquad {}^\dual X = X^\vee[m]
\]
are respectively the right and left adjoints of $X$ in $\LG$. These are objects of $\hmf(k[x,z], W-V)$, with the differentials spelled out in \eqref{eq:differentials_adjoints}. To prove that these $1$-morphisms are adjoint to $X$ we have to exhibit two pairs of evaluation and coevaluation maps, and prove that they satisfy the Zorro moves~\eqref{Zorros} and~\eqref{otherZorros}.

In Sections \ref{subsec:derivcoeval} and \ref{subsec:eval} below we define evaluation and coevaluation maps 
\begin{align}
\coev_X &: \Delta_V \lto X \otimes_{k[x]} {}^\dual X
\, , \qquad
\eval_X: {}^\dual X \otimes_{k[z]} X \lto \Delta_W \label{eq:derivcoeval1}\\
\widetilde\coev_X&: \Delta_W \lto X^\dual \otimes_{k[z]} X
\, , \qquad
\widetilde\eval_X: X \otimes_{k[x]} X^\dual \lto \Delta_V \label{eq:derivcoeval2}
\end{align}
and derive explicit formulas for these morphisms in terms of Atiyah classes. We already give these formulas below; the rest of the section will be spent justifying that these expressions do in fact define chain maps representing canonical morphisms \eqref{eq:derivcoeval1} and \eqref{eq:derivcoeval2} in the homotopy category of matrix factorisations. In Section~\ref{sec:Zorro} we will prove that these maps determine adjunctions ${}^\dual X \dashv X$ and $X \dashv X^\dual$ by proving the Zorro moves.

While the homotopy classes of these morphisms are canonical the formulas depend on a choice of homogeneous basis $\{ e_i \}_{i}$ for $X$ with dual basis $\{ e_i^* \}_{i}$. Representatives for the coevaluation morphisms are given by the chain maps
\begin{gather}
\widetilde\coev_X(\gamma) = \sum_{l \ge 0} \varepsilon \left( \gamma \wedge  (-1)^{l+nl} \Psi \Atlarrow_2(X^\dual)^l( \iota_X ) \right)\,, \label{eq:coev10}\\
\iota_X = \sum_{j} (-1)^{|e_j|} e_j^* \otimes e_j \in X^\dual \otimes_{k[z]} X \nonumber
\end{gather}
and
\begin{gather}
\coev_X( \gamma ) = \sum_{l \ge 0} \varepsilon \left( \gamma \wedge (-1)^{l+ml} \Psi \Atlarrow_2(X)^l( \iota'_X ) \right)\,,\label{eq:coev102}\\
\iota'_X = \sum_j (-1)^{m|e_j|} e_j \otimes e_j^* \in X \otimes_{k[x]} {}^\dual X\,. \nonumber
\end{gather}
where $\varepsilon$ is as defined in \eqref{eq:vareps}. The Atiyah classes involved in \eqref{eq:coev10} and \eqref{eq:coev102} are actually those of the $k[x]$-bimodule $X^\dual \otimes_{k[z]} X$ and the $k[z]$-bimodule $X \otimes_{k[x]} X^\dual$, respectively, but these simplify to the Atiyah classes given in the formulas; see the discussion preceeding Proposition \ref{prop:coev1}.

A representative for the evaluation morphism is given by
\be\label{eq:constructevalmap00}
\widetilde\eval_X( \eta \otimes \nu ) = \sum_{l \ge 0} (-1)^{\globalsigneval+l+n|\eta|} \Res_{k[x]/k} \left[ \frac{ \Psi \left( \lAt_2(X)^l( \Lambda^{(x)} \eta ), \nu \right) \, \underline{\operatorname{d}\!x}}{\partial_{x_1}W, \ldots, \partial_{x_n} W} \right]
\ee
where $\Lambda^{(x)} = (-1)^n \partial_{x_1}(d_X) \ldots \partial_{x_n}(d_X)$, $\underline{\operatorname{d}\!x} = \ud x_1 \ldots \ud x_n$ and $(\eta',\nu)$ denotes the evaluation of $\nu$ on an element $\eta'$ of $X$, with an appropriate Koszul sign and taking into account the fact that $\nu$ and $\eta$ involve different copies of the $z$-variables. For the precise statement see Proposition \ref{prop:constructevalmap}. 

In the other evaluation the Atiyah class of $X^{\vee}$ is applied to the functional $\nu \circ \Lambda$, and the resulting functional-valued form is evaluated on $\eta$. The formula is
\be\label{eq:constructevalmap01}
\eval_X( \nu \otimes \eta ) = \sum_{l \ge 0} (-1)^{\binom{m+1}{2}+l+m|\nu|} \Res_{k[z]/k} \left[ \frac{ \Psi \left( \lAt_2(X^{\vee})^l( \nu \circ \Lambda^{(z)} ), \eta\right) \, \underline{\operatorname{d}\!z}}{\partial_{z_1}V, \ldots, \partial_{z_m} V} \right]
\ee
where $\Lambda^{(z)} = \partial_{z_1}(d_X) \ldots \partial_{z_m}(d_X), \underline{\ud z} = \ud z_1 \ldots \ud z_m$. Here $(\nu', \eta)$ denotes the evaluation of a functional $\nu'$ on $\eta$, without signs but distinguishing between the two copies of the $x$-variables involved; a more careful explanation is given after Proposition \ref{prop:constructevalmapnontilde}. 

Throughout we write $R = k[x], S = k[z]$ and $\Re  = R_1 \otimes_k R_2$ with $R_i = R$ and $\Se  = S_1 \otimes_k S_2$ with $S_i = S$. Then $\Delta_W, X^\dual \otimes X, {}^\dual X \otimes X$ are $\Re $-linear factorisations of $\widetilde{W} = W \otimes 1 - 1 \otimes W$ and $\eval_X, \widetilde\coev_X$ are $\Re $-linear, while $\Delta_V$ and $X \otimes X^\dual, X \otimes {}^\dual X$ are $\Se$-linear factorisations of $\widetilde{V} = V \otimes 1 - 1 \otimes V$ and $\coev_X, \widetilde\eval_X$ are $\Se$-linear. The differentials on $\Delta_V, \Delta_W$ are given by \eqref{DeltaW}.

Throughout we make use of the canonical maps from Section \ref{section:canonicalmaps}.

\subsection{Coevaluation}\label{subsec:derivcoeval}

In this section we present the coevaluation morphism $\widetilde\coev_X$ and the derivation of its explicit chain level representative. Afterwards we consider the other coevaluation $\coev_X$, but since the derivation is almost identical we leave most of the details in this case to the reader.

\begin{proposition}\label{prop:isogivescoev} There is a canonical homotopy equivalence of $\mathbb{Z}_2$-graded $k$-complexes
\begin{equation}
\Hom_{\Re }( \Delta_W, X^\dual \otimes_{S} X ) \lto \Hom_{R \otimes_k S}(X,X)\,. \label{eq:isogivescoev}
\end{equation}
\end{proposition}

\begin{definition}\label{def:coeval} We define $\widetilde\coev_X$ to be the morphism in the category $\HMF(\Re, \widetilde{W})$ whose cohomology class maps to $1_X$ under the quasi-isomorphism \eqref{eq:isogivescoev}.
\end{definition}

\begin{proof}[Proof of Proposition \ref{prop:isogivescoev}]
Let $\Delta_W'$ denote the matrix factorisation of $- \widetilde{W}$ with the same underlying graded free module as $\Delta_W$ but the modified differential $d_{\Delta'} = - \delta_{+} + \delta_{-}$. This approximates the diagonal as a matrix factorisation of $- \widetilde{W}$ in the same way that $\Delta_W$ approximates it as a factorisation of $\widetilde{W}$. Also let~$K$ denote the $\mathbb{Z}_2$-graded complex with the same underlying graded module as $\Delta_W$, but the differential~$\delta_{-}$, so~$K$ is the usual Koszul complex of $x_1 - x'_1, \ldots, x_n - x'_n$.

The product in the exterior algebra gives rise to a morphism of $\mathbb{Z}_2$-graded complexes
\[
\Delta_W' \otimes_{\Re} \Delta_W \lto K\,.
\]
Composing with the morphism $\varepsilon: K \lto \Re[n]$ from~\eqref{eq:vareps} and taking the adjoint, we obtain the isomorphism of matrix factorisations
\be
\zeta: \Delta_W' \lto \Delta_W^\vee [n]\, , \qquad \omega \lmt \varepsilon( \omega \wedge - )\,.
\ee
Combining this with various canonical maps and the morphism $\pi: \Delta_W' \lto (\Delta_W')_0 = \Re \lto R$ we have the following diagram of $R_2$-linear morphisms:
\begin{equation}
\xymatrix{
\Hom_{\Re}( \Delta_W, X^\vee[n] \otimes_{S} X)\\
\Delta_W^\vee \otimes_{\Re} \left( X^\vee[n] \otimes_S X \right) \ar[u]^-{\cong}_-{\xi}\\
\Delta_W^\vee[n] \otimes_{\Re} \left( X^\vee \otimes_{S} X \right) \ar[u]^-{\cong}\\
\Delta_W' \otimes_{\Re} \left( X^\vee \otimes_S X \right) \ar[u]^-{\cong}_-{\zeta \otimes 1}\\
\left( X^\vee \otimes_{S} X \right) \otimes_{\Re} \Delta_W' \ar[u]^-{\cong}_-{\textup{swap}} \ar[d]^-{\pi}\\
\left( X^\vee \otimes_{S} X \right) \otimes_{\Re} R \ar[d]_-{\cong}\ar[d]^-{\kappa}\\
\Hom_{R \otimes_k S}(X,X)
}\label{eq:isogivescoev2}
\end{equation}
where for $f \in X^\vee$ and $g,g' \in X$ we define $\kappa( f \otimes g )(g') = (-1)^{|f||g|} f(g') \cdot g$. To complete the proof we need only to show that the step marked $\pi$ is a homotopy equivalence. But this is a consequence of Lemma \ref{lemma:koszulstab1}.
\end{proof}

To give a representative chain map for the coevaluation we begin with $1_X$ at the bottom of (\ref{eq:isogivescoev2}) and apply the various maps in turn; the only nontrivial step will be to apply the homotopy inverse for $\pi$ found in Section \ref{section:pertandhtpy}. The two different choices of inverse provided there will lead to two different (but homotopic) chain level representatives for the coevaluation.

With $\reprod$ as in (\ref{eq:attau}) we consider on the tensor product $\reprod_*(\Bar_{R/k}) \otimes_{\Re} ( X^\vee \otimes_{S} X)$ the operator
\[
\Atlarrow_{2}(X^\vee \otimes X) = [d, d_{X^\vee \otimes X}] = [d, d_{X^\vee}]
\]
with the second equality holding because $d$ is $R_2$-linear. For this reason we abuse notation and write $\Atlarrow_2(X^\vee)$ for $\Atlarrow_{2}(X^\vee \otimes X)$. Consider the map
\[
\xymatrix@C+1pc{
X^\vee \otimes_{S} X \ar[rr]^-{\Atlarrow_2(X^\vee)^m} & & \reprod_*(\Bar_{R/k}) \otimes_{\Re} ( X^\vee \otimes_{S} X ) \ar[r]^-{\Psi} & \Delta_W \otimes_{\Re} (X^\vee \otimes_{S} X)
}
\]
which we can apply to the tensor $\iota_X = \sum_{j} (-1)^{|e_j|} e_j^* \otimes e_j$.

\begin{proposition}\label{prop:coev1} A representative for the coevaluation morphism is the chain map \eqref{eq:coev10}.
\end{proposition}
\begin{proof}
We lift the identity $1_X$ upwards through the sequence of quasi-isomorphisms in \eqref{eq:isogivescoev2}, noting that $\iota_X = \kappa^{-1}(1_X)$. To take the inverse image of $\iota_X$ under the map induced by $\pi$ on cohomology, we may use either of the homotopy inverses $\pi_1^{-1}, \pi_2^{-1}$ of Corollary \ref{corollary:koszulstab2}. Using the second yields
\begin{equation}\label{eq:coev11}
\pi^{-1}_2 \kappa^{-1}(1_X) = \sum_{m \ge 0} (-1)^m \Psi \At_2(X^{\vee})^m( \iota_X )\,.
\end{equation}
Applying the swap isomorphism has the effect of exchanging $\At_2$ for $\Atlarrow_2$ in this formula, so that the cohomology class in $\Delta_W' \otimes_{\Re} (X^\vee \otimes_S X)$ mapping to $1_X$ is $\sum_{l \ge 0} (-1)^l \Psi \Atlarrow_2(X^{\vee})^l( \iota_X )$.

A tensor $\omega \otimes \alpha$ in $\Delta_W' \otimes_{\Re} (X^\vee \otimes_S X)$ is sent under the remaining maps in \eqref{eq:isogivescoev2} to the function $(-1)^{n(|\omega| + n)} \xi( \varepsilon( \omega \wedge - ) \otimes \alpha )$. This map sends a form $\gamma \in \Delta_W$ to
\begin{align}
\gamma &\mapsto (-1)^{n(|\omega| +n ) + (|\omega| + n)(|\alpha|+n)} \varepsilon( \omega \wedge \gamma ) \cdot \alpha\label{eq:signcalccoeval}\\
&= (-1)^{|\omega||\alpha| + n|\alpha| + |\omega||\gamma|} \varepsilon( \gamma \wedge \omega ) \cdot \alpha\,. \nonumber
\end{align}
This vanishes unless $|\gamma| + |\omega| = n$, so the final sign is $(-1)^{|\omega||\alpha| + n|\alpha| + n|\omega| + |\omega|}$. Since $\iota_X$ has degree zero the element $\sum_{m \ge 0} (-1)^m \Psi \Atlarrow_2(X^{\vee})^m( \iota_X )$ can be written as a sum of tensors $\omega \otimes \alpha$ with $|\omega| = |\alpha| = m$, in which case the sign on \eqref{eq:signcalccoeval} is just $+1$ and the formula \eqref{eq:coev10} follows.
\end{proof}

There is an alternative presentation which we will need. On $( X^\dual \otimes_S X) \otimes_{\Re } \Bar$ consider
\[
\At_{1}(X^\dual \otimes X) = [s, d_{X^\dual \otimes X}] = [s, d_X]
\]
with the second equality holding because $s$ is $R$-linear. We write $\At_{1}(X)$ for this operator. 

\begin{proposition} A representative for the coevaluation morphism is the chain map
\[
\widetilde\coev^R_X(\gamma) = \varepsilon \left( \sum_{l \ge 0} (-1)^{l+ln} \Psi \At_{1}(X)^l( \iota_X ) \wedge \gamma \right)\,.
\]
\end{proposition}

We conclude this section by explaining how one defines, and derives an explicit formula for, the other coevaluation. To begin with one constructs a canonical homotopy equivalence
\be\label{eq:othercoev11}
\Hom_{S^e}(\Delta_V, X \otimes_R X^{\vee}[m]) \lto \Hom_{R \otimes_k S}(X, X)
\ee
in the same way that we constructed the homotopy equivalence of Proposition \ref{prop:isogivescoev}. The coevaluation $\coev_X$ is defined to be the morphism $\Delta_V \lto X \otimes_R X^{\vee}[m]$ in $\HMF(S^e, \widetilde{V})$ whose cohomology class is the preimage of the identity under the quasi-isomorphism \eqref{eq:othercoev11}. Once again using the homotopy inverses computed in Corollary \ref{corollary:koszulstab2} one finds two different chain maps lifting the identity. 

The first representative is \eqref{eq:coev102} above: let us state this as a

\begin{proposition} A representative for $\coev_X$ is the chain map \eqref{eq:coev102}.
\end{proposition}

The second choice of homotopy inverse yields

\begin{proposition} A representative for $\coev_X$ is the chain map
\[
\coev^R_X( \gamma ) = \sum_{l \ge 0} (-1)^{l + lm} \varepsilon \left( \Psi \At_1(X^{\vee})^l( \iota'_X ) \wedge \gamma \right)\,.
\]
\end{proposition}

\subsection{Evaluation}\label{subsec:eval}
% see (eval)

In this section we construct the evaluation $\widetilde\eval_X$ by writing down a morphism $\widetilde\eval_0: X \otimes_R X^\dual \lto S$ and then lifting this via perturbation to a morphism into $\Delta_V$. The other evaluation $\eval_X$ is defined in a similar way, and will be treated briefly at the end of the section.

The partial derivatives $f_i = \partial_{x_i} W$ act null-homotopically on $X$ and we let $\lambda_i \in \Hom_{R \otimes_k S}(X,X)$ denote a degree-one map with $[d_X, \lambda_i] = f_i \cdot 1_X$. The construction will turn out to be independent of this choice up to homotopy, although this is not completely obvious; see Remark \ref{remark:indeptlambda}. We also set $\Lambda^{(x)} = \lambda_1 \ldots \lambda_n$. For example, taking $\lambda_i = -\partial_{x_i} d_X$ would do since by the Leibniz rule
\[
\partial_{x_i}(d_X) d_X + d_X \partial_{x_i}(d_X) = \partial_{x_i}( d_X^2 ) = - \partial_{x_i} W\,.
\]
However it will be important to allow other choices of null-homotopy.

To begin with there is a canonical map (an $\Se $-linear morphism of linear factorisations of $\widetilde{V}$)
\be\label{eq:defneval0_00}
X \otimes_R X^{\vee}[n] \cong (X \otimes_R X^{\vee})[n] \lto \left( X \otimes_R X^{\vee} \right)[n] \otimes_R \bar{R} \cong (\bar{X} \otimes_R \bar{X}^{\vee})[n]
\ee
where we write $\bar{R} = R/(f_1,\ldots,f_n)$ and $\bar{X} = X \otimes_R \bar{R}$. Then we compose with
\be\label{eq:defneval0_0}
\xymatrix@C+2.5pc{
(\bar{X} \otimes_R \bar{X}^{\vee})[n] \ar[r]^-{\Lambda^{(x)} \otimes 1} & \bar{X} \otimes_R \bar{X}^{\vee}
}
\ee
which is closed because $\Lambda^{(x)}$ is a closed map $X[n] \lto X$ modulo the $f_i$. Finally we compose with
\be\label{eq:defneval0_1}
\xymatrix@C+1.5pc{
\bar{X} \otimes_R \bar{X}^{\vee} \ar[r]^-{\can} & S \otimes_{S^e} ( \bar{X} \otimes_R \bar{X}^{\vee} ) \cong \Hom_{R \otimes_k S}(\bar{X}, \bar{X}) \ar[r]^-{\str} & \bar{R} \otimes_k S \ar[r]^-{\Res} & S
}
\ee
where the last map marked is the $S$-linear residue symbol (see Section~\ref{section:residuebackground})
\[
\Res_{R/k} \left[ \frac{ (-) \, \underline{\operatorname{d}\!x}}{\partial_{x_1}W, \ldots, \partial_{x_n} W} \right]: \bar{R} \otimes_k S \lto S\,.
\]
We note that the first map in~\eqref{eq:defneval0_1} and the second in \eqref{eq:defneval0_1} are as defined in Section \ref{section:canonicalmaps}.

\begin{definition} $\widetilde\eval_0$ is $(-1)^{\globalsigneval}$ times the composite of \eqref{eq:defneval0_00}, \eqref{eq:defneval0_0} and \eqref{eq:defneval0_1}.
\end{definition}

The global sign is determined by the Zorro move and the signs on the coevaluation. 

\begin{lemma}\label{lemma:morphismevalzero} $\widetilde\eval_0$ is a morphism of linear factorisations of $\widetilde{V}$ over $\Se $, and for $\eta \in X, \nu \in X^\vee$
\begin{align}
\widetilde\eval_0( \eta \otimes \nu ) &= (-1)^{\globalsigneval+n|\eta|} \Res_{R/k} \left[ \frac{ \str( \Lambda^{(x)} \circ \eta \circ \nu) \, \underline{\operatorname{d}\!x}}{\partial_{x_1}W, \ldots, \partial_{x_n} W} \right]\\
&= (-1)^{\globalsigneval+n|\eta| + |\nu|} \Res_{R/k} \left[ \frac{ \nu \Lambda^{(x)}( \eta ) \, \underline{\operatorname{d}\!x}}{\partial_{x_1}W, \ldots, \partial_{x_n} W} \right]
\end{align}
\end{lemma} 

There is a stabilisation morphism $\pi_\Delta: \Delta_V \lto S$, and while $X \otimes_R X^\dual$ is not free of finite rank over $\Se $, it is a direct summand of a finite-rank matrix factorisation in $\HMF(\Se , \widetilde{V})$. The unique lifting statement of Proposition \ref{prop:liftingresult} can be extended to summands (for the argument, see the proof of the proposition below) so there is up to homotopy a unique morphism of matrix factorisations $\widetilde\eval_X$ making the following diagram commute:
\be\label{eq:defineeval}
\xymatrix{
X \otimes_R X^\dual \ar[dr]_-{\widetilde\eval_0} \ar@{.>}[rr]^-{\widetilde\eval_X} & & \Delta_V \ar[dl]^-{\pi_\Delta} \\
& S
}\,.
\ee

\begin{definition} $\widetilde\eval_X$ is the unique morphism in $\HMF(\Se , \widetilde{V})$ making \eqref{eq:defineeval} commute.
\end{definition}

The formula for $\widetilde\eval_X$ involves the following $S^e \otimes_k R$-linear map
\begin{gather*}
(-,-): X \otimes_R X^{\vee} \lto S^e \otimes_k R\,,\\
( e_i, e_j^* ) = (-1)^{|e_i||e_j|} e_j^*( e_i ) = (-1)^{|e_i|} \delta_{ij}\,.
\end{gather*}
This pairing is simply evaluation (with an appropriate Koszul sign) modulo the fact that $X$ and its dual $X^{\vee}$ involve different copies of the ring $S$.

\begin{proposition}\label{prop:constructevalmap} A representative for $\widetilde\eval_X$ is the chain map \eqref{eq:constructevalmap00}.
\end{proposition}

To be clear, the numerator of the residue in \eqref{eq:constructevalmap00} is the image of $\Lambda^{(x)}(\eta) \otimes \nu$ under the map
\[
\xymatrix@C+1pc{
X \otimes_R X^{\vee} \ar[r]^-{\lAt_2(X)^l} & \Bar \otimes_{S^e}(X \otimes_R X^{\vee} ) \ar[r]^-{1 \otimes (-,-)} & \Bar \otimes_{S^e} ( S^e \otimes_k R ) \cong \Bar \otimes_k R \ar[r]^-{\Psi \otimes 1} & \Delta_V \otimes_k R
}
\]
where $\lAt_2(X) = [d, d_X], \Bar = \Bar_{S/k}$. The residue integrates out the $R$, leaving an element of $\Delta_V$.

As has already been mentioned, if $X \otimes_R X^\dual$ were finite-rank we could apply Proposition \ref{prop:liftingresult} directly to deduce a formula for the lifting $\widetilde\eval_X$. As this is not the case, we have to be a little indirect and appeal to the idempotent pushforward construction of \cite{dm1102.2957}. Specifically, with $\bar{R} = R/(f_1,\ldots,f_n)$, Theorem \ref{??} of \textsl{loc.cit.} shows that there is a diagram
\be\label{eq:idempotentpushdia}
\xymatrix@C+2pc{
X \otimes_R X^\vee[n] \cong (X \otimes_R X^\vee)[n] \ar@<-1ex>[r]_-{\vartheta} & \bar{X} \otimes_{\bar{R}} \bar{X}^\vee := (X \otimes_R X^\vee) \otimes_R \bar{R} \ar@<-1ex>[l]_-{\psi}
}
\ee
in $\HMF(\Se , \widetilde{V})$ with $\psi \circ \vartheta = 1$ in the homotopy category and $\vartheta = \Lambda^{(x)} \otimes 1$. Observe that since $W$ is a potential, $R/(f_1,\ldots,f_n)$ is a finitely generated free $k$-module and $\bar{X} \otimes_{R} \bar{X}^\vee$ is therefore a finite-rank matrix factorisation of $\widetilde{V}$ over $S^e$.

\begin{proof}[Proof of Proposition \ref{prop:constructevalmap}] To lift $\widetilde\eval_0: X \otimes_R X^{\vee}[n] \lto S$, consider the commutative diagram
\be\label{eq:proofconstructevalmap}
\xymatrix@C+2pc{
\Hom_{\Se }( X \otimes_R X^\vee[n], \Delta ) \ar[d]_{\pi^\bullet_\Delta} \ar@<-1ex>[r]_-{\psi_\bullet} & \Hom_{\Se }(\bar{X} \otimes_{R} \bar{X}^\vee, \Delta ) \ar@<-1ex>[l]_-{\vartheta_\bullet} \ar[d]^{\pi^\bullet_\Delta}\\
\Hom_{\Se }( X \otimes_R X^\vee[n], S ) \ar@<-1ex>[r]_-{\psi_\bullet} & \Hom_{\Se }( \bar{X} \otimes_{R} \bar{X}^\vee, S) \ar@<-1ex>[l]_-{\vartheta_\bullet}
}
\ee
where a ``$\bullet$'' as a superscript indicates postcomposition and as a subscript it denotes precomposition, and we leave the isomorphism $X \otimes_R X^\vee[n] \cong (X \otimes_R X)[n]$ implicit throughout. By Proposition \ref{prop:liftingresult} the right-hand vertical map is a homotopy equivalence and therefore so is the left-hand vertical map. This justifies why in \eqref{eq:defineeval} there is a unique morphism $\widetilde\eval_X$ making the diagram commute: it is the image in cohomology of $\widetilde\eval_0$ under the homotopy inverse of $\pi_{\Delta}^\bullet$.

Let $\widetilde\eval'_0$ denote $(-1)^{\globalsigneval}$ times the morphism in \eqref{eq:defneval0_1} so that $\widetilde\eval_0 = \vartheta_\bullet( \widetilde\eval'_0 )$, and let $\widetilde\eval'$ denote the morphism $\bar{X} \otimes_{\bar R} \bar{X}^\vee \lto \Delta$ lifting $\widetilde\eval'_0$ which is produced by Proposition \ref{prop:liftingresult}. To run the lifting construction we need to fix an $\Se $-basis for $\bar{X} \otimes_{\bar R} \bar{X}^\vee$. The basis we pick is $\{ g_\alpha e_i \otimes e_j ^* \}_{\alpha, i, j}$ where $g_\alpha$ gives a $k$-basis of $\bar{R}$. We define $\widetilde\eval_X = \vartheta_\bullet( \widetilde\eval' ) = \widetilde\eval' \circ (\Lambda^{(x)} \otimes 1)$. By construction this morphism makes \eqref{eq:defineeval} commute, and it just remains to compute it explicitly.

The statement of Proposition \ref{prop:liftingresult} gives us
\[
\widetilde\eval_X = \sum_{l \ge 0} (-1)^{l} \Psi \circ (1 \otimes \widetilde\eval''_0) \circ \lAt(X \otimes X^\vee)^l \circ ( \Lambda^{(x)} \otimes 1 )
\]
where $\widetilde\eval''_0: \bar{X} \otimes_{\bar{R}} \bar{X}^\vee \lto \Se $ is the $\Se $-linear map defined by $\widetilde\eval''_0( g_\alpha e_i \otimes e_j^* ) = 1 \otimes \widetilde\eval'_0( f_\alpha e_i \otimes e_j^* )$. In $d_{X \otimes X^\vee} = d_X \otimes 1 + 1 \otimes d_{X^\vee}$ only $d_X$ involves polynomials in $R_1$, so $\lAt(X \otimes X^\vee) = \lAt(X)$. We compute using \eqref{eq:leftatiyahexp} that for $g \in R$ (writing $D = d_X$, and putting in the sign $n|e_i|$ from the first map of \eqref{eq:defneval0_00})
\begin{align*}
\widetilde\eval_X( g e_i \otimes e_j^* ) &= \sum_{l \ge 0} (-1)^{l+n|e_i|} \Psi (1 \otimes \widetilde\eval''_0)\left( g \sum_k \lAt(X)^l( \Lambda^{(x)}_{ki} e_k \otimes e_j ^*) \right)\\
&= \sum_{l \ge 0} (-1)^{l+n|e_i|} \Psi (1 \otimes \widetilde\eval''_0)\left( g \sum_{k, k_1, \ldots, k_l} \Lambda^{(x)}_{ki} \cdot d( D_{k_1k} ) \ldots d( D_{k_l, k_{l-1}} ) \otimes e_{k_l} \otimes e_j^* \right)
\end{align*}
Applying $\widetilde\eval''_0$ involves taking the supertrace $\str( e_{k_l} \circ e_j^* ) = (-1)^{|e_j|}\delta_{jk_l}$ so that the above equals
\[
\sum_{l \ge 0} \sum_{k,k_1,\ldots,k_{l-1}} (-1)^{\globalsigneval+l+n|e_i|+|e_j|} \Lambda^{(x)}_{ki} \Psi( d(D_{k_1,k}) \ldots d(D_{j,k_{l-1}}) ) \Res_{R/k} \left[ \frac{ g \,  \underline{\operatorname{d}\!x}}{\partial_{x_1}W, \ldots, \partial_{x_n} W} \right]
\]
which agrees with \eqref{eq:constructevalmap00} for $\eta = g e_i$ and $\nu = e_j^*$, completing the proof.
\end{proof}

\begin{remark}\label{remark:indeptlambda} We claim that the morphism $\widetilde\eval_X$ is canonical in the sense that it is independent up to homotopy of the choices of null-homotopy $\lambda_i$. It clearly suffices to argue that $\widetilde\eval_0$ is independent of these choices, and for this we use a variant of the argument in \cite[Appendix A]{dm1102.2957}. 

Write $D(-)$ for $[d_X, -]$ and suppose $\lambda'_1$ is another degree-one map with $D(\lambda'_1) = f_1 \cdot 1_X$. Then as $S$-linear morphisms of complexes $\Hom_{R \otimes_k S}(X,X) \lto S$ we have, writing $\Lambda_{>1} = \lambda_2 \ldots \lambda_n$
\begin{align*}
&\Res_{R/k} \left[ \frac{ \str( \Lambda^{(x)} \circ -) \, \underline{\operatorname{d}\!x}}{f_1 \ldots f_n} \right]\\
&= \Res_{R/k} \left[ \frac{ \str( f_1 \lambda_1 \Lambda_{> 1} \circ -) \, \underline{\operatorname{d}\!x}}{f_1^2, f_2, \ldots, f_n} \right]\\
&= \Res_{R/k} \left[ \frac{ \str\big( D( \lambda_1') \lambda_1 \Lambda_{> 1} \circ -\big) \, \underline{\operatorname{d}\!x}}{f_1^2, f_2, \ldots, f_n} \right]\\
&= \Res_{R/k} \left[ \frac{ \str\big( \lambda_1' D( \lambda_1 \Lambda_{> 1} \circ -)\big) \, \underline{\operatorname{d}\!x}}{f_1^2, f_2, \ldots ,f_n} \right]\\
&= \Res_{R/k} \left[ \frac{ \str\big( \lambda_1' f_1 \Lambda_{> 1} \circ (-) - \lambda_1' \lambda_1 D( \Lambda_{> 1} ) \circ (-) + (-1)^n \lambda_1'\Lambda^{(x)} \circ D(-)\big) \, \underline{\operatorname{d}\!x}}{f_1^2, f_2, \ldots, f_n} \right]
\end{align*}
The term involving $D( \Lambda_{> 1} )$ belongs to the submodule $(f_2,\ldots,f_n)\Hom_{R \otimes_k S}(X,X)$, and since the $f_i$ with $i > 2$ annihilate with the residue we are left with
\begin{align*}
&= \Res_{R/k} \left[ \frac{ \str( \lambda_1' \Lambda_{> 1} \circ -) \, \underline{\operatorname{d}\!x}}{f_1, \ldots, f_n} \right] + (-1)^n\Res_{R/k} \left[ \frac{ \str\big(\lambda_1' \Lambda^{(x)} \circ D(-)\big) \, \underline{\operatorname{d}\!x}}{f_1^2, \ldots, f_n} \right]
\end{align*}
The second summand is a null-homotopic functional, so we have proven that $\widetilde\eval_0$ is homotopic to the map defined using $\lambda_1' \Lambda_{> 1}$ in place of $\Lambda^{(x)}$, as claimed. The argument for general $i$ is the same. In much the same way one can show that up to a sign $\widetilde\eval_X$ is also independent of the way in which we order the $\lambda_i$, but we will not need this.
\end{remark}

We conclude this section with a brief explanation of the derivation of $\eval_X$. The partial derivatives $\partial_{z_i} V$ act null-homotopically on $X$ and we let $\lambda_i^{(z)} \in \Hom_{R \otimes_k S}(X,X)$ denote a degree-one map with $[d_X, \lambda_i^{(z)}] = \partial_{z_i} V \cdot 1_X$. For example $\lambda_i^{(z)} = \partial_{z_i} d_X$ would do. We set $\Lambda^{(z)} = \lambda_1^{(z)} \ldots \lambda_m^{(z)}$.

We define an operator $\lambda_{i \bullet}^{(z)}$ on $X^{\vee}$ by
\[
\lambda_{i \bullet}^{(z)} {}(\alpha) = (-1)^{|\alpha|} \lambda_i^{(z)} \circ \alpha\,.
\]
It is easily checked that $[d_{X^\vee}, \lambda_{i \bullet}^{(z)}] = \partial_{z_i} V \cdot 1_{X^\vee}$, so this is a null-homotopy for the action of $\partial_{z_i} V$. With $\bar{S} = S/(\partial_{z_1} V, \ldots, \partial_{z_m} V)$ there is an $R^e$-linear morphism of linear factorisations of $\widetilde{W}$
\be\label{eq:ordinaryeval00}
\xymatrix{
X^{\vee}[m] \otimes_S X \ar[r] & ( X^{\vee}[m] \otimes_S X ) \otimes_S \bar{S} \cong \bar{X}^{\vee}[m] \otimes_S \bar{X} \ar[rrr]^-{\lambda^{(z)}_{m \bullet} \circ \cdots \circ \lambda^{(z)}_{1 \bullet}} & & & \bar{X}^{\vee} \otimes_S \bar{X}
}
\ee
which we can compose with
\be\label{eq:ordinaryeval01}
\xymatrix@C+1pc{
\bar{X}^{\vee} \otimes_S \bar{X} \ar[r]^-{\can} & S \otimes_{S^e}( \bar{X}^{\vee} \otimes_S \bar{X} ) \cong \Hom_{R \otimes_k S}(\bar{X}, \bar{X}) \ar[r]^-{\str} & R \otimes_k \bar{S} \ar[r]^-{\Res} & R
}
\ee
where the last map is the $R$-linear residue symbol (we write $\underline{\ud z} = \ud z_1 \ldots \ud z_m$)
\[
\Res_{S/k} \left[ \frac{ (-) \, \underline{\operatorname{d}\!z}}{\partial_{z_1}V, \ldots, \partial_{z_m} V} \right]: R \otimes_k \bar{S} \lto R\,.
\]

\begin{definition} $\eval_0$ is $(-1)^m$ times the composite of \eqref{eq:ordinaryeval00} and \eqref{eq:ordinaryeval01}.
\end{definition}

\begin{lemma} $\eval_0$ is a morphism of linear factorisations of $\widetilde W$ over $\Re$, and for $\eta \in X, \nu \in X^\vee$
\begin{align}
\eval_0( \nu \otimes \eta ) &= (-1)^{\binom{m+1}{2}+(m+1)|\nu|} \Res_{S/k}\left[ \frac{ \str( \Lambda^{(z)} \circ \eta \circ \nu ) \, \underline{\operatorname{d}\!z}}{\partial_{z_1}V, \ldots, \partial_{z_m} V} \right]\\
&= (-1)^{\binom{m+1}{2} + m|\nu|} \Res_{S/k} \left[ \frac{ \nu \Lambda^{(z)}( \eta ) \, \underline{\operatorname{d}\!z}}{\partial_{z_1}V, \ldots, \partial_{z_m} V} \right]\,.
\end{align}
\end{lemma}

We define $\eval_X$ to be the unique morphism in $\HMF(\Re, \widetilde{W})$ making the diagram
\be
\xymatrix{
{}^\dual X \otimes_S X \ar[dr]_-{\eval_0} \ar@{.>}[rr]^-{\eval_X} & & \Delta_W \ar[dl]^-{\pi_\Delta} \\
& R
}\,.
\ee
The existence of such a unique morphism, and the explicit formula for it, are established as before. First we have to write the infinite rank matrix factorisation $X^{\vee}[m] \otimes_S X$ as a direct summand of a finite rank factorisation as in \eqref{eq:idempotentpushdia}. Then Proposition \ref{prop:liftingresult} applies to this finite rank factorisation in which we have embedded to produce the desired lifting. The conclusion is that (\textbf{todo} give formula for $\eval_0$)

\begin{proposition}\label{prop:constructevalmapnontilde} A representative for $\eval_X$ is the chain map \eqref{eq:constructevalmap01}.
\end{proposition}

The formula for $\eval_X$ involves the $R^e \otimes_k S$-linear map
$$
(-,-): X^{\vee} \otimes_S X \lto R^e \otimes_k S 
\, , \qquad 
( e_i^*, e_j ) = e_i^*( e_j ) = \delta_{ij}\,.
$$
The numerator of the residue in \eqref{eq:constructevalmap01} is the image of $\nu \circ \Lambda^{(z)} \otimes \eta$ under the map
\[
\xymatrix@C+1pc{
X^{\vee} \otimes_S X \ar[r]^-{\lAt_2(X^{\vee})^l} & \Bar \otimes_{R^e}(X^{\vee} \otimes_S X ) \ar[r]^-{1 \otimes (-,-)} & \Bar \otimes_{R^e} ( R^e \otimes_k S ) \cong \Bar \otimes_k S \ar[r]^-{\Psi \otimes 1} & \Delta_W \otimes_k S
}
\]
The residue integrates out the $S$, leaving us with an element of $\Delta_W$. By the argument given in Remark \ref{remark:indeptlambda} the morphism $\eval_X$ is independent of the choices of null-homotopy $\lambda^{(z)}_i$.

\subsection{Alternative adjoints and their graphical presentation}\label{subsec:alternatives}

If adjoints exist for 1-morphisms in a bicategory then they are unique only up to isomorphism. For many purposes this freedom is irrelevant, but in practice a particular choice of isomorphism may prove more useful than another. As we saw above for a 1-morphism $X: (R,W) \lra (S,V)$ with $\LG$ with $R=k[x_1,\ldots,x_n]$ and $S=k[z_1,\ldots,z_m]$ its left and right adjoints~${}^\dagger X$ and~$X^\dagger$ are one natural choice. On the other hand, for technical computations like in Section~\ref{sec:pivotality} and for the translation between correlators and 2-morphisms discussed in Section~\ref{sec:ocTFT} we found that a slightly different choice for left adjoints is more suitable. In the remainder of this section we will introduce them together with their graphical presentation (which we basically borrow from~\cite{ct1007.2679}). 

We begin by observing that there are isomorphisms
\begin{align*}
X^\dual & = X^\vee[n] \lra (R \otimes_R X^\vee)[n] \lra R[n] \otimes_R X^\vee \, , \\
{}^\dual X & = X^\vee[m] \lra (X^\vee \otimes_S S)[m] \lra X^\vee \otimes_S S[m] \, . 
\end{align*}
By Lemma~\ref{lem:shiftsigns} the first map sends~$\nu$ to $1\otimes \nu$, and in this way we identify~$X^\dual$ and $R[n] \otimes_R X^\dual$, while the second map involves a sign: $\nu \lmt (-1)^{n |\nu|} 1\otimes \nu$. Thus the adjunction maps exhibiting $R[n] \otimes_R X^\vee$ as a right adjoint of~$X$ are precisely the maps $\widetilde\eval_X$ and $\widetilde\coev_X$ we derived earlier in this section, while the adjunction maps making $X^\vee \otimes_S S[m]$ into a left adjoint of~$X$ differ from $\eval_X$ and $\coev_X$ by signs. Namely, the former are 
\begin{align*}
& (X^\vee \otimes_S S[m]) \otimes_S X \ni \nu \otimes \eta \lmt 
\sum_{l \ge 0} (-1)^{\binom{m+1}{2}+l} \Res_{k[z]/k} \left[ \frac{ \Psi \left( \lAt_2(X^{\vee})^l( \nu \circ \Lambda^{(z)} ), \eta\right) \, \underline{\operatorname{d}\!z}}{\partial_{z_1}V, \ldots, \partial_{z_m} V} \right] \in \Delta_W \, , \\
& \Delta_V \ni \gamma  \lmt
\sum_{l \ge 0} \sum_J \varepsilon \left( \gamma \wedge (-1)^{l+ml} \Psi \Atlarrow_2(X)^l( e_j \otimes e_j^* ) \right) \in X \otimes_R (X^\vee \otimes_S S[m]) \, . 
\end{align*}
We will reflect this difference from~\eqref{eq:constructevalmap01} and~\eqref{eq:coev102} not by introducing new symbols for the above maps but by diagrammatical notation that makes the tensor factors of shifted rings explicit. To wit, while we continue to associate upwards oriented solid lines to 1-morphisms~$X$, we will use thin wiggly lines to represent rings shifted by the number of variables on which they depend. Accordingly we will use the following presentation for the above adjunction maps: 
\begin{align*}
&
%%%%%%%%%%%%%%%%%%%%%%
\begin{tikzpicture}[very thick,scale=1.0,color=blue!50!black, baseline=.3cm]
\draw[directed] (3,0) .. controls +(0,1) and +(0,1) .. (2,0);
\draw[very thin, decorate, decoration={snake, amplitude=0.2mm, segment length=1.0mm}] (2.1,0) .. controls +(0.1,0.5) and +(-0.5,-0.45) .. (2.5,0.75); 
\end{tikzpicture}
%%%%%%%%%%%%%%%%%%%%%%
: (X^\vee \otimes_S S[m]) \otimes_S X \lra \Delta_W
\, , \qquad
%%%%%%%%%%%%%%%%%%%%%%
\begin{tikzpicture}[very thick,scale=1.0,color=blue!50!black, baseline=-.4cm,rotate=180]
\draw[redirected] (3,0) .. controls +(0,1) and +(0,1) .. (2,0);
\draw[very thin, decorate, decoration={snake, amplitude=0.2mm, segment length=1.0mm}] (1.87,0) .. controls +(0.1,0.5) and +(-0.7,-0.15) .. (2.5,0.75); 
\end{tikzpicture}
%%%%%%%%%%%%%%%%%%%%%%
: \Delta_V \lra X \otimes_R (X^\vee \otimes_S S[m]) \, , \\
&
%%%%%%%%%%%%%%%%%%%%%%
\begin{tikzpicture}[very thick,scale=1.0,color=blue!50!black, baseline=.3cm]
\draw[redirected] (3,0) .. controls +(0,1) and +(0,1) .. (2,0);
\draw[very thin, decorate, decoration={snake, amplitude=0.2mm, segment length=1.0mm}] (2.9,0) .. controls +(-0.1,0.5) and +(0.5,-0.45) .. (2.5,0.75); 
\end{tikzpicture}
%%%%%%%%%%%%%%%%%%%%%%
: X \otimes_R ( R[n] \otimes_R X^\vee) \lra \Delta_V
\, , \qquad
%%%%%%%%%%%%%%%%%%%%%%
\begin{tikzpicture}[very thick,scale=1.0,color=blue!50!black, baseline=-.4cm,rotate=180]
\draw[directed] (3,0) .. controls +(0,1) and +(0,1) .. (2,0);
\draw[very thin, decorate, decoration={snake, amplitude=0.2mm, segment length=1.0mm}] (3.13,0) .. controls +(-0.1,0.5) and +(0.7,-0.15) .. (2.5,0.75); 
\end{tikzpicture}
%%%%%%%%%%%%%%%%%%%%%%
: \Delta_W \lra (R[n]\otimes_R X^\vee) \otimes_S  X \, . 
\end{align*}
In this notation the Zorro moves e.\,g.~for the left adjoints read
$$
%%%%%%%%%%%%%%%%%%%%%%
\begin{tikzpicture}[very thick,scale=1.0,color=blue!50!black, baseline=0cm]
\draw[directed] (0,0) .. controls +(0,-1) and +(0,-1) .. (-1,0);
\draw[directed] (1,0) .. controls +(0,1) and +(0,1) .. (0,0);
\draw (-1,0) -- (-1,1.25); 
\draw (1,0) -- (1,-1.25); 
\draw[very thin, decorate, decoration={snake, amplitude=0.2mm, segment length=1.0mm}] (-0.5,-0.75) .. controls +(0.75,0) and +(-0.5,-0.35) .. (0.5,0.75); 
\end{tikzpicture}
%%%%%%%%%%%%%%%%%%%%%%
\;=\;
%%%%%%%%%%%%%%%%%%%%%%
\begin{tikzpicture}[very thick,scale=1.0,color=blue!50!black, baseline=0cm]
\draw (0,-1.25) -- (0,1.25); 
\end{tikzpicture}
%%%%%%%%%%%%%%%%%%%%%%
\, , \qquad
%%%%%%%%%%%%%%%%%%%%%%
\begin{tikzpicture}[very thick,scale=1.0,color=blue!50!black, baseline=0cm]
\draw[directed] (0,0) .. controls +(0,1) and +(0,1) .. (-1,0);
\draw[directed] (1,0) .. controls +(0,-1) and +(0,-1) .. (0,0);
\draw (-1,0) -- (-1,-1.25); 
\draw (1,0) -- (1,1.25); 
\draw[very thin, decorate, decoration={snake, amplitude=0.2mm, segment length=1.0mm}] (-0.9,-1.25) .. controls +(0,0.5) and +(-0.5,-0.35) .. (-0.5,0.75); 
\draw[very thin, decorate, decoration={snake, amplitude=0.2mm, segment length=1.0mm}] (0.5,-0.75) .. controls +(0.8,0.05) and +(0,-1) .. (1.1,1.25); 
\end{tikzpicture}
%%%%%%%%%%%%%%%%%%%%%%
\;=\;
%%%%%%%%%%%%%%%%%%%%%%
\begin{tikzpicture}[very thick,scale=1.0,color=blue!50!black, baseline=0cm]
\draw (0,-1.25) -- (0,1.25); 
\draw[very thin, decorate, decoration={snake, amplitude=0.2mm, segment length=1.0mm}] (0.15,-1.25) -- (0.15,1.25); 
\end{tikzpicture} \, .
%%%%%%%%%%%%%%%%%%%%%%
$$
By definition a wiggly line in a domain associated to an object $(R,W)\in\LG$ represents the ring~$R$ shifted by the parity of its variables. Furthermore we adopt the convention that in diagrams with wiggly lines a downwards oriented solid line represents the dual factorisation~$X^\vee$ of the associated 1-morphism~$X$ (and not necessarily its left or right adjoint~${}^\dual X$ or~$X^\dual$). 

There are two ways for wiggly lines to meet in a diagram. One is that they ``annihilate'', i.\,e.~we have a map
$$
%%%%%%%%%%%%%%%%%%%%%%
\begin{tikzpicture}[very thick,scale=1.0,color=blue!50!black, baseline=0.3cm]
\draw[very thin, decorate, decoration={snake, amplitude=0.2mm, segment length=1.0mm}] (0,0) -- (0.5,0.75);
\draw[very thin, decorate, decoration={snake, amplitude=0.2mm, segment length=1.0mm}] (1,0) -- (0.5,0.75);
\end{tikzpicture}
%%%%%%%%%%%%%%%%%%%%%%
: R[n] \otimes_R R[n] \lra R \, , \qquad r_1\otimes r_2 \lmt r_1 r_2 \, . 
$$
Secondly, for a 1-morphism $X: (R,W) \lra (S,V)$ where~$R$ and~$S$ have the same number of variables~$n$ mod~2, wiggly lines can also ``cross over'' to another domain. By this we mean that there are maps 
\begin{align*}
& 
%%%%%%%%%%%%%%%%%%%%%%
\begin{tikzpicture}[very thick,scale=1.0,color=blue!50!black, baseline=0cm]
\draw[very thin, decorate, decoration={snake, amplitude=0.2mm, segment length=1.0mm}] (-0.3,-0.75) .. controls +(0,0.5) and +(0,-0.5) ..  (0.3,0.75);
\draw (0,-0.75) -- (0,0.75); 
\end{tikzpicture}
%%%%%%%%%%%%%%%%%%%%%%
: S[n] \otimes_S X \lra X \otimes_R R[n] \, , \qquad 1\otimes \eta \lmt (-1)^{n |\eta|} \eta \otimes 1 \, , \\
& 
%%%%%%%%%%%%%%%%%%%%%%
\begin{tikzpicture}[very thick,scale=1.0,color=blue!50!black, baseline=0cm]
\draw[very thin, decorate, decoration={snake, amplitude=0.2mm, segment length=1.0mm}] (0.3,-0.75) .. controls +(0,0.5) and +(0,-0.5) ..  (-0.3,0.75);
\draw (0,-0.75) -- (0,0.75); 
\end{tikzpicture}
%%%%%%%%%%%%%%%%%%%%%%
: X \otimes_R R[n] \lra S[n] \otimes_S X \, , \qquad \eta \otimes 1 \lmt (-1)^{n |\eta|} 1\otimes \eta \, . 
\end{align*}





\section{Zorro moves}\label{sec:Zorro}

In this section we will show that the bicategory $\LG$ of Landau-Ginzburg models has adjoints, by proving that the evaluation and coevaluation morphisms of Section \ref{sec:derivcoeval} satisfy the Zorro moves. Let us fix two arbitrary potentials $W\in k[x] = k[x_1,\ldots,x_n]$ and $V\in k[z] = k[z_1,\ldots,z_m]$. Then we want to prove that for any matrix factorisation $X\in \hmf(k[x,z], V-W)$ and its left and right adjoints ${}^\dual X, X^\dual \in \hmf(k[x,z], W-V)$ the Zorro moves~\eqref{Zorros} and~\eqref{otherZorros} are satisfied. 

Let us consider the first identity of~\eqref{otherZorros} in more detail: 
\be\label{Zorro1detail}
%%%%%%%%%%%%%%%%%%%%%%
\begin{tikzpicture}[very thick,scale=1.0,color=blue!50!black, baseline=0cm]

\fill (0.2,1.6) circle (0pt) node {{\small $\Delta_V$}};
\fill (-0.2,-1.6) circle (0pt) node {{\small $\Delta_W$}};

\fill (1,1.8) circle (2.5pt) node[right] {{\small $\lambda$}};
\fill (-1,-1.8) circle (2.5pt) node[left] {{\small $\rho^{-1}$}};

%\fill (-1.25,-2.25) circle (0pt) node {{\footnotesize $S$}};
%\fill (-0.75,-2.25) circle (0pt) node {{\footnotesize $R$}};

\fill (-1.25,0) circle (0pt) node {{\footnotesize $S\vphantom{S}$}};
\fill (-0.5,0) circle (0pt) node {{\footnotesize $R\vphantom{R}$}};
\fill (0.5,0) circle (0pt) node {{\footnotesize $S\vphantom{R}$}};
\fill (1.25,0) circle (0pt) node {{\footnotesize $R\vphantom{S}$}};

%\fill (1.25,2.25) circle (0pt) node {{\footnotesize $R$}};
%\fill (0.75,2.25) circle (0pt) node {{\footnotesize $S\vphantom{R}$}};

\draw[dashed] (-0.5,0.75) .. controls +(0,0.75) and +(-0.25,-0.75) .. (1,1.8);
\draw[dashed] (0.5,-0.75) .. controls +(0,-0.75) and +(0.25,0.75) .. (-1,-1.8);

\draw[line width=0] 
(1,2.7) node[line width=0pt] (A) {{\small $X$}}
(-1,-2.7) node[line width=0pt] (A2) {{\small $X$}}; 
\draw[redirected] (0,0) .. controls +(0,1) and +(0,1) .. (-1,0);
\draw[redirected] (1,0) .. controls +(0,-1) and +(0,-1) .. (0,0);
\draw (-1,0) -- (A2); 
\draw (1,0) -- (A); 
\end{tikzpicture}
=
\begin{tikzpicture}[very thick,scale=1.0,color=blue!50!black, baseline=0cm]
\draw[line width=0] 
(0,2.7) node[line width=0pt] (A) {{\small $X$}}
(0,-2.7) node[line width=0pt] (A2) {{\small $X$}}; 
\draw (A2) -- (A); 
\end{tikzpicture}
%%%%%%%%%%%%%%%%%%%%%%
\ee
Here we label the domains by the rings $R = k[x]$ and $S = k[z]$ pertaining to the two objects of $\LG$. We keep in mind that there are two $\Delta$'s which appear: one of which is an $R$-$R$-bimodule, and the other is an $S$-$S$-bimodule. We write $\Re = R_1 \otimes_k R_2$ with $R_i = R$ and $\Se  = S_1 \otimes_k S_2$ with $S_i = S$.


We call the left-hand side of (\ref{Zorro1detail}) the \textsl{Zorro map} and denote it $\mathcal Z$. It is the composite
\be\label{eq:zorro1a1}
\xymatrix@C+2pc{
X \ar[r]^-{\rho^{-1}} & X \otimes_R \Delta_W \ar[r]^-{1 \otimes\,\widetilde{\coev}} & X \otimes_R X^{\dagger} \otimes_S X \ar[r]^-{\widetilde{\eval}\, \otimes 1} & \Delta_V \otimes_S X \ar[r]^-{\lambda} & X\,,
}
\ee
which we prove is homotopic to the identity on $X$. Note that we continue to discard subscripts that are clear from the context, e.\,g.~$\widetilde\eval = \widetilde\eval_X$. Our first order of business is to establish the expression for the Zorro map given in the next lemma. For this we need to introduce the $R_2$-linear map
$$
\big\langle\!\big\langle - \big\rangle\!\big\rangle: \Bar \lto R_2 
\, ,\qquad
\big\langle\!\big\langle \alpha \big\rangle\!\big\rangle = \Res_{\Re /R_2} \left[ \frac{ \varepsilon\Psi( \alpha ) \, \underline{\operatorname{d}\!x}}{\partial_{x_1}W, \ldots, \partial_{x_n} W} \right] .
$$
We use the maps $\varepsilon$ and $\Psi$ of Section~\ref{sec:Background} and $\underline{\operatorname{d}\!x}=\operatorname{d}\!x_1\ldots \operatorname{d}\!x_n$. To be clear, this differential form and the $\partial_{x_i} W$ in the denominator of the residue are all viewed as elements of $R_1$, the copy of $R$ which is integrated out by the residue.

Let $\{ e_i \}_{i}$ be a homogeneous $(S \otimes_k R)$-basis of $X$, $\{ e_i^* \}_{i}$ the dual basis and $\{ e_{ij} = e_i \circ e_j^* \}_{i,j}$ the determined basis of $\End(X) = \Hom_{S \otimes_k R}(X,X)$. By the Atiyah class $\At = \At( \End(X) )$ of this complex we mean the operator on $\End(X) \otimes_{R_1} \Bar$ which would more properly be called the Atiyah class of $\End(X) \otimes_{R_1} \Re $ determined by the basis $e_{ij}$ using the convention of Remark \ref{remark:linearatother}.

As before let $\lambda_i$ be a null-homotopy on $X$ for the action of $\partial_{x_i} W$, for example $\lambda_i= -\partial_{x_i}d_X(x,z)$, and set $\Lambda = \lambda_1\ldots \lambda_n$. Postcomposition with $\Lambda$ gives a well-defined operator $\Lambda^\bullet$ on $\End(X) \otimes_{R_1} \Bar$.

\begin{lemma}\label{lemma:Zorrointermediate} We have
\be\label{Zorrointermediate}
\mathcal Z = \sum_j (-1)^{\binom{n}{2}+|e_j|} \big\langle\!\big\langle \str \big( \Lambda \circ \At^n (-\circ e_j^*) \big) \big\rangle\!\big\rangle \cdot e_j
\ee
where for $f \in X$, the endomorphism $f \circ e_j^*$ of $X$ sends $g$ to $e_j^*(g) \cdot f$.
\end{lemma}

(TODO: make the next two paragraphs a Remark?)
Before giving the proof of the lemma, let us roughly sketch how it leads to an argument that $\mathcal Z$ is homotopic to the identity. Although we will not formalise the argument in precisely these terms, the Atiyah class is ``anti-self-adjoint'' (\textbf{todo} still compatible with signs?) with respect to $\langle\!\langle \str( - ) \rangle\!\rangle$: 
\begin{equation}\label{eq:trueproofeq}
\big\langle\!\big\langle \str \big( \Lambda \circ \At^n (-)\big) \big\rangle\!\big\rangle \simeq (-1)^n \big\langle\!\big\langle \str \big( \At^n( \Lambda ) \circ -\big) \big\rangle\!\big\rangle\,.
\end{equation}
We will see that $\At^n( \Lambda )$ is a transition determinant in the sense of the calculus of residues, so that the right-hand side of~(\ref{eq:trueproofeq}) can be identified with $\str(-)$. Hence
\begin{equation}\label{eq:trueproofeq2}
\mathcal{Z} \simeq \sum_j (-1)^{|e_j|} \str(-\circ e_j^*) \cdot e_j = \sum_j \str(e_j^* \circ -) \cdot e_j = 1_X\,.
\end{equation}
The reader will appreciate the parallel with the argument for duality in the monoidal category of vector spaces, with the new feature here being the Atiyah class and its anti-self-adjointness.

There are two subtleties which complicate the proof. The first is that we must deal throughout with associative Atiyah classes and thus noncommutative differential forms (compare with the proof of nondegeneracy in Section~\ref{section:dualityadjointop} where only ordinary Atiyah classes are involved). The second is that the identites above only hold, in general, up to homotopy ``$\simeq$''.

For the proof of Lemma \ref{lemma:Zorrointermediate} let us recall the explicit formulas for the maps involved: 
\begin{align*}
\rho^{-1}: X \lto X \otimes_R \Delta_W \, , \qquad \rho^{-1} &= \sum_{l \ge 0}(-1)^l \Psi \At_2(X)^l \sigma_2\,,\\
\widetilde{\coev}: \Delta_W \lto X^{\dagger} \otimes_S X \, , \qquad \widetilde{\coev}(\gamma) &= \sum_{l,j} (-1)^{l+nl+|e_j|} \varepsilon \Big( \gamma \wedge \Psi\!\At_2(X^{\dual})^l( e_j^* ) \otimes e_j \Big)\\
\widetilde{\eval}: X \otimes_R X^{\dagger} \lto \Delta_V \, , \qquad \widetilde{\eval}(\eta \otimes \nu) &= (-1)^{\binom{n}{2}+n|\eta| + |\nu|}\Res_{R/k}\! \left[ \frac{ \nu \Lambda( \eta ) \underline{\operatorname{d}\! x}}{\partial_{x_1}W, \ldots, \partial_{x_n} W} \right] + \mathcal{O}(\theta)\,.
\end{align*}
Here $\mathcal{O}(\theta)$ denotes higher order terms which do not survive the map $\lambda$ in~\eqref{Zorro1detail} and therefore may be ignored for present purposes.

\begin{proof}[Proof of Lemma \ref{lemma:Zorrointermediate}]
Throughout we write $\At$ for $\At_2$, and Atiyah classes are taken with respect to $R/k$. The image of a basis element $e_q$ under the first two maps of \eqref{eq:zorro1a1} is computed by
\begin{align}
(1 \otimes \widetilde{\coev}) \circ \rho^{-1}( e_q ) &= (1 \otimes \widetilde{\coev}) \sum_{l \ge 0} (-1)^l \Psi \At(X)^l (e_q)\nonumber\\
&= \sum_{j} \sum_{l+l'=n} (-1)^{n+nl'+|e_j|} \varepsilon\Big( \Psi \At(X)^l( e_q ) \wedge \Psi \Atlarrow(X^{\dual})^{l'}( e_j^* ) \otimes e_j \Big) \nonumber\\
&= \sum_{j} \sum_{l+l' = n} (-1)^{n+nl'+|e_j|} \varepsilon\Psi\Big( \At(X)^l( e_q ) \times \Atlarrow(X^{\dual})^{l'}( e_j^* ) \otimes e_j \Big) \label{eq:zorro1a12}
\end{align} 
In the last step we use that $\Psi$ intertwines the shuffle product on $\Bar$ with the exterior product on~$\Delta$. Next we apply Lemma \ref{lemma:atshufat} to write this shuffle product of Atiyah classes as a single Atiyah class. As it stands in \eqref{eq:zorro1a12} the term inside the $\varepsilon \Psi ( - )$ belongs to $X \otimes \gamma_*(\Bar) \otimes X^\vee[n] \otimes X$. Moving the $\Bar$ to the right of the $X^\vee[n]$ introduces a sign $(-1)^{n( |e_j| + n + l' )}$. 

We account for this sign by rewriting \eqref{eq:zorro1a12} as
\begin{align*}
\sum_{j} \sum_{l+l' = n} (-1)^{|e_j|+n(|e_j| + n + l') + n|e_j|} \varepsilon\Psi\Big( \At(X)^l( e_q ) \times \Atlarrow(X^{\dual})^{l'}( e_j^* ) \otimes e_j \Big)\,.
\end{align*}
Then Lemma \ref{lemma:atshufat} applies directly to show that
\[
(1 \otimes \widetilde{\coev}) \circ \rho^{-1}( e_q ) = \sum_j (-1)^{|e_j| + n|e_j|} \varepsilon \Psi\Big( \At(X \otimes X^{\vee}[n])^n(e_q \otimes e_j^*) \otimes e_j \Big)\,.
\]
Applying the first isomorphism of \eqref{eq:defneval0_00} yields
\[
\sum_j (-1)^{(n+1)|e_j| + n|e_q|} \varepsilon \Psi \Big( \At((X \otimes X^\vee)[n])^n( e_q \otimes e_j^* ) \otimes e_j \Big)\,.
\]
Since the Atiyah class of $Y[n]$ agrees with that of $Y$, this is the same as
\[
\sum_j (-1)^{(n+1)|e_j|+n|e_q|} \varepsilon \Psi \Big( \At(X \otimes X^\vee)^n( e_q \otimes e_j^* ) \otimes e_j \Big)\,.
\]
The next step is to apply $(-) \otimes_{S^e} S$, that is, to identify the left and right actions of $S$, after which $X \otimes X^\vee$ becomes identified $\End(X)$, so we are left with
\[
\sum_j (-1)^{(n+1)|e_j|+n|e_q|} \varepsilon \Psi \Big( \At( \End(X) )^n( e_q \circ e_j^* ) \otimes e_j \Big)\,.
\]
To finish applying $\widetilde\eval_0$ it remains to compose with $\Lambda$, multiply by the global sign, supertrace and integrate:
\begin{align*}
\mathcal{Z}(e_q) &= \lambda \circ (\widetilde\eval \otimes 1) \circ (1 \otimes \widetilde\coev) \circ \rho^{-1}( e_q )\\
&= \sum_j (-1)^{\binom{n}{2} + (n+1)|e_j|+ n|e_q|}\Res_{\Re /R_2}\! \left[ \frac{ \varepsilon\Psi \str\Big( \Lambda \circ \At^n( e_q \circ e_j^* ) \Big) \underline{\operatorname{d}\! x} }{\partial_{x_1}W, \ldots, \partial_{x_n} W} \right] \cdot e_j\\
&= \sum_j (-1)^{\binom{n}{2} + (n+1)|e_j|+n|e_q|} \big\langle\!\big\langle \str \big( \Lambda \circ \At^n (e_q\circ e_j^*) \big) \big\rangle\!\big\rangle \cdot e_j\,.
\end{align*}
The only nonvanishing summands have $|e_q| = |e_j|$, so the sign in the last formula is in fact $(-1)^{\binom{n}{2}+(n+1)|e_j| + n|e_q|} = (-1)^{\binom{n}{2}+|e_j|}$. Since the Zorro map~$\mathcal Z$ is $S \otimes_k R$-linear it is fixed by its action on basis elements~$e_q$, which proves that $\mathcal{Z}$ is given by (\ref{Zorrointermediate}).
\end{proof}

The basic adjointness that we need is contained in the next lemma. Throughout we write $\psi^\bullet$ for the operator $\psi^\bullet(\alpha \otimes \omega) = \psi \circ \alpha \otimes \omega$ on $\End(X) \otimes_{R_1} \Bar$, and we denote $[d_X,-]$ by~$D$. 

\begin{lemma}\label{lemma:lameadjoints} Given homogeneous $\psi \in \End(X)$ and $\varphi \in \End(X) \otimes_{R_1} \Bar$
\begin{align*}
\big\langle\!\big\langle \str( \psi \circ \At(\varphi) ) \big\rangle\!\big\rangle = (-1)^{|\psi|+1} &\big\langle\!\big\langle d \str( D( \psi ) \circ \varphi ) \big\rangle\!\big\rangle + (-1)^{|\psi|} \dlangle \str\big( \big[ \psi^\bullet, d \big] ( D\varphi ) \big) \drangle\\
&- \dlangle \str\big( D(\psi) \circ d\varphi ) \big) \drangle \, . 
\end{align*}
\end{lemma}
\begin{proof}
The super Jacobi identity for operators on this tensor product gives 
\be\label{superJacobi0}
\big[ \psi^\bullet, \At \big] = \big[ \psi^\bullet, [d, D] \big] = \big[ D, \big[ \psi^\bullet, d \big] \big] - (-1)^{|\psi|} \big[ d, \big[ D, \psi^\bullet \big] \big]\,.
\ee
It is easy to see that $\str \circ d = d \circ \str$ and hence $\str \circ \At = 0$ so if we apply~\eqref{superJacobi0} to~$\varphi$
\begin{align*}
\dlangle \str\big( \psi \circ \At(\varphi) \big) \drangle &= \dlangle \str\big( \big[ D, \big[ \psi^\bullet, d \big] \big] \varphi \big) \drangle - (-1)^{|\psi|} \dlangle \str\big( \big[ d, \big[ D, \psi^\bullet \big] \big] \varphi \big) \drangle\\
&= (-1)^{|\psi|} \dlangle \str\big( \big[ \psi^\bullet, d \big] ( D\varphi ) \big) \drangle\\
&\qquad- (-1)^{|\psi|} \dlangle d \str\big( \big[ D, \psi^\bullet \big] \varphi \big) \drangle\\
&\qquad- \dlangle \str\big( \big[ D, \psi^\bullet \big]( d \varphi ) \big) \drangle \, .
\end{align*}
But since $[D, \psi^\bullet] = D(\psi)^\bullet$ this is what we wanted to show.
\end{proof}

Applying the previous lemma with $\psi = \Lambda$ and using that $D(\Lambda)$ is a linear combination of the $f_i$ which annihilate with the residue in $\langle\!\langle - \rangle\!\rangle$ we find that
\[
\big\langle\!\big\langle \str( \lambda_1 \ldots \lambda_n \circ \At(\varphi) ) \big\rangle\!\big\rangle = \sum_{i=1}^n (-1)^{n+i} \Big\langle\!\!\Big\langle d \left\{ f_i \str( \lambda_1 \ldots \widehat{\lambda_i} \ldots \lambda_n \circ \varphi ) \right\} \Big\rangle\!\!\Big\rangle + h_0( D(\varphi))
\]
for some function $h_0$, where as usual $\widehat{\lambda_i}$ means that $\lambda_i$ is omitted. Applying the lemma again will reduce the number of $\lambda$'s by one, and proceeding inductively we can use the lemma to interpolate between the left- and right-hand sides of (\ref{eq:trueproofeq}) (although we will never formalise exactly what the right hand side means! \textbf{todo}). The precise statement is contained in the next lemma, of which the previous formula is the $p = 0$ case.

We will need some notation: given a subset $\bs{i} = \{i_1,\ldots,i_p\}$ of $\{1, \ldots, n\}$ (we do not require that the $i_j$ be in ascending order) we write $\ell(\bs{i}) = p$ and
\be
|\bs{i}| = \sum_{1 \le a < b \le p} \delta_{i_a > i_b}\, , \qquad \gamma(\bs{i}) = |\bs{i}| + i_1 + \ldots + i_p + np\,.
\ee
The empty sequence $\emptyset$ is the unique sequence of length $\ell(\emptyset) = 0$. We write $\Lambda_{i_1,\ldots,i_p}$ for the product $\Lambda = \lambda_1 \ldots \lambda_n$ with the $\lambda_{i_1},\ldots,\lambda_{i_p}$ omitted.
%Let $D$ denote the differential on $\End(X)$, i.\,e.~the commutator with~$d_X$.

\begin{lemma}\label{lemma:zorromain} For $\varphi \in \End(X) \otimes_{R_1} \Bar$ and $0 \le p < n$ we have
\begin{align*}
\sum_{\ell(\bs{i}) = p} (-1)^{\gamma(\bs{i})} & \Big\langle\!\!\Big\langle d \circ f_{i_1} \circ \ldots \circ d \circ f_{i_p} \str\big( \Lambda_{i_1,\ldots,i_p} \circ \At(\varphi) \big) \Big\rangle\!\!\Big\rangle\\
&= \sum_{\ell(\bs{i}) = p+1} (-1)^{\gamma(\bs{i})} \Big\langle\!\!\Big\langle d \circ f_{i_1} \circ \ldots \circ d \circ f_{i_{p+1}} \str\big( \Lambda_{i_1,\ldots,i_p,i_{p+1}} \circ \varphi \big) \Big\rangle\!\!\Big\rangle + h_p( D \varphi )
\end{align*}
where $f_i = \partial_{x_i} W$, $D=[d_X,-]$ is the differential on $\End(X)$, and the sums are over all sequences of length $p, p+1$, respectively, and
\[
h_p( \alpha ) = \sum_{\ell(\bs{i}) = p} (-1)^{\gamma(\bs{i}) + n + p} \Big\langle\!\!\Big\langle d \circ f_{i_1} \circ \ldots \circ d \circ f_{i_p} \str\big( [ \Lambda_{i_1,\ldots,i_p}^\bullet, d](\alpha) \big) \Big\rangle\!\!\Big\rangle\,.
\]
\end{lemma}
\begin{proof}
Fix a sequence $\bs{i}$ of length $p$, set $\psi = \Lambda_{i_1,\ldots,i_p}$ and let $\eta$ be the operator $\eta = d \circ f_{i_1} \circ \ldots \circ d \circ f_{i_p}$ on $\Bar$, where the $f_i$ stand for multiplication (on the left) by $f_i$. Then by Lemma \ref{lemma:lameadjoints}, or rather the proof with $\dlangle - \drangle$ replaced by $\dlangle \eta( - ) \drangle$ we have
\begin{align}
\dlangle \eta\, \str\big( \psi \circ \At(\varphi) \big) \drangle &= (-1)^{n+p+1}\dlangle \eta\, d\, \str\big( D(\psi) \circ \varphi \big) \drangle \label{eq:hutch1}\\
&\qquad+ (-1)^{n+p} \dlangle \eta\, \str\big( [\psi^\bullet, d]( D\varphi ) \big) \drangle\label{eq:hutch2}\\
&\qquad- \dlangle \eta\, \str\big( D(\psi) \circ d\varphi \big) \drangle\label{eq:hutch3}
\end{align}
Since $D(\lambda_i) = f_i$ we have
\be\label{eq:expansiondpsi}
D(\psi) = \sum_{q \notin \{ i_1, \ldots, i_p \} } (-1)^{q+1 + \sum_{1 \le b \le p} \delta_{i_b<q}} f_q \cdot \Lambda_{i_1,\ldots,i_p,q}
\ee
where the sign counts the number of $\lambda$'s appearing to the left of $\lambda_q$ in $\Lambda_{i_1,\ldots,i_p}$. Substituting, the last summand (\ref{eq:hutch3}) in the above is, up to a sign, 
\[
\sum_{q \notin \{ i_1,\ldots, i_p \}} (-1)^{q+1 + \sum_{1 \le b \le p} \delta_{i_b<q}} \bigdlangle d \circ f_{i_1} \circ \ldots \circ d \circ f_{i_p} f_q \str\big( \Lambda_{i_1,\ldots,i_p,q} \circ d \varphi \big) \bigdrangle \, .
\]
If we sum over all sequences $\bs{i}$ with the signs given in the statement of the lemma this vanishes, because $f_{i_p} f_q$ and $f_q f_{i_p}$ appear in the sum with opposite signs. This completes the proof, since the sum over $\bs{i}$ of (\ref{eq:hutch1}) and (\ref{eq:hutch2}) gives us the right-hand side of the equation in the statement of the lemma. To see that the signs are correct, it suffices to observe that
\begin{align*}
\gamma(\{ i_1, \ldots, i_{p+1} \}) + \gamma(\bs{i}) &= \sum_{1 \le a \le p} \delta_{i_a>i_{p+1}} + i_{p+1} + n\\
&= \sum_{1 \le b \le p} \delta_{i_b < i_{p+1}} + p + i_{p+1} + n
\end{align*}
which matches the sign coming from \eqref{eq:expansiondpsi} with $q = i_{p+1}$, plus the $n+p+1$ from \eqref{eq:hutch1}.
\end{proof}

\begin{lemma}\label{lemma:langlecalc1} For $\varphi \in \End(X)$ we have
\[
\big\langle\!\big\langle \str \big( \Lambda \At^n (\varphi)\big) \big\rangle\!\big\rangle = (-1)^{\binom{n}{2}} \str(\varphi) + \sum_{p=0}^{n-1} h_i( D \At^{n-p-1}(\varphi) )\,.
\]
\end{lemma}
\begin{proof}
Applying Lemma \ref{lemma:zorromain} repeatedly yields
\begin{align*}
\big\langle\!\big\langle \str \big( \Lambda \At^n (\varphi)\big) \big\rangle\!\big\rangle &= \sum_{\ell(\bs{i}) = n} (-1)^{\gamma(\bs{i})} \big\langle\!\big\langle d \circ f_{i_1} \circ \ldots \circ d \circ f_{i_n} \str(\varphi) \big\rangle\!\big\rangle + H'\\
&= (-1)^{\binom{n}{2}}\sum_{\sigma \in S_n} (-1)^{|\sigma|} \big\langle\!\big\langle d \circ f_{\sigma(1)} \circ \ldots \circ d \circ f_{\sigma(n)} \str(\varphi) \big\rangle\!\big\rangle + H'\,.
\end{align*}
%where 
%$$
%\delta = 
%\det 
%\begin{pmatrix}
%df_1 & df_2 & \ldots & df_n \\
%\vdots & \vdots & & \vdots \\
%df_1 & df_2 & \ldots & df_n 
%\end{pmatrix} .
%$$
%This concludes the proof that the Zorro map~\eqref{Zorro1superdetail} is homotopic to the identity. 
where $H' = \sum_{p=0}^{n-1} h_i( D \At^{n-p-1}(\varphi) )$. By a small argument involving transition determinants (\textbf{todo}) this equals 
\[
(-1)^{\binom{n}{2}} \str( \varphi ) + \sum_{p=0}^{n-1} h_i( D \At^{n-p-1}(\varphi) ) \, .
\]
\end{proof}

The upshot is that, by the previous lemma and Lemma \ref{lemma:Zorrointermediate},
\begin{align*}
\mathcal{Z} &= \sum_j (-1)^{\binom{n}{2}+|e_j|} \big\langle\!\big\langle \str \big( \Lambda \circ \At^n (-\circ e_j^*) \big) \big\rangle\!\big\rangle \cdot e_j\\
&= \sum_j (-1)^{|e_j|} \str(-\circ e_j^*) \cdot e_j + H\\
&= \sum_j \str(e_j^* \circ -) \cdot e_j + H\\
&= 1_X + H
\end{align*}
where
\[
H = \sum_j \sum_{p=0}^{n-1} (-1)^{|e_j|} h_i( D \At^{n-p-1}(- \circ e_j^*) ) \cdot e_j\,.
\]
As the difference of two closed maps $H = \mathcal{Z} - 1_X$ is a closed endomorphism of $X$, and to complete the proof that $\mathcal{Z} \simeq 1_X$ it remains to prove that $H$ is null-homotopic (\textbf{todo} From here on do not trust, some errors in $H$ for example).

The idea is to use the nondegenerate pairing on the morphism spaces of $\hmf(k[x,z], V-W)$. If~$k$ is a field this is the Kapustin-Li pairing of \cite{??,??} but in general we use the contents of Section \ref{??} (\textbf{more}). The precise statement (Theorem \ref{??}) is that there is a homotopy equivalence of $\mathbb{Z}_2$-graded complexes over $k$
\be
\Hom_{k[x,z]}(X,X) \lto \Hom_k( \Hom_{k[x,z]}(X,X), k )[m+n]\,, \qquad \varphi \lmt \langle \varphi, - \rangle_X\,.
\ee
induced by the pairing
\begin{equation}\label{eq:bracketkl}
\langle -, - \rangle_X = \frac{1}{n!} \sum_{\sigma \in S_n} (-1)^{|\sigma|} \Res_{k[x,z]/k} \left[ \frac{ \str( - \circ - \circ \Lambda^{(x)} \Lambda^{(z)}) \, \underline{\operatorname{d}\!x} \underline{\operatorname{d}\!z}}{\partial_{x_1}W, \ldots, \partial_{x_n} W, \partial_{z_1} V, \ldots, \partial_{z_m} V} \right]
\end{equation}
where $\Lambda^{(x)}=\Lambda$ is the product of homotopies for the action of $\partial_{x_i}W$ introduced earlier, and $\Lambda^{(z)}$ is the product of homotopies for the action of $\partial_{z_j}V$ on~$X$. To prove that $\mathcal{Z} \simeq 1_X$ it is therefore enough to prove that the functionals $\langle \mathcal{Z}, - \rangle_X$ and $\langle 1_X, - \rangle_X$ on $\Hom_{k[x,z]}(X,X)$ are homotopic, or what is the same, that $\langle H, \varphi \rangle_X$ can be written as a function of $D(\varphi)=[d_X,\varphi]$. This will follow from a careful study of the map $\str( H \circ - )$.

We compute for a homogeneous element $\varphi \in \End(X)$ that
\begin{align}
\str(H \varphi) &= \str( \mathcal{Z} \varphi ) - \str(\varphi) \nonumber\\
&= \sum_i (-1)^{|e_i|} e^*_i \big( \mathcal Z(\varphi(e_i)) \big) - \str(\varphi)\nonumber \\
& = \sum_{i,j} (-1)^{|e_i| + |e_j| + n} e^*_i \Big( \big\langle\!\big\langle \str( \Lambda \At^n (\varphi(e_i)\otimes e^*_j \otimes e_j)) \big\rangle\!\big\rangle \Big) - \str(\varphi)\nonumber \\
& = \sum_{i,j} (-1)^{|e_i| + |e_j| + n + n|e_j|} e^*_i \Big( \big\langle\!\big\langle \str( \Lambda \At^n (\varphi(e_i)\otimes e^*_j )) \big\rangle\!\big\rangle e_j \Big) - \str(\varphi)\nonumber \\
& = \big\langle\!\big\langle \str \big( \Lambda \At^n \big(\sum_i \varphi(e_i)\otimes e^*_j \big)\big) \big\rangle\!\big\rangle - \str(\varphi)\nonumber \\
& = \big\langle\!\big\langle \str \big( \Lambda \At^n (\varphi)\big) \big\rangle\!\big\rangle - \str(\varphi) \nonumber\\
&= \sum_{p=0}^{n-1} h_i( D \At^{n-p-1}(\varphi) ) \label{strlambdaAtn}
\end{align}
where in the last step we use Lemma \ref{lemma:langlecalc1}. For the final steps in the argument we will need the following lemma.

\begin{lemma}\label{lem:PsiToTheRight}
Let $\omega\in\Omega(k[x_1,\ldots,x_n])$ be an $n$-form. Then $\Psi(\omega f)=\Psi(\omega) f(y)$ for any $f\in k[x]$. 
\end{lemma}

\begin{proof}
We may assume that $\omega$ is of the form $da_1\ldots da_n$ for some $a_i\in k[x]$. Then
\begin{align*}
\Psi(\omega f) & = \Psi \Big( \sum_{i=1}^n (-1)^{n-i} da_1\ldots da_{i-1} d(a_i a_{i+1}) da_{i+2}\ldots da_n df + (-1)^n a_1 da_2\ldots da_n df \Big) \\
& = \sum_{i=1}^n (-1)^{n-i} \partial_{[1]} a_1\ldots \partial_{[i-1]} a_{i-1} \partial_{[i]} (a_i a_{i+1}) \partial_{[i+1]} a_{i+2} \ldots \partial_{[n-1]} a_n \partial_{[n]} f \\
& \qquad + (-1)^n a_1 \partial_{[1]} a_2 \ldots \partial_{[n-1]} a_n \partial_{[n]} f \\
& = \sum_{i=1}^n (-1)^{n-i} \partial_{[1]} a_1\ldots \partial_{[i-1]} a_{i-1} \Big(\partial_{[i]} a_i {}	^{t_1\ldots t_i}a_{i+1} + {}^{t_1\ldots t_{i-1}} a_i \partial_{[i]} a_{i+1} \Big) \\
& \qquad \cdot  \partial_{[i+1]} a_{i+2} \ldots \partial_{[n-1]} a_n \partial_{[n]} f + (-1)^n a_1 \partial_{[1]} a_2 \ldots \partial_{[n-1]} a_n \partial_{[n]} f \\
& = \partial_{[1]} a_1 \ldots \partial_{[n]} a_n f(y) \\
& = \Psi(\omega) f(y) \, ,
\end{align*}
where we used Lemma~\ref{lem:LeibnizForDQO} in the third step. 
\end{proof}

\begin{theorem}\label{theorem:mainzorro} The Zorro map $\mathcal{Z}$ is homotopic to $1_X$.
\end{theorem}
\begin{proof}
As we have already mentioned, it suffices by nondegeneracy of \eqref{eq:bracketkl} to prove that $\langle H, - \rangle_X$ is a null-homotopic function $\Hom_{k[x,z]}(X,X) \lto k$. For $\varphi' \in \End(X)$ the numerator in the residue which computes $\langle H, \varphi' \rangle_X$ is, using \eqref{strlambdaAtn}, a differential form multiplied by
\[
\str( H \varphi' \Lambda^{(x)} \Lambda^{(z)}) = \sum_{p=0}^{n-1} h_p( D \At^{n-p-1}(\varphi' \Lambda^{(x)} \Lambda^{(z)}) )
\]
But given $\varphi' \in \End(X)$ if we set $\varphi = \varphi' \circ \Lambda^{(x)} \Lambda^{(z)}$ and substitute in (\ref{??}) we find that the two functionals $\langle \mathcal{Z}, - \rangle_X$ and $\langle 1_X, - \rangle_X$ differ by a term which is an expression in residues of
\begin{equation}\label{eq:mainzorro1}
h_p( D \At^{n-p-1}(\varphi) ) = h_p( D \At^{n-p-1}(\varphi' \Lambda^{(x)} \Lambda^{(z)}) )\,.
\end{equation}
But $\At$ is closed, so 
\begin{align}
h_p( D \At^{n-p-1}(\varphi' \Lambda^{(x)} \Lambda^{(z)}) ) &= h_p \At^{n-p-1}D(\varphi' \Lambda^{(x)} \Lambda^{(z)}) ) \nonumber\\
&= h_p \At^{n-p-1}\big( D(\varphi') \Lambda^{(x)} \Lambda^{(z)} )\\
&\qquad + (-1)^{|\phi'|} h_p \At^{n-p-1}\big( \varphi' D(\Lambda^{(x)}) \Lambda^{(z)} ) \label{eq:mainzorro4}\\
&\qquad + (-1)^{|\phi'| + n} h_p \At^{n-p-1}\big( \varphi' \Lambda^{(x)} D(\Lambda^{(z)}) ) \, . \label{eq:mainzorro5}
\end{align}
Now $D(\Lambda^{(z)})$ is a linear combination of terms divisible by the $\partial_{z_j} V$, and since $h_p$ and $\At$ are $k[z]$-linear these coefficients pass through to annihilate with the denominator in the residue in  (\ref{eq:bracketkl}). 

Similarly $\varphi' D(\Lambda^{(x)}) \Lambda^{(z)}$ is a sum of terms of the form $f_j \alpha$ for various $j$. But the Atiyah class and the commutator $[\Lambda_{i_1,\ldots,i_p}, \nabla]$ are right linear so
\begin{align*}
h_p \At^{n-p-1}\big( \alpha f_j ) &= h_p\left( \At^{n-p-1}(\alpha) \cdot f_j \right)\\
&= \sum_{\ell(\bs{i}) = p} (-1)^{\gamma(\bs{i}) +p} \Big\langle\!\!\Big\langle \nabla \left\{ f_{i_1} \ldots \nabla f_{i_p} \str\big( [ \Lambda_{i_1,\ldots,i_p}, \nabla]\At^{n-p-1}(\alpha) \big) \cdot f_j \right\} \Big\rangle\!\!\Big\rangle\,.
\end{align*}
But $\nabla( f_{i_1} \beta f_j ) = f_{i_1} \nabla( \beta f_j ) + df_{i_1} \beta f_j$. The first summand vanishes in the residue for obvious reasons, and the second vanishes since by Lemma~\ref{lem:PsiToTheRight}
\[
\dlangle - \cdot f_j \drangle = \dlangle - \drangle \cdot f_j(x')
\]
which vanishes in the residue.

The upshot is that (\ref{eq:mainzorro4}) and (\ref{eq:mainzorro5}) do not contribute under the residue, so that
\[
\langle H, - \rangle_X = 
\frac{1}{n!} \sum_{\sigma \in S_n}\sum_{p=0}^{n-1} (-1)^{|\sigma|} \Res_{k[x,z]/k} \left[ \frac{ h_p \At^{n-p-1}( D(-) \Lambda^{(x)} \Lambda^{(z)} ) \, \underline{\operatorname{d}\!x} \underline{\operatorname{d}\!z}}{\partial_{x_1}W, \ldots, \partial_{x_n} W, \partial_{z_1} V, \ldots, \partial_{z_m} V} \right]\,.
\]
This proves that $\langle H, - \rangle_X$ is null-homotopic, and thus $\langle \mathcal{Z}, - \rangle_X$ and $\langle 1_X, - \rangle_X$ are homotopic functionals. Hence by the nondegeneracy of Theorem~\ref{theorem:generalkl} we conclude that $\mathcal{Z}$ and $1_X$ are homotopic.
\end{proof}

The three other Zorro moves are proven analogously (\textbf{todo}).

\begin{theorem}
The bicategory $\LG$ has left and right adjoints. Explicit expressions for the evaluation and coevaluation maps are those of Section~\ref{sec:derivcoeval}. 
\end{theorem}

(\textbf{todo}) Mention Appendix \ref{app:adjoints} and summands.

\begin{remark}
The same result also holds for the graded bicategory $\LG^{\mathrm{gr}}$ of Remark~\ref{remark:gradedbicategory}. To see this all we have to check is that the evaluation and coevaluation 2-morphisms have $\nZ$-degree~0. For a potential~$W$ of degree $2c$ we fix the $\nZ$-grading of the unit $\Delta_W$ by assigning degree $-c$ to the~$\theta_i$ in~\eqref{DeltaW}; similarly an $n$-form in the bar complex~\eqref{BarComplex} has degree $-nc$. In this way both the left and right unit actions $\lambda, \rho$ as well as their homotopy inverses manifestly have degree~0. Furthermore, by Definition~\ref{def:coeval} the coevaluation is the image of the (degree~0) identity under the sequence of (degree~0) maps~\eqref{eq:isogivescoev2} and thus has degree~0. That the evaluation also has degree~0 then follows from the Zorro move. 
\end{remark}



\section{Pivotality}\label{sec:pivotality}

TODO: rewrite section; point out that at least the 1-endomorphism categories in $\LG$ are pivotal

We end this section by clarifying the relation between the adjunctions given by $\eval_X, \coev_X$ and $\widetilde\eval_X, \widetilde\coev_X$, and listing some of their properties. Of particular importance is the identity~\eqref{eq:monoidalproperty} which we will later need for applications in Sections~\ref{sec:defectaction} and~\ref{sec:CardyCondition}, e.\,g.~to prove the Cardy condition. 

For a map $\varphi: X \lra Y$ between matrix factorisations in $\hmf(k[x_1,\ldots,x_n,z_1,\ldots,z_m], V-W)$ 
%\begin{align*}
%Y & \in \hmf(k[x_1,\ldots,x_n,y_1,\ldots,y_m], W_2(y)-W_1(x)) \, , \\
%X & \in \hmf(k[y_1,\ldots,y_m,z_1,\ldots,z_p], W_3(z)-W_2(y))
%\end{align*}
we define its adjoints to be
$$
{}^\dual \varphi = \varphi^\vee[m] : {}^\dual Y \lra {}^\dual X \, , \qquad
\varphi^\dual = \varphi^\vee[n] : Y^\dual \lra X^\dual
$$

\begin{proposition}\label{prop.morphismmigration}
For a 2-morphism $\varphi: X \lra Y$ in $\LG$ we have
\begin{enumerate}
\item $\eval_Y \circ (1_{{}^\dual Y} \otimes \varphi) = \eval_X \circ ({}^\dual \varphi \otimes 1_X)$, 
\item $(\varphi \otimes 1_{{}^\dual X}) \circ \coev_X = (1_Y \otimes {}^\dual \varphi) \circ \coev_Y$, 
\item ${}^\dual \varphi = \lambda_{{}^\dual X} \circ (\eval_Y \otimes 1_{{}^\dual X}) \circ (1_{{}^\dual Y} \otimes \varphi \otimes 1_{{}^\dual X}) \circ (1_{{}^\dual Y} \otimes \coev_X) \circ \rho^{-1}_{{}^\dual Y}$
\end{enumerate}
and similarly for $\widetilde\eval$ and $\widetilde\coev$. Diagramatically the above identities read
$$
%%%%%%%%%%%%%%%%%%%%%%
\begin{tikzpicture}[very thick,scale=1.0,color=blue!50!black, baseline=.4cm]
\draw[line width=0pt] 
(3,0.5) node[line width=0pt] (D) {}
(2,0.5) node[line width=0pt] (s) {}; 
\draw[directed] (D) .. controls +(0,1) and +(0,1) .. (s);
\fill (3,0.5) circle (2.5pt) node[right] {{\small $\varphi$}};
\draw (3,0.65) -- (3,0)
node[below] {{{\small$X\vphantom{{}^\dual Y}$}}};
\draw (2,0.65) -- (2,0)
node[below] {{{\small${}^\dual Y$}}};
\end{tikzpicture} 
%%%%%%%%%%%%%%%%%%%%%%
=
%%%%%%%%%%%%%%%%%%%%%%
\begin{tikzpicture}[very thick,scale=1.0,color=blue!50!black, baseline=.4cm]
\draw[line width=0pt] 
(3,0.5) node[line width=0pt] (D) {}
(2,0.5) node[line width=0pt] (s) {}; 
\draw[directed] (D) .. controls +(0,1) and +(0,1) .. (s);
\fill (2,0.5) circle (2.5pt) node[left] {{\small ${}^\dual \varphi$}};
\draw (3,0.65) -- (3,0)
node[below] {{{\small$X\vphantom{{}^\dual Y}$}}};
\draw (2,0.65) -- (2,0)
node[below] {{{\small${}^\dual Y$}}};
\end{tikzpicture} 
%%%%%%%%%%%%%%%%%%%%%%
\, , \qquad
%%%%%%%%%%%%%%%%%%%%%%
\begin{tikzpicture}[very thick,scale=1.0,color=blue!50!black, baseline=.4cm]
\draw[line width=0pt] 
(3,0.5) node[line width=0pt] (D) {}
(2,0.5) node[line width=0pt] (s) {}; 
\draw[directed] (D) .. controls +(0,-1) and +(0,-1) .. (s);
\fill (2,0.5) circle (2.5pt) node[left] {{\small $\varphi$}};
\draw (3,0.35) -- (3,1)
node[above] {{{\small${}^\dual X$}}};
\draw (2,0.35) -- (2,1)
node[above] {{{\small$Y$}}};
\end{tikzpicture} 
%%%%%%%%%%%%%%%%%%%%%%
=
%%%%%%%%%%%%%%%%%%%%%%
\begin{tikzpicture}[very thick,scale=1.0,color=blue!50!black, baseline=.4cm]
\draw[line width=0pt] 
(3,0.5) node[line width=0pt] (D) {}
(2,0.5) node[line width=0pt] (s) {}; 
\draw[directed] (D) .. controls +(0,-1) and +(0,-1) .. (s);
\fill (3,0.5) circle (2.5pt) node[right] {{\small ${}^\dual \varphi$}};
\draw (3,0.35) -- (3,1)
node[above] {{{\small${}^\dual X$}}};
\draw (2,0.35) -- (2,1)
node[above] {{{\small$Y$}}};
\end{tikzpicture} 
%%%%%%%%%%%%%%%%%%%%%%
\, , \qquad
%%%%%%%%%%%%%%%%%%%%%%
\begin{tikzpicture}[very thick,scale=1.0,color=blue!50!black, baseline=.4cm]
\fill (3,0.5) circle (2.5pt) node[right] {{\small ${}^\dual \varphi$}};
\draw (3,0) -- (3,1.5)
node[above] {{{\small${}^\dual X$}}};
\draw (3,0) -- (3,-0.5)
node[below] {{{\small${}^\dual Y$}}};
\end{tikzpicture} 
%%%%%%%%%%%%%%%%%%%%%%
=
%%%%%%%%%%%%%%%%%%%%%%
\begin{tikzpicture}[very thick,scale=1.0,color=blue!50!black, baseline=.4cm]
\draw[line width=0pt] 
(3,0.5) node[line width=0pt] (D) {}
(2,0.5) node[line width=0pt] (s) {}; 
\draw[directed] (D) .. controls +(0,-1) and +(0,-1) .. (s);
\fill (2,0.5) circle (2.5pt) node[right] {{\small $\varphi$}};
\draw (2,0.32) -- (2,0.68);
\draw[line width=0pt] 
(2,0.5) node[line width=0pt] (D) {}
(1,0.5) node[line width=0pt] (s) {}; 
\draw[directed] (D) .. controls +(0,+1) and +(0,+1) .. (s);
\draw (3,0.35) -- (3,1.5)
node[above] {{{\small${}^\dual X$}}};
\draw (1,0.65) -- (1,-0.5)
node[below] {{{\small${}^\dual Y$}}};
\end{tikzpicture} 
%%%%%%%%%%%%%%%%%%%%%%
\, . 
$$
\end{proposition}

\begin{proof}
The first identity follows immediately from the explicit expression for the evaluation. To prove the third identity we only have to apply the Zorro move and use part~(i), and then the third identity follows by appending another upwards-oriented $Y$-line to the left and another Zorro move: 
$$
%%%%%%%%%%%%%%%%%%%%%%
\begin{tikzpicture}[very thick,scale=1.0,color=blue!50!black, baseline=.4cm]
\fill (3,0.5) circle (2.5pt) node[right] {{\small ${}^\dual \varphi$}};
\draw (3,0) -- (3,1.5)
node[above] {{{\small${}^\dual X$}}};
\draw (3,0) -- (3,-0.5)
node[below] {{{\small${}^\dual Y$}}};
\end{tikzpicture} 
%%%%%%%%%%%%%%%%%%%%%%
=
%%%%%%%%%%%%%%%%%%%%%%
\begin{tikzpicture}[very thick,scale=1.0,color=blue!50!black, baseline=.4cm]
\draw[line width=0pt] 
(3,0.5) node[line width=0pt] (D) {}
(2,0.5) node[line width=0pt] (s) {}; 
\draw[directed] (D) .. controls +(0,-1) and +(0,-1) .. (s);
\fill (1,0.5) circle (2.5pt) node[right] {{\small ${}^\dual \varphi$}};
\draw (2,0.32) -- (2,0.68);
\draw[line width=0pt] 
(2,0.5) node[line width=0pt] (D) {}
(1,0.5) node[line width=0pt] (s) {}; 
\draw[directed] (D) .. controls +(0,+1) and +(0,+1) .. (s);
\draw (3,0.35) -- (3,1.5)
node[above] {{{\small${}^\dual X$}}};
\draw (1,0.65) -- (1,-0.5)
node[below] {{{\small${}^\dual Y$}}};
\end{tikzpicture} 
%%%%%%%%%%%%%%%%%%%%%%
=
%%%%%%%%%%%%%%%%%%%%%%
\begin{tikzpicture}[very thick,scale=1.0,color=blue!50!black, baseline=.4cm]
\draw[line width=0pt] 
(3,0.5) node[line width=0pt] (D) {}
(2,0.5) node[line width=0pt] (s) {}; 
\draw[directed] (D) .. controls +(0,-1) and +(0,-1) .. (s);
\fill (2,0.5) circle (2.5pt) node[right] {{\small $\varphi$}};
\draw (2,0.32) -- (2,0.68);
\draw[line width=0pt] 
(2,0.5) node[line width=0pt] (D) {}
(1,0.5) node[line width=0pt] (s) {}; 
\draw[directed] (D) .. controls +(0,+1) and +(0,+1) .. (s);
\draw (3,0.35) -- (3,1.5)
node[above] {{{\small${}^\dual X$}}};
\draw (1,0.65) -- (1,-0.5)
node[below] {{{\small${}^\dual Y$}}};
\end{tikzpicture} 
%%%%%%%%%%%%%%%%%%%%%%
\Rightarrow
%%%%%%%%%%%%%%%%%%%%%%
\begin{tikzpicture}[very thick,scale=1.0,color=blue!50!black, baseline=.4cm]
\draw[line width=0pt] 
(3,0.5) node[line width=0pt] (D) {}
(2,0.5) node[line width=0pt] (s) {}; 
\draw[directed] (D) .. controls +(0,-1) and +(0,-1) .. (s);
\fill (3,0.5) circle (2.5pt) node[left] {{\small ${}^\dual \varphi$}};
\draw (3,0.35) -- (3,1.5)
node[above] {{{\small${}^\dual X$}}};
\draw (2,0.35) -- (2,1.5)
node[above] {{{\small$Y$}}};
\end{tikzpicture} 
%%%%%%%%%%%%%%%%%%%%%%
=
%%%%%%%%%%%%%%%%%%%%%%
\begin{tikzpicture}[very thick,scale=1.0,color=blue!50!black, baseline=.4cm]
\draw[line width=0pt] 
(3,0.5) node[line width=0pt] (D) {}
(2,0.5) node[line width=0pt] (s) {}; 
\draw[directed] (D) .. controls +(0,-1) and +(0,-1) .. (s);
\fill (2,0.5) circle (2.5pt) node[right] {{\small $\varphi$}};
\draw (2,0.32) -- (2,0.68);
\draw[line width=0pt] 
(2,0.5) node[line width=0pt] (D) {}
(1,0.5) node[line width=0pt] (s) {}; 
\draw[directed] (D) .. controls +(0,+1) and +(0,+1) .. (s);
\draw[line width=0pt] 
(1,0.5) node[line width=0pt] (D) {}
(0,0.5) node[line width=0pt] (s) {}; 
\draw[directed] (D) .. controls +(0,-1) and +(0,-1) .. (s);
\draw (1,0.32) -- (1,0.68);
\draw (3,0.35) -- (3,1.5)
node[above] {{{\small${}^\dual X$}}};
\draw (0,0.32) -- (0,1.5)
node[above] {{{\small$Y$}}};
\end{tikzpicture} 
%%%%%%%%%%%%%%%%%%%%%%
=
%%%%%%%%%%%%%%%%%%%%%%
\begin{tikzpicture}[very thick,scale=1.0,color=blue!50!black, baseline=.4cm]
\draw[line width=0pt] 
(3,0.5) node[line width=0pt] (D) {}
(2,0.5) node[line width=0pt] (s) {}; 
\draw[directed] (D) .. controls +(0,-1) and +(0,-1) .. (s);
\fill (2,0.5) circle (2.5pt) node[right] {{\small $\varphi$}};
\draw (3,0.35) -- (3,1.5)
node[above] {{{\small${}^\dual X$}}};
\draw (2,0.35) -- (2,1.5)
node[above] {{{\small$Y$}}};
\end{tikzpicture} 
%%%%%%%%%%%%%%%%%%%%%%
\, . 
$$
The obvious analogous identities for $\widetilde\eval$ and $\widetilde\coev$ follow in the same way. 
\end{proof}

\begin{definition}\label{def.pivotal}
We call a bicategory with right adjoints \textsl{pivotal} if for each 1-morphism~$X$ there is a natural monoidal 2-isomorphism $\delta_X: X\lra X^{\dual\dual}$. It then follows that~$X^\dual$ is also left adjoint to~$X$, with adjunction maps $\overline\eval_X = \widetilde\eval_{X^\dual}\circ (1_{X^\dual} \otimes \delta_{X})$ and $\overline\coev_X = (\delta_X^{-1} \otimes 1_{X^\dual} ) \circ \widetilde\coev_{X^\dual}$. By the condition for~$\delta$ to be natural monoidal we mean that the identities
$$
%%%%%%%%%%%%%%%%%%%%%%
\begin{tikzpicture}[very thick,scale=1.0,color=blue!50!black, baseline=0cm]
\draw[line width=0] 
(-1,1.25) node[line width=0pt] (A) {{\small $Z^\dual$}}
(1,-1.25) node[line width=0pt] (A2) {{\small $Y^\dual$}}; 
\fill (0,0) circle (2.5pt) node[left] {{\small $\varphi$}};
\draw[redirected] (0,0) .. controls +(0,-1) and +(0,-1) .. (-1,0);
\draw[redirected] (1,0) .. controls +(0,1) and +(0,1) .. (0,0);
\draw (-1,0) -- (A); 
\draw (1,0) -- (A2); 
\end{tikzpicture}
%%%%%%%%%%%%%%%%%%%%%%
=
%%%%%%%%%%%%%%%%%%%%%%
\begin{tikzpicture}[very thick,scale=1.0,color=blue!50!black, baseline=0cm]
\draw[line width=0] 
(1,1.25) node[line width=0pt] (A) {{\small $Z^\dual$}}
(-1,-1.25) node[line width=0pt] (A2) {{\small $Y^\dual$}}; 
\draw[directed] (0,0) .. controls +(0,1) and +(0,1) .. (-1,0);
\draw[directed] (1,0) .. controls +(0,-1) and +(0,-1) .. (0,0);
\fill (0,0) circle (2.5pt) node[right] {{\small $\varphi$}};
\draw (-1,0) -- (A2); 
\draw (1,0) -- (A); 
\end{tikzpicture}
%%%%%%%%%%%%%%%%%%%%%%
\, , \qquad 
%%%%%%%%%%%%%%%%%%%%%%
\begin{tikzpicture}[very thick,scale=0.8,color=blue!50!black, baseline=0.45cm]
\draw[line width=1pt] 
(-2,-1) node[line width=0pt] (X) {{\small $Y^\dual$}}
(-1,-1) node[line width=0pt] (Y) {{\small $X^\dual$}}
(2,2) node[line width=0pt] (XY) {{\small $(X\otimes Y)^\dual$}}; 
\draw[directed] (0,0) .. controls +(0,1) and +(0,1) .. (-1,0);
\draw[directed] (1,0) .. controls +(0,2) and +(0,2) .. (-2,0);
\draw[directed] (2,0) .. controls +(0,-1) and +(0,-1) .. (0.5,0);
\draw (-1,0) -- (Y);
\draw (-2,0) -- (X);
\draw[dotted] (0,0) -- (1,0);
\draw (2,0) -- (XY);
\end{tikzpicture}
%%%%%%%%%%%%%%%%%%%%%%
=
%%%%%%%%%%%%%%%%%%%%%%
\begin{tikzpicture}[very thick,scale=0.8,color=blue!50!black, baseline=0.45cm]
\draw[line width=1pt] 
(2,-1) node[line width=0pt] (Y) {{\small $Y^\dual$}}
(3,-1) node[line width=0pt] (X) {{\small $X^\dual$}}
(-1,2) node[line width=0pt] (XY) {{\small $(X\otimes Y)^\dual$}}; 
\draw[directed] (1,0) .. controls +(0,1) and +(0,1) .. (2,0);
\draw[directed] (0,0) .. controls +(0,2) and +(0,2) .. (3,0);
\draw[directed] (-1,0) .. controls +(0,-1) and +(0,-1) .. (0.5,0);
\draw (2,0) -- (Y);
\draw (3,0) -- (X);
\draw[dotted] (0,0) -- (1,0);
\draw (-1,0) -- (XY);
\end{tikzpicture}
%%%%%%%%%%%%%%%%%%%%%%
$$
hold for every 2-morphism $\varphi: Z \lra Y$, where above (and only there!) the caps and cups with left-oriented arrows represent $\overline\eval$ and $\overline\coev$. 
\end{definition}

In the bicategory $\LG$ the natural candidate for a pivotal structure is
\be\label{delta}
\delta_X = 
%%%%%%%%%%%%%%%%%%%%%%
\begin{tikzpicture}[very thick,scale=1.0,color=blue!50!black, baseline=.4cm]
\draw[line width=0pt] 
(3,0.5) node[line width=0pt] (D) {}
(2,0.5) node[line width=0pt] (s) {}; 
\draw[directed] (D) .. controls +(0,-1) and +(0,-1) .. (s);
\fill (2.6,-0.3) circle (0pt) node[below] {{\small $\coev_{X^\dual}$}};
\draw (2,0.32) -- (2,0.68);
\draw[line width=0pt] 
(1,0.5) node[line width=0pt] (D) {}
(2,0.5) node[line width=0pt] (s) {}; 
\draw[directed] (D) .. controls +(0,+1) and +(0,+1) .. (s);
\fill (1.6,1.23) circle (0pt) node[above] {{\small $\widetilde\eval_{X}$}};
\draw (3,0.35) -- (3,1.5)
node[above] {{{\small${}^\dual X^{\dual}$}}};
\draw (1,0.65) -- (1,-0.5)
node[below] {{{\small$X$}}};
\end{tikzpicture} 
%%%%%%%%%%%%%%%%%%%%%%
: X\stackrel{\cong}{\lra} {}^\dual X^\dual \cong X^{\vee\vee}[n+n] = X^{\vee\vee}
\, , \qquad 
\delta_X^{-1} = 
%%%%%%%%%%%%%%%%%%%%%%
\begin{tikzpicture}[very thick,scale=1.0,color=blue!50!black, baseline=.4cm]
\draw[line width=0pt] 
(3,0.5) node[line width=0pt] (D) {}
(2,0.5) node[line width=0pt] (s) {}; 
\draw[redirected] (D) .. controls +(0,-1) and +(0,-1) .. (s);
\fill (2.6,-0.28) circle (0pt) node[below] {{\small $\widetilde\coev_{X}$}};
\draw (2,0.32) -- (2,0.68);
\draw[line width=0pt] 
(1,0.5) node[line width=0pt] (D) {}
(2,0.5) node[line width=0pt] (s) {}; 
\draw[redirected] (D) .. controls +(0,+1) and +(0,+1) .. (s);
\fill (1.6,1.23) circle (0pt) node[above] {{\small $\eval_{X^\dual}$}};
\draw (3,0.35) -- (3,1.5)
node[above] {{{\small$X$}}};
\draw (1,0.65) -- (1,-0.5)
node[below] {{{\small${}^\dual X^{\dual}$}}};
\end{tikzpicture} 
%%%%%%%%%%%%%%%%%%%%%%
\ee
for each 1-morphism $X\in \hmf(k[x_1,\ldots,x_n,z_1,\ldots,z_m], V-W)$. Observing that~$\delta_X$ is simply the Zorro map~\eqref{eq:trueproofeq2} with~$e_j$ exchanged by $e_j^{**}$, one straightforwardly verifies that~$\delta_X$ is the canonical map given by $(\delta_X(a))(\Phi) = \sigma(\Phi(a))$ for all $a\in X, \Phi\in X^\dual$, where $\sigma: R\otimes_k S \lra S\otimes_k R$ is the linear map that swaps tensor factors. 

For~$\delta$ to describe a pivotal structure it must be an isomorphism between~$X$ and $X^{\dual\dual} = X^\vee[n]^\dual \cong X^{\vee\vee}[m+n]$. This is indeed true ``half of the time'', to wit whenever $m+n$ is even, and below we will establish that then it also satisfies the conditions in Definition~\ref{def.pivotal}. However, for $m+n$ odd~$X$ and $X^{\dual\dual}$ are not isomorphic in general (a counterexample being $W=0$, $V=z^d$, $X=S^{\oplus 2}$, $d_X = (\begin{smallmatrix}0&z^n\\z^{d-n}&0\end{smallmatrix})$ for $n\notin d/2$). Thus $\LG$ cannot be pivotal in the above sense. 

What is important for applications though is that identities like those in Definition~\ref{def.pivotal} hold. The precise statement is the following result, which can also be read as ``$\LG$ is pivotal up to shifts''. 

\begin{proposition}\label{prop:monoidalproperty}
\begin{enumerate}
\item For any map~$\varphi:Z\lra Y$ in $\hmf(k[x_1,\ldots,x_n,z_1,\ldots,z_m], V-W)$ we have 
$$
%%%%%%%%%%%%%%%%%%%%%%
\begin{tikzpicture}[very thick,scale=1.0,color=blue!50!black, baseline=0cm]
\draw[line width=0] 
(-1,1.25) node[line width=0pt] (A) {{\small $Z^\dual$}}
(1,-1.25) node[line width=0pt] (A2) {{\small $Y^\dual$}}; 
\fill (0,0) circle (2.5pt) node[left] {{\small $\varphi$}};
\draw[redirected] (0,0) .. controls +(0,-1) and +(0,-1) .. (-1,0);
\draw[redirected] (1,0) .. controls +(0,1) and +(0,1) .. (0,0);
\draw (-1,0) -- (A); 
\draw (1,0) -- (A2); 
\end{tikzpicture}
%%%%%%%%%%%%%%%%%%%%%%
=
%%%%%%%%%%%%%%%%%%%%%%
\begin{tikzpicture}[very thick,scale=1.0,color=blue!50!black, baseline=0cm]
\draw[line width=0] 
(1,1.25) node[line width=0pt] (A) {{\small ${}^\dual Z$}}
(-1,-1.25) node[line width=0pt] (A2) {{\small ${}^\dual Y$}}; 
\draw[directed] (0,0) .. controls +(0,1) and +(0,1) .. (-1,0);
\draw[directed] (1,0) .. controls +(0,-1) and +(0,-1) .. (0,0);
\fill (0,0) circle (2.5pt) node[right] {{\small $\varphi$}};
\draw (-1,0) -- (A2); 
\draw (1,0) -- (A); 
\end{tikzpicture}
%%%%%%%%%%%%%%%%%%%%%%
[m+n]
$$
where the right-hand side map is shifted to match the left-hand side's source and target. 
\item For two matrix factorisations $Y\in \hmf(k[x_1,\ldots,x_n,y_1,\ldots,y_m], W_2(y)-W_1(x))$ and $X\in \hmf(k[y_1,\ldots,y_m,z_1,\ldots,z_p], W_3(z)-W_2(y))$ we have
\be\label{eq:monoidalproperty}
%%%%%%%%%%%%%%%%%%%%%%
\begin{tikzpicture}[very thick,scale=0.8,color=blue!50!black, baseline]
\draw[line width=1pt] 
(-2,-1) node[line width=0pt] (X) {{\small ${}^\dual Y$}}
(-1,-1) node[line width=0pt] (Y) {{\small ${}^\dual X$}}
(2,2) node[line width=0pt] (XY) {{\small ${}^\dual (X\otimes Y)$}}; 
\draw[directed] (0,0) .. controls +(0,1) and +(0,1) .. (-1,0);
\draw[directed] (1,0) .. controls +(0,2) and +(0,2) .. (-2,0);
\draw[directed] (2,0) .. controls +(0,-1) and +(0,-1) .. (0.5,0);
\draw (-1,0) -- (Y);
\draw (-2,0) -- (X);
\draw[dotted] (0,0) -- (1,0);
\draw (2,0) -- (XY);
\end{tikzpicture}
%%%%%%%%%%%%%%%%%%%%%%
[n+p] \circ \tau
=
%%%%%%%%%%%%%%%%%%%%%%
\begin{tikzpicture}[very thick,scale=0.8,color=blue!50!black, baseline]
\draw[line width=1pt] 
(2,-1) node[line width=0pt] (Y) {{\small $Y^\dual$}}
(3,-1) node[line width=0pt] (X) {{\small $X^\dual$}}
(-1,2) node[line width=0pt] (XY) {{\small $(X\otimes Y)^\dual$}}; 
\draw[directed] (1,0) .. controls +(0,1) and +(0,1) .. (2,0);
\draw[directed] (0,0) .. controls +(0,2) and +(0,2) .. (3,0);
\draw[directed] (-1,0) .. controls +(0,-1) and +(0,-1) .. (0.5,0);
\draw (2,0) -- (Y);
\draw (3,0) -- (X);
\draw[dotted] (0,0) -- (1,0);
\draw (-1,0) -- (XY);
\end{tikzpicture}
%%%%%%%%%%%%%%%%%%%%%%
\ee
where~$\tau$ is the isomorphism $Y^\dual \otimes X^\dual = Y^\vee[n] \otimes X^\vee[m] \lra Y^\vee[m+n+p] \otimes X^\vee[p] = ({}^\dual Y \otimes {}^\dual X)[n+p]$. 
\end{enumerate}
\end{proposition}

\begin{proof}
Part (i) follows in a standard way by applying Zorro moves together with the ``morphism migration properties'' of Proposition~\ref{prop.morphismmigration}. 

[TODO: include signs!] To establish part (ii) we start by computing the left-hand side of~\eqref{eq:monoidalproperty}. As explained in the previous section one may choose a representative for $\coev_{Y\otimes X}$ where all Atiyah classes act only on the single factor~$Y$ in the tensor product $Y\otimes X \otimes (Y\otimes X)^\dual$. Hence the Zorro move for the $Y^\dual$-line can be performed, after which the only contribution left from $\coev_{Y\otimes X}$ is the sum $\sum_i e_i \otimes e_i^*$ for a basis $\{e_i\}_i$ of~$X$. Applying the remaining map $\lambda_{(Y\otimes X)^\dual} \circ (\eval_X \otimes 1_{(Y\otimes X)^\dual})$ we compute that the left-hand side of~\eqref{eq:monoidalproperty} maps $\alpha\otimes \beta \in X^\dual \otimes Y^\dual$ to 
$$
\sum_i \Res_{k[x,y,z]/k[x,z]} \left[ \frac{\str \big( (e_i \circ \alpha \circ \Lambda^{(y)}_X \big)\cdot e_i^* \otimes \beta \, \underline{\operatorname{d}\!y}}{\partial_{y_1} W_2, \ldots, \partial_{y_m} W_2} \right]
= 
\Res_{k[x,y,z]/k[x,z]} \left[ \frac{\str ( \alpha \circ \Lambda^{(y)}_X)(\alpha) \otimes \beta \, \underline{\operatorname{d}\!y}}{\partial_{y_1} W_2, \ldots, \partial_{y_m} W_2} \right] . 
$$
The argument for the right-hand side of~\eqref{eq:monoidalproperty} is analogous and we find 
\begin{align*}
%%%%%%%%%%%%%%%%%%%%%%
\begin{tikzpicture}[very thick,scale=0.8,color=blue!50!black, baseline]
\draw[line width=1pt] 
(-2,-1) node[line width=0pt] (X) {{\small ${}^\dual Y$}}
(-1,-1) node[line width=0pt] (Y) {{\small ${}^\dual X$}}
(2,2) node[line width=0pt] (XY) {{\small ${}^\dual (X\otimes Y)$}}; 
\draw[directed] (0,0) .. controls +(0,1) and +(0,1) .. (-1,0);
\draw[directed] (1,0) .. controls +(0,2) and +(0,2) .. (-2,0);
\draw[directed] (2,0) .. controls +(0,-1) and +(0,-1) .. (0.5,0);
\draw (-1,0) -- (Y);
\draw (-2,0) -- (X);
\draw[dotted] (0,0) -- (1,0);
\draw (2,0) -- (XY);
\end{tikzpicture}
%%%%%%%%%%%%%%%%%%%%%%
& =
\Res_{k[x,y,z]/k[x,z]} \left[ \frac{\str ( \Lambda^{(y)}_{{}^\dual X} \otimes 1) \, \underline{\operatorname{d}\!y}}{\partial_{y_1} W_2, \ldots. \partial_{y_m} W_2} \right] ,
\\
%%%%%%%%%%%%%%%%%%%%%%
\begin{tikzpicture}[very thick,scale=0.8,color=blue!50!black, baseline]
\draw[line width=1pt] 
(2,-1) node[line width=0pt] (Y) {{\small $Y^\dual$}}
(3,-1) node[line width=0pt] (X) {{\small $X^\dual$}}
(-1,2) node[line width=0pt] (XY) {{\small $(X\otimes Y)^\dual$}}; 
\draw[directed] (1,0) .. controls +(0,1) and +(0,1) .. (2,0);
\draw[directed] (0,0) .. controls +(0,2) and +(0,2) .. (3,0);
\draw[directed] (-1,0) .. controls +(0,-1) and +(0,-1) .. (0.5,0);
\draw (2,0) -- (Y);
\draw (3,0) -- (X);
\draw[dotted] (0,0) -- (1,0);
\draw (-1,0) -- (XY);
\end{tikzpicture}
%%%%%%%%%%%%%%%%%%%%%%
& = 
\Res_{k[x,y,z]/k[x,z]} \left[ \frac{\str ( 1 \otimes \Lambda^{(y)}_{Y^\dual}) \, \underline{\operatorname{d}\!y}}{\partial_{y_1} W_2, \ldots, \partial_{y_m} W_2} \right] .
\end{align*}

To show that the above two expressions are equal up to homotopy we again make use of the nondegeneracy of the Kapustin-Li pairing. To wit, we have to show that the difference~$\delta$ of the two maps paired with any closed map $\Phi\in \Hom((Y\otimes X)^\dual, X^\dual \otimes Y^\dual)$ gives zero, i.\,e.~$\langle \Phi, \delta \rangle_{X^\dual \otimes Y^\dual} = 0$. Evaluating this pairing gives
$$
\Res_{k[x,y,z]/k} \left[ \frac{\str \big( \Phi \circ (\Lambda^{(y)}_{X^\dual} \otimes 1 - 1 \otimes \Lambda^{(y)}_{Y^\dual})) \circ (\Lambda^{(x)}_{X^\dual} \otimes \Lambda^{(z)}_{Y^\dual}) \big) \,\underline{\operatorname{d}\!x} \underline{\operatorname{d}\!y} \underline{\operatorname{d}\!z}}{(\prod_{i=1}^n \partial_{x_i} W_1), (\prod_{j=1}^m \partial_{y_j} W_2), (\prod_{l=1}^p \partial_{z_l} W_3)} \right] , 
$$
but since $\partial_{y_j} d_{X^\dual} \otimes 1$ is as good a homotopy for the action of $\partial_{y_j} W_2$ on $X^\dual \otimes Y^\dual$ as $1\otimes \partial_{y_j} d_{X^\dual}$, the above residue indeed vanishes. 
\end{proof}


\section{Defect action on bulk fields}\label{sec:defectaction}

In any pivotal bicategory with adjoints there are natural maps between the endomorphism spaces of unit 1-morphisms. Roughly, these maps are constructed by capturing a 2-morphism of a unit 1-morphism inside a loop labelled by an arbitrary 1-morphism (and its adjoint). Below we present the details for the case of the bicategory $\LG$. We will also give the interpretation in terms of defect actions on bulk fields in Landau-Ginzburg models. 

Let $X\in \hmf(k[x_1,\ldots,x_n,z_1,\ldots,z_m], V-W)$ as before. In this section when we write $\Hom$ and $\End$ we mean the spaces of 2-(endo)morphisms in $\LG$. We define \textsl{defect operators} 
$$
\mathcal D_l(X): \End(\Delta_V) \longrightarrow \Hom(\Delta_W, \Delta_W[m+n]) \, , \qquad
\mathcal D_r(X): \End(\Delta_W) \longrightarrow \Hom(\Delta_V, \Delta_V[m+n])
$$
in terms of the morphisms encoding the monoidal and adjunction structures as follows. For $\phi\in \End(\Delta_V)$ and $\psi\in \End(\Delta_W)$ we set
\begin{align}
\mathcal D_l(X)(\phi) & = \eval_X[m+n] \circ (1_{X^\dual}\otimes (\lambda_X \circ (\phi\otimes 1_X)\circ \lambda_X^{-1})) \circ \widetilde\coev_X \, , \nonumber \\ 
\mathcal D_r(X)(\psi) & = \widetilde\eval_X[m+n] \circ (1_{X}\otimes (\rho_{X} \circ (1_X \otimes \psi)\circ \rho_{X}^{-1})) \circ \coev_X \, . \label{defectactionwrittenout}
\end{align}
Note that since $\Hom(\Delta,\Delta[1]) = 0$ in $\LG$ the operators $\mathcal D_l(X)$ and $\mathcal D_r(X)$ are zero if $m+n$ is odd, while for even $m+n$ they map to $\End(\Delta_W)$ and $\End(\Delta_V)$, respectively. 
In the special case $\phi=1$ and $\psi=1$ we obtain the left and right \textsl{quantum dimensions} $\operatorname{qdim}_l(X) = \mathcal D_l(X)(1)$ and $\operatorname{qdim}_r(X) = \mathcal D_r(X)(1)$. 

Diagrammatically these definitions read
\begin{align}
\label{leftdefectaction}
& \mathcal D_l(X)(\phi) =
%%%%%%%%%%%%%%%%%%%%%%%%%%%%
\begin{tikzpicture}[very thick,scale=0.6,color=blue!50!black, baseline,>=stealth]
\nicecolourscheme (0,0) circle (3.5);
\fill (-2.2,-2.2) circle (0pt) node[white] {{\small$W$}};
\nicepalecolourscheme (0,0) circle (2);
\fill (-1.1,-1.1) circle (0pt) node[white] {{\small$V$}};
%
%\fill (180:2) circle (0pt) node[left] {{\small$x\vphantom{z'}$}};
%\fill (180:2) circle (0pt) node[right] {{\small$z\vphantom{z'}$}};
%\fill (0:2) circle (0pt) node[left] {{\small$z'\vphantom{z'}$}};
%\fill (0:2) circle (0pt) node[right] {{\small$x'\vphantom{z'}$}};
\draw (0,0) circle (2);
\draw[<-, very thick] (0.100,-2) -- (-0.101,-2) node[above] {}; 
\draw[<-, very thick] (-0.100,2) -- (0.101,2) node[below] {}; 
\fill (135:0) circle (3.3pt) node[left] {{\small$\phi$}};
\fill (50:2) circle (3.3pt) node[right] {{\small$\lambda_{X}$}};
\fill (-50:2) circle (3.3pt) node[right] {{\small$\lambda^{-1}_{X}$}};
\draw[dashed] (135:0) .. controls +(0,1) and +(-0.5,-1) .. (50:2);
\fill (0.41,1.15) circle (0pt) node {{\small $\Delta_V$}};
\fill (0.41,-1.15) circle (0pt) node {{\small $\Delta_V$}};
\draw[dashed] (135:0) .. controls +(0,-1) and +(-0.5,1) .. (-50:2);
\draw[dashed] (270:2) -- (270:3.3)
node[near end,right] {{{\small$\Delta_W$}}};
\draw[dashed] (90:2) -- (90:3.3)
node[near end,right] {{{\small$\Delta_W$}}};
\end{tikzpicture} 
%%%%%%%%%%%%%%%%%%%%%%%%%%%% 
=
%%%%%%%%%%%%%%%%%%%%%%%%%%%%
\begin{tikzpicture}[very thick,scale=0.6,color=blue!50!black, baseline,>=stealth]
\nicecolourscheme (0,0) circle (3.5);
\fill (-2.2,-2.2) circle (0pt) node[white] {{\small$W$}};
\nicepalecolourscheme (0,0) circle (2);
\fill (-1.1,-1.1) circle (0pt) node[white] {{\small$V$}};
%
\draw[very thin, decorate, decoration={snake, amplitude=0.2mm, segment length=1.0mm}] (270:2) .. controls +(-3,0) and +(-3,-1) .. (90:2); 
\draw (0,0) circle (2);
\draw[<-, very thick] (0.100,-2) -- (-0.101,-2) node[above] {}; 
\draw[<-, very thick] (-0.100,2) -- (0.101,2) node[below] {}; 
\fill (135:0) circle (3.3pt) node[left] {{\small$\phi$}};
\fill (50:2) circle (3.3pt) node[right] {{\small$\lambda_{X}$}};
\fill (-50:2) circle (3.3pt) node[right] {{\small$\lambda^{-1}_{X}$}};
\draw[dashed] (135:0) .. controls +(0,1) and +(-0.5,-1) .. (50:2);
\fill (0.41,1.15) circle (0pt) node {{\small $\Delta_V$}};
\fill (0.41,-1.15) circle (0pt) node {{\small $\Delta_V$}};
\draw[dashed] (135:0) .. controls +(0,-1) and +(-0.5,1) .. (-50:2);
\draw[dashed] (270:2) -- (270:3.3)
node[near end,right] {{{\small$\Delta_W$}}};
\draw[dashed] (90:2) -- (90:3.3)
node[near end,right] {{{\small$\Delta_W$}}};
\end{tikzpicture} 
%%%%%%%%%%%%%%%%%%%%%%%%%%%% 
\, ,  \\
& \mathcal D_r(X)(\psi) = 
%%%%%%%%%%%%%%%%%%%%%%%%%%%%
\begin{tikzpicture}[very thick,scale=0.6,color=blue!50!black, baseline,>=stealth]
\nicepalecolourscheme (0,0) circle (3.5);
\fill (2.2,-2.2) circle (0pt) node[white] {{\small$V$}};
\nicecolourscheme (0,0) circle (2);
\fill (1.1,-1.1) circle (0pt) node[white] {{\small$W$}};
%
\draw (0,0) circle (2);
\draw[<-, very thick] (0.100,2) -- (-0.101,2) node[above] {}; 
\draw[<-, very thick] (-0.100,-2) -- (0.101,-2) node[below] {}; 
\fill (135:0) circle (3.3pt) node[right] {{\small$\psi$}};
\fill (130:2) circle (3.3pt) node[left] {{\small$\rho_X$}};
\fill (230:2) circle (3.3pt) node[left] {{\small$\rho^{-1}_{X}$}};
\draw[dashed] (135:0) .. controls +(0,1) and +(0.5,-1) .. (130:2);
\fill (-0.1,1.15) circle (0pt) node {{\small $\Delta_W$}};
\fill (-0.1,-1.15) circle (0pt) node {{\small $\Delta_W$}};
\draw[dashed] (135:0) .. controls +(0,-1) and +(0.5,1) .. (230:2);
\draw[dashed] (270:2) -- (270:3.3)
node[near end,right] {{{\small$\Delta_V$}}};
\draw[dashed] (90:2) -- (90:3.3)
node[near end,right] {{{\small$\Delta_V$}}};
\end{tikzpicture} 
%%%%%%%%%%%%%%%%%%%%%%%%%%%%
= 
%%%%%%%%%%%%%%%%%%%%%%%%%%%%
\begin{tikzpicture}[very thick,scale=0.6,color=blue!50!black, baseline,>=stealth]
\nicepalecolourscheme (0,0) circle (3.5);
\fill (2.2,-2.2) circle (0pt) node[white] {{\small$V$}};
\nicecolourscheme (0,0) circle (2);
\fill (1.1,-1.1) circle (0pt) node[white] {{\small$W$}};
%
\draw[very thin, decorate, decoration={snake, amplitude=0.2mm, segment length=1.0mm}] (270:2) .. controls +(3,0) and +(3,-1) .. (90:2); 
\draw (0,0) circle (2);
\draw[<-, very thick] (0.100,2) -- (-0.101,2) node[above] {}; 
\draw[<-, very thick] (-0.100,-2) -- (0.101,-2) node[below] {}; 
\fill (135:0) circle (3.3pt) node[right] {{\small$\psi$}};
\fill (130:2) circle (3.3pt) node[left] {{\small$\rho_X$}};
\fill (230:2) circle (3.3pt) node[left] {{\small$\rho^{-1}_{X}$}};
\draw[dashed] (135:0) .. controls +(0,1) and +(0.5,-1) .. (130:2);
\fill (-0.1,1.15) circle (0pt) node {{\small $\Delta_W$}};
\fill (-0.1,-1.15) circle (0pt) node {{\small $\Delta_W$}};
\draw[dashed] (135:0) .. controls +(0,-1) and +(0.5,1) .. (230:2);
\draw[dashed] (270:2) -- (270:3.3)
node[near end,right] {{{\small$\Delta_V$}}};
\draw[dashed] (90:2) -- (90:3.3)
node[near end,right] {{{\small$\Delta_V$}}};
\end{tikzpicture} 
%%%%%%%%%%%%%%%%%%%%%%%%%%%%
\label{rightdefectaction}
\end{align}
in the nontrivial case, where we use the adjunction maps and notation of Section~\ref{subsec:alternatives} on the right-hand side. 

\begin{remark}
$\End(\Delta_W) = k[x]/(\partial_{x_i}W)$ is the Hochschild cohomology of $\hmf(k[x], W)$~\cite{d0904.4713}. This space also precisely describes bulk fields of Landau-Ginzburg models with potential~$W$, it is a commutative Frobenius algebra whose nondegenerate pairing
\be\label{bulktopmet}
\langle \phi, \psi \rangle_W = 
(-1)^{n\choose 2} \Res_{k[x]/k} \left[ \frac{\phi \psi \, \underline{\operatorname{d}\!x}}{\partial_{x_1}W, \ldots, \partial_{x_n} W}\right]
\ee
describes 2-point correlators on the sphere, see also Section~\ref{sec:ocTFT}. Furthermore, matrix factorisations of $V-W$ describe defect conditions between different Landau-Ginzburg models. Hence the maps~\eqref{leftdefectaction} and~\eqref{rightdefectaction} have the natural interpretation in terms of defect operators on bulk fields: for example, a bulk field~$\phi$ in the theory with potential~$V$ is mapped to the bulk field $\mathcal D_l(X)(\phi)$ in the theory with potential~$W$ by wrapping around its insertion on the worldsheet a defect line labelled by~$X$, and then collapsing this loop onto the insertion point. This limiting process is non-singular as the bicategory $\LG$ describes the purely topological sector of Landau-Ginzburg models. 

%As already observed in Section~\ref{sec:Introduction}, for the special case $V=0$ the right defect action in~\eqref{defectaction} is precisely the disk correlator of the bulk field~$\psi$ in the presence of the boundary condition~$X$. If we allow for boundary field insertions on the $X$-line we naturally recover the general disk correlator as a 2-morphism in our setting, as follows from the proof of Proposition~\ref{prop:defectaction}. This is another example of the general fact that string diagrams in the bicategory $\LG$ exactly compute the correlators of worldsheets with insertions that they intuitively depict. 
\end{remark}

Using the ``folding trick'', which relates defects to boundary conditions in a product theory, one can argue for explicit expressions for $\mathcal D_l(X)$ and $\mathcal D_r(X)$ (in particular, from this perspective it is also immediate that $\mathcal D_l(X)$ and $\mathcal D_r(X)$ should vanish for odd $m+n$). This was done in~\cite{cr1006.5609} for the case $V=W$ in one variable. Here we use our adjunction formulas to directly prove it for the general case. In addition, we want to allow for the additional decoration with a map $\Phi: X \lra X$ for which we define the following variants of the operators $\mathcal D_l(X)$ and $\mathcal D_r(X)$: 
\be\label{defectactionwithfieldinsertion}
\mathcal D_l^\Phi(X)(\phi) =
%%%%%%%%%%%%%%%%%%%%%%%%%%%%
\begin{tikzpicture}[very thick,scale=0.6,color=blue!50!black, baseline,>=stealth]
\nicecolourscheme (0,0) circle (3.5);
\fill (-2.2,-2.2) circle (0pt) node[white] {{\small$W$}};
\nicepalecolourscheme (0,0) circle (2);
\fill (-1.1,-1.1) circle (0pt) node[white] {{\small$V$}};
%
%\fill (180:2) circle (0pt) node[left] {{\small$x\vphantom{z'}$}};
%\fill (180:2) circle (0pt) node[right] {{\small$z\vphantom{z'}$}};
%\fill (0:2) circle (0pt) node[left] {{\small$z'\vphantom{z'}$}};
%\fill (0:2) circle (0pt) node[right] {{\small$x'\vphantom{z'}$}};
\draw (0,0) circle (2);
\draw[<-, very thick] (0.100,-2) -- (-0.101,-2) node[above] {}; 
\draw[<-, very thick] (-0.100,2) -- (0.101,2) node[below] {}; 
\fill (135:0) circle (3.3pt) node[left] {{\small$\phi$}};
\fill (0:2) circle (3.3pt) node[right] {{\small$\Phi$}};
\fill (50:2) circle (3.3pt) node[right] {{\small$\lambda_{X}$}};
\fill (-50:2) circle (3.3pt) node[right] {{\small$\lambda^{-1}_{X}$}};
\draw[dashed] (135:0) .. controls +(0,1) and +(-0.5,-1) .. (50:2);
\fill (0.41,1.15) circle (0pt) node {{\small $\Delta_V$}};
\fill (0.41,-1.15) circle (0pt) node {{\small $\Delta_V$}};
\draw[dashed] (135:0) .. controls +(0,-1) and +(-0.5,1) .. (-50:2);
\draw[dashed] (270:2) -- (270:3.3)
node[near end,right] {{{\small$\Delta_W$}}};
\draw[dashed] (90:2) -- (90:3.3)
node[near end,right] {{{\small$\Delta_W$}}};
\end{tikzpicture} 
%%%%%%%%%%%%%%%%%%%%%%%%%%%% 
\, , \qquad
\mathcal D_r^\Phi(X)(\psi) = 
%%%%%%%%%%%%%%%%%%%%%%%%%%%%
\begin{tikzpicture}[very thick,scale=0.6,color=blue!50!black, baseline,>=stealth]
\nicepalecolourscheme (0,0) circle (3.5);
\fill (2.2,-2.2) circle (0pt) node[white] {{\small$V$}};
\nicecolourscheme (0,0) circle (2);
\fill (1.1,-1.1) circle (0pt) node[white] {{\small$W$}};
%
\draw (0,0) circle (2);
\draw[<-, very thick] (0.100,2) -- (-0.101,2) node[above] {}; 
\draw[<-, very thick] (-0.100,-2) -- (0.101,-2) node[below] {}; 
\fill (135:0) circle (3.3pt) node[right] {{\small$\psi$}};
\fill (180:2) circle (3.3pt) node[left] {{\small$\Phi$}};
\fill (130:2) circle (3.3pt) node[left] {{\small$\rho_X$}};
\fill (230:2) circle (3.3pt) node[left] {{\small$\rho^{-1}_{X}$}};
\draw[dashed] (135:0) .. controls +(0,1) and +(0.5,-1) .. (130:2);
\fill (-0.1,1.15) circle (0pt) node {{\small $\Delta_W$}};
\fill (-0.1,-1.15) circle (0pt) node {{\small $\Delta_W$}};
\draw[dashed] (135:0) .. controls +(0,-1) and +(0.5,1) .. (230:2);
\draw[dashed] (270:2) -- (270:3.3)
node[near end,right] {{{\small$\Delta_V$}}};
\draw[dashed] (90:2) -- (90:3.3)
node[near end,right] {{{\small$\Delta_V$}}};
\end{tikzpicture} 
%%%%%%%%%%%%%%%%%%%%%%%%%%%%
\, . 
\ee

\begin{proposition}\label{prop:defectaction}
For any $X\in \hmf(k[x_1,\ldots,x_n,z_1,\ldots,z_m], V-W)$, $\Phi \in \End(X)$, $\phi\in \End(\Delta_V)$ and $\psi\in \End(\Delta_W)$ we have
\begin{align*}
\mathcal D_l^\Phi(X)(\phi) & 
= (-1)^{{m\choose 2} + {n\choose 2}}\Res_{k[x,z]/k[x]} \left[ \frac{\phi(z) \str\big( \Phi \, \partial_{x_1} d_{X}\ldots \partial_{x_n} d_{X} \partial_{z_1} d_{X}\ldots \partial_{z_m} d_{X} \big) \underline{\operatorname{d}\! z}}{\partial_{z_1} V, \ldots, \partial_{z_m} V} \right] \, , \\
\mathcal D_r^\Phi(X)(\psi) & = (-1)^{{m\choose 2} + {n\choose 2}}  \Res_{k[x,z]/k[z]} \left[ \frac{\psi(x) \str\big( \Phi \, \partial_{x_1} d_{X}\ldots \partial_{x_n} d_{X} \partial_{z_1} d_{X}\ldots \partial_{z_m} d_{X} \big) \underline{\operatorname{d}\! x}}{\partial_{x_1} W, \ldots, \partial_{x_n} W} \right] \, . 
\end{align*}
\end{proposition}

\begin{proof}
We treat the case of $\mathcal D_r^\Phi(X)$ in detail, the argument for $\mathcal D_l^\Phi(X)$ works similarly. Since $\End(\Delta_W) = k[x]/(\partial_{x_i}W)$ and $\End(\Delta_V) = k[z]/(\partial_{z_i}V)$ we are free to to identify the variables on both sides of the unit 1-endomorphisms at appropriate places. Furthermore, $\rho_X$ will project out all non-zero degree contributions coming from the action of $\rho_{X}^{-1}$, so $\rho_{X} \circ (1_X \otimes \psi)\circ \rho_{X}^{-1}$ is simply multiplication by the polynomial $\psi(z)$. 

In the lower part of the expression for $\mathcal D_r^\Phi(X)(\psi)$ in~\eqref{defectactionwithfieldinsertion} we have 
$$
\coev_X(1) = \sum_j (-1)^{{m\choose 2} + m|e_j|} \partial_{[1]} d_X \ldots \partial_{[m]} d_X e_j \otimes e_j^* \, . 
$$
%\begin{align}
%\widetilde\coev_X (1) & = \sum_j (-1)^{|e_j|} (\varepsilon\Psi) \left( (-\At_{X})^n (e_j^*\otimes e_j) \right) \nonumber \\
%& = \sum_j (-1)^{|e_j| + n} (-1)^{(|e_j| + 1)+\ldots +(|e_j| + n) + n|e_j|} e^*_{j} \otimes e_{k_n} \otimes (\varepsilon\Psi) \left( d(d_{X})_{k_n k_{n-1}} \ldots d(d_{X})_{k_1 j} \right) \nonumber \\
%& = \sum_j (-1)^{|e_j|(|e_j| + n) + {n+1\choose 2} + n} e_j^* \otimes e_{k_n}\big( \partial_{[1]} d_{X} \ldots \partial_{[n]} d_{X} \big)_{k_n j} \nonumber \\
%& = \sum_j (-1)^{|e_j|(|e_j| + n) + {n\choose 2}} e_j^* \otimes \big( \partial_{x_1} d_{X} \ldots \partial_{x_n} d_{X} \big) (e_{j}) \label{coevtilde1}
%\end{align}
%which we identify with $(-1)^{{n\choose 2}} \partial_{x_1} d_{X} \ldots \partial_{x_n} d_{X}$ in $\End(X)$. Note that in the last step leading to~\eqref{coevtilde1} we set $\partial_{[i]} d_{X}(x,z) = \partial_{x_i} d_{X}(x,z)$ since $x=x'$ in $\End(\Delta_W)$. 
Next we apply~$\Phi$, $\psi$ and $\widetilde\eval_X$. Since the latter maps back to~$\Delta_V$ we may set $\partial_{[i]} d_X = \partial_{z_i} d_X$. Thus we obtain
$$
\mathcal D_r^\Phi(X)(\psi) = (-1)^{{m\choose 2} + {n\choose 2}} \Res_{k[x,z]/k[z]} \left[ \frac{\psi(x) \str\big( \partial_{x_1} d_{X}\ldots \partial_{x_n} d_{X} \partial_{z_1} d_{X}\ldots \partial_{z_m} d_{X} \big) \underline{\operatorname{d}\! x}}{\partial_{x_1} W, \ldots, \partial_{x_n} W} \right] + \mathcal O(\theta) \, . 
$$
Here we collectively denote the contributions from $\widetilde\eval_X$ of non-zero degree in the Koszul complex $\Delta_V$ by $\mathcal O(\theta)$. Since we know that $\mathcal D_r^\Phi(X)(\psi)$ is a morphism in $\End(\Delta_V) = k[z]/(\partial_{z_i}V)$ it follows that $\mathcal O(\theta)$ must be null-homotopic, thus concluding the proof. 
\end{proof}

\begin{corollary}
For any $X\in \hmf(k[x,z], V-W)$ the operators $\mathcal D_l(X)$ and $\mathcal D_r(X)$ are adjoint with respect to the pairings~\eqref{bulktopmet}, i.\,e.~we have
\be\label{Dadjoint}
\big\langle \mathcal D_l(X)(\phi), \psi \big\rangle_W = \big\langle \phi , \mathcal D_r(X)(\psi) \big\rangle_V
\ee
for all $\phi\in \End(\Delta_V)$ and $\psi\in \End(\Delta_W)$. 
\end{corollary}

\begin{proof}
This directly follows from the explicit expressions for $\mathcal D_l(X)$, $\mathcal D_r(X)$ and $\big\langle -, -\big\rangle_W$, $\big\langle -, -\big\rangle_V$, together with the transitivity rule for residues. 
\end{proof}


\begin{remark}
We recall the physical interpretation of the relation~\eqref{Dadjoint}. Both sides of this equation are 2-point correlators on the Riemann sphere, with a defect line labelled by~$X$ wrapped around counterclockwise the bulk field~$\phi$, or wrapped around~$\psi$ in clockwise fashion. That both correlators should be equal follows from the fact that the topological defect can be moved around the sphere at no cost: 
$$
\left\langle
%%%%%%%%%%%%%%%%%%%%%%%%%%%%
\begin{tikzpicture}[baseline=-0.1cm]
\def\R{1.85}
\def\angEl{45}
\filldraw[ball color= white!77!blue,draw=white] (0,0) circle (\R);
\DrawLatitudeCircleU[\R,rotate=130,very thick, blue]{65}
\fill (-0.95,-0.83) circle (1.3pt) node[above] {{\small$\phi$}}; 
\fill (0.95,-0.83) circle (1.3pt) node[above] {{\small$\psi$}}; 
\end{tikzpicture}
%%%%%%%%%%%%%%%%%%%%%%%%%%%%
\right\rangle
=
\left\langle
%%%%%%%%%%%%%%%%%%%%%%%%%%%%
\begin{tikzpicture}[baseline=-0.1cm]
\def\R{1.85}
\def\angEl{45}
\filldraw[ball color= white!77!blue,draw=white] (0,0) circle (\R);
\DrawLongitudeCircle[\R]{80}
\fill (-0.95,-0.83) circle (1.3pt) node[above] {{\small$\phi$}}; 
\fill (0.95,-0.83) circle (1.3pt) node[above] {{\small$\psi$}}; 
\end{tikzpicture}
%%%%%%%%%%%%%%%%%%%%%%%%%%%%
\right\rangle
=
\left\langle
%%%%%%%%%%%%%%%%%%%%%%%%%%%%
\begin{tikzpicture}[baseline=-0.1cm]
\def\R{1.85}
\def\angEl{45}
\filldraw[ball color= white!77!blue,draw=white] (0,0) circle (\R);
\DrawLatitudeCircle[\R,rotate=-130, very thick, blue]{65}
\fill (-0.95,-0.83) circle (1.3pt) node[above] {{\small$\phi$}}; 
\fill (0.95,-0.83) circle (1.3pt) node[above] {{\small$\psi$}}; 
\end{tikzpicture}
%%%%%%%%%%%%%%%%%%%%%%%%%%%%
\right\rangle .
$$
\end{remark}

The defect operators satisfy the following compatibility conditions: 

\begin{proposition}
\begin{enumerate}
\item $\mathcal D_l(\Delta) = 1 = \mathcal D_r(\Delta)$ (TODO: actually, we get $(-1)^{n\choose 2}$ at the moment!), 
\item $\mathcal D_l(X) = \mathcal D_r(X^\dual)$, 
\item $\mathcal D_l(Y \otimes X) = \mathcal D_l(X) \circ \mathcal D_l(Y)$, 
\item $\mathcal D_r(Y \otimes X) = \mathcal D_r(Y) \circ \mathcal D_r(X)$. 
\end{enumerate}
\label{prop:defectactionproperties}
\end{proposition}

\begin{proof}
(i) TODO: either cite Polishchuk-Vaintrob and/or use $\coev_\Delta= \lambda^{-1}_\Delta$ etc. (well, actually PV tell us that we get $(-1)^{n\choose 2}$)

(ii): In the proof of the explicit formula for $\mathcal D_l(X)$ in Proposition~\ref{prop:defectaction} we could also have used the alternative representations for $\eval_X$ and $\widetilde\coev_X$ in terms of $d_{X^\dual}$. Then the assertion immediately follows. 

(iii) \& (iv): These relations follow from Proposition~\ref{prop:monoidalproperty} and a straightforward calculation with string diagrams, see e.\,g.~the proof of~\cite[Lemma~3.1]{cr1006.5609}. (TODO: do we want to show this calculation with wiggly lines?)
\end{proof}


\section{Open/closed topological field theory}\label{sec:ocTFT}

In this section we describe how the structure of two-dimensional open/closed topological field theory (TFT) is completely  captured in the bicategory $\LG$. We will see how all TFT data are naturally encoded in string diagrams that directly translate the physical intuition into straightforwardly computable 2-morphisms. More generally, this allows for the computation of any correlator (of worldsheets of arbitrary genus, possibly with boundaries and defect lines). As an example we will compute an annulus correlator, thereby providing a new, simple proof of the Cardy condition. 

Recall from~\cite{l0312, ms0609042} that one way to present a two-dimensional open/closed TFT is by the data of 
\begin{itemize}
\item a commutative Frobenius algebra~$C$, 
\item a Calabi-Yau category~$\mathcal O$, 
\item \textsl{bulk-boundary maps} $\beta_A: C \lra \End_{\mathcal O}(A)$ and \textsl{boundary-bulk maps} $\beta^A: \End_{\mathcal O}(A) \lra C$ for all $A\in \mathcal O$. 
\end{itemize}
These data are subject to the following conditions. 
\begin{itemize}
\item The bulk-boundary maps $\beta_A$ are morphisms of unital algebras that map into the centre of $\End_{\mathcal O}(A)$. 
\item $\beta_A$ and $\beta^A$ are mutually adjoint with respect to the nondegenerate pairings $\langle -,- \rangle$ on~$C$ and $\langle -,- \rangle_A$ on $\End_{\mathcal O}(A)$ (which are part of the Frobenius and Calabi-Yau structure): 
$$
\langle \beta_A(\phi) , \psi \rangle_A = \langle \phi, \beta^A(\psi) \rangle
$$
for all $\phi\in C$ and $\psi\in \End_{\mathcal O}(A)$. 
\item The \textsl{Cardy condition} is satisfied, i.\,e.~we have
$$
\str ( {}_\psi m_\varphi ) = \langle \beta^A(\varphi), \beta^B(\psi) \rangle
$$
for all $\varphi: A \lra A$, $\psi: B \lra B$ where ${}_\psi m_\varphi (\alpha) = \psi\alpha\varphi$ for all $\alpha \in \Hom_{\mathcal O}(A,B)$. 
\end{itemize}

Every Landau-Ginzburg model with potential $W\in R = k[x_1,\ldots,x_n]$ gives rise to an open/closed TFT with $C = R/(\partial W)$, $\mathcal O = \hmf(R,W)$, 
\be\label{bubobobu}
\beta_X : \phi \lmt \phi \cdot 1_X \, , \qquad  
\beta^X : \psi \lmt (-1)^{n\choose 2} \str(\psi \, \partial_{x_1} d_X \ldots \partial_{x_n} d_X) 
\ee
and the bulk and boundary pairings 
\begin{align}
\langle \phi_1 , \phi_2 \rangle 
& = 
(-1)^{n\choose 2} \Res_{k[x]/k} \left[ \frac{\phi_1 \phi_2 \, \underline{\operatorname{d}\!x}}{\partial_{x_1}W, \ldots, \partial_{x_n} W}\right] , \label{bulkpairing} \\
\langle \psi_1 , \psi_2 \rangle_X 
& = (-1)^{n\choose 2} \Res_{k[x]/k} \left[ \frac{ \str (\psi_1 \psi_2 \, \partial_{x_1} d_X \ldots \partial_{x_n} d_X) \,  \underline{\operatorname{d}\!x}}{\partial_{x_1}W, \ldots, \partial_{x_n} W}\right] . \label{KapuLi}
\end{align}
The hardest part in establishing this result is to prove the nondegeneracy of the Kapustin-Li pairing~\eqref{KapuLi} and that the Cardy condition holds (the fact that~\eqref{bulkpairing} is nondegenerate is a classical result in residue theory, and checking the remaining axioms is obvious or straightforward); this was first done in~\cite{m0912.1629} and~\cite{pv1002.2116}, respectively, in the case where~$k$ is a field. In Section~\ref{section:dualityadjointop} and at the end of this section we will give (new) proofs for the general case. 

Before turning to the Cardy condition we wish to explain how the above data can be extracted from the bicategory $\LG$ using the powerful and suggestive string diagram language.\footnote{Note however that $\LG$ contains much more information than just the structure of open/closed TFT.} For $X\in \hmf(R,W)$ the bulk-boundary and boundary-bulk maps~\eqref{bubobobu} are given by
$$
\beta_X(\phi) = 
%%%%%%%%%%%%%%%%%%%%%%%%%%%%
\begin{tikzpicture}[very thick,scale=0.6,color=blue!50!black, baseline,>=stealth]
\nicecolourscheme (-2.6,-3.3) rectangle (0,3.3);
\fill (-2.2,-2.2) circle (0pt) node[white] {{\small$W$}};
%
\draw[->, very thick] (0,-3.3) -- (0,0); 
\draw[very thick] (0,0) -- (0,3.3);
\draw[dashed] (0,-2) .. controls +(-0.5,1) and +(0,-1) .. (-1,0);
\draw[dashed] (-1,0) .. controls +(0,1) and +(-0.5,-1) .. (0,2);
\fill (0,0) circle (0pt) node[right] {{\small$X$}};
\fill (-1,0) circle (3.3pt) node[left] {{\small$\phi$}};
\end{tikzpicture} 
%%%%%%%%%%%%%%%%%%%%%%%%%%%% 
\, , \qquad 
\beta^X(\psi) = 
%%%%%%%%%%%%%%%%%%%%%%%%%%%%
\begin{tikzpicture}[very thick,scale=0.6,color=blue!50!black, baseline,>=stealth]
\nicecolourscheme (0,0) circle (3.5);
\fill (-2.2,-2.2) circle (0pt) node[white] {{\small$W$}};
%
\shadedraw[top color=white, bottom color=white, draw=white] (0,0) circle (2);
\draw (0,0) circle (2);
\draw[->, very thick] (0.100,-2) -- (-0.101,-2) node[above] {}; 
\draw[->, very thick] (-0.100,2) -- (0.101,2) node[below] {}; 
\fill (90:2) circle (0pt) node[below] {{\small$X$}};
\fill (180:2) circle (3.3pt) node[left] {{\small$\psi$}};
\draw[dashed] (270:2) -- (270:3.3)
node[near end,right] {{{\small$\Delta_W$}}};
\draw[dashed] (90:2) -- (90:3.3)
node[near end,right] {{{\small$\Delta_W$}}};
\end{tikzpicture} 
%%%%%%%%%%%%%%%%%%%%%%%%%%%% 
$$
where we used the identification $R/(\partial W) \cong \End_{\hmf(R,W)}(\Delta_W)$ and a special case of Proposition~\ref{prop:defectaction}. Another special case of the same proposition allows us to recover the Kapustin-Li pairing as the obvious 2-morphism representing the disk correlator, 
$$
\langle \psi_1 , \psi_2 \rangle_X = 
%%%%%%%%%%%%%%%%%%%%%%%%%%%%
\begin{tikzpicture}[very thick,scale=0.6,color=blue!50!black, baseline,>=stealth]
\nicecolourscheme (0,0) circle (2);
\fill (-1.1,-1.1) circle (0pt) node[white] {{\small$W$}};
%
\draw (0,0) circle (2);
\draw[<-, very thick] (0.100,-2) -- (-0.101,-2) node[above] {}; 
\draw[<-, very thick] (-0.100,2) -- (0.101,2) node[below] {}; 
\fill (-22.5:2) circle (3.3pt) node[right] {{\small$\psi_1$}};
\fill (22.5:2) circle (3.3pt) node[right] {{\small$\psi_2$}};
\fill (270:2) circle (0pt) node[below] {{\small$X$}};
\end{tikzpicture} 
%%%%%%%%%%%%%%%%%%%%%%%%%%%% 
\, .
$$
Note that here and from now on we do no longer display dashed lines for the unit 1-endomorphism~$\Delta_W$. 

The bulk pairing $\langle -,- \rangle = \langle -,- \rangle_W$ describes the 2-point \textsl{sphere} correlator; flattening the sphere suggests the identity
\be\label{DeltaDisk}
\langle \phi_1, \phi_2 \rangle_W = 
%%%%%%%%%%%%%%%%%%%%%%%%%%%%
\begin{tikzpicture}[very thick,scale=0.6,color=blue!50!black, baseline,>=stealth]
\nicecolourscheme (0,0) circle (2);
\fill (0,0) circle (0pt) node[white] {{\small$W(x)-W(y)$}};
%
\draw (0,0) circle (2);
\draw[<-, very thick] (0.100,-2) -- (-0.101,-2) node[above] {}; 
\draw[<-, very thick] (-0.100,2) -- (0.101,2) node[below] {}; 
%\draw[dashed] (135:0) .. controls +(0,1) and +(-0.5,-1) .. (50:2);
%\draw[dashed] (135:0) .. controls +(0,-1) and +(-0.5,1) .. (-50:2);
\fill (-22.5:2) circle (3.3pt) node[right] {{\small$\phi_1(x)$}};
\fill (22.5:2) circle (3.3pt) node[right] {{\small$\phi_2(x)$}};
\fill (270:2) circle (0pt) node[below] {{\small$\Delta_W$}};
\end{tikzpicture} 
%%%%%%%%%%%%%%%%%%%%%%%%%%%% 
\ee
where we view $\Delta_W$ as a 1-morphism $(k,0) \lra (\Re, \widetilde W)$, i.\,e.~as a boundary condition of the doubled theory with potential $W(x)-W(y)$. The above equality indeed holds as follows from our explicit expressions for the adjunction maps together with~\cite[Proposition\,4.1.2]{pv1002.2116}. A more conceptual derivation of~\eqref{DeltaDisk} would involve to endow the bicategory $\LG$ with a monoidal structure (so as to give rigorous meaning to the process of folding worldsheets together). 

We have just seen how the structure of open/closed TFT embeds into $\LG$. In principle this is enough to compute arbitrary correlators using the factorisation property of TFT. However, one can also compute more general correlators directly in $\LG$: all we have to do is to interpret the physical picture of a worldsheet~$\Sigma$ with insertions and defect lines as the associated string diagram representing a 2-morphism $k \lra k$ which is the value of the correlator of~$\Sigma$. 

As an illustrative example let us consider the genus-3 correlator of the worldsheet with two phases governed by the potentials~$V$ and~$W$, depicted in the upper part of Figure~\ref{fig:worldsheet}. Here we label boundaries, defect lines, bulk fields, boundary fields, defect fields by $Y, X, \phi, \psi, \Phi$ with appropriate indices, respectively. The string diagram computing this correlator is shown in the lower part of Figure~\ref{fig:worldsheet}. It is exactly the same as the worldsheet picture, apart form the fact that we interpret the relevant parts as evaluation and coevaluation maps, and if the numbers~$m$ and~$n$ of variables on which~$V$ and~$W$ depend are both odd then we use the wiggly line notation explained in Section~\ref{subsec:alternatives}.\footnote{Note that if~$m$ and~$n$ have opposite parity then by the results of Section~\ref{sec:defectaction} the correlator vanishes as it corresponds to a map in $\Hom_{\hmf(k,0)}(k,k[1]) = 0$.} 
We point out that in general the nature of the 1-morphisms/defects involved determines uniquely whether a given cup or cap is a tilded or untilded adjunction map. There may be ambiguities how to make wiggly lines meet, but the value of the 2-morphism obtained is independent of this choice. 

\begin{figure}[h!]
$$
%%%%%%%%%%%%%%%%%%%%%%%%%%%%
\begin{tikzpicture}[very thick,scale=0.95,color=blue!50!black, baseline,>=stealth]
%
% big circle
\nicecolourscheme[
	decoration={markings, mark=at position 0.25 with {\arrow{>}}}, postaction={decorate}
	] 
	(0,0) circle (5.2);
\draw (0,0) circle (5.2);
%
% draw 4 circles inside big one: 
\nicepalecolourscheme (-2,-2) circle (1.5);
\shadedraw[top color=white, bottom color=white, draw=white] (-2,2) circle (1.5);
\shadedraw[top color=white, bottom color=white, draw=white] (2,2) circle (1.5);
\shadedraw[top color=white, bottom color=white, draw=white] (2,-2) circle (1.5);
%
\draw[
	decoration={markings, mark=at position 0.25 with {\arrow{>}}}, postaction={decorate}
	] 
	(-2,-2) circle (1.5);
\draw[
	decoration={markings, mark=at position 0.25 with {\arrow{<}}}, postaction={decorate}
	]
	(2,-2) circle (1.5);
\draw[
	decoration={markings, mark=at position 0.25 with {\arrow{<}}}, postaction={decorate}
	]
	 (-2,2) circle (1.5);
\draw[
	decoration={markings, mark=at position 0.25 with {\arrow{<}}}, postaction={decorate}
	]
	 (2,2) circle (1.5);
%
% potentials: 
\fill (0.5,-4.5) circle (0pt) node[white] {{\small$W$}};
\fill (-1.3,-2.8) circle (0pt) node[white] {{\small$V$}};
%
% field insertions: 
\fill (225:5.2) circle (2.2pt) node[left] {{\small$\Psi_1$}};
\coordinate (Psi2) at ($ (-2,2) + (45:1.5) $);
\fill (Psi2) circle (2.2pt) node[right] {{\small$\Psi_2$}};
\fill (35:5.2) circle (2.2pt) node[right] {{\small$\Psi_3$}};
\fill (75:5.2) circle (2.2pt) node[above] {{\small$\Phi_1$}};
\coordinate (Phi2) at ($ (2,2) + (130:1.5) $);
\fill (Phi2) circle (2.2pt) node[left] {{\small$\Phi_2$}};
\coordinate (Phi3) at ($ (2,2) + (-10:1.5) $);
\fill (Phi3) circle (2.2pt) node[right] {{\small$\Phi_3$}};
\fill (4.2,0) circle (2.2pt) node[right] {{\small$\Phi_4$}};
\coordinate (Phi5) at ($ (2,-2) + (10:1.5) $);
\fill (Phi5) circle (2.2pt) node[right] {{\small$\Phi_5$}};
\fill (-2,-2) circle (2.2pt) node[right] {{\small$\phi_1$}};
\fill (-4,-1) circle (2.2pt) node[left] {{\small$\phi_2$}};
%
% more defects: 
\draw[
	decoration={markings, mark=at position 0.5 with {\arrow{>}}}, postaction={decorate}
	]
	 (Phi2) .. controls +(-0.2,1) and +(0.3,-0.5) .. (75:5.2); % X_2
\draw[
	decoration={markings, mark=at position 0.5 with {\arrow{>}}}, postaction={decorate}
	]
	 (4.2,0) .. controls +(0,1) and +(0.2,-0.5) .. (Phi3); 	% X_3
\draw[
	decoration={markings, mark=at position 0.5 with {\arrow{>}}}, postaction={decorate}
	]
	 (Phi5) .. controls +(0.2,0.5) and +(0,-1) .. (4.2,0); 	% X_4
%
% label arrowheads: 
\fill (0,5.45) circle (0pt) node {{\small$Y_1$}}; %Y_1
%\fill (0,-5.5) circle (0pt) node {{\small$\widetilde\coev_{Y_1}$}};
\fill (-2,3.8) circle (0pt) node {{\small$Y_2$}}; %Y_2
%\fill (-2,0.2) circle (0pt) node {{\small$\coev_{Y_2}$}};
\fill (2,3.8) circle (0pt) node {{\small$Y_3$}}; %Y_3
%\fill (2,0.2) circle (0pt) node {{\small$\coev_{Y_3}$}};
\fill (2,-0.2) circle (0pt) node {{\small$Y_4$}}; %Y_4
%\fill (2,-3.8) circle (0pt) node {{\small$\coev_{Y_4}$}};
%
\fill (-2,-0.2) circle (0pt) node {{\small$X_1$}}; %X_1
%\fill (-2,-3.8) circle (0pt) node {{\small$\widetilde\coev_{X_1}$}};
\fill (0.75,4.1) circle (0pt) node {{\small$X_2$}}; % X_2
\fill (3.7,0.8) circle (0pt) node {{\small$X_3$}}; % X_3
\fill (3.7,-0.8) circle (0pt) node {{\small$X_4$}}; % X_4
\end{tikzpicture} 
%%%%%%%%%%%%%%%%%%%%%%%%%%%%
$$
\vspace{0.05cm}
$$
%%%%%%%%%%%%%%%%%%%%%%%%%%%%
\begin{tikzpicture}[very thick,scale=0.95,color=blue!50!black, baseline,>=stealth]
%
% big circle
\nicecolourscheme[
	decoration={markings, mark=at position 0.25 with {\arrow{>}}, mark=at position 0.75 with {\arrow{>}}}, postaction={decorate}
	] 
	(0,0) circle (5.2);
\draw (0,0) circle (5.2);
%
% draw 4 circles inside big one: 
\nicepalecolourscheme (-2,-2) circle (1.5);
\shadedraw[top color=white, bottom color=white, draw=white] (-2,2) circle (1.5);
\shadedraw[top color=white, bottom color=white, draw=white] (2,2) circle (1.5);
\shadedraw[top color=white, bottom color=white, draw=white] (2,-2) circle (1.5);
%
\draw[
	decoration={markings, mark=at position 0.25 with {\arrow{>}}, mark=at position 0.75 with {\arrow{>}}}, postaction={decorate}
	] 
	(-2,-2) circle (1.5);
\draw[
	decoration={markings, mark=at position 0.25 with {\arrow{<}}, mark=at position 0.75 with {\arrow{<}}}, postaction={decorate}
	]
	(2,-2) circle (1.5);
\draw[
	decoration={markings, mark=at position 0.25 with {\arrow{<}}, mark=at position 0.75 with {\arrow{<}}}, postaction={decorate}
	]
	 (-2,2) circle (1.5);
\draw[
	decoration={markings, mark=at position 0.25 with {\arrow{<}}, mark=at position 0.75 with {\arrow{<}}}, postaction={decorate}
	]
	 (2,2) circle (1.5);
%
% potentials: 
\fill (0.5,-4.5) circle (0pt) node[white] {{\small$W$}};
\fill (-1.3,-2.8) circle (0pt) node[white] {{\small$V$}};
%
% field insertions: 
\fill (225:5.2) circle (2.2pt) node[left] {{\small$\Psi_1$}};
\coordinate (Psi2) at ($ (-2,2) + (45:1.5) $);
\fill (Psi2) circle (2.2pt) node[right] {{\small$\Psi_2$}};
\fill (35:5.2) circle (2.2pt) node[right] {{\small$\Psi_3$}};
\fill (75:5.2) circle (2.2pt) node[above] {{\small$\Phi_1$}};
\coordinate (Phi2) at ($ (2,2) + (130:1.5) $);
\fill (Phi2) circle (2.2pt) node[left] {{\small$\Phi_2$}};
\coordinate (Phi3) at ($ (2,2) + (-10:1.5) $);
\fill (Phi3) circle (2.2pt) node[right] {{\small$\Phi_3$}};
\fill (4.2,0) circle (2.2pt) node[right] {{\small$\Phi_4$}};
\coordinate (Phi5) at ($ (2,-2) + (10:1.5) $);
\fill (Phi5) circle (2.2pt) node[right] {{\small$\Phi_5$}};
\fill (-2,-2) circle (2.2pt) node[right] {{\small$\phi_1$}};
\fill (-4,-1) circle (2.2pt) node[left] {{\small$\phi_2$}};
%
% more defects: 
\draw[
	decoration={markings, mark=at position 0.5 with {\arrow{>}}}, postaction={decorate}
	]
	 (Phi2) .. controls +(-0.2,1) and +(0.3,-0.5) .. (75:5.2); % X_2
\draw[
	decoration={markings, mark=at position 0.5 with {\arrow{>}}}, postaction={decorate}
	]
	 (4.2,0) .. controls +(0,1) and +(0.2,-0.5) .. (Phi3); 	% X_3
\draw[
	decoration={markings, mark=at position 0.5 with {\arrow{>}}}, postaction={decorate}
	]
	 (Phi5) .. controls +(0.2,0.5) and +(0,-1) .. (4.2,0); 	% X_4
%
% wiggly lines:
\draw[very thin, decorate, decoration={snake, amplitude=0.2mm, segment length=1.0mm}] (-2,0.5) .. controls +(-3.5,0) and +(-4,-0.5) .. (0,5.2); 
\draw[very thin, decorate, decoration={snake, amplitude=0.2mm, segment length=1.0mm}] (-2,-3.5) .. controls +(-2.5,0) and +(-2,-1) .. (-2,-0.5); 
\draw[very thin, decorate, decoration={snake, amplitude=0.2mm, segment length=1.0mm}] (2,-3.5) .. controls +(-3,0) and +(0,-1) .. (0,3); 
\draw[very thin, decorate, decoration={snake, amplitude=0.2mm, segment length=1.0mm}] (2,0.5) .. controls +(-1.7,0) and +(0,-1) .. (0,3); 
%
% label arrowheads: 
\fill (0,5.4) circle (0pt) node {{\small$\eval_{Y_1}$}}; %Y_1
\fill (0,-5.5) circle (0pt) node {{\small$\widetilde\coev_{Y_1}$}};
\fill (-2,3.8) circle (0pt) node {{\small$\widetilde\eval_{Y_2}$}}; %Y_2
\fill (-2,0.2) circle (0pt) node {{\small$\coev_{Y_2}$}};
\fill (2,3.8) circle (0pt) node {{\small$\widetilde\eval_{Y_3}$}}; %Y_3
\fill (2,0.2) circle (0pt) node {{\small$\coev_{Y_3}$}};
\fill (2,-0.2) circle (0pt) node {{\small$\widetilde\eval_{Y_4}$}}; %Y_4
\fill (2,-3.8) circle (0pt) node {{\small$\coev_{Y_4}$}};
%
\fill (-2,-0.25) circle (0pt) node {{\small$\eval_{X_1}$}}; %X_1
\fill (-2,-3.8) circle (0pt) node {{\small$\widetilde\coev_{X_1}$}};
\fill (0.75,4.1) circle (0pt) node {{\small$X_2$}}; % X_2
\fill (3.7,0.8) circle (0pt) node {{\small$X_3$}}; % X_3
\fill (3.7,-0.8) circle (0pt) node {{\small$X_4$}}; % X_4
\end{tikzpicture} 
%%%%%%%%%%%%%%%%%%%%%%%%%%%%
$$
\caption{A genus-3 worldsheet (top) and the 2-morphism (bottom) in $\LG$ computing its correlator} 
\label{fig:worldsheet} 
\end{figure}


\subsection{Cardy condition}

Diagrams much simpler than the one in Figure~\ref{fig:worldsheet} can already lead to relevant insight. As another application of the adjunctions in $\LG$ we will now give a novel proof of the Cardy condition. For Landau-Ginzburg models this deep result long eluded rigirous proofs, which were only recently given in~\cite{pv1002.2116, dm1102.2957} using rather heavy machinery. With our expressions for evaluation and coevaluation maps in $\LG$ it will follow rather effortlessly for arbitrary commutative noetherian $\nQ$-algebras~$k$, e.\,g.~$k=\nC[t_1,\ldots,t_d]$, simply from considerations of one single diagram. 

Let us again fix a potential $W\in k[x] = k[x_1,\ldots,x_n]$, two matrix factorisations $X,Y \in \hmf(k[x], W)$, and also two maps $\varphi: X\lra X$, $\psi: Y\lra Y$. We interpret~$X$ and~$Y$ as 1-morphisms $(k,0)\lra (k[x],W)$ in $\LG$, and in this way we obtain the 2-morphism
\be\label{annulus}
%%%%%%%%%%%%%%%%%%%%%%%%%%%%
\begin{tikzpicture}[very thick,scale=0.8,color=blue!50!black, baseline,>=stealth]
\nicecolourscheme (0,0) circle (2);
\nicereallynocolourscheme (0,0) circle (1);
\fill (1.5,0) circle (0pt) node[white] {{\small$W$}};
\draw (0,0) circle (2);
\draw[->, very thick] (-0.100,2) -- (-0.101,2) node[above] {}; %{{\small$\eval_Y$}}; 
%\draw[->, very thick] (0.100,-2) -- (0.101,-2) node[below] {}; % {{\small$\widetilde\coev_Y$}}; 
\fill (45:2) circle (2.5pt) node[right] {{\small$\psi$}};
%
\draw (0,0) circle (1);
\draw[->, very thick] (0.100,1) -- (0.101,1) node[above] {}; % {{\small$\widetilde\eval_X$}}; 
%\draw[->, very thick] (-0.100,-1) -- (-0.101,-1) node[below] {}; % {{\small$\coev_X$}}; 
\fill (135:1) circle (2.5pt) node[left] {{\small$\varphi$}};
\fill (180:0.9) circle (0pt) node[left] {{\small$X$}};
\fill (0:2.7) circle (0pt) node[left] {{\small$Y$}};
\end{tikzpicture} 
%%%%%%%%%%%%%%%%%%%%%%%%%%%%
\equiv \; 
%%%%%%%%%%%%%%%%%%%%%%%%%%
\begin{tikzpicture}[very thick,scale=0.8,color=blue!50!black, baseline,>=stealth]
\nicecolourscheme (0,0) circle (2);
\nicereallynocolourscheme (0,0) circle (1);
\fill (1.5,0) circle (0pt) node[white] {{\small$W$}};
\draw (0,0) circle (2);
\draw[->, very thick] (-0.100,2) -- (-0.101,2) node[above] {{\small$\eval_Y$}}; 
\draw[->, very thick] (0.100,-2) -- (0.101,-2) node[below] {{\small$\widetilde\coev_Y$}}; 
\fill (45:2) circle (2.5pt) node[right] {{\small$\psi$}};
%
\draw (0,0) circle (1);
\draw[->, very thick] (0.100,1) -- (0.101,1) node[above] {{\small$\widetilde\eval_X$}}; 
\draw[->, very thick] (-0.100,-1) -- (-0.101,-1) node[below] {{\small$\coev_X$}}; 
\fill (135:1) circle (2.5pt) node[left] {{\small$\varphi$}};
%
\draw[very thin, decorate, decoration={snake, amplitude=0.2mm, segment length=1.0mm}] (270:1) .. controls +(-2,0) and +(-2,-0.2) .. (90:2); %decorate, decoration={zigzag}
\end{tikzpicture} 
%%%%%%%%%%%%%%%%%%%%%%%%%% 
\ee
which is a map $k\lra k$ that we call~$C$. We can think of~$C$ as the value of the annulus correlator with boundary conditions $X,Y$ and boundary fields $\varphi,\psi$. 

To put ourselves in a position to better understand and manipulate~\eqref{annulus} let us spell out its constituents. From the general expressions in Section~\ref{sec:derivcoeval} for evaluation and coevaluation maps we find
\begin{align}
& \widetilde\eval_X (e_i \otimes e_j^*) = (-1)^{|e_j|} \delta_{ij} + \mathcal O(\theta) \, , \qquad 
\eval_Y (f_i^* \otimes f_j) = (-1)^{{n+1\choose 2} + n |f_j|} \Res_{k[x]/k} \left[ \frac{ (\Lambda_Y)_{ij} \,\underline{\operatorname{d}\! x}}{\partial_1 W, \ldots, \partial_n W} \right] , \nonumber \\
& \coev_X (1) = \sum_j (-1)^{{n\choose 2} + n |e_j|} \bar\Lambda_X e_j \otimes e_j^* \, , \qquad 
\widetilde\coev_Y (1) = \sum_j (-1)^{|f_j|} f_j^* \otimes f_j \label{evcoeveasy}
\end{align}
where $\bar\Lambda_X = \partial_{[1]} d_X\ldots \partial_{[n]} d_X$ and $\Lambda_Y = \partial_1 d_Y\ldots \partial_n d_Y$, and $\{ e_i \}_i$ and $\{ f_j \}_j$ are homogeneous bases of~$X$ and~$Y$, respectively. 

The diagram~\eqref{annulus} can be computed in two different ways; equating the respective results amounts to the Cardy condition. One possibility to compute~\eqref{annulus} is to first collapse the $X$-loop to obtain a ``$(k[x],W)$-bubble'', and then collapse its boundary loop~$Y$ in turn to arrive at an expression for $C:k \lra k$. It follows from the general results of Section~\ref{sec:defectaction} or directly from~\eqref{evcoeveasy} that the first step of collapsing~$X$ produces the 2-morphism $(-1)^{n\choose 2} \str(\varphi \Lambda_X)\in\End(\Delta_W)$. Similarly, the second step of collapsing~$Y$ produces a residue
$$
C 
= 
%%%%%%%%%%%%%%%%%%%%%%
\begin{tikzpicture}[very thick,scale=0.6,color=blue!50!black, baseline,>=stealth]
\nicecolourscheme (0,0) circle (2);
\nicereallynocolourscheme (0,0) circle (1);
\fill (1.5,0) circle (0pt) node[white] {{\small$W$}};
\draw (0,0) circle (2);
\draw[->, very thick] (-0.100,2) -- (-0.101,2) node[above] {}; 
\draw[->, very thick] (0.100,-2) -- (0.101,-2) node[below] {}; 
\fill (45:2) circle (2.5pt) node[right] {{\small$\psi$}};
%
\draw (0,0) circle (1);
\draw[->, very thick] (0.100,1) -- (0.101,1) node[above] {}; 
\draw[->, very thick] (-0.100,-1) -- (-0.101,-1) node[below] {}; 
\fill (135:1) circle (2.5pt) node[left] {{\small$\varphi$}};
\end{tikzpicture} 
%%%%%%%%%%%%%%%%%%%%%%
= 
(-1)^{n\choose 2} \,  
%%%%%%%%%%%%%%%%%%%%%%
\begin{tikzpicture}[very thick,scale=0.6,color=blue!50!black, baseline,>=stealth]
\nicecolourscheme (0,0) circle (2);
\fill (1.5,0) circle (0pt) node[white] {{\small$W$}};
\draw (0,0) circle (2);
\draw[->, very thick] (-0.100,2) -- (-0.101,2) node[above] {}; 
\draw[->, very thick] (0.100,-2) -- (0.101,-2) node[below] {}; 
\fill (45:2) circle (2.5pt) node[right] {{\small$\psi$}};
%
\fill (45:0) circle (2.5pt) node[below] {{\small$\str(\varphi\Lambda_X)$}};
\end{tikzpicture} 
%%%%%%%%%%%%%%%%%%%%%%
= 
\Res_{k[x]/k} \left[ \frac{\str\! \big( \str( \varphi \Lambda_X) \psi \Lambda_Y \big) \underline{\operatorname{d}\! x}}{\partial_1 W, \ldots, \partial_n W} \right] .
$$
Using the expressions $\beta^X(\varphi) = (-1)^{n\choose 2} \str(\varphi \Lambda_X)$ and $\beta^Y(\psi) = (-1)^{n\choose 2} \str(\psi \Lambda_Y)$ for the boundary-bulk maps we thus find from the above that
\be\label{C1}
C = 
\Res_{k[x]/k} \left[ \frac{\beta^X(\varphi) \, \beta^Y (\psi) \, \underline{\operatorname{d}\! x}}{\partial_1 W, \ldots, \partial_n W} \right] .
\ee

On the other hand, we may compute~$C$ by first fusing the two loops in~\eqref{annulus} together to obtain a single loop labelled by ${}^\dual X\otimes Y = \Hom(X,Y)[n] \in \End_{\LG}( (k,0) )$ (TODO: check signs from the shift). Then we use
$$
\eval_{{}^\dual X \otimes Y} ((e_i^* \otimes f_j)^* \otimes (e_a^* \otimes f_b)) = \delta_{ia} \, \delta_{jb} 
\, , \qquad 
\widetilde\coev_{{}^\dual X \otimes Y} (1) = \sum_{i,j} (-1)^{|e_i| + |f_j|} (e_i^* \otimes f_j)^* \otimes (e_i^* \otimes f_j)
$$
to arrive at
\be\label{C2}
C 
= 
%%%%%%%%%%%%%%%%%%%%%%
\begin{tikzpicture}[very thick,scale=0.6,color=blue!50!black, baseline,>=stealth]
\nicecolourscheme (0,0) circle (2);
\nicereallynocolourscheme (0,0) circle (1);
\fill (1.5,0) circle (0pt) node[white] {{\small$W$}};
\draw (0,0) circle (2);
\draw[->, very thick] (-0.100,2) -- (-0.101,2) node[above] {}; 
\draw[->, very thick] (0.100,-2) -- (0.101,-2) node[below] {}; 
\fill (45:2) circle (2.5pt) node[right] {{\small$\psi$}};
%
\draw (0,0) circle (1);
\draw[->, very thick] (0.100,1) -- (0.101,1) node[above] {}; 
\draw[->, very thick] (-0.100,-1) -- (-0.101,-1) node[below] {}; 
\fill (135:1) circle (2.5pt) node[left] {{\small$\varphi$}};
\end{tikzpicture} 
%%%%%%%%%%%%%%%%%%%%%%
= 
(-1)^n 
%%%%%%%%%%%%%%%%%%%%%%
\begin{tikzpicture}[very thick,scale=0.6,color=blue!50!black, baseline,>=stealth]
\draw[ultra thick] (0,0) circle (2);
\draw[->, ultra thick] (-0.100,2) -- (-0.101,2) node[above] {}; 
\draw[->, ultra thick] (0.100,-2) -- (0.101,-2) node[below] {}; 
\fill (45:2) circle (2.5pt) node[right] {{\small$1\otimes\psi$}};
\fill (135:2) circle (2.5pt) node[left] {{\small$1\otimes\varphi$}};
\end{tikzpicture} 
%%%%%%%%%%%%%%%%%%%%%%
= 
(-1)^n \str ( {}_\psi m_\varphi ) 
\ee
where $ {}_\psi m_\varphi \in \End_k(\Hom_{\hmf(k[x],W)}(X,Y))$ is the operator that sends~$\alpha$ to $ \psi\circ\alpha\circ\varphi$. 
The second equality in~\eqref{C2} follows from Proposition~\ref{prop:defectactionproperties}(iii) (whose proof in turn relies on Proposition~\ref{prop:monoidalproperty}). Comparing~\eqref{C1} and~\eqref{C2} we thus observe: 

\begin{theorem}\label{thm:CardyCondition}
The Cardy condition holds in $\LG$: for 1-morphisms $X,Y \in \hmf( k[x_1,\ldots,x_n], W)$ and 2-morphisms $\varphi: X\lra X$, $\psi: Y\lra Y$ we have (TODO: sign)
$$
\str ( {}_\psi m_\varphi ) 
= 
\Res_{k[x]/k} \left[ \frac{\beta^X(\varphi) \, \beta^Y (\psi) \, \underline{\operatorname{d}\! x}}{\partial_1 W. \ldots, \partial_n W} \right] 
$$
with ${}_\psi m_\varphi(\alpha) = \psi\circ\alpha\circ\varphi$ for $\alpha\in \Hom_{\hmf(k[x],W)}(X,Y)$. 
\end{theorem}

\begin{remark}
The Cardy condition also holds if one or both of the maps $\varphi, \psi$ are odd (which we present as maps such as $X\lra X[1]$ in $\LG$). The proof of this fact proceeds similarly to the above. 
\end{remark}


\section{Shadows}\label{sec:shadows}

The adjunctions in $\LG$ afford us the construction of a bicategorical trace in terms of shadow functors~\cite{p0807.1471}. We will also see that shadows allow to recover and generalise the boundary-bulk and bulk-boundary maps of the two-dimensional topological field theories based on Landau-Ginzburg models. 

\begin{definition}
A bicategory~$\mathcal B$ \textsl{has shadows} if there is a category~$\mathcal C$ together with functors
$$
\langle\!\langle - \rangle\!\rangle : \mathcal B(A,A) \lra \mathcal C 
$$
for every object $A\in \mathcal B$ such that there are natural isomorphisms $\theta : \langle\!\langle X \otimes Y \rangle\!\rangle \lra \langle\!\langle Y \otimes X \rangle\!\rangle$ for every pair of composable 1-morphisms $X,Y$, and the diagrams
$$
\xymatrix{%
\langle\!\langle (X \otimes Y) \otimes Z \rangle\!\rangle \ar[r]^-{\theta} \ar[d]_-{\langle\!\langle \alpha \rangle\!\rangle} & 
\langle\!\langle Z \otimes (X \otimes Y) \rangle\!\rangle \ar[r]^-{\langle\!\langle \alpha^{-1} \rangle\!\rangle} & 
\langle\!\langle (Z \otimes X) \otimes Y \rangle\!\rangle \\
\langle\!\langle X \otimes (Y \otimes Z) \rangle\!\rangle \ar[r]^-{\theta} & 
\langle\!\langle (Y \otimes Z) \otimes X \rangle\!\rangle \ar[r]^-{\langle\!\langle \alpha \rangle\!\rangle} & 
\langle\!\langle Y \otimes (Z \otimes X) \rangle\!\rangle \ar[u]_-{\theta}
}%
$$
and
$$
\xymatrix{%
\langle\!\langle X \otimes 1_A \rangle\!\rangle \ar[r]^-{\theta} \ar[dr]_-{\langle\!\langle \rho \rangle\!\rangle} & 
\langle\!\langle 1_A \otimes X \rangle\!\rangle \ar[r]^-{\theta} \ar[d]^-{\langle\!\langle \lambda \rangle\!\rangle} & 
\langle\!\langle X \otimes 1_A \rangle\!\rangle \ar[dl]^-{\langle\!\langle \rho \rangle\!\rangle}  \\
 & 
\langle\!\langle X \rangle\!\rangle & 
}%
$$
commute whenever they make sense. 
\end{definition}

\begin{proposition}
The bicategory $\LG$ has shadows given by 
\begin{align*}
\langle\!\langle - \rangle\!\rangle : \LG \big( (R,W), (R,W) \big) & \lra \hmf(k,0) \, , \\
Z & \lmt Z\otimes_{\Re} R
\end{align*}
with the isomorphism $\theta: \langle\!\langle X \otimes Y \rangle\!\rangle \lra \langle\!\langle Y\otimes X  \rangle\!\rangle$ induced by the graded swap map $X \otimes Y \lra Y \otimes X$. 
\end{proposition}

The proof is a straightforward check of the axioms, made especially easy by the fact that $\langle\!\langle - \rangle\!\rangle$ is simply defined as tensoring with the actual diagonal~$R$. Note however that this is homotopy equivalent to tensoring with the unit matrix factorisation $\Delta_W$. 

Since $\LG$ is a bicategory with adjoints and shadows it is automatically equipped with a 2-categorical trace operation as introduced and discussed at length in~\cite{p0807.1471, ps0910.1306}. We only quote the definition: 

\begin{definition}
Let~$\mathcal B$ be a bicategory with shadows and 1-morphism~$Y$ with left adjoint~${}^\dual Y$. Then the \textsl{trace} of a 2-morphism $\psi: X\otimes Y \lra Y \otimes Z$ is the map 
$$
\xymatrix{%
\langle\!\langle X \rangle\!\rangle \ar[rr]^-{\langle\!\langle1\otimes \coev_Y\rangle\!\rangle} && 
\langle\!\langle X \otimes Y \otimes {}^\dual Y \rangle\!\rangle \ar[r]^-{\langle\!\langle\psi \otimes 1\rangle\!\rangle} & 
\langle\!\langle Y \otimes Z \otimes {}^\dual Y \rangle\!\rangle \ar[r]^-{\theta} &
\langle\!\langle {}^\dual Y \otimes Y \otimes Z \rangle\!\rangle \ar[rr]^-{\langle\!\langle\eval_Y \otimes 1\rangle\!\rangle} && 
\langle\!\langle Z \rangle\!\rangle 
}%
\, . 
$$
\end{definition}

Next we wish to point out a connection between shadows and the structure of two-dimensional open/closed topological field theory (TFT) for Landau-Ginzburg models. For this we merely observe that the bulk-boundary and boundary-bulk maps~\eqref{bubobobu} can also be recovered from the adjunction and shadow structure of $\LG$ as follows. On the one hand we have 
$$
\langle\!\langle \Delta_W \rangle\!\rangle = R/(\partial W)[n]
$$
since $\langle\!\langle \Delta_W \rangle\!\rangle = \Delta_W \otimes_{\Re} R = (\bigwedge (\bigoplus_{i=1}^n R \theta_i), \sum_{i=1}^n \partial_{x_i}W \theta_i)$ which is homotopy equivalent (and therefore equal in $\hmf(k,0)$) to $R/(\partial W)[n]$. On the other hand $\langle\!\langle X^\dual \otimes X \rangle\!\rangle = X^\vee \otimes_k X \otimes_{\Re} R = \End_R(X)$ for $X\in \hmf(R,W)$. Thus from the explicit expressions in Section~\ref{sec:derivcoeval} we find that $\beta_X = \langle\!\langle \widetilde\coev_X \rangle\!\rangle$ and $\beta^X = \langle\!\langle \eval_X \rangle\!\rangle$. 

Motivated by the above this construction can be extended to any 1-morphism in $\LG$: for $X \in \hmf(k[x_1,\ldots,x_n,z_1,\ldots,z_m], V-W)$ we define the \textsl{generalised bulk-boundary} and \textsl{boundary-bulk maps} to be
$$
\beta_X= \langle\!\langle \widetilde\coev_X \rangle\!\rangle : \langle\!\langle \Delta_{W} \rangle\!\rangle \lra \langle\!\langle X^\dual \otimes X \rangle\!\rangle
\, , \qquad 
\beta^X= \langle\!\langle \eval_X \rangle\!\rangle : \langle\!\langle {}^\dual X \otimes X \rangle\!\rangle \lra \langle\!\langle \Delta_{W} \rangle\!\rangle \, , 
$$
respectively. Substituting the expressions for the adjiunction maps in Section~\ref{sec:derivcoeval} we find that the form of $\beta_X$ stays the same while for $m\neq 0$ the generalised boundary-bulk map $\beta^X$  involves a residue and additional derivatives $\partial_{z_i} d_X$: 
\begin{align*}
\beta_X : k[x]/(\partial V)[n] & \lra \End_{k[x,z]}(X)[m] \, , \\
\phi & \lmt \phi \cdot 1_{X}[m] \, , \\
\beta^X : \End_{k[x,z]}(X)[m] & \lra k[x]/(\partial V)[n] \, , \\
\psi & \lmt (-1)^{{m\choose 2} + {n\choose 2}} \Res_{k[x,z]/k[x]} \left[ \frac{\str\left( \psi\,  \partial_{x_1} d_X \ldots \partial_{x_n} d_X \partial_{z_1} d_X \ldots \partial_{z_m} d_X \right) \underline{\operatorname{d}\!x}}{\partial_{x_1} W, \ldots, \partial_{x_n} W} \right] \, . 
\end{align*}


\section{Duality via adjoint operators}\label{section:dualityadjointop}

Let $k$ be a noetherian $\nQ$-algebra and $R = k[x_1,\ldots,x_n]$. If $k$ is a field then we know from \cite{??,??} that for any polynomial $W$ with an isolated singularity, the Kapustin-Li formula gives a nondegenerate pairing on the mapping spaces of the triangulated category of matrix factorisations of $W$ over $R$. In this section we prove the analogue of this result for an arbitrary base $k$. In particular, the mapping spaces of the triangulated category will not be flat over $k$, so we work instead with dg-categories.

Given $W \in R$ and two finite-rank matrix factorisations $X,Y$ of $W$ let $\Hom(X,Y) = \Hom_R(X,Y)$ denote the $\mathbb{Z}_2$-graded Hom-complex. We define a natural $k$-bilinear pairing
\[
\langle -, - \rangle: \Hom(X,Y) \otimes_k \Hom(Y,X) \lto k[n]
\]
by the following formula, where $\lambda_i = \partial_{x_i}d_Y, \omega = \ud x_1 \ldots \ud x_n$ and $f_i = \partial_{x_i} W$: 
\begin{equation}\label{eq:generalklpairing}
\langle \varphi, \psi \rangle = \frac{1}{n!} \sum_{\sigma \in S_n} (-1)^{|\sigma|} \Ress{R/k}\Bigg[ \frac{ \str( \varphi \circ \psi \circ \lambda_{\sigma(1)} \circ \ldots \circ \lambda_{\sigma(n)} )\, \omega }{ f_1, \ldots, f_n } \Bigg]\,.
\end{equation}
In order to be able to define this residue we require that the partial derivatives $f_i$ form a regular sequence $f = \{f_1,\ldots,f_n\}$ in $R$ and that $\bar{R} = R/(f_1,\ldots,f_n)R$ is a finitely generated free $k$-module.

Such a pairing is \textsl{homotopically nondegenerate} if the adjoint morphism of $\mathbb{Z}_2$-graded complexes
\begin{align*}
\Hom(X,Y) & \lto \Hom_k( \Hom(Y,X), k )[n]\,,\\
\varphi & \lmt \langle \varphi, - \rangle
\end{align*}
is a homotopy equivalence over $k$. The main theorem here is that in complete generality 

\begin{theorem}\label{theorem:generalkl} The pairing $\langle -, - \rangle$ is homotopically nondegenerate.
\end{theorem}

%This nondegeneracy is used in Section~\ref{sec:Zorro} to prove that specified evaluation and coevaluation morphisms give rise to adjoint $1$-morphisms in the bicategory of Landau-Ginzburg models over $k$.

There is also version when $X,Y$ are graded matrix factorisations; see Section \ref{section:gradedduality}. The proof of the theorem takes up the rest of this section. We argue that $\Hom(X,Y)[n]$ and $\Hom_k(\Hom(Y,X),k)$ are homotopy equivalent by showing that they respectively split idempotents on
\[
\Hom(\bar{X},\bar{Y}) = \Hom(X,Y) \otimes_R \bar{R}
\]
and
\[
\Hom_k( \Hom(\bar{Y}, \bar{X}), k) = \Hom_k( \Hom(Y,X) \otimes_R \bar{R}, k )
\]
in the homotopy category of $\mathbb{Z}_2$-graded $k$-complexes. There is an isomorphism
\begin{equation}\label{eq:frobnondeg}
\Hom(\bar{X}, \bar{Y}) \cong \Hom_k( \Hom(\bar{Y}, \bar{X}), k)
\end{equation}
of complexes which identifies these idempotents, and from this the theorem will follow. The existence of the isomorphism (\ref{eq:frobnondeg}) is an easy consequence of the fact that $\bar{R}$ is Frobenius over $k$. The hard work lies in showing that this isomorphism identifies the two idempotents, and for this we need to carefully study adjointness between operators on dg-categories. Thus the first part of the proof, in Section \ref{section:adjointopdg}, consists in formalising this kind of adjointness.

\subsection{Adjoint operators on dg-categories}\label{section:adjointopdg}
% TODO: Point out that these Atiyah classes are from connections over k[f], not k like in the associative case.

In this section $\otimes$ denotes $\otimes_k$. Let $\cat{C}$ be a $\mathbb{Z}_2$-graded dg-category over $k$ equipped with the data of a $k$-linear morphism of complexes 
\[
c_{XY}: \cat{C}(X,Y) \otimes \cat{C}(Y,X) \lto k
\]
for each pair of objects $X,Y$ in $\cat{C}$. 
When it is convenient we write $\langle \alpha, \beta \rangle$ for $c_{XY}(\alpha \otimes \beta)$. Throughout the differentials on the $\cat{C}(X,Y)$ and their tensor products are denoted $D$.

\begin{definition}\label{defn:nondegpair} We say that the family of pairings $\{ c_{XY} \}_{X,Y \in \cat{C}}$ is
\begin{itemize}
\item[(i)] \textsl{cyclic} if for all $X,Y$ the diagram
\[
\xymatrix{
\cat{C}(X,Y) \otimes \cat{C}(Y,X) \ar[dr]_{c_{XY}}\ar[rr]^\tau & & \cat{C}(Y,X) \otimes \cat{C}(X,Y) \ar[dl]^{c_{YX}}\\
& k
}
\]
commutes, where $\tau$ is the graded twist $\tau( \varphi \otimes \psi) = (-1)^{|\varphi||\psi|} \psi \circ \varphi$; 
\item[(ii)] \textsl{nondegenerate} if for all $X,Y$ the morphism
\begin{align*}
\zeta_{XY}: \cat{C}(X,Y) & \lto \Hom_k( \cat{C}(Y,X), k)\,,\\
\varphi & \lmt c_{XY}( \varphi \otimes - )
\end{align*}
is an isomorphism of complexes.
\end{itemize}
\end{definition}

From now on we assume $\cat{C}$ is equipped with a cyclic nondegenerate pairing. In this section an \textsl{operator} is a closed homogeneous $k$-linear operator on some mapping complex $\cat{C}(X,Y)$ in $\cat{C}$, that is, a closed homogeneous element of the complex $\Hom_k( \cat{C}(X,Y), \cat{C}(X,Y) )$. We are interested in linear operators on the $\cat{C}(X,Y)$ and adjunctions between them, with respect to the pairing.

Recall that if $\Psi$ is homogeneous then $1 \otimes \Psi$ acts on tensors with Koszul signs.

\begin{definition}\label{defn:adjointop} An operator $\Phi$ on $\cat{C}(X,Y)$ is \textsl{adjoint} to an operator $\Psi$ on $\cat{C}(Y,X)$ if both~$\Phi$ and~$\Psi$ are homogeneous of the same degree and the diagram
\[
\xymatrix@C+1pc@R+0.5pc{
\cat{C}(X,Y) \otimes \cat{C}(Y,X) \ar[d]_{\Phi \otimes 1} \ar[r]^{1 \otimes \Psi} & \cat{C}(X,Y) \otimes \cat{C}(Y,X) \ar[d]^{c_{XY}}\\
\cat{C}(X,Y) \otimes \cat{C}(Y,X) \ar[r]_-{c_{XY}} & k
}
\]
commutes up to homotopy. Equivalently, there is a $k$-linear degree $|\Phi|+1$ morphism
\[
\mu: \cat{C}(X,Y) \otimes \cat{C}(Y,X) \lto k
\]
with the property that $[D, \mu] = c_{XY} \circ (1 \otimes \Psi) - c_{XY} \circ ( \Phi \otimes 1)$. Evaluated on homogeneous morphisms $\alpha, \beta$ this identity reads
\begin{equation}\label{eq:adjointopeq}
(-1)^{|\Phi|} \mu D(\alpha \otimes \beta) = (-1)^{|\alpha||\Psi|}\langle \alpha, \Psi(\beta) \rangle - \langle \Phi(\alpha) , \beta \rangle \, .
\end{equation}
\end{definition}

We will show in a moment that this type of adjointness is symmetric in $\Phi, \Psi$ so there is no need to distinguish between left and right adjoints. If $\Psi$ is a homogeneous operator on $\cat{C}(Y,X)$ then $\Psi^*$ is the operator on $\Hom_k(\cat{C}(Y,X),k)$ defined by $\Psi^*(f) = (-1)^{|f||\Psi|} f \circ \Psi$. 

\begin{lemma} An operator $\Phi$ is adjoint to $\Psi$ if and only if the diagram
\begin{equation}\label{eq:lemadjointdia}
\xymatrix@C+2pc{
\cat{C}(X,Y) \ar[d]_\Phi \ar[r]^-{\zeta} & \Hom_k( \cat{C}(Y,X), k ) \ar[d]^{\Psi^*}\\
\cat{C}(X,Y) \ar[r]_-{\zeta} & \Hom_k( \cat{C}(Y,X), k)
}
\end{equation}
commutes up to homotopy.
\end{lemma}

\begin{lemma} An operator $\Phi$ is adjoint to $\Psi$ if and only if $\Psi$ is adjoint to $\Phi$.
\end{lemma}
\begin{proof}
By naturality of the graded twist $\tau$ and the cyclicity axiom for $c_{XY}$.
\end{proof}

\begin{lemma} Any operator $\Phi$ admits an adjoint, which is unique up to homotopy. 
\end{lemma}
\begin{proof}
This follows from commutativity of (\ref{eq:lemadjointdia}).
\end{proof}

\begin{definition} Given an operator $\Phi$ we denote by $\Phi^{\dagger}$ the adjoint of $\Phi$.
\end{definition}

Since the adjoint is only well-defined up to homotopy it is implicit that in any identities involving the dagger notation we are working with homogeneous operators up to homotopy (to be clear, odd operators are homotopic if they differ by $[D,\alpha]$ for an even operator $\alpha$). 

The following basic properties of adjoints are easily checked.

\begin{lemma} 
\begin{enumerate}
\item Let $\Phi_1, \Phi_2$ be operators on $\cat{C}(X,Y)$ and $\cat{C}(Y,Z)$ respectively. Then
\[
(\Phi_2 \circ \Phi_1)^{\dagger} = (-1)^{|\Phi_1||\Phi_2|} \Phi_1^{\dagger} \circ \Phi_2^{\dagger}\,.
\]
\item If $\Phi_1, \Phi_2$ are operators on $\cat{C}(X,Y)$ of the same degree, then $(\Phi_1 + \Phi_2)^{\dagger} = \Phi_1^{\dagger} + \Phi_2^{\dagger}$.
\end{enumerate}
\end{lemma}

\subsection{Skew-symmetry of the Atiyah class}

Now we apply this general theory to the dg-category of matrix factorisations. In order to introduce the Atiyah classes, which are the most interesting example of adjoint operators on this dg-category, we need $R$ to be equipped with a connection. For $R = k[x_1,\ldots,x_n]$ it may be necessary to pass to the $(f_1,\ldots,f_n)$-adic completion to ensure the existence of a connection, so for the next two sections we work instead with an arbitrary noetherian $k$-algebra $R$, $W \in R$, satisfying:
\begin{itemize}
\item[(H1)] $f = \{ f_1,\ldots,f_n \}$ is any regular sequence in $R$ with the property that multiplication by $f_i$ is a null-homotopic endomorphism of $X,Y \in \hmf(R,W)$ for $1 \le i \le n$.
\item[(H2)] $\bar{R} = R/(f_1,\ldots,f_n)R$ is a finitely generated free $k$-module.
\item[(H3)] $R$ admits a flat $k$-linear connection
\[
\nabla: R \lto R \otimes_{k[f]} \Omega^1_{k[f]/k}
\]
as a $k[f]$-module, which is standard in the sense of \cite[Definition 8.6]{dm1102.2957}.
\item[(H4)] There is an $n$-form $\omega \in \Omega^n_{R/k}$ such that the $R$-linear map
\begin{align*}
\bar{R} & \lto \Hom_k( \bar{R}, k)\, , \\
r & \lmt \Ress{R/k}\Bigg[ \frac{r \cdot (-) \cdot \omega}{ f_1, \ldots, f_n} \Bigg]
\end{align*}
is an isomorphism.
\end{itemize}
In this situation the pairing (\ref{eq:generalklpairing}) still makes sense and we prove that it is homotopically nondegenerate in Proposition \ref{prop:almostkltheorem}. The theorem follows by passing from $k[x_1,\ldots,x_n]$ to the completion, which satisfies (H1)-(H4) above.

We let $\cat{C}$ denote the dg-category whose objects are finite-rank matrix factorisations of $W$ and whose mapping complexes are given by the quotients
\[
\cat{C}(X,Y) = \Hom(\bar{X}, \bar{Y}) = \Hom(X,Y) \otimes_R \bar{R}\,.
\]
While the ordinary dg-category of matrix factorisations only admits a pairing which is homotopically nondegenerate, this (linear) quotient $\cat{C}$ admits a cyclic nondegenerate pairing in the stronger sense explained above. To define it, let
\[
\langle - \rangle: R \lto k
\]
be the $k$-linear map
\[
\langle r \rangle = \Ress{R/k}\Bigg[ \frac{r \cdot \omega}{ f_1, \ldots, f_n} \Bigg]\,.
\]
This map annihilates the ideal $(f_1,\ldots,f_n)R$ and therefore factors via a $k$-linear map $\bar{R} \lto k$.

% see dviat
\begin{proposition}\label{prop:barnondeg} The pairing on $\cat{C}$ defined by
\begin{align*}
\langle -, - \rangle: \Hom(\bar{X}, \bar{Y}) \otimes \Hom(\bar{Y}, \bar{X}) \lto k\,, \qquad
\langle \varphi, \psi \rangle = \langle \str( \varphi \circ \psi ) \rangle
\end{align*}
is cyclic and nondegenerate in the sense of Definition \ref{defn:nondegpair}.
\end{proposition}
\begin{proof}
The pairing factors as
\begin{equation}\label{eq:barnondeg}
\xymatrix@C+1pc{
\Hom(\bar{X}, \bar{Y}) \otimes \Hom(\bar{Y}, \bar{X}) \ar[r]^-{-\circ-} & \Hom(\bar{Y}, \bar{Y}) \ar[r]^-{\str} & \bar{R} \ar[r]^-{\langle - \rangle} & k
}
\end{equation}
so it is clear that it is a closed $k$-linear map, and moreover cyclicity follows from the cyclicity of the supertrace. Nondegeneracy follows from hypothesis (H4) and the following calculation, in which the first step is adjoint to the composite of the first two maps in (\ref{eq:barnondeg}) with $\otimes$ replaced by $\otimes_{\bar{R}}$
\begin{align*}
\Hom_R(\bar{X}, \bar{Y}) &\cong \Hom_R( \Hom(\bar{Y}, \bar{X}), \bar{R} )\,\\
&\cong \Hom_R( \Hom(\bar{Y}, \bar{X}), \Hom_k( \bar{R}, k) )\\
&\cong \Hom_k( \Hom(\bar{Y}, \bar{X}), k )\,.
\end{align*}
\end{proof}

We have in mind two special classes of operators on the dg-category $\cat{C}$. The first arises because the $f_i$ act as zero on the cohomology of $\cat{C}$ but via nonzero maps on the dg-level. In what follows let $X,Y$ denote finite-rank matrix factorisations of $W$.

\begin{definition} Given a null-homotopy $\lambda_i$ for the action of $f_i$, that is, a map $\lambda_i \in \Hom(Y,Y)$ of degree one with $[d_Y, \lambda_i] = f_i \cdot 1_Y$, we define the odd operator $\lambda_i^\bullet$ on $\Hom(\bar{X},\bar{Y})$ by
\[
\lambda_i^\bullet(\varphi) = \lambda_i \circ \varphi\,
\]
and the odd operator ${\lambda_i}_\bullet$ on $\Hom(\bar{Y}, \bar{X})$ by
\[
{\lambda_i}_\bullet(\varphi) = (-1)^{|\varphi|} \varphi \circ \lambda_i \,.
\]
\end{definition}

Observe that composition with $\lambda_i$ is not a closed map on $\Hom(X,Y)$ but is closed as an operator on $\Hom(\bar{X}, \bar{Y})$. These operators give the simplest example of an adjoint pair: 

\begin{lemma} The operator $\lambda_i^\bullet$ is adjoint to ${\lambda_i}_\bullet$.
\end{lemma}
\begin{proof}
The identity (\ref{eq:adjointopeq}) holds with $\mu = 0$, since
\begin{align*}
\langle {\lambda_i}_\bullet(\alpha), \beta \rangle &= \langle \str( {\lambda_i}_\bullet(\alpha) \circ \beta ) \rangle\\
&= (-1)^{|\varphi|} \langle \str( \alpha \circ \lambda_i \circ \beta ) \rangle\\
&= (-1)^{|\varphi|} \langle \alpha, \lambda_i^\bullet(\beta) \rangle\,.
\end{align*}
\end{proof}

The second class of operators are the components of the Atiyah class. Our reference for Atiyah classes is \cite{buchweitz03}, see also \cite[\S 9]{dm1102.2957}. Recall that by hypothesis (H2) the ring $R$ admits a flat $k$-linear connection $\nabla$ as a $k[f]$-module. The components of $\nabla$ define $k$-linear operators $\partial_{f_i} = (\ud f_i)^* \circ \nabla$ on $R$ with the property that $[\partial_{f_i}, f_j] = \delta_{ij}$.

Any free $R$-module admits a $k$-linear connection over $k[f]$. For convenience choose homogeneous $R$-bases $\{ e_i \}_{i}$ for~$X$ and $\{ e_j \}_{j}$ for~$Y$ respectively. Then the maps $e_{ji} = e_j \circ e_i^*$ form an $R$-basis for $\Hom(X,Y)$ and the induced $k$-linear connection over $k[f]$ is defined by
\begin{align*}
\nabla = \nabla_{XY}: \Hom(X,Y) & \lto \Hom(X,Y) \otimes_{k[f]} \Omega^1_{k[f]/k}\,,\\
r e_{ji} & \lmt e_{ji} \otimes \nabla(r)\,.
\end{align*}
In contrast to Section \ref{section:connections} the connections here do not have sign contributions from the $\mathbb{Z}_2$-grading, so $\nabla$ satisfies the usual Leibniz rule rather than the graded Leibniz rule, and $\nabla$ is given degree zero for the purpose of computing graded commutators.

The connection has components $\partial_{f_i}$ defined by $\partial_{f_i}( r e_{ji} ) = \partial_{f_i}(r) \cdot e_{ji}$ which are $k$-linear operators on $\Hom(X,Y)$. The \textsl{Atiyah class} of $\Hom(X,Y)$ is the commutator
\[
\At = \At_{XY} = [D, \nabla] = D \circ \nabla - \nabla \circ D: \Hom(X,Y) \lto \Hom(X,Y) \otimes_{k[f]} \Omega^1_{k[f]/k}
\]
where $D$ is the differential on $\Hom(X,Y)$. The Atiyah class is a $k[f]$-linear closed map of degree one whose homotopy class is independent of the choice of connection, and therefore independent of the basis chosen for $X,Y$. It is important to distinguish this kind of Atiyah class from the associative Atiyah classes introduced in Section \ref{section:atiyahclasses} and used in the rest of the paper; the distinction is that here we use ordinary commutative differential forms, whereas associative Atiyah classes are defined using noncommutative forms.

In terms of the components: 

\begin{definition} For $1 \le i \le n$ the components $\At_i = [D, \partial_{f_i}]$ of the Atiyah class define $k$-linear closed operators of degree one on $\Hom(\bar{X}, \bar{Y})$ which are canonically defined up to homotopy.
\end{definition}

Conceptually, duality in the dg-category of matrix factorisations arises from the fact that these operators are skew-symmetric. To prove this we need the following Leibniz rule for Atiyah classes.

\begin{lemma}\label{lemma:leibnizatiyah} For homogeneous $\alpha \in \Hom(Y,Z)$ and $\beta \in \Hom(X,Y)$
\begin{equation}\label{eq:leibnizcompos}
\At_i( \alpha \circ \beta ) = \At_i(\alpha) \circ \beta + (-1)^{|\alpha|} \alpha \circ \At_i(\beta) + [g,D]( \alpha \otimes \beta)
\end{equation}
where $g( \alpha \otimes \beta ) = \partial_i(\alpha) \circ \beta + \alpha \circ \partial_i(\beta) - \partial_i( \alpha \circ \beta )$.
\end{lemma}
\begin{proof}
If $\nabla_{YZ}, \nabla_{XY}$ respectively denote the connections on $\Hom(Y,Z)$ and $\Hom(X,Y)$ then
\[
\nabla_{YZ,XY}(\alpha \otimes \beta) = \nabla_{YZ}(\alpha) \otimes \beta + \alpha \otimes \nabla_{XY}(\beta)
\]
defines a connection on $\Hom(Y,Z) \otimes_{k[f]} \Hom(X,Y)$. With $\At_{YZ,XY} = [D, \nabla_{YZ,XY}]$ it follows from naturality of the Atiyah class that the diagram
\[
\xymatrix{
\Hom(Y,Z) \otimes_{k[f]} \Hom(X,Y) \ar[d]_{\At_{YZ,XY}}\ar[r]^-{\kappa} & \Hom(X,Z) \ar[d]^{\At_{XZ}}\\
\Hom(Y,Z) \otimes_{k[f]} \Hom(X,Y) \otimes_{k[f]} \Omega^1_{k[f]/k} \ar[r]_-{\kappa} & \Hom(X,Z) \otimes_{k[f]} \Omega^1_{k[f]/k}
}
\]
commutes up to homotopy, where $\kappa$ is the composition map. Specifically, if $g = [\kappa, \nabla] = \kappa \circ \nabla_{YZ,XY} - \nabla_{XZ} \circ \kappa$ then a simple calculation using the graded Jacobi identity shows that
\[
\At_{XZ} \circ \kappa - \kappa \circ \At_{YZ,XY} = [g,D]\,.
\]
Applying both sides to $\alpha \otimes \beta$ and projecting to the $\ud f_i$ component yields (\ref{eq:leibnizcompos}).
\end{proof}

\begin{lemma}\label{lemma:stratiyahzero} For any $\alpha \in \Hom(X,X)$ we have $\str( \At_i( \alpha ) ) = 0$ in $R$.
\end{lemma}
\begin{proof}
Consider the diagram
\[
\xymatrix@C+1pc{
\Hom(X,X) \ar[d]_{\At_{XX}} \ar[r]^-{\str} & R \ar[d]^{\At = 0}\\
\Hom(X,X) \otimes_{k[f]} \Omega^1_{k[f]/k} \ar[r]_-{\str} & R \otimes_{k[f]} \Omega^1_{k[f]/k}\,.
}
\]
By the graded Jacobi identity $\str \circ \At_{XX} = [\str, \At] = [g,D]$ where $g: \Hom(X,X) \lto R \otimes_{k[f]} \Omega^1_{k[f]/k}$ is $g = [\str, \nabla]$. But from the way we have defined our connections, it is clear that $\str$ and $\nabla$ commute, so $g = 0$ and $\str \circ \At_{XX} = 0$.
\end{proof}

\begin{proposition} The operator $\At_i$ on $\Hom(\bar{X}, \bar{Y})$ is adjoint to $-\At_i$ on $\Hom(\bar{Y}, \bar{X})$.
\end{proposition}
\begin{proof}
If we apply $\langle \str( - ) \rangle$ to both sides of (\ref{eq:leibnizcompos}) and use Lemma \ref{lemma:stratiyahzero} we find that
\begin{align*}
\langle \At_i(\alpha), \beta \rangle = -(-1)^{|\alpha|} \langle \alpha,  \At_i(\beta) \rangle - \langle \str( gD(\alpha \otimes \beta) ) \rangle\,.
\end{align*}
So $\mu( \alpha \otimes \beta ) = \langle \str( g( \alpha \otimes \beta ) ) \rangle$ is a homotopy expressing $\At_i$ as adjoint to $-\At_i$.
\end{proof}

\subsection{Idempotents}\label{section:idempkl}

In the previous section we constructed operators $\lambda_i^\bullet, {\lambda_i}_\bullet$ and $\At_i$ on the dg-category $\cat{C}$. Now we use these operators to define idempotent endomorphisms of the complex
\[
\cat{C}(X,Y) = \Hom(\bar{X}, \bar{Y})
\]
which split in the homotopy category of $\mathbb{Z}_2$-graded $k$-complexes to give $\Hom(X,Y)$. In this way the dg-category of matrix factorisations can be recovered from the quotient $\cat{C}$ and the nondegenerate pairing defined above induces the homotopically nondegenerate pairing $\langle -, - \rangle$.

The main result of \cite{dm1102.2957} is that if $V$ is the free $k$-module on the basis $\theta_1,\ldots,\theta_n$ then there is a homotopy equivalence
\[
\Hom(\bar{X}, \bar{Y}) \cong \Hom(X,Y) \otimes_k \bigwedge V\,.
\]
There are consequently $2^n$ ways to embed $\Hom(X,Y)$ in the homotopy category of $k$-complexes as a direct summand in $\Hom(\bar{X}, \bar{Y})$. The ``top degree'' embedding, corresponding to the form $\theta_1 \ldots \theta_n$, is determined by the following idempotent endomorphism of $\Hom(\bar{X}, \bar{Y})$: 
\[
e = \frac{1}{(n!)^2} (-1)^{\binom{n+1}{2}}\sum_{\sigma,\tau \in S_n} (-1)^{|\sigma\tau|} \cdot \lambda_{\sigma(1)}^\bullet \ldots \lambda_{\sigma(n)}^\bullet \At_{\tau(1)} \ldots \At_{\tau(n)}\,.
\]
The details are recalled in Appendix \ref{section:symidem}, see \eqref{eq:ordinaryidem}. The embedding corresponding to the $0$-form $1$ in $\bigwedge V$ was not discussed in \cite{dm1102.2957} but we give the details in the appendix. The upshot is that by using the relation $[ \lambda_i, \At_j ] = - \delta_{ij}$ one can show (Proposition \ref{prop:otheridempotent}) that this embedding is determined by the idempotent
\[
e' = \frac{1}{(n!)^2} (-1)^{\binom{n+1}{2}}\sum_{\sigma, \tau \in S_n} (-1)^{|\sigma\tau|} \cdot \At_{\tau(1)} \ldots \At_{\tau(n)} {\lambda_{\sigma(1)}}_\bullet \ldots {\lambda_{\sigma(n)}}_\bullet\,.
\]
To be precise, taking $Z = \Hom(X,Y)$ in \eqref{eq:finalidempotents2} we have a diagram of degree zero $k$-linear morphisms of complexes
\[
\xymatrix@C+2pc{
\Hom(\bar{X}, \bar{Y}) \ar@<-0.7ex>[r]_-{\psi} & \Hom(X,Y)[n] \ar@<-0.7ex>[l]_-{\vartheta}
}
\]
with $\psi \circ \vartheta = 1$ and $\vartheta \circ \psi = e$ (equalities meaning equal up to $k$-linear homotopy) and a diagram
\[
\xymatrix@C+2pc{
\Hom(\bar{Y}, \bar{X}) \ar@<-0.7ex>[r]_-{\psi'} & \Hom(Y,X) \ar@<-0.7ex>[l]_-{\kappa}
}
\]
with $\psi' \circ \kappa = 1$ and $\kappa \circ \psi' = e'$. A concrete description of $\psi, \psi'$ is not important for us, but we will need to know that $\kappa$ is simply the quotient map, and that
\[
\vartheta = \frac{1}{n!} (-1)^n \sum_{\sigma \in S_n} (-1)^{|\sigma|} \lambda_{\sigma(1)}^\bullet \ldots \lambda_{\sigma(n)}^{\bullet}\,.
\]

\begin{proposition}\label{prop:eadjointeprime} The idempotent $e$ is adjoint to $e'$. Equivalently, the diagram
\begin{equation}\label{eq:eadjointeprime}
\xymatrix@C+2pc{
\Hom(\bar{X}, \bar{Y}) \ar[d]_-{e} \ar[r]^-{\zeta}_-{\cong} & \Hom_k(\Hom(\bar{Y},\bar{X}),k) \ar[d]^-{(e')^*}\\
\Hom(\bar{X}, \bar{Y}) \ar[r]_-{\zeta}^-{\cong} & \Hom_k( \Hom(\bar{Y}, \bar{X}), k )
}
\end{equation}
commutes up to homotopy, where $\zeta(\varphi) = \langle \varphi, - \rangle = \langle \str( \varphi \circ - ) \rangle$.
\end{proposition}
\begin{proof} 
Observe that by the (anti-)self-adjointness established in the previous section
\begin{align*}
\left( \lambda_{\sigma(1)}^\bullet \ldots \lambda_{\sigma(n)}^\bullet \At_{\tau(1)} \ldots \At_{\tau(n)} \right)^{\dagger} &= (-1)^n \At^\dagger_{\tau(n)} \ldots \At^\dagger_{\tau(1)} (\lambda_{\sigma(n)}^\bullet)^\dagger \ldots (\lambda_{\sigma(1)}^\bullet)^{\dagger}\\
&= \At_{\tau(n)} \ldots \At_{\tau(1)} {\lambda_{\sigma(n)}}_{\bullet} \ldots {\lambda_{\sigma(1)}}_{\bullet}\,,
\end{align*}
from which it is immediate that $e^{\dagger} = e'$.
\end{proof}

% the core of all this is in dviat4 and dviat5
\begin{proposition}\label{prop:almostkltheorem} The pairing (\ref{eq:generalklpairing}) is homotopically nondegenerate.
\end{proposition}
\begin{proof} Consider the diagram
\[
\xymatrix@C+2pc{
\Hom(\bar{X},\bar{Y}) \ar[d]_-{\zeta}^-{\cong} \ar@<-0.7ex>[r]_-{\psi} & \Hom(X,Y)[n] \ar@<-0.7ex>[l]_-{\vartheta} \ar@{.>}[d]^{\chi}\\
\Hom_k(\Hom(\bar{Y}, \bar{X},k) \ar@<-0.7ex>[r]_-{\kappa^*} & \Hom_k( \Hom(Y,X), k) \ar@<-0.7ex>[l]_-{(\psi')^*}
}
\]
where $\chi = \kappa^* \circ \zeta \circ \vartheta$. It is immediate from commutativity of~\eqref{eq:eadjointeprime} up to homotopy that $\chi$ is a homotopy equivalence with inverse $\psi \circ \zeta^{-1} \circ (\psi')^*$. To prove the theorem it only remains to observe that $\chi(\alpha)$ is the functional $\langle \alpha, - \rangle$ (\textbf{todo} distinguish from other pairing). But
\begin{align*}
\chi(\alpha) &= \kappa^* \zeta \vartheta( \alpha )\\
&= \kappa^* \zeta\left( \frac{1}{n!} (-1)^n \sum_{\sigma \in S_n} (-1)^{|\sigma|} \lambda_{\sigma(1)} \circ \ldots \circ \lambda_{\sigma(n)} \circ \alpha \right)\\
&= \frac{1}{n!} (-1)^n \sum_{\sigma \in S_n} (-1)^{|\sigma|} \langle \str( \lambda_{\sigma(1)} \circ \ldots \circ \lambda_{\sigma(n)} \circ \alpha \circ - ) \rangle\\
&= \frac{1}{n!} \sum_{\sigma \in S_n} (-1)^{|\sigma|} \langle \str(  \alpha \circ - \circ \lambda_{\sigma(1)} \circ \ldots \circ \lambda_{\sigma(n)} ) \rangle\\
&= \langle \alpha, - \rangle
\end{align*}
which completes the proof.
\end{proof}

\subsection{Proof of the theorem}

Let us now abandon the general setting, and return to the situation of the main theorem where $R = k[x_1,\ldots,x_n]$, $f_i = \partial_{x_i} W$ and $\omega = \ud x_1 \ldots \ud x_n$. The main nontrivial input that we need is:

\begin{theorem}\label{theorem:nonlocalduality} The $R$-linear map
\begin{align*}
\bar{R} & \lto \Hom_k( \bar{R}, k)\, , \\
r & \lmt \Ress{R/k}\Bigg[ \frac{r \cdot (-) \, \underline{\ud x}}{ f_1, \ldots, f_n} \Bigg]
\end{align*}
is an isomorphism.
\end{theorem}
\begin{proof}
For local rings $R$ this is the statement of local duality. We give a proof which works in the present level of generality in \cite{??}.
\end{proof}

Set $I = (f_1,\ldots,f_n)R$ and let $\widehat{R}$ denote the $I$-adic completion of $R$. The axioms (H1)--(H4) hold for $\widehat{R}$, since by \cite[Appendix B]{dm1102.2957} this algebra admits a flat standard connection as a $k[f]$-module, and (H4) follows from from Theorem \ref{theorem:nonlocalduality}.

Note that in (H1) we only need that the $f_i$ act null-homotopically on the matrix factorisations
\[
\widehat{X} = X \otimes_R \widehat{R} \, , \qquad \widehat{Y} = Y \otimes_R \widehat{R}
\]
extended from $R$

\begin{proof}[Proof of Theorem \ref{theorem:generalkl}] Consider the diagram
\[
\xymatrix@C+2pc{
\Hom_R(X,Y) \ar[d]_{\can} \ar[r] & \Hom_k( \Hom_R(Y,X), k )[n]\\
\Hom_{\widehat{R}}(\widehat{X}, \widehat{Y}) \ar[r] & \Hom_k( \Hom_{\widehat{R}}(\widehat{X}, \widehat{Y}), k)[n] \ar[u]_{\can}
}
\]
where the columns are the canonical maps, and the rows are adjoint to the two versions of the pairing (\ref{eq:generalklpairing}). It is easy to see that this diagram commutes, and as the bottom row is a homotopy equivalence by Proposition \ref{prop:almostkltheorem}, to prove the theorem it is enough to argue that the canonical map
\[
\Hom_R(X,Y) \lto \Hom_{\widehat{R}}(\widehat{X}, \widehat{Y})
\]
is a homotopy equivalence of $\mathbb{Z}_2$-graded complexes over $k$. But this is a consequence of the general fact that the ``pushforward commutes with flat base change'', see \cite[Remark 7.7]{dm1102.2957}.
\end{proof}

\subsection{Graded duality}\label{section:gradedduality}

In this section we collect together the modifications necessary in order to make Theorem \ref{theorem:generalkl} compatible with an additional $\mathbb{Z}$-grading on matrix factorisations. Our conventions for graded modules and matrix factorisations are contained in Section \ref{??}.

Let $k$ be a noetherian $\nQ$-algebra which is graded, and let $R = k[x_1,\ldots,x_n]$ be a graded ring in such a way that the structural map $k \lto R$ has degree zero. Let $W \in R$ be given, homogeneous of degree $|W| = 2c$ and let $X,Y$ be finite-rank graded matrix factorisations of $W$ over $R$. Define
\[
a = \sum_{i=1}^n |x_i| - nc \, . 
\]
Then the graded analogue of the earlier theorem is:

\begin{theorem}\label{theorem:generalklgraded} 
Under the conditions (H1)--(H3) of Theorem \ref{theorem:generalkl} the pairing $\langle -, - \rangle$ is adjoint to a homotopy equivalence of $\mathbb{Z} \times \mathbb{Z}_2$-graded complexes over $k$
\[
\Hom(X,Y) \lto \Hom^{\operatorname{gr}}_k( \Hom(Y,X), k )[n](a)\,.
\]
\end{theorem}

Note that $\Hom(Y,X)$ is unlikely to be finitely generated over $k$, so there is a need to distinguish between $\Hom_k$ and $\Hom_k^{\operatorname{gr}}$. For the proof we need the following fact about residues.

\begin{lemma}\label{lemma:residuesarehomog} If $f_1,\ldots,f_n$ is a regular sequence in $R$ of homogeneous elements, then the map
\[
\Ress{R/k}\Bigg[ \frac{(-) \, \ud \underline{x}}{ f_1, \ldots, f_n} \Bigg]: R \lto k
\]
is homogeneous of degree $\sum_{i=1}^n( |x_i| - |f_i| )$.
\end{lemma}
\begin{proof}
(\textbf{todo}) Follows from the determinantal formula for residues in Lipman's book.
\end{proof}

\begin{proof}[Proof of Theorem \ref{theorem:generalklgraded}]
Consider the following diagram of $\mathbb{Z}_2$-graded complexes
\[
\xymatrix@C+2pc{
\Hom(X,Y) \ar@{.>}[dr]_-{\zeta_{\operatorname{gr}}} \ar[r]^-{\xi} & \Hom(\Hom(Y,X),k)[n]\\
& \Hom^{gr}_k( \Hom(Y,X), k)[n](a) \ar[u]_{\iota = \inc}
}
\]
where $\xi$ is adjoint to (\ref{eq:generalklpairing}), and is therefore a homotopy equivalence by the earlier theorem. It is clear from the explicit formula for the pairing and Lemma \ref{lemma:residuesarehomog} that the image of $\xi$ lies in the subspace of homogeneous maps, and we let $\zeta_{\operatorname{gr}}$ denote this factorisation. The value of $a$ is chosen such that $\zeta_{gr}$ is homogeneous of degree zero. It is enough to prove that $\zeta_{\operatorname{gr}}$ is a homotopy equivalence of $\mathbb{Z}_2$-graded complexes (forgetting the $\mathbb{Z}$-grading) since then by an elementary argument we can find a homotopy inverse which is degree zero, and graded homotopies.

It suffices to prove that the inclusion $\iota$ is a homotopy equivalence of $\mathbb{Z}_2$-graded complexes. This would be a tautology if $\Hom(Y,X)$ were a finitely generated $k$-module, and in general we use the fact that $\Hom(Y,X)$ embeds up to homotopy in a finitely generated complex. That is: by \cite[Section ??]{blah} we have a diagram
\begin{equation}\label{eq:generalklgraded}
\xymatrix@C+2pc{
\Hom(\bar{Y}, \bar{X}) \ar@<-0.5ex>[r]_-{\psi} & \Hom(Y,X)[n] \ar@<-0.5ex>[l]_-{\vartheta}
}
\end{equation}
of $\mathbb{Z}_2$-graded complexes over $k$ with $\psi \circ \vartheta \simeq 1$. But if one examines the explicit expressions for $\vartheta, \psi$ given in loc.\,cit.~it is clear that $\vartheta$ has degree $-a$, $\psi$ has degree $a$, and the homotopy between $\psi \circ \vartheta$ and the identity is homogeneous. If we apply both $\Hom_k(-,k)$ and $\Hom_k^{\operatorname{gr}}(-,k)$ to (\ref{eq:generalklgraded}) we obtain a diagram
\[
\xymatrix@C+2pc{
\Hom_k( \Hom(\bar{Y}, \bar{X}), k ) \ar@<-0.5ex>[r]_-{\vartheta^*} & \Hom_k(\Hom(Y,X),k)[n] \ar@<-0.5ex>[l]_-{\psi^*}\\
\Hom_k^{\operatorname{gr}}(\Hom(\bar{Y}, \bar{X}), k) \ar[u]^{=} \ar@<-0.5ex>[r]_-{\vartheta^*} & \Hom_k^{\operatorname{gr}}(\Hom(Y,X),k)[n](a) \ar@<-0.5ex>[l]_-{\psi^*} \ar[u]_{\iota}
}
\]
in which both of the implict squares commute up to homotopy, and from this we deduce that $\iota$ is a homotopy equivalence with inverse $\vartheta^* \circ \psi^*$.
\end{proof}


\appendix

\section{Symmetrised idempotents}\label{section:symidem}

In Section \ref{section:dualityadjointop} we used the idempotent pushforward construction of \cite{dm1102.2957} to establish a relative form of duality for matrix factorisations. We needed slightly more than what is stated in \textsl{loc.cit.} and in this appendix we provide the necessary additions. With later applications in mind we work in the same generality as the original construction, but the reader should keep in mind that for Section~\ref{section:dualityadjointop} we only need the case $Z = \Hom_R(X,Y)$ and $W = 0$ of the following.

Let $k$ be a $\nQ$-algebra and $R$ a $k$-algebra with a quasi-regular sequence $\{ f_1, \ldots, f_n \}$ such that $\bar{R} = R/(f_1,\ldots,f_n)R$ is a finitely generated projective $k$-module. Let $W \in k$ be given and let $Z$ be a finite-rank matrix factorisation of $W$ over $R$ with each $f_i$ acting null-homotopically on $Z$, and moreover let $\lambda_i \in \Hom_R(Z,Z)$ be odd maps with $[D, \lambda_i] = f_i \cdot 1_Z$ for each $1 \le i \le n$ where $D = d_Z$. We fix a homogeneous basis for $Z$ and, unless specified otherwise, $\otimes = \otimes_R$.

The aim is to write $Z$, up to $k$-linear homotopy equivalence, as the splitting of an idempotent on the quotient $\bar{Z} = Z \otimes \bar{R}$. Suppose that $R$ admits a flat $k$-linear connection $\nabla$ as a $k[f]$-module which is standard in the sense of \cite[Definition 8.6]{dm1102.2957}. The components of $\nabla$ are denoted $\partial_{f_i}$. If $K(f) = K(f_1,\ldots,f_n)$ denotes the usual Koszul complex then by hypothesis the map $K(f) \lto \bar{R}$ is a quasi-isomorphism and by \cite[Section 10]{dm1102.2957} the projection
\begin{equation}\label{eq:symidem1}
\pi: Z \otimes K(f) \lto Z \otimes \bar{R} = \bar{Z}
\end{equation}
is a homotopy equivalence over $k$. 

As a graded $R$-algebra $K(f)$ is the exterior algebra $\bigwedge F$ where $F$ is $R$-free on symbols $\theta_1,\ldots,\theta_n$ placed in degree $-1$. Since the $f_i$ all act null-homotopically on $Z$, if we think of $Z \otimes K(f)$ as being constructed by iterated mapping cones it is easy to see that this complex is homotopy equivalent over $R$ to the tensor product $Z \otimes \bigwedge F$, with no differential placed on the exterior component; see \cite[\S $4$]{dm1102.2957}. This is just a direct sum of $2^n$ copies of $Z$ and $Z[1]$. In light of (\ref{eq:symidem1}) this means that in the homotopy category of linear factorisations of $W$ over $k$, there is an isomorphism
\[
\bar{Z} \cong Z \otimes_k \bigwedge (k\theta_1 \oplus \ldots \oplus k\theta_n)
\]
between $\bar{Z}$ and a direct sum of shifted copies of $Z$, and to this direct sum decomposition corresponds $2^n$ orthogonal $k$-linear idempotents on $\bar{Z}$. Our aim is to describe the idempotents corresponding to the top degree summand $Z \theta_1 \ldots \theta_n$ and the bottom degree summand $Z \cdot 1$.

The strategy is to construct a pair of $R$-linear idempotents on $Z \otimes K(f)$, and then transfer these idempotents to $\bar{Z}$. More precisely, we will construct a diagram
\begin{equation}\label{eq:twoidempotents}
\xymatrix@C+2pc{
Z[n] \ar@<0.7ex>[r]^-{\vartheta'} & Z \otimes K(f) \ar@<0.7ex>[l]^-{\varepsilon}\ar@<0.7ex>[r]^-{\rho} & Z \ar@<0.7ex>[l]^-{\kappa'}
}
\end{equation}
in which the maps are $R$-linear morphisms of linear factorisations satisfying $\rho \circ \kappa' = 1$ and $\varepsilon \circ \vartheta' = 1$. The composites $\kappa' \circ \rho$ and $\vartheta' \circ \varepsilon$ will be the desired pair of idempotents on $Z \otimes K(f)$ corresponding to top and bottom degree.

To this end consider the case of a single $f_i$, so there is an exact sequence
\[
\xymatrix{
0 \ar[r] & Z \ar[r]^-{\kappa'_i} & Z \otimes K(f_i) \ar[r]^-{\varepsilon_i} & Z\theta_i \ar[r] & 0
}
\]
where $Z\theta_i = Z[1]$, $\kappa'_i(x) = x \otimes 1$ is the inclusion and $\varepsilon_i( x \theta_i ) = (-1)^{|x|} x$. This sequence is split exact; choosing a splitting is equivalent to choosing a null-homotopy on $X$ for the action of $f_i$. The maps
\[
\rho_i(x + y \theta_i) = x + (-1)^{|y|} \lambda_i(y), \qquad \vartheta'_i(x) = (-1)^{|x|} x \theta_i - \lambda_i(x)
\]
satisfy $\kappa'_i \circ \rho_i + \vartheta'_i \circ \varepsilon_i = 1$ and $\rho_i \circ \kappa'_i = 1, \varepsilon_i \circ \theta_i = 1$ and $\rho_i \circ \vartheta'_i = 0$. That is, they equip $Z \otimes K(f_i)$ with the structure of a biproduct $Z \oplus Z\theta_i$. It will be convenient to have a more compact definition of $\vartheta'_i$. Let us agree that left multiplication by $\theta_i$ on $X \otimes K(f)$ comes with signs $\theta_i \cdot (x \otimes \eta) = (-1)^{|x|} x \otimes \theta_i \eta$ so that multiplication by $\theta_i$ defines a map $\theta_i: X \lto X \otimes K(f_i)$ and $\vartheta'_i = \theta_i - \lambda_i$.

To construct idempotents on $Z \otimes K(f)$ we can inductively apply either of the projections $\varepsilon$ or $\rho$ until we reach form degree zero, e.g. in the first step we can choose to ``keep'' $\theta_n$
\[
Z \otimes K(f_1,\ldots,f_n) \cong (Z \otimes K(f_1,\ldots,f_{n-1})) \otimes K(f_n) \xlto{\varepsilon} Z \otimes K(f_1,\ldots,f_{n-1})\theta_n\,,
\]
or we can choose to project out $\theta_n$ by applying $\rho$ instead. If after keeping $\theta_n$ we proceed to apply the $\varepsilon$ projections for the rest of the $\theta_i$, we obtain a split epimorphism $\varepsilon: Z \otimes K(f) \lto Z[n]$ defined by $\varepsilon(x \theta_1 \ldots \theta_n) = (-1)^{n|x|} x$ with left inverse $\vartheta'$ defined by concatenating all the $\vartheta'_i$. This left inverse depends on the order in which we project out the $\theta_i$, but if we symmetrise over all permutations we obtain the following left inverse to $\varepsilon$
\[
\vartheta' = \frac{1}{n!} \sum_{\sigma \in S_n} (-1)^{|\sigma|} (\theta_{\sigma(1)} - \lambda_{\sigma(1)}) \circ \ldots \circ (\theta_{\sigma(n)} - \lambda_{\sigma(n)})\,.
\]
The reader may safely ignore this derivation, and instead check directly that $\vartheta'$ commutes with the differentials and satisfies $\varepsilon \circ \vartheta' = 1$, so that $\vartheta' \circ \varepsilon$ is an idempotent on $Z \otimes K(f)$ splitting to $Z[n]$.

At the other extreme we can choose at each stage to project out the $\theta_i$ using $\rho$ to obtain a split epimorphism $Z \otimes K(f) \lto Z$ with left inverse the inclusion $\kappa'(x) = x \otimes 1$. The precise formula for this epimorphism depends on the order in which we choose to project out the $\theta_i$. If we symmetrise over all permutations we arrive at the morphism
\begin{equation}\label{eq:rhoformula1}
\rho: Z \otimes K(f) \lto Z
\end{equation}
defined for $i_1 < \ldots < i_p$ by
% see syme3
\begin{equation}\label{eq:rhoformula2}
\rho( x \theta_{i_1} \ldots \theta_{i_p} ) = \frac{1}{p!} (-1)^{p|x|} \sum_{\sigma \in S_p} (-1)^{|\sigma|} \lambda_{i_{\sigma(1)}} \ldots \lambda_{i_{\sigma(p)}}(x)\,.
\end{equation}
Again one may ignore this derivation and check directly that $\rho$ commutes with the differentials and satisfies $\rho \circ \kappa' = 1$, so that $\kappa' \circ \rho$ is an idempotent splitting to $Z$.

Having constructed the maps in \eqref{eq:twoidempotents} we now proceed to transfer these two idempotents to $\bar{Z}$. We do this via the $k$-linear homotopy inverse $\sigma_\infty$ to $\pi$ which is described in \cite[Section 10]{dm1102.2957} (with $S = k$ and $X = Z$). In the notation given there
\[
\sigma_\infty = \sum_{m \ge 0} (-1)^m (\tau^{-1} \nabla_{gr} D)^m \sigma\,.
\]
Here $\nabla_{gr} = \sum_{j=1}^n \partial_{f_j} \circ \theta_j$ where $\theta_j$ denotes left multiplication on $Z \otimes K(f)$ with attendant Koszul sign, and $\partial_{f_j}$ is the operator extended to $Z$ using our chosen homogeneous basis. Consider now the diagram
\be\label{eq:finalidempotents}
\xymatrix@C+2pc{
& & Z[n] \ar@<-0.7ex>[dl]_-{\vartheta}\\
\bar{Z} \ar@<-0.7ex>[r]_-{\sigma_\infty} & Z \otimes K(f) \ar@<-0.7ex>[ur]_-{\varepsilon}\ar@<-0.7ex>[l]_-{\pi} \ar@<-0.7ex>[dr]_-{\rho}\\
& & Z \ar@<-0.7ex>[ul]_-{\kappa}
}\,.
\ee
From this we deduce two diagrams of $k$-linear factorisations of $W$, writing $\vartheta = \pi \circ \vartheta'$ and $\kappa = \pi \circ \kappa'$
\be\label{eq:finalidempotents2}
\xymatrix@C+2pc{
\bar{Z} \ar@<-0.7ex>[r]_-{\varepsilon \circ \sigma_\infty} & Z[n] \ar@<-0.7ex>[l]_-{\vartheta}
}, \qquad
\xymatrix@C+2pc{
\bar{Z} \ar@<-0.7ex>[r]_-{\rho \circ \sigma_\infty} & Z \ar@<-0.7ex>[l]_-{\kappa}
}\,.
\ee
The $k$-linear idempotent on $\bar{Z}$ corresponding to top degree is then $e = \vartheta \circ \varepsilon \circ \sigma_\infty$ while the bottom degree idempotent is $e' = \kappa \circ \rho \circ \sigma_\infty$. The former is described by symmetrising the content of \cite[Corollary 10.4]{dm1102.2957}
\begin{equation}\label{eq:ordinaryidem}
e = \frac{1}{(n!)^2} (-1)^{\binom{n+1}{2}} \sum_{\sigma, \tau \in S_n} (-1)^{|\sigma\tau|} \lambda_{\sigma(1)} \ldots \lambda_{\sigma(n)} \At_{\tau(1)} \ldots \At_{\tau(n)}
\end{equation}
where $\At_i = [D, \partial_{f_i}]$ is the $i$th component of the Atiyah class.

\begin{proposition}\label{prop:otheridempotent} There is a $k$-linear homotopy
\begin{equation}\label{eq:propotheridempotent}
e' \simeq \frac{1}{(n!)^2} (-1)^{\binom{n+1}{2}} \sum_{\sigma, \tau \in S_n} (-1)^{|\sigma\tau|} \At_{\tau(1)} \ldots \At_{\tau(n)} \lambda_{\sigma(1)} \ldots \lambda_{\sigma(n)}\,.
\end{equation}
\end{proposition}
\begin{proof}
Working modulo the $f_i$ and using the tricks in the discussion preceeding \cite[Theorem 10.1]{dm1102.2957} we have
\begin{align*}
\sigma_\infty &= \sum_{m \ge 0} (-1)^m \frac{1}{m!} \big[ D, \nabla_{gr} \big]^m \\
\end{align*}
where $\nabla_{gr}$ has odd degree for the purpose of writing commutators. Expanding each of these commutators in its components $\At_i$ yields a sum
\begin{align*}
& \sum_{m \ge 0} \sum_{\sigma \in S_{m,n}} (-1)^{\binom{m+1}{2}} \frac{1}{m!} \At_{\sigma(1)} \ldots \At_{\sigma(m)} \theta_{\sigma(1)} \ldots \theta_{\sigma(m)}
\end{align*}
over all injective functions $S_{m,n}$ from $[1,m]$ to $[1,n]$. Using \eqref{eq:rhoformula2}
% see syme2
\begin{align*}
e' &= \pi \circ \kappa \circ \rho \circ \sigma_\infty\\
&= \sum_{m \ge 0} \frac{1}{m!} (-1)^{\binom{m+1}{2}} \sum_{\sigma \in S_{m,n}} \rho \At_{\sigma(1)} \ldots \At_{\sigma(m)} \theta_{\sigma(1)} \ldots \theta_{\sigma(m)}\\
&= \sum_{m \ge 0} \frac{1}{(m!)^2} (-1)^{\binom{m}{2}} \sum_{\sigma \in S_{m,n}} \sum_{\tau \in S_m} (-1)^{|\tau|} \lambda_{\sigma \tau(1)} \ldots \lambda_{\sigma \tau(m)} \At_{\sigma(1)} \ldots \At_{\sigma(m)}(x)\,.
\end{align*}
Using the relation $[ \lambda_i, \At_j ] = - \delta_{ij}$ up to homotopy given in the next lemma, we can commute all the Atiyah classes through to the left. Taking care of some combinatorics (and some delicate signs) the result is $e'$ is homotopic to the expression on the right hand side of \eqref{eq:propotheridempotent}.
\end{proof}
% see syme2

\begin{lemma} As operators on $\bar{Z}$ we have
\[
\lambda_i \At_j + \At_j \lambda_i = - \delta_{ij} + \big[ D, \big[ \partial_j, \lambda_i \big] \big]
\]
and in particular $[ \lambda_i, \At_j ] = - \delta_{ij}$ up to homotopy.
\end{lemma}
\begin{proof}
\textbf{todo}.
\end{proof}

\section{Remarks on adjoints}\label{app:adjoints}
% see bicat2, adjbi, adjbi2

Let $\cat{B}$ be a bicategory. We address the following problem encountered in the construction of adjoints in the bicategory of Landau-Ginzburg models: suppose that for each $1$-morphism $f: A \lto B$ we are given a pair of $1$-morphisms ${}^\dual f, f^\dual: B \lto A$ and that for those $f$ belonging to some subcategory $\cat{C}_{AB} \subseteq \cat{B}(A, B)$ we are given evaluation and coevaluation maps exhibiting ${}^\dual f, f^\dual$ as adjoints of $f$.

Provided every object of $\cat{B}(A,B)$ is a direct summand of an object of $\cat{C}_{AB}$ we explain below that there is an essentially unique way of extending the definition of evaluation and coevaluation maps from $\cat{C}$ to all of $\cat{B}(A,B)$.

But before getting into that we need to recall some more background on adjoints in bicategories. Specifically, the way in which the adjoints ${}^\dual f$ and $f^\dual$ are functorial with respect to $2$-morphisms. This is an adaptation of standard material for monoidal categories, so we will be brief.

Throughout this section we write $\Hom(f,g)$ for $\Hom_{\cat{B}(A,B)}(f,g)$.

\subsection{Functoriality of adjoints}\label{section:funcadjoint}

The following lemma recasts the definition of adjunction given in Section \ref{subsec:bicat} in terms of a natural isomorphism. Observe that we insist on naturality with respect to both $1$ and $2$-morphisms. 

\begin{lemma} Let $f: A \lto B$ and $g: B \lto A$ be $1$-morphisms. Specifying $2$-morphisms $\eval, \coev$ satisfying \eqref{uglyZorro1}, \eqref{uglyZorro2} is equivalent to specifying a family of bijections
\[
\Phi_{h, l}: \Hom(g \otimes h, l) \lto \Hom(h, f \otimes l),
\]
indexed by pairs of $1$-morphisms $h: C \lto B$ and $l: C \lto A$, subject to the following conditions:
\begin{itemize}
\item[(i)] The bijections $\Phi$ are natural with respect to $2$-morphisms in the variables $h, l$, and
\item[(ii)] The bijections $\Phi$ are natural with respect to $1$-morphisms in the sense that for any $h': C' \lto C$ the diagram
\[
\xymatrix@C+2.5pc{
\Hom(g \otimes h, l) \ar[d]_-{(-) \otimes 1_{h'}}\ar[r]^-{\Phi_{h,l}} & \Hom(h, f \otimes l) \ar[d]^-{(-) \otimes 1_{h'}}\\
\Hom( (g \otimes h) \otimes h', l \otimes h') \ar[d]_\alpha & \Hom(h \otimes h', (f \otimes l) \otimes h') \ar[d]^\alpha\\
\Hom( g \otimes (h \otimes h'), l \otimes h') \ar[r]_-{\Phi_{h \otimes h', l \otimes h'}} & \Hom(h \otimes h', f \otimes (l \otimes h') )
}
\]
commutes.
\end{itemize}
\end{lemma}
\begin{proof}
The proof is essentially the same as the usual proof for adjoint functors, so we omit it.
\end{proof}

\begin{remark}
For the careful reader we mention that when working with bicategories one often needs to check commutativity of diagrams constructed from the structure morphisms $\lambda, \rho, \alpha$. Just as for monoidal categories this commutativity is, in most cases, an easy consequence of the coherence axioms. The only exceptions relevant for this paper are the diagrams $(5), (7), (10)$ of \cite{kelly}, whose commutativity follows from the coherence axioms but not in a completely obvious way.
\end{remark}

Let $f_1,f_2: A \lto B$ be $1$-morphisms with left adjoints ${}^\dual f_i$. By the previous lemma there is an isomorphism
\be\label{eq:2isolemma}
\Hom({}^\dual f_2, {}^\dual f_1) \stackrel{\rho}{\cong} \Hom({}^\dual f_2 \otimes \Delta_B, {}^\dual f_1) \stackrel{\Phi}{\cong} \Hom( \Delta_B, f_2 \otimes {}^\dual f_1)\,.
\ee
\begin{definition} Given a $2$-morphism $\zeta: f_1 \lto f_2$ we define ${}^\dual \zeta: {}^\dual f_2 \lto {}^\dual f_1$ to be the unique $2$-morphism making the diagram
\be\label{eq:dualzeta}
\xymatrix@C+1pc@R+1pc{
\Delta_B \ar[r]^-{\coev} \ar[d]_-{\coev} & f_1 \otimes {}^\dual f_1 \ar[d]^{\zeta \otimes 1}\\
f_2 \otimes {}^\dual f_2 \ar[r]_-{1 \otimes {}^\dual \zeta} & f_2 \otimes {}^\dual f_1
}
\ee
commute.
\end{definition}

\begin{remark}
\begin{itemize}
\item[(i)] The construction is functorial, in the sense that ${}^\dual(1_f) = 1_{{}^\dual f}$ and ${}^\dual( \xi \circ \zeta ) = {}^\dual \zeta \circ {}^\dual \xi$.
\item[(ii)] Using commutativity of \eqref{eq:dualzeta} and the Zorro identities one checks that
\be\label{eq:dualzeta2}
\xymatrix@C+1pc@R+1pc
{
{}^\dual f_2 \otimes f_1 \ar[r]^-{{}^\dual \zeta \otimes 1}\ar[d]_-{1 \otimes \zeta} & {}^\dual f_1 \otimes f_1 \ar[d]^-{\eval}\\
{}^\dual f_2 \otimes f_2 \ar[r]_{\eval} & \Delta_A
}
\ee
commutes.
\item[(iii)] In diagrammatic language the explicit construction of the $2$-morphism ${}^\dual \zeta$ can be presented in terms of the diagram:

The commutativity of \eqref{eq:dualzeta} and \eqref{eq:dualzeta2} expresses the following two identities, which tell us that we can ``slide'' $2$-morphism around cups and caps:
\end{itemize}
\end{remark}

Let $f_1,f_2: A \lto B$ be $1$-morphisms with right adjoints $f_i^\dual$. There is an isomorphism
\[
\Hom(f_2^\dual, f_1^\dual) \stackrel{\rho}{\cong} \Hom(f_2^\dual, f_1^\dual \otimes \Delta_B) \stackrel{\Phi}{\cong} \Hom(f_1 \otimes f_2^\dual, \Delta_B)\,.
\]

\begin{definition} Given a $2$-morphism $\zeta: f_1 \lto f_2$ we define $\zeta^\dual: f_2^\dual \lto f_1^\dual$ to be the unique $2$-morphism making the diagram
\be\label{eq:funcrightdual}
\xymatrix@C+1pc@R+1pc{
f_1 \otimes f_2^\dual \ar[r]^-{\zeta \otimes 1}\ar[d]_-{1 \otimes \zeta^\dual} & f_2 \otimes f_2^\dual \ar[d]^-{\eval}\\
f_1 \otimes f_1^\dual \ar[r]_{\eval} & \Delta_B
}
\ee
commute.
\end{definition}

Again the construction is functorial, and the diagram
\be\label{eq:funcrightdual2}
\xymatrix@C+1pc@R+1pc{
\Delta_A \ar[d]_-{\coev} \ar[r]^-{\coev} & f_1^\dual \otimes f_1 \ar[d]^-{1 \otimes \zeta}\\
f_2^\dual \otimes f_2 \ar[r]_-{\zeta^\dual \otimes 1} & f_1^\dual \otimes f_2
}
\ee
commutes.

\subsection{Extending duality to the idempotent closure}

Let $\cat{B}$ be a bicategory and suppose that for fixed objects $A,B$ there is a contravariant functor
\be\label{eq:extenddual1}
(-)^\dual: \cat{B}(A,B) \lto \cat{B}(B,A)
\ee
and a full subcategory $\cat{C} \subseteq \cat{B}(A,B)$ together with, for each $f \in \cat{C}$, a pair of $2$-morphisms
\be\label{eq:extenddual2}
\eval_f: f \otimes f^\dual \lto \Delta_B \, , \qquad \coev_f: \Delta_A \lto f^\dual \otimes f
\ee
making $f$ into the left adjoint of $f^\dual$. Suppose that the following two conditions are satisfied:
\begin{itemize}
\item[(i)] every object in the category $\cat{B}(A,B)$ is a direct summand of an object in $\cat{C}$, and
\item[(ii)] for every morphism $\zeta: f_1 \lto f_2$ in $\cat{C}$ the morphism $\zeta^\dual: f_2^\dual \lto f_1^\dual$ defined in Section \ref{section:funcadjoint} agrees with the image of $\zeta$ under the functor $\eqref{eq:extenddual1}$. Equivalently, the image makes \eqref{eq:funcrightdual} commute.
\end{itemize}

Having been given duality in the subcategory $\cat{C}$ we show how to exend to the idempotent closure. The argument for extending left adjoints is identical and we leave it to the reader.

\begin{proposition}\label{prop:uniquext} There is a \textsl{unique} way of defining for each $x \in \cat{B}(A,B)$ a pair of $2$-morphisms
\[
\eval_x: x \otimes x^\dual \lto \Delta_B \, , \qquad \coev_x: \Delta_A \lto x^\dual \otimes x
\]
satisfying the conditions
\begin{itemize}
\item[(i)] if $x \in \cat{C}$ then $\eval_x, \coev_x$ are as given in \eqref{eq:extenddual2},
\item[(ii)] $\eval_x, \coev_x$ make $x$ the left adjoint of $x^\dual$, and
\item[(iii)] the diagrams
\be\label{eq:propuniquext}
\xymatrix@C+1pc@R+1pc{
\Delta_A \ar[d]_-{\coev_{x_2}} \ar[r]^-{\coev_{x_1}} & x_1^\dual \otimes x_1 \ar[d]^-{1 \otimes \zeta}\\
x_2^\dual \otimes x_2 \ar[r]_-{\zeta^\dual \otimes 1} & x_1^\dual \otimes x_2
}
 \qquad
\xymatrix@C+1pc@R+1pc{
x_1 \otimes x_2^\dual \ar[r]^-{\zeta \otimes 1}\ar[d]_-{1 \otimes \zeta^\dual} & x_2 \otimes x_2^\dual \ar[d]^-{\eval_{x_2}}\\
x_1 \otimes x_1^\dual \ar[r]_{\eval_{x_1}} & \Delta_B
}
\ee
commute for every morphism $\zeta: x_1 \lto x_2$ in $\cat{B}(A,B)$.
\end{itemize}
\end{proposition}

The proof occupies the rest of this section. To begin with we choose for each $x \in \cat{B}(A,B)$ an object $f \in \cat{C}$ (depending on $x$) and $2$-morphisms $q,p$ such that $p \circ q = 1_x$ as in the diagram
\[
\xymatrix@C+2pc{
x \ar@<-1ex>[r]_-{q} & f \ar@<-1ex>[l]_-{p}
}\,.
\]
Using these choices we define the $2$-morphisms $\eval_x, \coev_x$, which will turn out to be independent of the choice of $f,p,q$. Applying the functor $(-)^\dual$ to the morphisms $p,q$ of $\cat{B}(A,B)$ we have summands
\[
\xymatrix@C+2pc{
x^\dual \otimes x \ar@<-1ex>[r]_-{p^\dual \otimes q} & f^\dual \otimes f \ar@<-1ex>[l]_-{p^\dual \otimes q}
}, \qquad (q^\dual \otimes p) \circ (q^\dual \otimes p) = 1_{x^\dual \otimes x}
\]
and
\[
\xymatrix@C+2pc{
x \otimes x^\dual \ar@<-1ex>[r]_-{q \otimes p^\dual} & f \otimes f^\dual \ar@<-1ex>[l]_-{p \otimes q^\dual}
}, \qquad (p \otimes q^\dual) \circ (q \otimes p^\dual) = 1_{x \otimes x^\dual}\,.
\]
Using these maps we define $\coev_x$ to be the composite
\[
\coev_x = \xymatrix@C+1pc{ \Delta_A \ar[r]^-{\coev_f} & f^\dual \otimes f \ar[r]^-{q^\dual \otimes p} & x^\dual \otimes x }
\]
and we define $\eval_x$ to be
\[
\eval_x = \xymatrix@C+1pc{ x \otimes x^\dual \ar[r]^-{q \otimes p^\dual} & f \otimes f^\dual \ar[r]^-{\eval_f} & \Delta_B }\,.
\]
First we check that $\eval_x, \coev_x$ make $x$ into a left adjoint of $x^\dual$.

\begin{lemma} $\eval_x, \coev_x$ satisfy \eqref{uglyZorro1} and \eqref{uglyZorro2}.
\end{lemma}
\begin{proof}
We explain the proof of \eqref{uglyZorro2}, the other verification is similar. Consider the following diagram, where we omit an associator to avoid clutter
\[
\xymatrix@C+2pc@R+1pc{
f \ar[d]_-{p} \ar[r]^-{\rho_f^{-1}} & f \otimes \Delta_A \ar[d]_-{p \otimes 1}\ar[r]^-{1 \otimes \coev_f} & f \otimes f^\dual \otimes f \ar@/_1pc/[d]_-{p \otimes q^\dual \otimes p} \ar[r]^-{\eval_f \otimes 1} & \Delta_A \otimes f \ar[r]^-{\lambda_f} & f\\
x \ar[r]_-{\rho_x^{-1}} & x \otimes \Delta_A \ar[r]_-{1 \otimes \coev_x} & x \otimes x^\dual \otimes x \ar@/_1pc/[u]_-{q \otimes p^\dual \otimes q} \ar[r]_-{\eval_x \otimes 1} & \Delta_A \otimes x \ar[u]_-{1 \otimes q}\ar[r]_-{\lambda_x} & x \ar[u]_-{q}
}\,.
\]
Let $Z$ denote the bottom row. By definition each of the squares in this diagram commute, so
\begin{align*}
q \circ Z \circ p &= \lambda_f \circ ( \eval_f \otimes 1 ) \circ ( qp \otimes (qp)^\dual \otimes qp ) \circ (1 \otimes \coev_f ) \circ \rho_f^{-1}\\
&= \lambda_f \circ (\eval_f \otimes 1) \circ ((qp)^2 \otimes 1 \otimes qp ) \circ (1 \otimes \coev_f ) \circ \rho_f^{-1}\\
&= \lambda_f \circ (\eval_f \otimes 1) \circ (qp \otimes 1 \otimes qp ) \circ (1 \otimes \coev_f ) \circ \rho_f^{-1}\\
&= qp \circ \lambda_f \circ (\eval_f \otimes 1) \circ (1 \otimes \coev_f ) \circ \rho_f^{-1} \circ qp\\
&= (qp)^2 = q \circ 1_x \circ p\,,
\end{align*}
and hence $Z = 1_x$ as claimed. In the second step we use the fact that $qp$ is a morphism in $\cat{C}$, so by naturality of the evaluation $\eval_f \circ ( qp \otimes 1 ) = \eval_f \circ ( 1 \otimes (qp)^\dual )$.
\end{proof}

Using naturality \eqref{eq:funcrightdual}, \eqref{eq:funcrightdual2} of $\eval_f, \coev_f$ it is straightforward to check that the diagrams \eqref{eq:propuniquext} commute. If we apply the same argument to two presentations of a $1$-morphism $x$ as a summand of $f,f'$ in $\cat{C}$ we see that $\eval_x, \coev_x$ are independent of the choice of presentation, and in particular if $x \in \cat{C}$ then these maps agree with the given ones.

To complete the proof of the proposition it only remains to argue that this family of $2$-morphisms $\{ \eval_x, \coev_x \}_{x \in \cat{B}(A,B)}$ is \textsl{unique}. Suppose that we are given $\{\eval'_x, \coev'_x\}_{x \in \cat{B}(A,B)}$ satisfying (i),(ii),(iii) of the proposition. Given $x$ with presentation as above, the solid square in the diagram
\[
\xymatrix@C+1pc{
f \otimes x^\dual \ar[d]_-{p \otimes 1}\ar[r]^-{1 \otimes p^\dual} & f \otimes f^\dual \ar[d]^-{\eval_f}\\
x \otimes x^\dual \ar@{.>}@/_1pc/[u]_{q \otimes 1}\ar[r]_-{\eval'_x} & \Delta_B
}
\]
commutes by (iii) and (i). But then
\[
\eval'_x = \eval'_x \circ (p \otimes 1) \circ (q \otimes 1) = \eval_f \circ (1 \otimes p^\dual) \circ (q \otimes 1) = \eval_x
\]
and similarly for $\coev'_x$.

\section{The map $\varepsilon \Psi$}\label{appendix:mapeppsi}
% see divop

In this section $k$ is any commutative ring, $R = k[x_1,\ldots,x_n]$. We write $\Re = R \otimes_k R = k[x] \otimes_k k[x']$. In the notation of Section \ref{subsec:Bar} both the Koszul complex $(\Delta, \delta_{-})$ and $(\Bar, b')$ give $\Re$-resolutions of the diagonal $R$, and $\Psi$ is the unique chain map up to homotopy making the triangle in the diagram
\[
\xymatrix{
\Bar \ar[dr]_-{\pi} \ar[r]^-{\Psi} & \Delta \ar[d]^-{\pi} \ar[r]^\varepsilon & \Re[n]\\
& R
}
\]
commute. As indicated in this diagram there is a second canonical chain map $\varepsilon: (\Delta, \delta_{-}) \lto \Re[n]$, see \eqref{eq:vareps}, and the composite $\varepsilon \Psi$ gives a cocycle in Hochschild cohomology
\[
\varepsilon \Psi \in H^n \Hom_{\Re}( \Bar, \Re ) = H^n( R, \Re )\,.
\]
This Hochschild cohomology module is a free $R$-module of rank one, and $\varepsilon \Psi$ gives a generator. In this appendix we explain that $\varepsilon \Psi$ is a product in Hochschild cohomology of the derivations
\[
\partial_{(i)}: R \lra \Re \, , \qquad f \lmt \frac{f - {}^{t_i}f}{x_i-x'_i} \, ,
\]
where we use the notation ${}^{t_i}(-)$ of Section \ref{subsec:bicatLG}. 

\begin{definition} Let $\Re_{(i)}$ denote the $R$-bimodule which is $\Re$ with the left and right actions
\[
r_1 * s * r_2 = r_1 \cdot s \cdot {}^{t_i} r_2\,, \qquad r_i \in R, s \in \Re
\]
where ``$\cdot$'' denotes ring multiplication in $\Re$.
\end{definition}

\begin{lemma} $\partial_{(i)}$ is a $k$-linear derivation of $R$ into the bimodule $\Re_{(i)}$.
\end{lemma}
\begin{proof}
Immediate from the identity in Lemma \ref{lem:LeibnizForDQO}.
\end{proof}

By the usual correspondence between bimodule derivations and Hochschild $1$-cocycles, each $\partial_{(i)}$ determines a class $[ D_i ] \in H^1(R, \Re_{(i)})$. Taking the product yields a class in
\[
[ D_1 ] * \cdots * [D_n] \in H^n(R, \Re_{(1)} \otimes_R \cdots \otimes_R \Re_{(n)})\,.
\]
Next we observe that the map
\begin{gather*}
\gamma: \Re_{(1)} \otimes_R \cdots \otimes_R \Re_{(n)} \lto \Re\,\\
\gamma( r_1 \otimes \cdots \otimes r_n ) = r_1 \cdot {}^{t_1} r_2 \cdot {}^{t_1t_2} r_3 \cdots {}^{t_1 \ldots t_{n-1}} r_n
\end{gather*}
is a morphism of $R$-bimodules.

\begin{lemma} $\varepsilon \Psi$ is the image of $[D_1] * \cdots * [D_n]$ under the induced map
\[
H^n(R, \gamma): H^n(R, \Re_{(1)} \otimes_R \cdots \otimes_R \Re_{(n)}) \lto H^n(R, \Re)\,.
\]
That is, in Hochschild cohomology
\[
\varepsilon \Psi = H^n(R, \gamma)( [ D_1 ] * \cdots * [D_n] )\,.
\]
\end{lemma}
\begin{proof}
We have
\begin{align*}
H^n(R, \gamma)( [ D_1 ] * \cdots * [D_n] )( df_1 \ldots df_p ) &= \gamma\left( \partial_{(1)}f_1 \otimes \partial_{(2)}f_2 \cdots \otimes \partial_{(n)}f_n \right)\\
&= \partial_{(1)}f_1 {}^{t_1} \partial_{(2)}f_2 \cdots \otimes {}^{t_1 \ldots t_{n-1}}\partial_{(n)}f_n\\
&= \partial_{[1]}f_1 \partial_{[2]}f_2 \cdots \otimes \partial_{[n]}f_n\\
&= \varepsilon \Psi( df_1 \ldots df_p )\,.
\end{align*}
\end{proof}

\newcommand{\etalchar}[1]{$^{#1}$}
\providecommand{\href}[2]{#2}
\begin{thebibliography}{FYH{\etalchar{+}}85}

\bibitem[Arn89]{arnold}
V.~I.~Arnold, \textsl{Mathematical methods of classical mechanics}, Graduate Texts in Mathematics Vol. 60, Springer-Verlag 1989.

\bibitem[ABCDG]{MB2}
P.~S.~Aspinwall, T.~Bridgeland, A.~Craw, M.~R.~Douglas, and M.~Gross, \textsl{Dirichlet Branes and Mirror Symmetry}, Clay Mathematics Monographs, AMS, 2009.

%\bibitem[BN]{bnKhovanov11crossings}
%D.~Bar-Natan, \textsl{Khovanov {H}omology for {K}nots and {L}inks with up to 11
%  {C}rossings}, available at
%  \href{http://www.math.toronto.edu/drorbn/papers/KHTables/KHTables.pdf}{http:%
%//www.math.toronto.edu/drorbn/papers/KHTables/KHTables.pdf}.
%
%\bibitem[Bec]{b1105.0702}
%H.~Becker, \textsl{Khovanov-Rozansky homology via Cohen-Macaulay approximations and Soergel bimodules},
%  \href{http://arxiv.org/abs/1105.0702}{[arXiv:1105.0702]}.

\bibitem[Ben67]{benabou}
J.~B\'{e}nabou, \textsl{Introduction to bicategories}, Reports of the Midwest Category
Seminar, pages 1�-77. Springer, Berlin, 1967.

\bibitem[Bor94]{bor94}
F.~Borceux, \textsl{Handbook of categorical algebra $1$}, volume $50$ of \textsl{Encyclopedia of Mathematics and its Applications}, Cambridge University Press, Cambridge, 1994.

\bibitem[BR07]{br0707.0922}
I.~Brunner and D.~Roggenkamp, \textsl{B-type defects in {L}andau-{G}inzburg
  models}, JHEP \textbf{0708} (2007), 093,
  \href{http://arxiv.org/abs/0707.0922}{[arXiv:0707.0922]}.

\bibitem[BF03]{buchweitz03}
R.-O. Buchweitz and H.~Flenner, \textsl{A semiregularity map for modules and
  applications to deformations}, Compositio Math. \textbf{137} (2003),
  135--210.

%\bibitem[CF94]{cf9405183}
%L.~Crane and I.~B. Frenkel, \textsl{Four dimensional topological quantum field
%  theory, {H}opf categories, and the canonical bases}, J. Math. Phys.
%  \textbf{35} (1994), 5136--5154,
%  \href{http://arxiv.org/abs/hep-th/9405183}{[hep-th/9405183]}.
%
%\bibitem[CK08a]{ck0701194}
%S.~Cautis and J.~Kamnitzer, \textsl{Knot homology via derived categories of
%  coherent sheaves {I}, {$sl(2)$} case}, Duke Math. J. \textbf{142} (2008),
%  511--588, \href{http://arxiv.org/abs/math/0701194}{[math.AG/0701194]}.
%
%\bibitem[CK08b]{ck0710.3216}
%S.~Cautis and J.~Kamnitzer, \textsl{Knot homology via derived categories of coherent sheaves {II},
%  {$sl(m)$} case}, Invent. Math. \textbf{174} (2008), 165--232,
%  \href{http://arxiv.org/abs/math/0710.3216}{[math.AG/0710.3216]}.
%
%\bibitem[CM]{cmWebCompileCode}
%N.~Carqueville and D.~Murfet, \textsl{Code to compute {K}hovanov-{R}ozansky
%  homology and defect fusion in {L}andau-{G}inzburg models},
%  \href{http://www.carqueville.net/nils/webCompilations}{http://www.carqueville.net/nils/webCompilations}.

\bibitem[CW10]{ct1007.2679}
A.~{C\u ald\u araru} and S.~Willerton, \textsl{The Mukai pairing, I: a categorical approach},
New York Journal of Mathematics \textbf{16} (2010) 61--98, 
  \href{http://arxiv.org/abs/0707.2052}{[arXiv:0707.2052]}.

\bibitem[CR10]{cr0909.4381}
N.~Carqueville and I.~Runkel, \textsl{On the monoidal structure of matrix bi-factorisations}, J. Phys.
  A: Math. Theor. \textbf{43} (2010), 275401,
  \href{http://arxiv.org/abs/0909.4381}{[arXiv:0909.4381]}.

\bibitem[CR12]{cr1006.5609}
N.~Carqueville and I.~Runkel, \textsl{Rigidity and defect actions in
  Landau-Ginzburg models}, Comm. Math. Phys. \textbf{310} (2012) 135--179, 
  \href{http://arxiv.org/abs/1006.5609}{[arXiv:1006.5609]}.

\bibitem[CR]{genorb}
N.~Carqueville and I.~Runkel, TODO: \textsl{Generalised orbifolds of Landau-Ginzburg models}, in preparation. 

\bibitem[Cra]{c0403266}
M.~Crainic, \textsl{On the perturbation lemma, and deformations},
  \href{http://arxiv.org/abs/math/0403266}{[math.AT/0403266]}.

%\bibitem[DGR06]{dgr0505662}
%N.~M. Dunfield, S.~Gukov, and J.~Rasmussen, \textsl{The {S}uperpolynomial for
%  {K}not {H}omologies}, Experimental Math. \textbf{15} (2006), 129--159,
%  \href{http://arxiv.org/abs/math/0505662}{[math.GT/0505662]}.

\bibitem[CQ95]{cuntzquillen}
J.~Cuntz and D.~Quillen, \textsl{Algebra extensions and nonsingularity}, Journal of the AMS, Volume 8, No. 2, 1995.

\bibitem[DKR]{dkr1107.0495}
A.~Davydov, L.~Kong, and I.~Runkel, \textsl{Field theories with defects and the
  centre functor}, Mathematical Foundations of Quantum Field Theory and Perturbative String Theory, 
  Proceedings of Symposia in Pure Mathematics, AMS, 2011, \href{http://arxiv.org/abs/1107.0495}{[arXiv:1107.0495]}.

%\bibitem[DBM{\etalchar{+}}11]{dbmmss1106.4305}
%P.~Dunin-Barkowski, A.~Mironov, A.~Morozov, A.~Sleptsov, A.~Smirnov, \textsl{Superpolynomials for toric knots from evolution induced by cut-and-join operators},
%  \href{http://arxiv.org/abs/1106.4305}{[arXiv:1106.4305]}. 
%  
\bibitem[Dyc11]{d0904.4713}
T.~Dyckerhoff, \textsl{Compact generators in categories of matrix factorizations},
  Duke Math. J. \textbf{159} (2011), 223--274,
  \href{http://arxiv.org/abs/0904.4713}{[arXiv:0904.4713]}.

\bibitem[DM]{dm1102.2957}
T.~Dyckerhoff and D.~Murfet, \textsl{Pushing forward matrix factorisations},
  \href{http://arxiv.org/abs/1102.2957}{[arXiv:1102.2957]}.

\bibitem[FRS02]{tft1}
J.~Fuchs, I.~Runkel and C.~Schweigert,
\textsl{TFT construction of RCFT correlators. I: Partition functions},
Nucl.~Phys.~B {\bf 646} (2002), 353--497, \href{http://arxiv.org/abs/hep-th/0204148}{[hep-th/0204148]}.

\bibitem[FFRS09]{ffrs0909.5013}
J.~Fr\"ohlich, J.~Fuchs, I.~Runkel and C.~Schweigert,
\textsl{Defect lines, dualities, and generalised orbifolds},
Proceedings of the XVI International Congress on Mathematical Physics, Prague, August 3--8, 2009, \href{http://arxiv.org/abs/0909.5013}{[arXiv:0909.5013]}.

\bibitem[Gray74]{gray}
J.~W.~Gray, \textsl{Formal category theory: adjointness for 2-categories}, Springer-Verlag, Berlin, 1974.

%\bibitem[FFRS07]{ffrs0607247}
%J.~Fr\"ohlich, J.~Fuchs, I.~Runkel, and C.~Schweigert, \textsl{Duality and
%  defects in rational conformal field theory}, Nucl. Phys. B \textbf{763}
%  (2007), 354--430,
%  \href{http://arxiv.org/abs/hep-th/0607247}{[hep-th/0607247]}.
%
%\bibitem[FYH{\etalchar{+}}85]{Homfly}
%P.~Freyd, D.~Yetter, J.~Hoste, W.~B.~R. Lickorish, K.~Millett, and A.~Ocneanu,
%  \textsl{A new polynomial invariant of knots and links}, Bull. Amer. Math. Soc.
%  \textbf{12} (1985), 239--246.
%
%\bibitem[GIKV10]{gikv0705.1368}
%S.~Gukov, A.~Iqbal, C.~Koz\c{c}az, and C.~Vafa, \textsl{Link {H}omologies and the
%  {R}efined {T}opological {V}ertex}, Comm. Math. Phys. \textbf{298} (2010),
%  757--785, \href{http://arxiv.org/abs/0705.1368}{[arXiv:0705.1368]}.
%
%\bibitem[GSV05]{gsv0412243}
%S.~Gukov, A.~Schwarz, and C.~Vafa, \textsl{Khovanov-{R}ozansky {H}omology and
%  {T}opological {S}trings}, Lett. Math. Phys. \textbf{74} (2005), 53--74,
%  \href{http://arxiv.org/abs/hep-th/0412243}{[hep-th/0412243]}.
%
%\bibitem[GV99]{gv9811131}
%R.~Gopakumar and C.~Vafa, \textsl{On the {G}auge {T}heory/{G}eometry
%  {C}orrespondence}, Adv. Theor. Math. Phys. \textbf{3} (1999), 1415--1443,
%  \href{http://arxiv.org/abs/hep-th/9811131}{[hep-th/9811131]}.
%
%\bibitem[GW]{gw0512298}
%S.~Gukov and J.~Walcher, \textsl{Matrix {F}actorizations and {K}auffman
%  {H}omology}, \href{http://arxiv.org/abs/hep-th/0512298}{[hep-th/0512298]}.
%
%\bibitem[Jae]{j1101.3302}
%T.~C. Jaeger, \textsl{Khovanov-{R}ozansky {H}omology and {C}onway {M}utation},
%  \href{http://arxiv.org/abs/1101.3302}{[arXiv:1101.3302]}.
%
%\bibitem[Jon85]{JonesPolynomialPaper}
%V.~F.~R. Jones, \textsl{A polynomial invariant for knots via von {N}eumann
%  algebras}, Bull. Amer. Math. Soc. \textbf{12} (1985), 103--111.

\bibitem[JS91]{JSGoTCI}
A.~Joyal and R.~Street, \textsl{The geometry of tensor calculus I}, Advances in Math. \textbf{88} (1991), 55--112.

\bibitem[JS]{JSGoTCII}
A.~Joyal and R.~Street, \textsl{The geometry of tensor calculus II}, 
draft available at 
\href{http://maths.mq.edu.au/~street/GTCII.pdf}{http://maths.mq.edu.au/\textasciitilde street/GTCII.pdf}

\bibitem[Ka]{k1004.2307}
A.~Kapustin, \textsl{Topological {F}ield {T}heory, {H}igher {C}ategories, and
  {T}heir {A}pplications},
  \href{http://arxiv.org/abs/1004.2307}{[arXiv:1004.2307]}.

\bibitem[Kel64]{kelly}
G.~M.~Kelly, \textsl{On {M}ac{L}ane's conditions for coherence of natural associativities, commutativities, etc.}, Journal of Algebra, 1, 397--402 (1964).

\bibitem[KS74]{kellystreet}
G.~M.~Kelly and R.~Street, \textsl{Review of the elements of 2-categories}, In Category Seminar (Proc. Sem., Sydney, 1972/1973), pages 75�103. Lecture Notes in Math., Vol. 420. Springer, Berlin, 1974.

\bibitem[Kho10]{khovdia}
M.~Khovanov, \textsl{Categorifications from planar diagrammatics}, Japanese J. of Mathematics 5, 153--181 (2010), \href{http://arxiv.org/abs/1008.5084}{[arXiv:1008.5084v1]}.

%\bibitem[Kaw96]{kawauchibook}
%A.~Kawauchi, \textsl{A {S}urvey of {K}not {T}heory}, Birkh\"auser, 1996.
%
%\bibitem[Kho00]{k9908171}
%M.~Khovanov, \textsl{A categorification of the {J}ones polynomial}, Duke Math. J.
%  \textbf{101} (2000), 359--426,
%  \href{http://arxiv.org/abs/math/9908171}{[math.QA/9908171]}.
%
%\bibitem[Kho07]{k0510265}
%M.~Khovanov, \textsl{Triply-graded link homology and Hochschild homology of Soergel bimodules},
%  Int. Journal of Math. \textbf{18} (2007), 869--885,
%  \href{http://arxiv.org/abs/math/0510265}{[math.GT/0510265]}.
%
%\bibitem[KR07a]{kr0701333}
%M.~Khovanov and L.~Rozansky, \textsl{Virtual crossings, convolutions and a
%  categorification of the {$\operatorname{SO}(2N)$} {K}auffman polynomial},
%  Journal of G\"okova Geometry Topology \textbf{1} (2007), 116--214,
%  \href{http://arxiv.org/abs/math/0701333}{[math.QA/0701333]}.
%
%\bibitem[KR07b]{kr0404189}
%M.~Khovanov and L.~Rozansky, \textsl{Topological Landau-Ginzburg models on the world-sheet foam},
%  Adv. Theor. Math. Phys. \textbf{11} (2007), 233--259,
%  \href{http://arxiv.org/abs/hep-th/0404189}{[hep-th/0404189]}.
%
%\bibitem[KR08a]{kr0401268}
%M.~Khovanov and L.~Rozansky, \textsl{Matrix factorizations and link homology}, Fund. Math.
%  \textbf{199} (2008), 1--91,
%  \href{http://arxiv.org/abs/math/0401268}{[math/0401268]}.
%
%\bibitem[KR08b]{kr0505056}
%M.~Khovanov and L.~Rozansky, \textsl{Matrix factorizations and link homology {II}}, Geometry \&
%  Topology \textbf{12} (2008), 1387--1425,
%  \href{http://arxiv.org/abs/math/0505056}{[math.QA/0505056]}.
%
%\bibitem[Ko]{Kock}
%J.~Kock, \textsl{Frobenius Algebras and 2D Topological Quantum Field Theories}, Cambridge University Press, 2003.
%
%\bibitem[Lam86]{LambekRingsModules}
%J.~Lambek, \textsl{Lectures on rings and modules}, AMS Chelsea Publishing, 1986.

\bibitem[Lack]{lack}
S.~Lack, \textsl{A $2$-categories companion}, \href{http://arxiv.org/abs/math/0702535}{[arXiv:math/0702535v1]}.

\bibitem[Lau12]{ladia}
A.~D.~Lauda, \textsl{An introduction to diagrammatic algebra and categorified quantum $\mathfrak{sl}_2$}, Bulletin of the Institute of Mathematics Academia Sinica (New Series), Vol. \textbf{7} (2012), No. 2, pp. 165--270, \href{http://arxiv.org/abs/1106.2128}{[arXiv:1106.2128v2]}.

\bibitem[Laz05]{l0312}
C.~I. Lazaroiu, \textsl{On the boundary coupling of topological {L}andau-{G}inzburg
  models}, JHEP \textbf{0505} (2005), 037,
  \href{http://www.arxiv.org/abs/hep-th/0312286}{[hep-th/0312286]}.

\bibitem[LM]{Calinetal}
C.~I. Lazaroiu and D.~McNamee, unpublished.

\bibitem[Lo]{Loday}
J.-L. Loday, \textsl{Cyclic homology}, Springer, 1997.

%\bibitem[LMnV00]{lmv0010102}
%J.~M.~F. Labastida, M.~Mari\~{n}o, and C.~Vafa, \textsl{Knot {I}nvariants and
%  {T}opological {S}trings}, JHEP \textbf{0011} (2000), 007,
%  \href{http://arxiv.org/abs/hep-th/0010102}{[hep-th/0010102]}.
%
%\bibitem[MSV09]{msv0708.2228}
%M.~Mackaay, M.~Sto\v{s}i\'{c}, and P.~Vaz, \textsl{$\mathfrak{sl}(N)$ link homology ($N\geq 4$) using foams and the Kapustin-Li formula}, Geometry \& Topology \textbf{13} (2009), 1075--1128,
%  \href{http://arxiv.org/abs/0708.2228}{[arXiv:0708.2228]}.
%
%\bibitem[Man07]{m0601629}
%C.~Manolescu, \textsl{Link homology theories from symplectic geometry}, Adv. in
%  Math. \textbf{211} (2007), 363--416,
%  \href{http://www.arxiv.org/abs/math.AG/0601629}{[math.SG/0601629]}.

\bibitem[McN09]{McNameethesis}
D.~McNamee, \textsl{On the mathematical structure of topological defects in
  {L}andau-{G}inzburg models}, MSc Thesis, Trinity College Dublin, 2009.
  
\bibitem[Mu]{m0912.1629}
D.~Murfet, \textsl{Residues and duality for singularity categories of isolated {G}orenstein singularities},
  \href{http://arxiv.org/abs/0912.1629}{[arXiv:0912.1629]}.  

%\bibitem[Mn05]{marinoknotbook}
%M.~Mari\~{n}o, \textsl{Chern-{S}imons {T}heory, {M}atrix {M}odels, and
%  {T}opological s{t}rings}, Oxford University Press, 2005.
%
%\bibitem[MOY98]{moy1998}
%H.~Murakami, T.~Ohtsuki, and S.~Yamada, \textsl{Homfly polynomial via an
%  invariant of colored plane graphs}, Enseign. Math. \textbf{44} (1998),
%  325--360.

\bibitem[MS]{ms0609042}
G.~W. Moore and G.~Segal, \textsl{D-branes and {K}-theory in {2D} topological
  field theory}, \href{http://arxiv.org/abs/hep-th/0609042}{[hep-th/0609042]}.

%\bibitem[MS]{ms0709.1971}
%V.~Manzorchuck and C.~Stroppel, \textsl{A combinatorial approach to functorial
%  quantum {$\mathfrak{sl}_k$} knot invariants},
%  \href{http://arxiv.org/abs/0709.1971}{[arXiv:0709.1971]}.
%
%\bibitem[OV00]{ov9912123}
%H.~Ooguri and C.~Vafa, \textsl{Knot {I}nvariants and {T}opological {S}trings},
%  Nucl. Phys. B \textbf{577} (2000), 419--438,
%  \href{http://arxiv.org/abs/hep-th/9912123}{[hep-th/9912123]}.

\bibitem[P10]{p0807.1471}
K.~Ponto, \textsl{Shadows and traces in bicategories}, 
Ast\'erisque, (333), 2010, 
\href{http://arxiv.org/abs/0807.1471}{[arXiv:0807.1471]}. 

\bibitem[PS]{ps0910.1306}
K.~Ponto and M.~Shulman, \textsl{Shadows and traces in bicategories}, 
\href{http://arxiv.org/abs/0910.1306}{[arXiv:0910.1306]}. 

\bibitem[PV]{pv1002.2116}
A.~Polishchuk and A.~Vaintrob, \textsl{Chern characters and {H}irzebruch-{R}iemann-{R}och formula for matrix factorizations}, 
\href{http://arxiv.org/abs/1002.2116}{[arXiv:1002.2116]}. 


%\bibitem[PT87]{HomflyPT}
%J.~Przytycki and P.~Traczyk, \textsl{Conway algebras and skein equivalence of
%  links}, Proc. Amer. Math. Soc. \textbf{100} (1987), 744--748.
%
%\bibitem[Ras]{r0607544}
%J.~Rasmussen, \textsl{Some differentials on {K}hovanov-{R}ozansky homology},
%  \href{http://arxiv.org/abs/math/0607544}{[math.GT/0607544]}.
%
%\bibitem[Ras07]{r0508510}
%J.~Rasmussen, \textsl{Khovanov-{R}ozansky homology of two-bridge knots and links},
%  Duke Math. J. \textbf{136} (2007), 551--583,
%  \href{http://arxiv.org/abs/math/0508510}{[math.GT/0508510]}.
%  
%\bibitem[Ric94]{rickard}
%J.~Rickard, \textsl{Translation functors and equivalences of derived categories for blocks of algebraic groups}, in “Finite dimensional algebras and related topics”, Kluwer (1994), 255-–264.
%
%\bibitem[Rou06]{RouquierMexico}
%R.~Rouquier, \textsl{Categorification of {$\mathfrak{sl}_{2}$} and braid groups},
%  Trends in representation theory of algebras and related topics (2006),
%  137--167.
%
%\bibitem[RT90]{RT1990}
%N.~Reshetikhin and V.~Turaev, \textsl{Ribbon graphs and their invariants derived
%  from quantum groups}, Comm. Math. Phys. \textbf{127} (190), 1--26.
%
%\bibitem[RT91]{RT1991}
%N.~Reshetikhin and V.~Turaev, \textsl{Invariants of 3-manifolds via link polynomials and quantum
%  groups}, Invent. Math. \textbf{103} (1991), 547--597.
%
%\bibitem[SS06]{ss0405089}
%P.~Seidel and I.~Smith, \textsl{A link invariant from the symplectic geometry of
%  nilpotent slices}, Duke Math. J. \textbf{134} (2006), 453--514,
%  \href{http://arxiv.org/abs/math/0405089}{[math.SG/0405089]}.

\bibitem[Qui85]{superquillen}
D.~Quillen, \textsl{Superconnections and the {C}hern character}, Topology Vol. 24, No. 1, 89--95, 1985.

\bibitem[SW11]{sw0911.0917}
C.~V.~Shepler and S.~Witherspoon, \textsl{Quantum differentiation and chain maps of bimodule complexes}, Algebra and Number Theory \textbf{5}-3 (2011), 339--360, 
\href{http://arxiv.org/abs/0911.0917}{[arXiv:0911.0917]}. 

%\bibitem[Str05]{sCatTLcTCpf}
%C.~Stroppel, \textsl{Categorification of the {T}emperley-{L}ieb category,
%  tangles, and cobordisms via projective functors}, Duke Math. J. \textbf{126}
%  (2005), 547--596.
%
%\bibitem[Sus]{s0701045}
%J.~Sussan, \textsl{Category {$\mathcal O$} and {$\mathfrak{sl}_k$} link
%  invariants}, \href{http://arxiv.org/abs/math/0701045}{[math.QA/0701045]}.
%
%\bibitem[Tur88]{t1988YB}
%V.~Turaev, \textsl{The {Y}ang-{B}axter equation and invariants of links}, Invent.
%  Math. \textbf{92} (1988), 527--553.
%
%\bibitem[Tur10]{turaevbook}
%V.~Turaev, \textsl{Quantum invariants of knots and 3-manifolds}, de Gruyter, 2010,
%  2nd edition.
%
%\bibitem[Weba]{w0610650}
%B.~Webster, \textsl{Khovanov-Rozansky homology via a canopolis formalism},
%  \href{http://arxiv.org/abs/math/0610650}{[math.GT/0610650]}.
%
%\bibitem[Webb]{w1005.4559}
%B.~Webster, \textsl{Knot invariants and higher representation theory {II}: the
%  categorification of quantum knot invariants},
%  \href{http://arxiv.org/abs/1005.4559}{[arXiv:1005.4559]}.
%
%\bibitem[Wit]{w1101.3216}
%E.~Witten, \textsl{Fivebranes and {K}nots},
%  \href{http://arxiv.org/abs/1101.3216}{[arXiv:1101.3216]}.
%
%\bibitem[Wit89]{wittenjones}
%E.~Witten, \textsl{Quantum field theory and the {J}ones polynomial}, Comm. Math.
%  Phys. \textbf{121} (1989), 351--399.
%
%\bibitem[Wit95]{w9207094}
%E.~Witten, \textsl{Chern-{S}imons {G}auge {T}heory {A}s {A} {S}tring {T}heory},
%  Prog. Math. \textbf{133} (1995), 637--678,
%  \href{http://arxiv.org/abs/hep-th/9207094}{[hep-th/9207094]}.
%
%\bibitem[Wu]{w0907.0695}
%H.~Wu, \textsl{A colored {$\mathfrak{sl}(N)$}-homology for links in {$S^3$}},
%  \href{http://arxiv.org/abs/0907.0695}{[arXiv:0907.0695]}.
%
%\bibitem[Wu08]{w0508064}
%H.~Wu, \textsl{Braids, {T}ransversal links and the {K}hovanov-{R}ozansky {T}heory},
%  \href{http://arxiv.org/abs/math/0508064}{[math.GT/0508064]}, Trans. Amer. Math. Soc. \textbf{360}
%  (2008), 3365--3389.
%
%\bibitem[Yon]{y0906.0220}
%Y.~Yonezawa, \textsl{Quantum {$(\mathfrak{sl}_n, \wedge V_n)$} link invariant and
%  matrix factorizations},
%  \href{http://arxiv.org/abs/0906.0220}{[arXiv:0906.0220]}.

\end{thebibliography}

\end{document}
