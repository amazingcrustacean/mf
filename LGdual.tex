 % Authors:  Nils Carqueville, Daniel Murfet
 
\documentclass{compositio}
\usepackage{stmaryrd}
\usepackage{amsmath, amscd, amssymb, mathrsfs, accents, amsfonts}
\usepackage{url}
\usepackage[all]{xy}
\usepackage{longtable}
\usepackage{dsfont}
\usepackage{tikz}
\def\nicenocolourscheme{\shadedraw[top color=gray!2, bottom color=gray!25, draw=gray!50!black, dashed]}
\definecolor{Myblue}{rgb}{0,0,0.6}
\usepackage[a4paper,colorlinks,citecolor=Myblue,linkcolor=Myblue,urlcolor=Myblue,pdfpagemode=None]{hyperref}

\SelectTips{cm}{}

\newtheorem{theorem}{Theorem}[section]
\newtheorem{proposition}[theorem]{Proposition}
\newtheorem{lemma}[theorem]{Lemma}
\newtheorem{corollary}[theorem]{Corollary}
\newtheorem*{theoremn}{Theorem}

\theoremstyle{definition}
\newtheorem{definition}[theorem]{Definition}
\newtheorem{example}[theorem]{Example}
\newtheorem{remark}[theorem]{Remark}
\newtheorem{s}[theorem]{}
\newtheorem*{setup}{Setup}
\newtheorem*{propositionn}{Proposition}

\numberwithin{equation}{section}

% Operators
\def\eval{\operatorname{ev}}
\def\res{\operatorname{Res}}
\def\sg{\operatorname{sg}}
\def\Inj{\operatorname{Inj}}
\def\inc{\operatorname{inc}}
\def\Proj{\operatorname{Proj}}
\def\Coker{\operatorname{Coker}}
\def\Ker{\operatorname{Ker}}
\def\Im{\operatorname{Im}}
\def\free{\operatorname{free}}
\def\can{\operatorname{can}}
\def\ac{\operatorname{ac}}
\def\HH{\operatorname{HH}}
\def\K{\mathbf{K}}
\def\D{\mathbf{D}}
\def\N{\mathbf{N}}
\def\sing{\operatorname{Sg}}
\def\Hom{\operatorname{Hom}}
\def\uHom{\underline{\Hom}}
\def\modd{\operatorname{mod}}
\def\Modd{\operatorname{Mod}}
\def\Grmodd{\operatorname{GrMod}}
\def\CM{\operatorname{CM}}
\def\Ker{\operatorname{Ker}}
\def\Spec{\operatorname{Spec}}
\def\straightK{\operatorname{K}}
\def\straightC{\operatorname{C}}
\def\holim{\operatorname{hocolim}}
\DeclareMathOperator{\Ext}{Ext}
\DeclareMathOperator{\coh}{coh}
\DeclareMathOperator{\serre}{S}
\DeclareMathOperator{\Flat}{Flat}
\DeclareMathOperator{\qc}{qc}
\DeclareMathOperator{\Perf}{Perf}
\DeclareMathOperator{\Map}{Map}
\DeclareMathOperator{\Qco}{Qco}
\DeclareMathOperator{\Tr}{Tr}
\DeclareMathOperator{\End}{End}
\DeclareMathOperator{\rank}{rank}
\DeclareMathOperator{\tot}{Tot}
\DeclareMathOperator{\skos}{K}
\DeclareMathOperator{\hht}{ht}
\DeclareMathOperator{\depth}{depth}
\DeclareMathOperator{\STr}{STr}
\DeclareMathOperator{\tr}{tr}
\DeclareMathOperator{\ch}{ch}
\DeclareMathOperator{\str}{str}
\DeclareMathOperator{\hmf}{hmf}
\DeclareMathOperator{\HMF}{HMF}
\DeclareMathOperator{\HF}{HF}
\DeclareMathOperator{\pr}{pr}
\DeclareMathOperator{\At}{At}
\DeclareMathOperator{\mff}{mf}
\DeclareMathOperator{\MF}{MF}
\begin{document}

% Commands
\def\Res{\res\!}
\newcommand{\cat}[1]{\mathcal{#1}}
\newcommand{\lto}{\longrightarrow}
\newcommand{\xlto}[1]{\stackrel{#1}\lto}
\newcommand{\mf}[1]{\mathfrak{#1}}
\newcommand{\md}[1]{\mathscr{#1}}
\newcommand{\intvar}{\bs{x}_{\textup{int}}}
\newcommand{\extvar}{\bs{x}_{\textup{ext}}}
\newcommand{\qderu}[2]{\mathbf{D}^{#1}(#2)}
\newcommand{\ud}{\mathrm{d}}
\def\l{\,|\,}
\def\cf{\boldsymbol{cf}}
\def\bx{\boldsymbol{x}}
\def\by{\boldsymbol{y}}
\def\ba{\boldsymbol{a}}
\def\bb{\boldsymbol{b}}
\def\totimes{\otimes}
\def\di{Q}
\newcommand{\cotimes}[1]{\,\widehat{\otimes}_{#1}\,}
\def\QQ{\mathds{Q}}
\def\krc{C}
\def\diffm{d}
\def\diffh{d_{\chi}}
\def\redh{\overline{H}}
\def\ZZ{\mathds{Z}}
\def\bs{\boldsymbol}
\def\Ztwo{\mathds{Z}_2}
\def\mdual{^{\vee}}
\def\KR{\operatorname{KR}}
\def\I{\!\operatorname{i}\!}
\def\E{\operatorname{e}\!}
\def\sln{\mathfrak{sl}(N)}
\def\nN{\mathds{N}}
\def\nZ{\mathds{Z}}
\def\nQ{\mathds{Q}}
\def\nR{\mathds{R}}
\def\nC{\mathds{C}}
\def\idem{\epsilon}
\def\Xcirc{%
\begin{tikzpicture}[inner sep=0mm]
\node (X) at (0,0) {$X$};
\node (0) at (0,0) [circle,inner sep=0.99pt, thin,draw=black,fill= white] {};
\end{tikzpicture}%
}
\def\Xbul{%
\begin{tikzpicture}[inner sep=0mm]
\node (X) at (0,0) {$X$};
\node (0) at (0,0) [circle,inner sep=0.99pt, thin,draw=black,fill= black] {};
\end{tikzpicture}%
}

\title{Rise of the Planet of the Coevaluations}
\author{Nils Carqueville}
\email{nils.carqueville@physik.uni-muenchen.de}
\address{Arnold Sommerfeld Center for Theoretical Physics, LMU M\"unchen \& Excellence Cluster Universe}

\author{Daniel Murfet}
\email{daniel.murfet@math.ucla.edu}
\address{Department of Mathematics, UCLA}

\classification{TODO}

\begin{abstract}
They take over. 
\end{abstract}

\maketitle

\section{Zorro moves}

bla



%\newcommand{\etalchar}[1]{$^{#1}$}
%\providecommand{\href}[2]{#2}
%\begin{thebibliography}{FYH{\etalchar{+}}85}
%
%\bibitem[AS]{as1105.5117}
%M.~Aganagic and S.~Shakirov, \emph{Knot {H}omology from {R}efined
%  {C}hern-{S}imons {T}heory},
%  \href{http://arxiv.org/abs/1105.5117}{[arXiv:1105.5117]}.
%
%\bibitem[BN]{bnKhovanov11crossings}
%D.~Bar-Natan, \emph{Khovanov {H}omology for {K}nots and {L}inks with up to 11
%  {C}rossings}, available at
%  \href{http://www.math.toronto.edu/drorbn/papers/KHTables/KHTables.pdf}{http:%
%//www.math.toronto.edu/drorbn/papers/KHTables/KHTables.pdf}.
%
%\bibitem[Bec]{b1105.0702}
%H.~Becker, \emph{Khovanov-Rozansky homology via Cohen-Macaulay approximations and Soergel bimodules},
%  \href{http://arxiv.org/abs/1105.0702}{[arXiv:1105.0702]}.
%
%\bibitem[BR07]{br0707.0922}
%I.~Brunner and D.~Roggenkamp, \emph{B-type defects in {L}andau-{G}inzburg
%  models}, JHEP \textbf{0708} (2007), 093,
%  \href{http://arxiv.org/abs/0707.0922}{[arXiv:0707.0922]}.
%
%\bibitem[CF94]{cf9405183}
%L.~Crane and I.~B. Frenkel, \emph{Four dimensional topological quantum field
%  theory, {H}opf categories, and the canonical bases}, J. Math. Phys.
%  \textbf{35} (1994), 5136--5154,
%  \href{http://arxiv.org/abs/hep-th/9405183}{[hep-th/9405183]}.
%
%\bibitem[CK08a]{ck0701194}
%S.~Cautis and J.~Kamnitzer, \emph{Knot homology via derived categories of
%  coherent sheaves {I}, {$sl(2)$} case}, Duke Math. J. \textbf{142} (2008),
%  511--588, \href{http://arxiv.org/abs/math/0701194}{[math.AG/0701194]}.
%
%\bibitem[CK08b]{ck0710.3216}
%S.~Cautis and J.~Kamnitzer, \emph{Knot homology via derived categories of coherent sheaves {II},
%  {$sl(m)$} case}, Invent. Math. \textbf{174} (2008), 165--232,
%  \href{http://arxiv.org/abs/math/0710.3216}{[math.AG/0710.3216]}.
%
%\bibitem[CM]{cmWebCompileCode}
%N.~Carqueville and D.~Murfet, \emph{Code to compute {K}hovanov-{R}ozansky
%  homology and defect fusion in {L}andau-{G}inzburg models},
%  \href{http://www.carqueville.net/nils/webCompilations}{http://www.carqueville.net/nils/webCompilations}.
%
%\bibitem[CR]{cr1006.5609}
%N.~Carqueville and I.~Runkel, \emph{Rigidity and defect actions in
%  Landau-Ginzburg models},
%  \href{http://arxiv.org/abs/1006.5609}{[arXiv:1006.5609]}.
%
%\bibitem[CR10]{cr0909.4381}
%N.~Carqueville and I.~Runkel, \emph{On the monoidal structure of matrix bi-factorisations}, J. Phys.
%  A: Math. Theor. \textbf{43} (2010), 275401,
%  \href{http://arxiv.org/abs/0909.4381}{[arXiv:0909.4381]}.
%
%\bibitem[Cra]{c0403266}
%M.~Crainic, \emph{On the perturbation lemma, and deformations},
%  \href{http://arxiv.org/abs/math/0403266}{[math.AT/0403266]}.
%
%\bibitem[DGR06]{dgr0505662}
%N.~M. Dunfield, S.~Gukov, and J.~Rasmussen, \emph{The {S}uperpolynomial for
%  {K}not {H}omologies}, Experimental Math. \textbf{15} (2006), 129--159,
%  \href{http://arxiv.org/abs/math/0505662}{[math.GT/0505662]}.
%
%\bibitem[DKR]{dkr1107.0495}
%A.~Davydov, L.~Kong, and I.~Runkel, \emph{Field theories with defects and the
%  centre functor}, \href{http://arxiv.org/abs/1107.0495}{[arXiv:1107.0495]}.
%
%\bibitem[DBM{\etalchar{+}}11]{dbmmss1106.4305}
%P.~Dunin-Barkowski, A.~Mironov, A.~Morozov, A.~Sleptsov, A.~Smirnov, \emph{Superpolynomials for toric knots from evolution induced by cut-and-join operators},
%  \href{http://arxiv.org/abs/1106.4305}{[arXiv:1106.4305]}. 
%  
%\bibitem[Dyc]{d0904.4713}
%T.~Dyckerhoff, \emph{Compact generators in categories of matrix factorizations},
%  Duke Math. J. \textbf{159} (2011), 223--274,
%  \href{http://arxiv.org/abs/0904.4713}{[arXiv:0904.4713]}.
%
%\bibitem[DM]{dm1102.2957}
%T.~Dyckerhoff and D.~Murfet, \emph{Pushing forward matrix factorisations},
%  \href{http://arxiv.org/abs/1102.2957}{[arXiv:1102.2957]}.
%
%\bibitem[FFRS07]{ffrs0607247}
%J.~Fr\"ohlich, J.~Fuchs, I.~Runkel, and C.~Schweigert, \emph{Duality and
%  defects in rational conformal field theory}, Nucl. Phys. B \textbf{763}
%  (2007), 354--430,
%  \href{http://arxiv.org/abs/hep-th/0607247}{[hep-th/0607247]}.
%
%\bibitem[FYH{\etalchar{+}}85]{Homfly}
%P.~Freyd, D.~Yetter, J.~Hoste, W.~B.~R. Lickorish, K.~Millett, and A.~Ocneanu,
%  \emph{A new polynomial invariant of knots and links}, Bull. Amer. Math. Soc.
%  \textbf{12} (1985), 239--246.
%
%\bibitem[GIKV10]{gikv0705.1368}
%S.~Gukov, A.~Iqbal, C.~Koz\c{c}az, and C.~Vafa, \emph{Link {H}omologies and the
%  {R}efined {T}opological {V}ertex}, Comm. Math. Phys. \textbf{298} (2010),
%  757--785, \href{http://arxiv.org/abs/0705.1368}{[arXiv:0705.1368]}.
%
%\bibitem[GSV05]{gsv0412243}
%S.~Gukov, A.~Schwarz, and C.~Vafa, \emph{Khovanov-{R}ozansky {H}omology and
%  {T}opological {S}trings}, Lett. Math. Phys. \textbf{74} (2005), 53--74,
%  \href{http://arxiv.org/abs/hep-th/0412243}{[hep-th/0412243]}.
%
%\bibitem[GV99]{gv9811131}
%R.~Gopakumar and C.~Vafa, \emph{On the {G}auge {T}heory/{G}eometry
%  {C}orrespondence}, Adv. Theor. Math. Phys. \textbf{3} (1999), 1415--1443,
%  \href{http://arxiv.org/abs/hep-th/9811131}{[hep-th/9811131]}.
%
%\bibitem[GW]{gw0512298}
%S.~Gukov and J.~Walcher, \emph{Matrix {F}actorizations and {K}auffman
%  {H}omology}, \href{http://arxiv.org/abs/hep-th/0512298}{[hep-th/0512298]}.
%
%\bibitem[Jae]{j1101.3302}
%T.~C. Jaeger, \emph{Khovanov-{R}ozansky {H}omology and {C}onway {M}utation},
%  \href{http://arxiv.org/abs/1101.3302}{[arXiv:1101.3302]}.
%
%\bibitem[Jon85]{JonesPolynomialPaper}
%V.~F.~R. Jones, \emph{A polynomial invariant for knots via von {N}eumann
%  algebras}, Bull. Amer. Math. Soc. \textbf{12} (1985), 103--111.
%
%\bibitem[Kap]{k1004.2307}
%A.~Kapustin, \emph{Topological {F}ield {T}heory, {H}igher {C}ategories, and
%  {T}heir {A}pplications},
%  \href{http://arxiv.org/abs/1004.2307}{[arXiv:1004.2307]}.
%
%\bibitem[Kaw96]{kawauchibook}
%A.~Kawauchi, \emph{A {S}urvey of {K}not {T}heory}, Birkh\"auser, 1996.
%
%\bibitem[Kho00]{k9908171}
%M.~Khovanov, \emph{A categorification of the {J}ones polynomial}, Duke Math. J.
%  \textbf{101} (2000), 359--426,
%  \href{http://arxiv.org/abs/math/9908171}{[math.QA/9908171]}.
%
%\bibitem[Kho07]{k0510265}
%M.~Khovanov, \emph{Triply-graded link homology and Hochschild homology of Soergel bimodules},
%  Int. Journal of Math. \textbf{18} (2007), 869--885,
%  \href{http://arxiv.org/abs/math/0510265}{[math.GT/0510265]}.
%
%\bibitem[KR07a]{kr0701333}
%M.~Khovanov and L.~Rozansky, \emph{Virtual crossings, convolutions and a
%  categorification of the {$\operatorname{SO}(2N)$} {K}auffman polynomial},
%  Journal of G\"okova Geometry Topology \textbf{1} (2007), 116--214,
%  \href{http://arxiv.org/abs/math/0701333}{[math.QA/0701333]}.
%
%\bibitem[KR07b]{kr0404189}
%M.~Khovanov and L.~Rozansky, \emph{Topological Landau-Ginzburg models on the world-sheet foam},
%  Adv. Theor. Math. Phys. \textbf{11} (2007), 233--259,
%  \href{http://arxiv.org/abs/hep-th/0404189}{[hep-th/0404189]}.
%
%\bibitem[KR08a]{kr0401268}
%M.~Khovanov and L.~Rozansky, \emph{Matrix factorizations and link homology}, Fund. Math.
%  \textbf{199} (2008), 1--91,
%  \href{http://arxiv.org/abs/math/0401268}{[math/0401268]}.
%
%\bibitem[KR08b]{kr0505056}
%M.~Khovanov and L.~Rozansky, \emph{Matrix factorizations and link homology {II}}, Geometry \&
%  Topology \textbf{12} (2008), 1387--1425,
%  \href{http://arxiv.org/abs/math/0505056}{[math.QA/0505056]}.
%
%\bibitem[Lam86]{LambekRingsModules}
%J.~Lambek, \emph{Lectures on rings and modules}, AMS Chelsea Publishing, 1986.
%
%\bibitem[LM]{Calinetal2}
%C.~I. Lazaroiu and D.~McNamee, unpublished.
%
%\bibitem[LMnV00]{lmv0010102}
%J.~M.~F. Labastida, M.~Mari\~{n}o, and C.~Vafa, \emph{Knot {I}nvariants and
%  {T}opological {S}trings}, JHEP \textbf{0011} (2000), 007,
%  \href{http://arxiv.org/abs/hep-th/0010102}{[hep-th/0010102]}.
%
%\bibitem[MSV09]{msv0708.2228}
%M.~Mackaay, M.~Sto\v{s}i\'{c}, and P.~Vaz, \emph{$\mathfrak{sl}(N)$ link homology ($N\geq 4$) using foams and the Kapustin-Li formula}, Geometry \& Topology \textbf{13} (2009), 1075--1128,
%  \href{http://arxiv.org/abs/0708.2228}{[arXiv:0708.2228]}.
%
%\bibitem[Man07]{m0601629}
%C.~Manolescu, \emph{Link homology theories from symplectic geometry}, Adv. in
%  Math. \textbf{211} (2007), 363--416,
%  \href{http://www.arxiv.org/abs/math.AG/0601629}{[math.SG/0601629]}.
%
%\bibitem[McN09]{McNameethesis}
%D.~McNamee, \emph{On the mathematical structure of topological defects in
%  {L}andau-{G}inzburg models}, MSc Thesis, Trinity College Dublin, 2009.
%
%\bibitem[Mn05]{marinoknotbook}
%M.~Mari\~{n}o, \emph{Chern-{S}imons {T}heory, {M}atrix {M}odels, and
%  {T}opological s{t}rings}, Oxford University Press, 2005.
%
%\bibitem[MOY98]{moy1998}
%H.~Murakami, T.~Ohtsuki, and S.~Yamada, \emph{Homfly polynomial via an
%  invariant of colored plane graphs}, Enseign. Math. \textbf{44} (1998),
%  325--360.
%
%\bibitem[MS]{ms0709.1971}
%V.~Manzorchuck and C.~Stroppel, \emph{A combinatorial approach to functorial
%  quantum {$\mathfrak{sl}_k$} knot invariants},
%  \href{http://arxiv.org/abs/0709.1971}{[arXiv:0709.1971]}.
%
%\bibitem[OV00]{ov9912123}
%H.~Ooguri and C.~Vafa, \emph{Knot {I}nvariants and {T}opological {S}trings},
%  Nucl. Phys. B \textbf{577} (2000), 419--438,
%  \href{http://arxiv.org/abs/hep-th/9912123}{[hep-th/9912123]}.
%
%\bibitem[PT87]{HomflyPT}
%J.~Przytycki and P.~Traczyk, \emph{Conway algebras and skein equivalence of
%  links}, Proc. Amer. Math. Soc. \textbf{100} (1987), 744--748.
%
%\bibitem[Ras]{r0607544}
%J.~Rasmussen, \emph{Some differentials on {K}hovanov-{R}ozansky homology},
%  \href{http://arxiv.org/abs/math/0607544}{[math.GT/0607544]}.
%
%\bibitem[Ras07]{r0508510}
%J.~Rasmussen, \emph{Khovanov-{R}ozansky homology of two-bridge knots and links},
%  Duke Math. J. \textbf{136} (2007), 551--583,
%  \href{http://arxiv.org/abs/math/0508510}{[math.GT/0508510]}.
%  
%\bibitem[Ric94]{rickard}
%J.~Rickard, \emph{Translation functors and equivalences of derived categories for blocks of algebraic groups}, in “Finite dimensional algebras and related topics”, Kluwer (1994), 255-–264.
%
%\bibitem[Rou06]{RouquierMexico}
%R.~Rouquier, \emph{Categorification of {$\mathfrak{sl}_{2}$} and braid groups},
%  Trends in representation theory of algebras and related topics (2006),
%  137--167.
%
%\bibitem[RT90]{RT1990}
%N.~Reshetikhin and V.~Turaev, \emph{Ribbon graphs and their invariants derived
%  from quantum groups}, Comm. Math. Phys. \textbf{127} (190), 1--26.
%
%\bibitem[RT91]{RT1991}
%N.~Reshetikhin and V.~Turaev, \emph{Invariants of 3-manifolds via link polynomials and quantum
%  groups}, Invent. Math. \textbf{103} (1991), 547--597.
%
%\bibitem[SS06]{ss0405089}
%P.~Seidel and I.~Smith, \emph{A link invariant from the symplectic geometry of
%  nilpotent slices}, Duke Math. J. \textbf{134} (2006), 453--514,
%  \href{http://arxiv.org/abs/math/0405089}{[math.SG/0405089]}.
%
%\bibitem[Str05]{sCatTLcTCpf}
%C.~Stroppel, \emph{Categorification of the {T}emperley-{L}ieb category,
%  tangles, and cobordisms via projective functors}, Duke Math. J. \textbf{126}
%  (2005), 547--596.
%
%\bibitem[Sus]{s0701045}
%J.~Sussan, \emph{Category {$\mathcal O$} and {$\mathfrak{sl}_k$} link
%  invariants}, \href{http://arxiv.org/abs/math/0701045}{[math.QA/0701045]}.
%
%\bibitem[Tur88]{t1988YB}
%V.~Turaev, \emph{The {Y}ang-{B}axter equation and invariants of links}, Invent.
%  Math. \textbf{92} (1988), 527--553.
%
%\bibitem[Tur10]{turaevbook}
%V.~Turaev, \emph{Quantum invariants of knots and 3-manifolds}, de Gruyter, 2010,
%  2nd edition.
%
%\bibitem[Weba]{w0610650}
%B.~Webster, \emph{Khovanov-Rozansky homology via a canopolis formalism},
%  \href{http://arxiv.org/abs/math/0610650}{[math.GT/0610650]}.
%
%\bibitem[Webb]{w1005.4559}
%B.~Webster, \emph{Knot invariants and higher representation theory {II}: the
%  categorification of quantum knot invariants},
%  \href{http://arxiv.org/abs/1005.4559}{[arXiv:1005.4559]}.
%
%\bibitem[Wit]{w1101.3216}
%E.~Witten, \emph{Fivebranes and {K}nots},
%  \href{http://arxiv.org/abs/1101.3216}{[arXiv:1101.3216]}.
%
%\bibitem[Wit89]{wittenjones}
%E.~Witten, \emph{Quantum field theory and the {J}ones polynomial}, Comm. Math.
%  Phys. \textbf{121} (1989), 351--399.
%
%\bibitem[Wit95]{w9207094}
%E.~Witten, \emph{Chern-{S}imons {G}auge {T}heory {A}s {A} {S}tring {T}heory},
%  Prog. Math. \textbf{133} (1995), 637--678,
%  \href{http://arxiv.org/abs/hep-th/9207094}{[hep-th/9207094]}.
%
%\bibitem[Wu]{w0907.0695}
%H.~Wu, \emph{A colored {$\mathfrak{sl}(N)$}-homology for links in {$S^3$}},
%  \href{http://arxiv.org/abs/0907.0695}{[arXiv:0907.0695]}.
%
%\bibitem[Wu08]{w0508064}
%H.~Wu, \emph{Braids, {T}ransversal links and the {K}hovanov-{R}ozansky {T}heory},
%  \href{http://arxiv.org/abs/math/0508064}{[math.GT/0508064]}, Trans. Amer. Math. Soc. \textbf{360}
%  (2008), 3365--3389.
%
%\bibitem[Yon]{y0906.0220}
%Y.~Yonezawa, \emph{Quantum {$(\mathfrak{sl}_n, \wedge V_n)$} link invariant and
%  matrix factorizations},
%  \href{http://arxiv.org/abs/0906.0220}{[arXiv:0906.0220]}.
%
%\end{thebibliography}

\end{document}
