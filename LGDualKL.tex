 % Authors:  Nils Carqueville, Daniel Murfet
 
\documentclass{compositio}
\usepackage{stmaryrd}
\usepackage{amsmath, amscd, amssymb, mathrsfs, accents, amsfonts}
\usepackage{url}
\usepackage[all]{xy}
\usepackage{longtable}
\usepackage{dsfont}

\SelectTips{cm}{}

\newtheorem{theorem}{Theorem}[section]
\newtheorem{proposition}[theorem]{Proposition}
\newtheorem{lemma}[theorem]{Lemma}
\newtheorem{corollary}[theorem]{Corollary}
\newtheorem*{theoremn}{Theorem}

\theoremstyle{definition}
\newtheorem{definition}[theorem]{Definition}
\newtheorem{example}[theorem]{Example}
\newtheorem{remark}[theorem]{Remark}
\newtheorem{s}[theorem]{}
\newtheorem*{setup}{Setup}
\newtheorem*{propositionn}{Proposition}

\numberwithin{equation}{section}

% Operators
\def\Res{\res\!}
\newcommand{\Ress}[1]{\res_{#1}\!}
\def\eval{\operatorname{ev}}
\def\coev{\operatorname{coev}}
\def\res{\operatorname{Res}}
\def\sg{\operatorname{sg}}
\def\Inj{\operatorname{Inj}}
\def\inc{\operatorname{inc}}
\def\Proj{\operatorname{Proj}}
\def\Coker{\operatorname{Coker}}
\def\Ker{\operatorname{Ker}}
\def\Im{\operatorname{Im}}
\def\free{\operatorname{free}}
\def\can{\operatorname{can}}
\def\ac{\operatorname{ac}}
\def\HH{\operatorname{HH}}
\def\K{\mathbf{K}}
\def\D{\mathbf{D}}
\def\N{\mathbf{N}}
\def\sgn{\operatorname{sgn}}
\def\Hom{\operatorname{Hom}}
\def\uHom{\underline{\Hom}}
\def\modd{\operatorname{mod}}
\def\Modd{\operatorname{Mod}}
\def\Grmodd{\operatorname{GrMod}}
\def\CM{\operatorname{CM}}
\def\Ker{\operatorname{Ker}}
\def\Spec{\operatorname{Spec}}
\def\straightK{\operatorname{K}}
\def\straightC{\operatorname{C}}
\def\holim{\operatorname{hocolim}}
\DeclareMathOperator{\Ext}{Ext}
\DeclareMathOperator{\coh}{coh}
\DeclareMathOperator{\serre}{S}
\DeclareMathOperator{\Flat}{Flat}
\DeclareMathOperator{\qc}{qc}
\DeclareMathOperator{\Perf}{Perf}
\DeclareMathOperator{\Map}{Map}
\DeclareMathOperator{\Qco}{Qco}
\DeclareMathOperator{\Tr}{Tr}
\DeclareMathOperator{\End}{End}
\DeclareMathOperator{\rank}{rank}
\DeclareMathOperator{\tot}{Tot}
\DeclareMathOperator{\skos}{K}
\DeclareMathOperator{\hht}{ht}
\DeclareMathOperator{\depth}{depth}
\DeclareMathOperator{\STr}{STr}
\DeclareMathOperator{\tr}{tr}
\DeclareMathOperator{\ch}{ch}
\DeclareMathOperator{\str}{str}
\DeclareMathOperator{\hmf}{hmf}
\DeclareMathOperator{\HMF}{HMF}
\DeclareMathOperator{\HF}{HF}
\DeclareMathOperator{\pr}{pr}
\DeclareMathOperator{\At}{At}
\DeclareMathOperator{\mff}{mf}
\DeclareMathOperator{\MF}{MF}
\DeclareMathOperator{\Sh}{Sh}

\begin{document}

% Commands
\def\Res{\res\!}
\newcommand{\cat}[1]{\mathcal{#1}}
\newcommand{\lto}{\longrightarrow}
\newcommand{\xlto}[1]{\stackrel{#1}\lto}
\newcommand{\mf}[1]{\mathfrak{#1}}
\newcommand{\md}[1]{\mathscr{#1}}
\newcommand{\intvar}{\bs{x}_{\textup{int}}}
\newcommand{\extvar}{\bs{x}_{\textup{ext}}}
\newcommand{\qderu}[2]{\mathbf{D}^{#1}(#2)}
\newcommand{\ud}{\mathrm{d}}
\def\l{\,|\,}
\def\cf{\boldsymbol{cf}}
\def\bx{\boldsymbol{x}}
\def\by{\boldsymbol{y}}
\def\ba{\boldsymbol{a}}
\def\bb{\boldsymbol{b}}
\def\totimes{\otimes}
\def\di{Q}
\newcommand{\cotimes}[1]{\,\widehat{\otimes}_{#1}\,}
\def\QQ{\mathds{Q}}
\def\krc{C}
\def\diffm{d}
\def\diffh{d_{\chi}}
\def\redh{\overline{H}}
\def\ZZ{\mathds{Z}}
\def\bs{\boldsymbol}
\def\Ztwo{\mathds{Z}_2}
\def\mdual{^{\vee}}
\def\KR{\operatorname{KR}}
\def\I{\!\operatorname{i}\!}
\def\E{\operatorname{e}\!}
\def\sln{\mathfrak{sl}(N)}
\def\nN{\mathds{N}}
\def\nZ{\mathds{Z}}
\def\nQ{\mathds{Q}}
\def\nR{\mathds{R}}
\def\nC{\mathds{C}}
\def\Bar{\mathds{B}}
\def\cBar{\widehat{\mathds{B}}}
\def\Ae{A^{\operatorname{e}}}

\title{General Kapustin-Li}

\maketitle

\section{Nondegenerate pairings}

Let $k$ be a commutative $\mathbb{Q}$-algebra and $R = k[x_1,\ldots,x_n]$. If $k$ is a field then we know from \cite{??,??} that for any polynomial $W$ with an isolated singularity, the Kapustin-Li formula gives a nondegenerate pairing on the morphism spaces of the triangulated category of matrix factorisations of $W$ over $R$. In this section we prove a relative version of the Kapustin-Li formula, which works even if the base $k$ is not a field; in particular, since the morphism spaces of the triangulated category need not be flat over $k$, we must work instead with dg-categories instead of triangulated categories.

Given $W \in R$ and two finite rank matrix factorisations $X,Y$ of $W$ let $\Hom(X,Y) = \Hom_R(X,Y)$ denote the $\mathbb{Z}_2$-graded Hom-complex. We can define a natural $k$-bilinear pairing
\[
\langle -, - \rangle_{\textup{KL}}: \Hom(X,Y) \otimes_k \Hom(Y,X) \lto k[n]
\]
by the following formula, where $\lambda_i = \partial_{x_i}(D_Y)$ and $\ud \underline{x} = \ud x_1 \cdots \ud x_n$
\[
\langle \varphi, \psi \rangle_{\textup{KL}} = \frac{1}{n!} \sum_{\sigma \in S_n} \sgn(\sigma) \Ress{R/k}\Bigg[ \frac{ \str( \varphi \circ \psi \circ \lambda_{\sigma(1)} \circ \cdots \circ \lambda_{\sigma(n)} )\, \ud \underline{x} }{ \partial_{x_1} W, \ldots, \partial_{x_n} W } \Bigg]\,.
\]
We say this pairing is \emph{homotopically nondegenerate} if the adjoint morphism of $\mathbb{Z}_2$-graded complexes
\begin{gather*}
\Hom(X,Y) \lto \Hom_k( \Hom(Y,X), k )[n]\,,\\
\varphi \mapsto \langle \varphi, - \rangle_{\textup{KL}}
\end{gather*}
is a homotopy equivalence over $k$. The main theorem of this section is that under some reasonable hypotheses on $R$ and $W$, the pairing $\langle -, - \rangle_{\textup{KL}}$ is homotopically nondegenerate. This nondegeneracy is used in Section \ref{??} to prove that specified evaluation and coevaluation morphisms give rise to adjoint $1$-morphisms in the bicategory of Landau-Ginzburg models over $k$.

\begin{theorem}\label{theorem:generalkl} Suppose that the following conditions are satisfied
\begin{itemize}
\item[(H1)] The partial derivatives $f_i = \partial_{x_i} W$ form a quasi-regular sequence $f = \{f_1,\ldots,f_n\}$ in $R$.
\item[(H2)] As a module over $k[f] = k[f_1,\ldots,f_n]$, $R$ admits a flat $k$-linear connection
\[
\nabla: R \lto R \otimes_{k[f]} \Omega^1_{k[f]/k}
\]
which is standard.
\item[(H3)] $\overline{R} = R/(f_1,\ldots,f_n)R$ is a finitely generated projective $k$-module.
\item[(H4)] The $R$-linear map
\begin{gather*}
\overline{R} \lto \Hom_k( \overline{R}, k)\,\\
r \mapsto \Ress{R/k}\Bigg[ \frac{r \cdot (-) \, \ud \underline{x}}{ f_1, \ldots, f_n} \Bigg]
\end{gather*}
is an isomorphism.
\end{itemize}
Then the pairing $\langle -, - \rangle_{\textup{KL}}$ is homotopically nondegenerate.
\end{theorem}

%\begin{remark} If $k$ is a field we need the Jacobi algebra $\overline{R}$ to be Frobenius over $k$, and the important part of the above assumptions is that this continues to be true; also everything would work equally well if $R = k\llbracket \bs{x} \rrbracket$ (TODO).
%\end{remark}

For the rest of this section, assume that (H1)-(H4) hold. The proof of the theorem will be given in Section \ref{??} after we have prepared some background material. The key point is that up to homotopy equivalence over $k$, the complex $\Hom(X,Y)[n]$ can be recovered from the mapping complex between the quotients $\overline{X} = X \otimes_R \overline{R}$ and $\overline{Y} = Y \otimes_R \overline{R}$, namely
\[
\Hom(\overline{X},\overline{Y}) = \Hom(X,Y) \otimes_R \overline{R}
\]
as the splitting of a certain idempotent $e$ on this complex. Similarly $\Hom(Y,X)$ splits an idempotent $e'$ on the complex $\Hom(\overline{Y}, \overline{X})$ and we prove the theorem by showing that there is an isomorphism
\begin{equation}\label{eq:frobnondeg}
\Hom(\overline{X}, \overline{Y}) \cong \Hom_k( \Hom(\overline{Y}, \overline{X}), k)
\end{equation}
of complexes, under which the idempotent $e$ is identified up to homotopy with $(e')^*$. From this the theorem follows immediately. The existence of the isomorphism (\ref{eq:frobnondeg}) is an easy consequence of the fact that $\overline{R}$ is Frobenius over $k$. The hard work lies in showing that $e$ corresponds under this isomorphism to the dual of $e'$, and for this we need to carefully study adjointness between operators on the dg-category formed by the complexes $\Hom(\overline{X}, \overline{Y})$.

\subsection{Adjoint operators on dg-categories}

In this section $\otimes$ denotes $\otimes_k$. Let $\cat{C}$ be a $\mathbb{Z}_2$-graded dg-category over $k$ equipped with the data of a $k$-linear morphism of complexes for each pair of objects $X,Y$ in $\cat{C}$
\[
c_{XY}: \cat{C}(X,Y) \otimes \cat{C}(Y,X) \lto k\,.
\]
When it is convenient we write $\langle \alpha, \beta \rangle$ for $c_{XY}(\alpha \otimes \beta)$. Throughout the differentials on the $\cat{C}(X,Y)$ and their tensor products are denoted $D$, with additional subscripts if necessary.

\begin{definition}\label{defn:nondegpair} We say that the pairing $\{ c_{XY} \}_{X,Y \in \cat{C}}$ is
\begin{itemize}
\item[(i)] \emph{Cyclic} if for all $X,Y$ the diagram
\[
\xymatrix{
\cat{C}(X,Y) \otimes \cat{C}(Y,X) \ar[dr]_{c_{XY}}\ar[rr]^\tau & & \cat{C}(Y,X) \otimes \cat{C}(X,Y) \ar[dl]^{c_{YX}}\\
& k
}
\]
commutes, where $\tau$ is the graded twist $\tau( \varphi \otimes \psi) = (-1)^{|\varphi||\psi|} \psi \circ \varphi$. 
\item[(ii)] \emph{Nondegenerate} if for all $X,Y$ the morphism
\begin{gather*}
\zeta_{XY}: \cat{C}(X,Y) \lto \Hom_k( \cat{C}(Y,X), k)\,,\\
\varphi \mapsto c_{XY}( \varphi \otimes - )
\end{gather*}
is an isomorphism of complexes.
\end{itemize}
\end{definition}

%\begin{remark} \textbf{todo}. Probably this all works for a weaker form of nondegeneracy, where $\zeta$ is just a homotopy equivalence, but we don't need this. In what generality should we state things, then? Maybe we also only need cyclicity up to homotopy.
%\end{remark}

From now on we assume $\cat{C}$ is equipped with a cyclic nondegenerate pairing. In this section an \emph{operator} is a closed homogeneous $k$-linear operator on some mapping complex $\cat{C}(X,Y)$ in $\cat{C}$, that is, a closed homogeneous element of the complex $\Hom_k( \cat{C}(X,Y), \cat{C}(X,Y) )$. We are interested in linear operators on the $\cat{C}(X,Y)$ and adjunctions between them, with respect to the pairing.

Recall that if $\Psi$ is homogeneous then $1 \otimes \Psi$ acts on tensors with Koszul signs.

\begin{definition}\label{defn:adjointop} An operator $\Phi$ on $\cat{C}(X,Y)$ is \emph{adjoint} to an operator $\Psi$ on $\cat{C}(Y,X)$ if both $\Phi$ and $\Psi$ are homogeneous of the same degree and the diagram
\[
\xymatrix@C+1pc@R+0.5pc{
\cat{C}(X,Y) \otimes \cat{C}(Y,X) \ar[d]_{\Phi \otimes 1} \ar[r]^{1 \otimes \Psi} & \cat{C}(X,Y) \otimes \cat{C}(Y,X) \ar[d]^{c_{XY}}\\
\cat{C}(X,Y) \otimes \cat{C}(Y,X) \ar[r]_-{c_{XY}} & k
}
\]
commutes up to homotopy. Equivalently, there is a $k$-linear degree $|\Phi|+1$ morphism
\[
\mu: \cat{C}(X,Y) \otimes \cat{C}(Y,X) \lto k
\]
with the property that $[D, \mu] = c_{XY} \circ (1 \otimes \Psi) - c_{XY} \circ ( \Phi \otimes 1)$. Evaluated on homogeneous morphisms $\alpha, \beta$ this identity reads
\begin{equation}\label{eq:adjointopeq}
(-1)^{|\Phi|} \mu D(\alpha \otimes \beta) = (-1)^{|\alpha||\Psi|}\langle \alpha, \Psi(\beta) \rangle - \langle \Phi(\alpha) , \beta \rangle.
\end{equation}
\end{definition}

We will show in a moment that adjointness is symmetric in $\Phi, \Psi$ so there is no need to distinguish between left and right adjoints. If $\Psi$ is a homogeneous operator on $\cat{C}(Y,X)$ then $\Psi^*$ is the operator on $\Hom_k(\cat{C}(Y,X),k)$ defined by $\Psi^*(f) = (-1)^{|f||\Psi|} f \circ \Psi$. It is easy to see that

\begin{lemma} An operator $\Phi$ is adjoint to $\Psi$ if and only if the diagram
\begin{equation}\label{eq:lemadjointdia}
\xymatrix@C+2pc{
\cat{C}(X,Y) \ar[d]_\Phi \ar[r]^-{\zeta} & \Hom_k( \cat{C}(Y,X), k ) \ar[d]^{\Psi^*}\\
\cat{C}(X,Y) \ar[r]_-{\zeta} & \Hom_k( \cat{C}(Y,X), k)
}
\end{equation}
commutes up to homotopy.
\end{lemma}

\begin{lemma} An operator $\Phi$ is adjoint to $\Psi$ if and only if $\Psi$ is adjoint to $\Phi$.
\end{lemma}
\begin{proof}
This follows from naturality of the graded twist $\tau$ and the cyclicity axiom for $c_{XY}$.
\end{proof}

\begin{lemma} Any operator $\Phi$ admits an adjoint, which is unique up to homotopy. 
\end{lemma}
\begin{proof}
If $\Phi$ is adjoint to $\Psi$ then from the previous lemma and (\ref{eq:lemadjointdia}) we deduce that $\Psi$ is homotopic to $\zeta^{-1} \circ \Phi^* \circ \zeta$ and from this both claims are clear.
\end{proof}

\begin{definition} Given an operator $\Phi$ we denote by $\Phi^{\dagger}$ the adjoint of $\Phi$.
\end{definition}

Since the adjoint is only well-defined up to homotopy it is implicit that in any identities involving the dagger notation we are working in the homotopy category of $\mathbb{Z}_2$-graded complexes. The following basic properties of adjoints are easily checked.

\begin{lemma} 
\begin{itemize}
\item[(i)] Let $\Phi_1, \Phi_2$ be operators on $\cat{C}(X,Y)$ and $\cat{C}(Y,Z)$ respectively. Then
\[
(\Phi_2 \circ \Phi_1)^{\dagger} = (-1)^{|\Phi_1||\Phi_2|} \Phi_1^{\dagger} \circ \Phi_2^{\dagger}\,.
\]
\item[(ii)] If $\Phi_1, \Phi_2$ are operators on $\cat{C}(X,Y)$ of the same degree, then $(\Phi_1 + \Phi_2)^{\dagger} = \Phi_1^{\dagger} + \Phi_2^{\dagger}$.
\end{itemize}
\end{lemma}

So much for the general theory; from now on $\cat{C}$ denotes the dg-category whose objects are finite rank matrix factorisations of $W$ and whose mapping complexes are given by the quotients
\[
\cat{C}(X,Y) = \Hom(\overline{X}, \overline{Y}) = \Hom(X,Y) \otimes_R \overline{R}\,.
\]
While the ordinary dg-category of matrix factorisations only admits a pairing which is homotopically nondegenerate, this (linear) quotient $\cat{C}$ admits a cyclic nondegenerate pairing in the stronger sense explained above. To define it, let
\[
\langle - \rangle: R \lto k
\]
denote the $k$-linear map
\[
\langle r \rangle = \Ress{R/k}\Bigg[ \frac{r \cdot \ud \underline{x}}{ f_1, \ldots, f_n} \Bigg]\,.
\]
This annihilates the ideal $(f_1,\ldots,f_n)R$ and therefore factors via a $k$-linear map $\overline{R} \lto k$.

\begin{proposition}\label{prop:barnondeg} The pairing on $\cat{C}$ defined by
\begin{gather*}
\langle -, - \rangle: \Hom(\overline{X}, \overline{Y}) \otimes \Hom(\overline{Y}, \overline{X}) \lto k\,,\\
\langle \varphi, \psi \rangle = \langle \str( \varphi \circ \psi ) \rangle
\end{gather*}
is cyclic and nondegenerate in the sense of Definition \ref{defn:nondegpair}.
\end{proposition}
\begin{proof}
The pairing factors as
\begin{equation}\label{eq:barnondeg}
\xymatrix@C+1pc{
\Hom(\overline{X}, \overline{Y}) \otimes \Hom(\overline{Y}, \overline{X}) \ar[r]^-{-\circ-} & \Hom(\overline{Y}, \overline{Y}) \ar[r]^-{\str} & \overline{R} \ar[r]^-{\langle - \rangle} & k
}
\end{equation}
so it is clear that it is a closed $k$-linear map, and moreover cyclicity follows from the cyclicity of the supertrace. Nondegeneracy follows from hypothesis (H4) and the following calculation, in which the first step is adjoint to the composite of the first two maps in (\ref{eq:barnondeg})
\begin{align*}
\Hom_R(\overline{X}, \overline{Y}) &\cong \Hom_R( \Hom(\overline{Y}, \overline{X}), \overline{R} )\,\\
&\cong \Hom_R( \Hom(\overline{Y}, \overline{X}), \Hom_k( \overline{R}, k) )\\
&\cong \Hom_k( \Hom(\overline{Y}, \overline{X}), k )\,.
\end{align*}
\end{proof}

We have in mind two special classes of operators on the dg-category $\cat{C}$. The first arise because the partial derivatives $f_i$ act as zero on the cohomology of $\cat{C}$ but via nonzero maps on the dg-level. In what follows let $X,Y$ denote finite rank matrix factorisations of $W$.

\begin{definition} Given a null-homotopy $\lambda_i$ for the action of $f_i$ on $Y$, that is, a map $\lambda_i \in \Hom(Y,Y)$ of degree one with $[D, \lambda_i] = f_i \cdot 1_Y$, we define the odd operator $\lambda_i^\bullet$ on $\Hom(\overline{X},\overline{Y})$ by
\[
\lambda_i^\bullet(\varphi) = \lambda_i \circ \varphi\,
\]
and the odd operator ${\lambda_i}_\bullet$ on $\Hom(\overline{Y}, \overline{X})$ by
\[
{\lambda_i}_\bullet(\varphi) = (-1)^{|\varphi|} \varphi \circ \lambda_i \,.
\]
\end{definition}

Observe that composition with $\lambda_i$ is not a closed map on $\Hom(X,Y)$ but is closed as an operator on $\Hom(\overline{X}, \overline{Y})$. These operators give the simplest example of an adjoint pair.

\begin{lemma} The operator $\lambda_i^\bullet$ is adjoint to ${\lambda_i}_\bullet$.
\end{lemma}
\begin{proof}
The identity (\ref{eq:adjointopeq}) holds with $\mu = 0$, since
\begin{align*}
\langle {\lambda_i}_\bullet(\alpha), \beta \rangle &= \langle \str( {\lambda_i}_\bullet(\alpha) \circ \beta ) \rangle\\
&= (-1)^{|\varphi|} \langle \str( \alpha \circ \lambda_i \circ \beta ) \rangle\\
&= (-1)^{|\varphi|} \langle \alpha, \lambda_i^\bullet(\beta) \rangle\,.
\end{align*}
\end{proof}

The second class of operators are the components of the Atiyah class. Recall that by hypothesis (H2) the ring $R$ admits a flat $k$-linear connection $\nabla$ as a $k[f]$-module. The $n$ components of this connection define $k$-linear operators $\partial_{f_i} = (\ud f_i)^* \circ \nabla$ on $R$ with the property that $[\partial_{f_i}, f_j] = \delta_{ij}$.

Any free $R$-module admits a $k$-linear connection over $k[f]$. For convenience choose homogeneous $R$-bases $\{ e_i \}_{i \in I}$ for $X$ and $\{ e_j \}_{j \in J}$ for $Y$ respectively. Then the maps $e_{ji} = e_j \circ e_i^*$ form an $R$-basis for $\Hom(X,Y)$ and the induced $k$-linear connection over $k[f]$ is defined by
\begin{gather*}
\nabla = \nabla_{XY}: \Hom(X,Y) \lto \Hom(X,Y) \otimes_{k[f]} \Omega^1_{k[f]/k}\,,\\
\nabla_{XY}( r e_{ji} ) = e_{ji} \otimes \nabla(r)\,.
\end{gather*}
This connection connection has components $\partial_{f_i}$ defined by $\partial_{f_i}( r e_{ji} ) = \partial_{f_i}(r) \cdot e_{ji}$ which are $k$-linear operators on $\Hom(X,Y)$. The \emph{Atiyah class} of $\Hom(X,Y)$ is the commutator
\[
\At = \At_{XY} = [D, \nabla] = D \circ \nabla - \nabla \circ D: \Hom(X,Y) \lto \Hom(X,Y) \otimes_{k[f]} \Omega^1_{k[f]/k}\,.
\]
The Atiyah class is $k[f]$-linear and is easily seen to be a morphism of complexes from $\Hom(X,Y)$ to the shift $\Hom(X,Y) \otimes_{k[f]} \Omega^1_{k[f]/k}[1]$. Moreover the homotopy class of this morphism is independent of the choice of connection, and in this sense choosing a basis for $X,Y$ is harmless. In terms of the components this means


\begin{definition} For $1 \le i \le n$ the components $\At_i = [D, \partial_{f_i}]$ of the Atiyah class define $k$-linear closed operators of degree one on $\Hom(\overline{X}, \overline{Y})$.
\end{definition}

For each pair of objects $X,Y$ in $\cat{C}$ we have defined a family of operators $\At_i$ on $\cat{C}(X,Y)$, defined using a choice of basis but independent of this choice up to homotopy. Conceptually, duality in the dg-category of matrix factorisations rests on the fact that these operators are anti-self-adjoint. To prove this we need the following Leibniz rule for Atiyah classes.

\begin{lemma}\label{lemma:leibnizatiyah} Leibniz rule.
\end{lemma}
\begin{proof}
Cite Ragnar's paper for basic facts, apply to composition map.
\end{proof}

\begin{lemma}\label{lemma:stratiyahzero} $\langle \str( \At_i( - ) ) \rangle = 0$.
\end{lemma}

\begin{proposition} The operator $\At_i$ on $\Hom(\overline{X}, \overline{Y})$ is adjoint to $-\At_i$ on $\Hom(\overline{Y}, \overline{X})$.
\end{proposition}
\begin{proof}
By Lemma \ref{lemma:leibnizatiyah} we have for $\alpha, \beta$ homogeneous composable maps the identity
\[
\At_i( \alpha \circ \beta ) = \At_i(\alpha) \circ \beta + (-1)^{|\alpha|} \alpha \circ \At_i(\beta) + [g,D]( \alpha \otimes \beta)\,.
\]
If we apply $\langle \str( - ) \rangle$ to both sides and use Lemma \ref{lemma:stratiyahzero} this yields
\begin{align*}
\langle \At_i(\alpha), \beta \rangle = -(-1)^{|\alpha|} \langle \alpha,  \At_i(\beta) \rangle - \langle \str( gD(\alpha \otimes \beta) ) \rangle\,.
\end{align*}
So $\mu( \alpha \otimes \beta ) = \langle \str( g( \alpha \otimes \beta ) ) \rangle$ is a homotopy expressing $\At_i$ as adjoint to $-\At_i$.
\end{proof}

\subsection{Idempotents}

In the previous section we constructed operators $\lambda_i^\bullet, {\lambda_i}_\bullet$ and $\At_i$ on the dg-category $\cat{C}$. Now we use these operators to define idempotent endomorphisms of the complex
\[
\cat{C}(X,Y) = \Hom(\overline{X}, \overline{Y})
\]
which split in the homotopy category of $\mathbb{Z}_2$-graded $k$-complexes to give $\Hom(X,Y)$. In this way the dg-category of matrix factorisations can be recovered from the quotient $\cat{C}$ and the nondegenerate pairing defined above induces the homotopically nondegenerate pairing $\langle -, - \rangle_{\textup{KL}}$.

The main result of \cite{??} is that if $V$ is the free $k$-module on the basis $\theta_1,\ldots,\theta_n$ then there is a homotopy equivalence
\[
\Hom(\overline{X}, \overline{Y}) \cong \Hom(X,Y) \otimes_k \bigwedge V\,.
\]
There are consequently $2^n$ ways to embed $\Hom(X,Y)$ in the homotopy category of $k$-complexes as a direct summand in $\Hom(\overline{X}, \overline{Y})$. The ``top degree'' embedding, corresponding to the form $\theta_1 \cdots \theta_n$, is determined by the following idempotent endomorphism of $\Hom(\overline{X}, \overline{Y})$
\[
e = \frac{1}{(n!)^2} (-1)^{\binom{n}{2}}\sum_{\sigma,\tau \in S_n} \textup{sgn}(\sigma\tau) \cdot \lambda_{\sigma(1)}^\bullet \cdots \lambda_{\sigma(n)}^\bullet \At_{\tau(1)} \cdots \At_{\tau(n)}\,.
\]
The embedding corresponding to the $0$-form $1$ in $\bigwedge V$ was not discussed in \cite{??} but we give the details in Appendix \ref{??}. The upshot is that this embedding is determined by the idempotent
\[
e' = \frac{1}{(n!)^2} (-1)^{\binom{n+1}{2}}\sum_{\sigma, \tau \in S_n} \textup{sgn}(\sigma\tau) \cdot \At_{\tau(1)} \cdots \At_{\tau(n)} {\lambda_{\sigma(1)}}_\bullet \cdots {\lambda_{\sigma(n)}}_\bullet\,.
\]
To be precise, there is a diagram of degree zero $k$-linear morphisms of complexes
\[
\xymatrix@C+2pc{
\Hom(\overline{X}, \overline{Y}) \ar@<-0.5ex>[r]_-{\psi} & \Hom(X,Y)[n] \ar@<-0.5ex>[l]_-{\vartheta}
}
\]
with $\psi \circ \vartheta = 1$ and $\vartheta \circ \psi = e$ (equalities meaning equal up to $k$-linear homotopy) and a diagram
\[
\xymatrix@C+2pc{
\Hom(\overline{Y}, \overline{X}) \ar@<-0.5ex>[r]_-{\psi'} & \Hom(Y,X) \ar@<-0.5ex>[l]_-{\vartheta'}
}
\]
with $\psi' \circ \vartheta' = 1$ and $\vartheta' \circ \psi' = e'$. The precise form of $\psi, \psi'$ is not important for us, but we will need to know that $\vartheta'$ is simply the quotient map, and
\[
\vartheta = \frac{1}{n!} \sum_{\sigma \in S_n} \textup{sgn}(\sigma) \lambda_{\sigma(1)}^\bullet \cdots \lambda_{\sigma(n)}^{\bullet}\,.
\]

\begin{proposition}\label{prop:eadjointeprime} The idempotent $e$ is adjoint to $(-1)^n e'$. Equivalently, the diagram
\begin{equation}\label{eq:eadjointeprime}
\xymatrix@C+2pc{
\Hom(\overline{X}, \overline{Y}) \ar[d]_-{e} \ar[r]^-{\zeta}_-{\cong} & \Hom_k(\Hom(\overline{Y},\overline{X}),k) \ar[d]^-{(-1)^n (e')^*}\\
\Hom(\overline{X}, \overline{Y}) \ar[r]_-{\zeta}^-{\cong} & \Hom_k( \Hom(\overline{Y}, \overline{X}), k )
}
\end{equation}
commutes up to homotopy.
\end{proposition}
\begin{proof} 
Observe that by the (anti-)self-adjointness established in the previous section
\begin{align*}
\left( \lambda_{\sigma(1)}^\bullet \cdots \lambda_{\sigma(n)}^\bullet \At_{\tau(1)} \cdots \At_{\tau(n)} \right)^{\dagger} &= (-1)^n \At^\dagger_{\tau(n)} \cdots \At^\dagger_{\tau(1)} (\lambda_{\sigma(n)}^\bullet)^\dagger \cdots (\lambda_{\sigma(1)}^\bullet)^{\dagger}\\
&= \At_{\tau(n)} \cdots \At_{\tau(1)} {\lambda_{\sigma(n)}}_{\bullet} \cdots {\lambda_{\sigma(1)}}_{\bullet}\,,
\end{align*}
from which it is immediate that $e^{\dagger} = (-1)^n e'$.
\end{proof}

Commutativity of (\ref{eq:eadjointeprime}) means that $e$ and $(-1)^n (e')^*$ may be viewed as homotopic idempotents on the same complex $\Hom(\overline{X}, \overline{Y})$. It follows that the splittings of these idempotents are homotopy equivalent, and this yields the desired homotopic nondegeneracy of the Kapustin-Li pairing.

\begin{proof}[Proof of Theorem \ref{theorem:generalkl}] Consider the diagram
\[
\xymatrix@C+2pc{
\Hom(\overline{X},\overline{Y}) \ar[d]_-{\zeta}^-{\cong} \ar@<-0.5ex>[r]_-{\psi} & \Hom(X,Y)[n] \ar@<-0.5ex>[l]_-{\vartheta} \ar@{.>}[d]^{\zeta_{\textup{KL}}}\\
\Hom_k(\Hom(\overline{Y}, \overline{X},k) \ar@<-0.5ex>[r]_-{(\vartheta')^*} & \Hom_k( \Hom(Y,X), k) \ar@<-0.5ex>[l]_-{(\psi')^*}
}
\]
where $\zeta_{KL} = (\vartheta')^* \circ \zeta \circ \vartheta$. It is immediate from commutativity of (\ref{eq:eadjointeprime}) up to homotopy that $\zeta_{\textup{KL}}$ is a homotopy equivalence with inverse $(-1)^n \psi \circ \zeta^{-1} \circ (\psi')^*$. To prove the theorem it only remains to observe that $\zeta_{\textup{KL}}(\alpha)$ is the functional $\langle \alpha, - \rangle_{\textup{KL}}$. But
\begin{align*}
\zeta_{\textup{KL}}(\alpha) &= (\vartheta')^* \zeta \vartheta( \alpha )\\
&= (\vartheta')^* \zeta\left( \frac{1}{n!} \sum_{\sigma \in S_n} \textup{sgn}(\sigma) \lambda_{\sigma(1)} \circ \cdots \circ \lambda_{\sigma(n)} \circ \alpha \right)\\
&= \frac{1}{n!} (-1)^n \sum_{\sigma \in S_n} \textup{sgn}(\sigma) \langle \str( \lambda_{\sigma(1)} \circ \cdots \circ \lambda_{\sigma(n)} \circ \alpha \circ (-) ) \rangle\\
&= \frac{1}{n!} \sum_{\sigma \in S_n} \textup{sgn}(\sigma) \langle \str(  \alpha \circ (-) \circ \lambda_{\sigma(1)} \circ \cdots \circ \lambda_{\sigma(n)} ) \rangle\\
&= \langle \alpha, - \rangle_{\textup{KL}}
\end{align*}
which completes the proof.
\end{proof}

\begin{remark} The expression for $e$ is really straight from the old paper, but $e'$ needs some serious effort. This should go in an Appendix (maybe $3$ pages).
\end{remark}

\newpage

\section{Questions}

\begin{itemize}
\item How to write dg-categories?
\end{itemize}

\newcommand{\etalchar}[1]{$^{#1}$}
\providecommand{\href}[2]{#2}
\begin{thebibliography}{FYH{\etalchar{+}}85}

%\bibitem[AS]{as1105.5117}
%M.~Aganagic and S.~Shakirov, \emph{Knot {H}omology from {R}efined
%  {C}hern-{S}imons {T}heory},
%  \href{http://arxiv.org/abs/1105.5117}{[arXiv:1105.5117]}.
%
%\bibitem[BN]{bnKhovanov11crossings}
%D.~Bar-Natan, \emph{Khovanov {H}omology for {K}nots and {L}inks with up to 11
%  {C}rossings}, available at
%  \href{http://www.math.toronto.edu/drorbn/papers/KHTables/KHTables.pdf}{http:%
%//www.math.toronto.edu/drorbn/papers/KHTables/KHTables.pdf}.
%
%\bibitem[Bec]{b1105.0702}
%H.~Becker, \emph{Khovanov-Rozansky homology via Cohen-Macaulay approximations and Soergel bimodules},
%  \href{http://arxiv.org/abs/1105.0702}{[arXiv:1105.0702]}.
%
%\bibitem[BR07]{br0707.0922}
%I.~Brunner and D.~Roggenkamp, \emph{B-type defects in {L}andau-{G}inzburg
%  models}, JHEP \textbf{0708} (2007), 093,
%  \href{http://arxiv.org/abs/0707.0922}{[arXiv:0707.0922]}.
%
%\bibitem[CF94]{cf9405183}
%L.~Crane and I.~B. Frenkel, \emph{Four dimensional topological quantum field
%  theory, {H}opf categories, and the canonical bases}, J. Math. Phys.
%  \textbf{35} (1994), 5136--5154,
%  \href{http://arxiv.org/abs/hep-th/9405183}{[hep-th/9405183]}.
%
%\bibitem[CK08a]{ck0701194}
%S.~Cautis and J.~Kamnitzer, \emph{Knot homology via derived categories of
%  coherent sheaves {I}, {$sl(2)$} case}, Duke Math. J. \textbf{142} (2008),
%  511--588, \href{http://arxiv.org/abs/math/0701194}{[math.AG/0701194]}.
%
%\bibitem[CK08b]{ck0710.3216}
%S.~Cautis and J.~Kamnitzer, \emph{Knot homology via derived categories of coherent sheaves {II},
%  {$sl(m)$} case}, Invent. Math. \textbf{174} (2008), 165--232,
%  \href{http://arxiv.org/abs/math/0710.3216}{[math.AG/0710.3216]}.
%
%\bibitem[CM]{cmWebCompileCode}
%N.~Carqueville and D.~Murfet, \emph{Code to compute {K}hovanov-{R}ozansky
%  homology and defect fusion in {L}andau-{G}inzburg models},
%  \href{http://www.carqueville.net/nils/webCompilations}{http://www.carqueville.net/nils/webCompilations}.

\bibitem[CR]{cr1006.5609}
N.~Carqueville and I.~Runkel, \emph{Rigidity and defect actions in
  Landau-Ginzburg models}, Comm. Math. Phys. \textbf{310} (2012) 135--179, 
  \href{http://arxiv.org/abs/1006.5609}{[arXiv:1006.5609]}.

%\bibitem[CR10]{cr0909.4381}
%N.~Carqueville and I.~Runkel, \emph{On the monoidal structure of matrix bi-factorisations}, J. Phys.
%  A: Math. Theor. \textbf{43} (2010), 275401,
%  \href{http://arxiv.org/abs/0909.4381}{[arXiv:0909.4381]}.
%
%\bibitem[Cra]{c0403266}
%M.~Crainic, \emph{On the perturbation lemma, and deformations},
%  \href{http://arxiv.org/abs/math/0403266}{[math.AT/0403266]}.
%
%\bibitem[DGR06]{dgr0505662}
%N.~M. Dunfield, S.~Gukov, and J.~Rasmussen, \emph{The {S}uperpolynomial for
%  {K}not {H}omologies}, Experimental Math. \textbf{15} (2006), 129--159,
%  \href{http://arxiv.org/abs/math/0505662}{[math.GT/0505662]}.
%
%\bibitem[DKR]{dkr1107.0495}
%A.~Davydov, L.~Kong, and I.~Runkel, \emph{Field theories with defects and the
%  centre functor}, \href{http://arxiv.org/abs/1107.0495}{[arXiv:1107.0495]}.
%
%\bibitem[DBM{\etalchar{+}}11]{dbmmss1106.4305}
%P.~Dunin-Barkowski, A.~Mironov, A.~Morozov, A.~Sleptsov, A.~Smirnov, \emph{Superpolynomials for toric knots from evolution induced by cut-and-join operators},
%  \href{http://arxiv.org/abs/1106.4305}{[arXiv:1106.4305]}. 
%  
\bibitem[Dyc]{d0904.4713}
T.~Dyckerhoff, \emph{Compact generators in categories of matrix factorizations},
  Duke Math. J. \textbf{159} (2011), 223--274,
  \href{http://arxiv.org/abs/0904.4713}{[arXiv:0904.4713]}.

%\bibitem[DM]{dm1102.2957}
%T.~Dyckerhoff and D.~Murfet, \emph{Pushing forward matrix factorisations},
%  \href{http://arxiv.org/abs/1102.2957}{[arXiv:1102.2957]}.
%
%\bibitem[FFRS07]{ffrs0607247}
%J.~Fr\"ohlich, J.~Fuchs, I.~Runkel, and C.~Schweigert, \emph{Duality and
%  defects in rational conformal field theory}, Nucl. Phys. B \textbf{763}
%  (2007), 354--430,
%  \href{http://arxiv.org/abs/hep-th/0607247}{[hep-th/0607247]}.
%
%\bibitem[FYH{\etalchar{+}}85]{Homfly}
%P.~Freyd, D.~Yetter, J.~Hoste, W.~B.~R. Lickorish, K.~Millett, and A.~Ocneanu,
%  \emph{A new polynomial invariant of knots and links}, Bull. Amer. Math. Soc.
%  \textbf{12} (1985), 239--246.
%
%\bibitem[GIKV10]{gikv0705.1368}
%S.~Gukov, A.~Iqbal, C.~Koz\c{c}az, and C.~Vafa, \emph{Link {H}omologies and the
%  {R}efined {T}opological {V}ertex}, Comm. Math. Phys. \textbf{298} (2010),
%  757--785, \href{http://arxiv.org/abs/0705.1368}{[arXiv:0705.1368]}.
%
%\bibitem[GSV05]{gsv0412243}
%S.~Gukov, A.~Schwarz, and C.~Vafa, \emph{Khovanov-{R}ozansky {H}omology and
%  {T}opological {S}trings}, Lett. Math. Phys. \textbf{74} (2005), 53--74,
%  \href{http://arxiv.org/abs/hep-th/0412243}{[hep-th/0412243]}.
%
%\bibitem[GV99]{gv9811131}
%R.~Gopakumar and C.~Vafa, \emph{On the {G}auge {T}heory/{G}eometry
%  {C}orrespondence}, Adv. Theor. Math. Phys. \textbf{3} (1999), 1415--1443,
%  \href{http://arxiv.org/abs/hep-th/9811131}{[hep-th/9811131]}.
%
%\bibitem[GW]{gw0512298}
%S.~Gukov and J.~Walcher, \emph{Matrix {F}actorizations and {K}auffman
%  {H}omology}, \href{http://arxiv.org/abs/hep-th/0512298}{[hep-th/0512298]}.
%
%\bibitem[Jae]{j1101.3302}
%T.~C. Jaeger, \emph{Khovanov-{R}ozansky {H}omology and {C}onway {M}utation},
%  \href{http://arxiv.org/abs/1101.3302}{[arXiv:1101.3302]}.
%
%\bibitem[Jon85]{JonesPolynomialPaper}
%V.~F.~R. Jones, \emph{A polynomial invariant for knots via von {N}eumann
%  algebras}, Bull. Amer. Math. Soc. \textbf{12} (1985), 103--111.
%
%\bibitem[Kap]{k1004.2307}
%A.~Kapustin, \emph{Topological {F}ield {T}heory, {H}igher {C}ategories, and
%  {T}heir {A}pplications},
%  \href{http://arxiv.org/abs/1004.2307}{[arXiv:1004.2307]}.
%
%\bibitem[Kaw96]{kawauchibook}
%A.~Kawauchi, \emph{A {S}urvey of {K}not {T}heory}, Birkh\"auser, 1996.
%
%\bibitem[Kho00]{k9908171}
%M.~Khovanov, \emph{A categorification of the {J}ones polynomial}, Duke Math. J.
%  \textbf{101} (2000), 359--426,
%  \href{http://arxiv.org/abs/math/9908171}{[math.QA/9908171]}.
%
%\bibitem[Kho07]{k0510265}
%M.~Khovanov, \emph{Triply-graded link homology and Hochschild homology of Soergel bimodules},
%  Int. Journal of Math. \textbf{18} (2007), 869--885,
%  \href{http://arxiv.org/abs/math/0510265}{[math.GT/0510265]}.
%
%\bibitem[KR07a]{kr0701333}
%M.~Khovanov and L.~Rozansky, \emph{Virtual crossings, convolutions and a
%  categorification of the {$\operatorname{SO}(2N)$} {K}auffman polynomial},
%  Journal of G\"okova Geometry Topology \textbf{1} (2007), 116--214,
%  \href{http://arxiv.org/abs/math/0701333}{[math.QA/0701333]}.
%
%\bibitem[KR07b]{kr0404189}
%M.~Khovanov and L.~Rozansky, \emph{Topological Landau-Ginzburg models on the world-sheet foam},
%  Adv. Theor. Math. Phys. \textbf{11} (2007), 233--259,
%  \href{http://arxiv.org/abs/hep-th/0404189}{[hep-th/0404189]}.
%
%\bibitem[KR08a]{kr0401268}
%M.~Khovanov and L.~Rozansky, \emph{Matrix factorizations and link homology}, Fund. Math.
%  \textbf{199} (2008), 1--91,
%  \href{http://arxiv.org/abs/math/0401268}{[math/0401268]}.
%
%\bibitem[KR08b]{kr0505056}
%M.~Khovanov and L.~Rozansky, \emph{Matrix factorizations and link homology {II}}, Geometry \&
%  Topology \textbf{12} (2008), 1387--1425,
%  \href{http://arxiv.org/abs/math/0505056}{[math.QA/0505056]}.
%
%\bibitem[Lam86]{LambekRingsModules}
%J.~Lambek, \emph{Lectures on rings and modules}, AMS Chelsea Publishing, 1986.
%
%\bibitem[LM]{Calinetal2}
%C.~I. Lazaroiu and D.~McNamee, unpublished.
%
%\bibitem[LMnV00]{lmv0010102}
%J.~M.~F. Labastida, M.~Mari\~{n}o, and C.~Vafa, \emph{Knot {I}nvariants and
%  {T}opological {S}trings}, JHEP \textbf{0011} (2000), 007,
%  \href{http://arxiv.org/abs/hep-th/0010102}{[hep-th/0010102]}.
%
%\bibitem[MSV09]{msv0708.2228}
%M.~Mackaay, M.~Sto\v{s}i\'{c}, and P.~Vaz, \emph{$\mathfrak{sl}(N)$ link homology ($N\geq 4$) using foams and the Kapustin-Li formula}, Geometry \& Topology \textbf{13} (2009), 1075--1128,
%  \href{http://arxiv.org/abs/0708.2228}{[arXiv:0708.2228]}.
%
%\bibitem[Man07]{m0601629}
%C.~Manolescu, \emph{Link homology theories from symplectic geometry}, Adv. in
%  Math. \textbf{211} (2007), 363--416,
%  \href{http://www.arxiv.org/abs/math.AG/0601629}{[math.SG/0601629]}.
%
%\bibitem[McN09]{McNameethesis}
%D.~McNamee, \emph{On the mathematical structure of topological defects in
%  {L}andau-{G}inzburg models}, MSc Thesis, Trinity College Dublin, 2009.
%
%\bibitem[Mn05]{marinoknotbook}
%M.~Mari\~{n}o, \emph{Chern-{S}imons {T}heory, {M}atrix {M}odels, and
%  {T}opological s{t}rings}, Oxford University Press, 2005.
%
%\bibitem[MOY98]{moy1998}
%H.~Murakami, T.~Ohtsuki, and S.~Yamada, \emph{Homfly polynomial via an
%  invariant of colored plane graphs}, Enseign. Math. \textbf{44} (1998),
%  325--360.
%
%\bibitem[MS]{ms0709.1971}
%V.~Manzorchuck and C.~Stroppel, \emph{A combinatorial approach to functorial
%  quantum {$\mathfrak{sl}_k$} knot invariants},
%  \href{http://arxiv.org/abs/0709.1971}{[arXiv:0709.1971]}.
%
%\bibitem[OV00]{ov9912123}
%H.~Ooguri and C.~Vafa, \emph{Knot {I}nvariants and {T}opological {S}trings},
%  Nucl. Phys. B \textbf{577} (2000), 419--438,
%  \href{http://arxiv.org/abs/hep-th/9912123}{[hep-th/9912123]}.
%
%\bibitem[PT87]{HomflyPT}
%J.~Przytycki and P.~Traczyk, \emph{Conway algebras and skein equivalence of
%  links}, Proc. Amer. Math. Soc. \textbf{100} (1987), 744--748.
%
%\bibitem[Ras]{r0607544}
%J.~Rasmussen, \emph{Some differentials on {K}hovanov-{R}ozansky homology},
%  \href{http://arxiv.org/abs/math/0607544}{[math.GT/0607544]}.
%
%\bibitem[Ras07]{r0508510}
%J.~Rasmussen, \emph{Khovanov-{R}ozansky homology of two-bridge knots and links},
%  Duke Math. J. \textbf{136} (2007), 551--583,
%  \href{http://arxiv.org/abs/math/0508510}{[math.GT/0508510]}.
%  
%\bibitem[Ric94]{rickard}
%J.~Rickard, \emph{Translation functors and equivalences of derived categories for blocks of algebraic groups}, in “Finite dimensional algebras and related topics”, Kluwer (1994), 255-–264.
%
%\bibitem[Rou06]{RouquierMexico}
%R.~Rouquier, \emph{Categorification of {$\mathfrak{sl}_{2}$} and braid groups},
%  Trends in representation theory of algebras and related topics (2006),
%  137--167.
%
%\bibitem[RT90]{RT1990}
%N.~Reshetikhin and V.~Turaev, \emph{Ribbon graphs and their invariants derived
%  from quantum groups}, Comm. Math. Phys. \textbf{127} (190), 1--26.
%
%\bibitem[RT91]{RT1991}
%N.~Reshetikhin and V.~Turaev, \emph{Invariants of 3-manifolds via link polynomials and quantum
%  groups}, Invent. Math. \textbf{103} (1991), 547--597.
%
%\bibitem[SS06]{ss0405089}
%P.~Seidel and I.~Smith, \emph{A link invariant from the symplectic geometry of
%  nilpotent slices}, Duke Math. J. \textbf{134} (2006), 453--514,
%  \href{http://arxiv.org/abs/math/0405089}{[math.SG/0405089]}.
%
%\bibitem[Str05]{sCatTLcTCpf}
%C.~Stroppel, \emph{Categorification of the {T}emperley-{L}ieb category,
%  tangles, and cobordisms via projective functors}, Duke Math. J. \textbf{126}
%  (2005), 547--596.
%
%\bibitem[Sus]{s0701045}
%J.~Sussan, \emph{Category {$\mathcal O$} and {$\mathfrak{sl}_k$} link
%  invariants}, \href{http://arxiv.org/abs/math/0701045}{[math.QA/0701045]}.
%
%\bibitem[Tur88]{t1988YB}
%V.~Turaev, \emph{The {Y}ang-{B}axter equation and invariants of links}, Invent.
%  Math. \textbf{92} (1988), 527--553.
%
%\bibitem[Tur10]{turaevbook}
%V.~Turaev, \emph{Quantum invariants of knots and 3-manifolds}, de Gruyter, 2010,
%  2nd edition.
%
%\bibitem[Weba]{w0610650}
%B.~Webster, \emph{Khovanov-Rozansky homology via a canopolis formalism},
%  \href{http://arxiv.org/abs/math/0610650}{[math.GT/0610650]}.
%
%\bibitem[Webb]{w1005.4559}
%B.~Webster, \emph{Knot invariants and higher representation theory {II}: the
%  categorification of quantum knot invariants},
%  \href{http://arxiv.org/abs/1005.4559}{[arXiv:1005.4559]}.
%
%\bibitem[Wit]{w1101.3216}
%E.~Witten, \emph{Fivebranes and {K}nots},
%  \href{http://arxiv.org/abs/1101.3216}{[arXiv:1101.3216]}.
%
%\bibitem[Wit89]{wittenjones}
%E.~Witten, \emph{Quantum field theory and the {J}ones polynomial}, Comm. Math.
%  Phys. \textbf{121} (1989), 351--399.
%
%\bibitem[Wit95]{w9207094}
%E.~Witten, \emph{Chern-{S}imons {G}auge {T}heory {A}s {A} {S}tring {T}heory},
%  Prog. Math. \textbf{133} (1995), 637--678,
%  \href{http://arxiv.org/abs/hep-th/9207094}{[hep-th/9207094]}.
%
%\bibitem[Wu]{w0907.0695}
%H.~Wu, \emph{A colored {$\mathfrak{sl}(N)$}-homology for links in {$S^3$}},
%  \href{http://arxiv.org/abs/0907.0695}{[arXiv:0907.0695]}.
%
%\bibitem[Wu08]{w0508064}
%H.~Wu, \emph{Braids, {T}ransversal links and the {K}hovanov-{R}ozansky {T}heory},
%  \href{http://arxiv.org/abs/math/0508064}{[math.GT/0508064]}, Trans. Amer. Math. Soc. \textbf{360}
%  (2008), 3365--3389.
%
%\bibitem[Yon]{y0906.0220}
%Y.~Yonezawa, \emph{Quantum {$(\mathfrak{sl}_n, \wedge V_n)$} link invariant and
%  matrix factorizations},
%  \href{http://arxiv.org/abs/0906.0220}{[arXiv:0906.0220]}.

\end{thebibliography}

\end{document}
